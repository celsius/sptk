\name{step}{generate step sequence}{signal generation}

\begin{synopsis}
\item[step] [ --l $L$ ] [ --n $N$ ]
\end{synopsis}

\begin{qsection}{DESCRIPTION}
This command generates a unit step sequence of length $L$ or order $N$,
and sends the results to the standard output.
That is, the sequence 
\begin{displaymath}
\underbrace{1, 1, 1, \ldots, 1}_{L}
\end{displaymath}
is outputed in float format.
\end{qsection}

\begin{options}
	\argm{l}{L}{length\\
                        In the case $L \le 0$ then values will be
                        generated indefinitely.}{256}
	\argm{n}{N}{order}{255}
\end{options}

\begin{qsection}{EXAMPLE}
In the following example, the unit step sequence passed through
a digital filter and sent to the standard output:
\begin{quote}
\verb!step | dfs -a 1 -0.8 | dmp!
\end{quote}
\end{qsection}

\begin{qsection}{SEE ALSO}
  impulse, train, ramp, sin
\end{qsection}
