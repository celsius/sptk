% ----------------------------------------------------------------- %
%             The Speech Signal Processing Toolkit (SPTK)           %
%             developed by SPTK Working Group                       %
%             http://sp-tk.sourceforge.net/                         %
% ----------------------------------------------------------------- %
%                                                                   %
%  Copyright (c) 1984-2007  Tokyo Institute of Technology           %
%                           Interdisciplinary Graduate School of    %
%                           Science and Engineering                 %
%                                                                   %
%                1996-2010  Nagoya Institute of Technology          %
%                           Department of Computer Science          %
%                                                                   %
% All rights reserved.                                              %
%                                                                   %
% Redistribution and use in source and binary forms, with or        %
% without modification, are permitted provided that the following   %
% conditions are met:                                               %
%                                                                   %
% - Redistributions of source code must retain the above copyright  %
%   notice, this list of conditions and the following disclaimer.   %
% - Redistributions in binary form must reproduce the above         %
%   copyright notice, this list of conditions and the following     %
%   disclaimer in the documentation and/or other materials provided %
%   with the distribution.                                          %
% - Neither the name of the SPTK working group nor the names of its %
%   contributors may be used to endorse or promote products derived %
%   from this software without specific prior written permission.   %
%                                                                   %
% THIS SOFTWARE IS PROVIDED BY THE COPYRIGHT HOLDERS AND            %
% CONTRIBUTORS "AS IS" AND ANY EXPRESS OR IMPLIED WARRANTIES,       %
% INCLUDING, BUT NOT LIMITED TO, THE IMPLIED WARRANTIES OF          %
% MERCHANTABILITY AND FITNESS FOR A PARTICULAR PURPOSE ARE          %
% DISCLAIMED. IN NO EVENT SHALL THE COPYRIGHT OWNER OR CONTRIBUTORS %
% BE LIABLE FOR ANY DIRECT, INDIRECT, INCIDENTAL, SPECIAL,          %
% EXEMPLARY, OR CONSEQUENTIAL DAMAGES (INCLUDING, BUT NOT LIMITED   %
% TO, PROCUREMENT OF SUBSTITUTE GOODS OR SERVICES; LOSS OF USE,     %
% DATA, OR PROFITS; OR BUSINESS INTERRUPTION) HOWEVER CAUSED AND ON %
% ANY THEORY OF LIABILITY, WHETHER IN CONTRACT, STRICT LIABILITY,   %
% OR TORT (INCLUDING NEGLIGENCE OR OTHERWISE) ARISING IN ANY WAY    %
% OUT OF THE USE OF THIS SOFTWARE, EVEN IF ADVISED OF THE           %
% POSSIBILITY OF SUCH DAMAGE.                                       %
% ----------------------------------------------------------------- %
\hypertarget{bcut}{}
\name{bcut}{binary file cut}{data operation}

\begin{synopsis}
\item[bcut] [ --s $S$ ] [ --e $E$ ] [ --l $L$ ] [ --n $N$ ] [ +{\em type} ] 
	    [ {\em infile} ] 
\end{synopsis}

\begin{qsection}{DESCRIPTION}
{\em bcut} copies a selected portion of {\em infile} (or standard input) 
to standard output.
\end{qsection}

\begin{options}
	\argm{s}{S}{start number}{0}
	\argm{e}{E}{end number}{EOF}
	\argm{l}{L}{block length}{1}
	\argm{n}{N}{block order}{L-1}
	\argp{t}{input data format\\ 
		\begin{tabular}{llcll} \\[-1ex]
         c & char (1 byte) & \quad &
         C & unsigned char (1 byte) \\
         s & short (2 bytes) & \quad &
                     S & unsigned short (2 bytes) \\
         i3 & int (3 bytes) & \quad &
                     I3 & unsigned int (3 bytes) \\
         i & int (4 bytes) & \quad &
                     I & unsigned int (4 bytes) \\
         l & long (4 bytes) & \quad &
                     L & unsigned long (4 bytes) \\
         le & long long (8 bytes) & \quad &
                     LE & unsigned long long (8 bytes) \\
         f & float (4 bytes) & \quad &
                     d & double (8 bytes)
		\end{tabular}\\\hspace*{\fill}}{f}
\end{options}

\begin{qsection}{EXAMPLE}
In the example below, the input file {\em data.f} in float format
is cut from the 3rd to the 5th float point:
\begin{quote}
 \verb!bcut +f -s 3 -e 5 data.f > data.cut!
\end{quote}
For example, if the file {\em data.f} had the following data
\begin{displaymath}
  1, 2, 3, 4, 5, 6, 7
\end{displaymath}
the output file {\em data.cut} would be 
\begin{displaymath}
  4, 5, 6.
\end{displaymath}
\par
If the block length is assigned:
\begin{quote}
 \verb!bcut +f -l 2 data.f -s 1 -e 2 > data.cut!
\end{quote}
then, the output file would contain the following data,
\begin{displaymath}
  3, 4, 5, 6
\end{displaymath}
\par
If the stationary part, say from the sample 100,
of the output of a digital filter excited with
pulse train is desired, then the following command can be used:
\begin{quote}
  \verb!train -p 10 -l 256 | dfs -a 1 0.8 0.6 | bcut +f -s 100 > data.cut!
\end{quote}
In this case, the file {\em data.cut} will contain 156 points.
\par
If we generate a {\em data.f} file passing a sinusoidal signal
through a 256-length window as follows
\begin{quote}
  \verb!sin -p 30 -l 2000 | window > data.f!
\end{quote}
and we want to take only the third window output,
we could use the following command:
\begin{quote}
  \verb!bcut +f -l 256 -s 3 -e 3 < data.f > data.cut!
\end{quote}
\end{qsection}

\begin{qsection}{SEE ALSO}
\hyperlink{bcp}{bcp},
\hyperlink{merge}{merge},
\hyperlink{reverse}{reverse}
\end{qsection}
