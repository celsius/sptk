\name{bcut}{binary file cut}{data operation}

\begin{synopsis}
\item[bcut] [ --s $S$ ] [ --e $E$ ] [ --l $L$ ] [ --n $N$ ] [ +{\em type} ] 
	    [ {\em infile} ] 
\end{synopsis}

\begin{qsection}{DESCRIPTION}
The {\em bcut} command cuts the assigned part of the input file.
If the input file is not assigend, data is read from the starndard
input.
\end{qsection}

\begin{options}
	\argm{s}{S}{start number}{0}
	\argm{e}{E}{end number}{EOF}
	\argm{l}{L}{block length}{1}
	\argm{n}{N}{block order}{L-1}
	\argp{t}{input data format\\ 
		\begin{tabular}{llcll} \\[-1zh]
			c & char (1byte) & \quad &
			s & short (2bytes) \\
			i & int (4bytes) & \quad &
			l & long (4bytes) \\
			f & float (4bytes) & \quad &
			d & double (8bytes) 
		\end{tabular}\\\hspace*{\fill}}{f}
\end{options}

\begin{qsection}{EXAMPLE}
In the example below, the input file {\em data.f} in float format
is cut from the 3rd to the 5th float point:
\begin{quote}
 \verb!bcut -s 3 -e 5 data.f > data.cut!
\end{quote}
For example, if the file {\em data.f} had the following data
\begin{displaymath}
  1, 2, 3, 4, 5, 6, 7
\end{displaymath}
the output file {\em data.cut} would be 
\begin{displaymath}
  4, 5, 6.
\end{displaymath}
\par
If block lenght is assigned:
\begin{quote}
 \verb!bcut -l 2 data.f -s 1 -e 2 > data.cut!
\end{quote}
then, the output file would contain following data,
\begin{displaymath}
  3, 4, 5, 6
\end{displaymath}
\par
If the stationary part, say from the sample 100,
of the output of digital filter excited with
pulse train is desired following command can be used:
\begin{quote}
  \verb!train -p 10 -l 256 | dfs -a 1 0.8 0.6 | bcut -s 100 > data.cut!
\end{quote}
In this case, the file {\em data.cut} will contain 156 points.
\par
If we generate a {\em data.f} file passing a sinusoidal signal
through a 256-length window as follows
\begin{quote}
  \verb!sin -p 30 -l 2000 | window > data.f!
\end{quote}
and we want to take only the 3rd window output,
we could use the following command:
\begin{quote}
  \verb!bcut -l 256 -s 3 -e 3 < data.f > data.cut!
\end{quote}
\end{qsection}

\begin{qsection}{SEE ALSO}
bcp,merge,reverse
\end{qsection}
