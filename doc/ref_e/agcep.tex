% ----------------------------------------------------------------- %
%             The Speech Signal Processing Toolkit (SPTK)           %
%             developed by SPTK Working Group                       %
%             http://sp-tk.sourceforge.net/                         %
% ----------------------------------------------------------------- %
%                                                                   %
%  Copyright (c) 1984-2007  Tokyo Institute of Technology           %
%                           Interdisciplinary Graduate School of    %
%                           Science and Engineering                 %
%                                                                   %
%                1996-2008  Nagoya Institute of Technology          %
%                           Department of Computer Science          %
%                                                                   %
% All rights reserved.                                              %
%                                                                   %
% Redistribution and use in source and binary forms, with or        %
% without modification, are permitted provided that the following   %
% conditions are met:                                               %
%                                                                   %
% - Redistributions of source code must retain the above copyright  %
%   notice, this list of conditions and the following disclaimer.   %
% - Redistributions in binary form must reproduce the above         %
%   copyright notice, this list of conditions and the following     %
%   disclaimer in the documentation and/or other materials provided %
%   with the distribution.                                          %
% - Neither the name of the SPTK working group nor the names of its %
%   contributors may be used to endorse or promote products derived %
%   from this software without specific prior written permission.   %
%                                                                   %
% THIS SOFTWARE IS PROVIDED BY THE COPYRIGHT HOLDERS AND            %
% CONTRIBUTORS "AS IS" AND ANY EXPRESS OR IMPLIED WARRANTIES,       %
% INCLUDING, BUT NOT LIMITED TO, THE IMPLIED WARRANTIES OF          %
% MERCHANTABILITY AND FITNESS FOR A PARTICULAR PURPOSE ARE          %
% DISCLAIMED. IN NO EVENT SHALL THE COPYRIGHT OWNER OR CONTRIBUTORS %
% BE LIABLE FOR ANY DIRECT, INDIRECT, INCIDENTAL, SPECIAL,          %
% EXEMPLARY, OR CONSEQUENTIAL DAMAGES (INCLUDING, BUT NOT LIMITED   %
% TO, PROCUREMENT OF SUBSTITUTE GOODS OR SERVICES; LOSS OF USE,     %
% DATA, OR PROFITS; OR BUSINESS INTERRUPTION) HOWEVER CAUSED AND ON %
% ANY THEORY OF LIABILITY, WHETHER IN CONTRACT, STRICT LIABILITY,   %
% OR TORT (INCLUDING NEGLIGENCE OR OTHERWISE) ARISING IN ANY WAY    %
% OUT OF THE USE OF THIS SOFTWARE, EVEN IF ADVISED OF THE           %
% POSSIBILITY OF SUCH DAMAGE.                                       %
% ----------------------------------------------------------------- %
\hypertarget{agcep}{}
\name[ref:agcep-IEICEtaikai90s]{agcep}{adaptive generalized cepstral analysis}%
{speech analysis}

\begin{synopsis}
\item [agcep] [ --m $M$ ] [ --g $G$ ] [ --l $L$ ] [ --t $T$] [ --k $K$ ]
	      [ --p $P$ ]
\item [\ ~~~~~~]  [ --s ] [ --n ] [ --e $E$ ] [ {\em pefile} ] $<$ {\em infile}
\end{synopsis}

\begin{qsection}{DESCRIPTION}
	{\em agcep} uses adaptive generalized cepstral analysis
	\cite{ref:agcep-IEICEtaikai90s}
	to calculate cepstral coefficients $c_\gamma(m)$ 
	from unframed float data from standard input,
	sending the result to standard output. 
	If {\em pefile} is given, 
	{\em agcep} writes the prediction error to that file.

	The format for input and output data is float.

	The algorithm to calculate recursively the
        adaptive generalized cepstral coefficients is 
 %
\begin{align}
  \bc_\gamma^{(n+1)} &= \bc_\gamma^{(n)} 
     - \mu^{(n)} \hat{\nabla} \varepsilon_{\tau}^{(n)} \notag \\
  \hat{\nabla} \varepsilon_{0}^{(n)} &= -2 e_\gamma (n) \be_\gamma^{(n)}
  \qquad ( \tau = 0 ) \notag \\
  \hat{\nabla} \varepsilon_{\tau}^{(n)} &= -2 (1 - \tau) \sum_{i=-\infty}^{n}
  \tau^{n-i} e_\gamma (i) \be_\gamma^{(i)} \qquad ( 0 \le \tau < 1 ) \notag \\
  \hat{\nabla} \varepsilon_{\tau}^{(n)} &= \tau \hat{\nabla}  
  \varepsilon_{\tau}^{(n-1)} - 2 (1 - \tau) e_\gamma (n) \be_\gamma^{(n)}
  \notag \\
  \mu^{(n)} &= \frac{k}{M \varepsilon^{(n)}} \notag \\
  \varepsilon^{(n)} &= \lambda \varepsilon^{(n-1)}
     + (1-\lambda)e_\gamma^2(n) \notag
\end{align}	

where
$\bc_\gamma = [c_\gamma(1),\ldots,c_\gamma(M)]^\top$,
$\be_\gamma = [e_\gamma(n-1),\ldots,e_\gamma(n-M)]^\top$.
The signal $e_\gamma(n)$ is obtained passing the input signal
 $x(n)$ through the filter $(1+\gamma F(z))^{-\frac{1}{\gamma}-1}$,
where 
\begin{displaymath}
F(z) = \sum_{m=1}^{M}c_\gamma(m)z^{-m}.
\end{displaymath}
\par
In the case $\gamma = -1/n$ , where $n$ is a natural number,
the adaptive generalized cepstral analysis system is shown in 
Figure \ref{fig:agcep_block}.
In the case $n=1$, the adaptive generalized cepstral
analysis is equivalent to the LMS linear predictor.
In the case $n \rightarrow \infty$,
the adaptive generalized cepstral
analysis is equivalent to the 
adaptive cepstral analysis.

%\def\topfraction{.8}
%\def\textfraction{.1}
%\def\floatpagefraction{.8}
\setcounter{figure}{0}
\newpage
\begin{figure}[t]
\begin{center}
\setlength{\unitlength}{0.01200in}%
%
\begingroup\makeatletter\ifx\SetFigFont\undefined
% extract first six characters in \fmtname
\def\x#1#2#3#4#5#6#7\relax{\def\x{#1#2#3#4#5#6}}%
\expandafter\x\fmtname xxxxxx\relax \def\y{splain}%
\ifx\x\y   % LaTeX or SliTeX?
\gdef\SetFigFont#1#2#3{%
  \ifnum #1<17\tiny\else \ifnum #1<20\small\else
  \ifnum #1<24\normalsize\else \ifnum #1<29\large\else
  \ifnum #1<34\Large\else \ifnum #1<41\LARGE\else
     \huge\fi\fi\fi\fi\fi\fi
  \csname #3\endcsname}%
\else
\gdef\SetFigFont#1#2#3{\begingroup
  \count@#1\relax \ifnum 25<\count@\count@25\fi
  \def\x{\endgroup\@setsize\SetFigFont{#2pt}}%
  \expandafter\x
    \csname \romannumeral\the\count@ pt\expandafter\endcsname
    \csname @\romannumeral\the\count@ pt\endcsname
  \csname #3\endcsname}%
\fi
\fi\endgroup
\begin{picture}(440,323)(114,420)
\thinlines
\put(440,475){\vector( 1, 0){ 60}}
\put(340,455){\framebox(100,40){}}
\put(280,475){\line( 1, 0){ 60}}
\put(390,470){\makebox(0,0)[b]{\smash{\SetFigFont{12}{14.4}{rm}$\exp F(z)$}}}
\put(300,485){\makebox(0,0)[b]{\smash{\SetFigFont{12}{14.4}{rm}$x(n)$}}}
\put(500,485){\makebox(0,0)[b]{\smash{\SetFigFont{12}{14.4}{rm}$e(n)=e_\gamma(n)$}}}
\put(280,600){\line( 1, 0){ 60}}
\put(340,580){\framebox(100,40){}}
\put(440,600){\vector( 1, 0){ 60}}
\put(480,610){\makebox(0,0)[b]{\smash{\SetFigFont{12}{14.4}{rm}$e(n)$}}}
\put(280,610){\makebox(0,0)[b]{\smash{\SetFigFont{12}{14.4}{rm}$x(n)=e_\gamma(n)$}}}
\put(390,595){\makebox(0,0)[b]{\smash{\SetFigFont{12}{14.4}{rm}$1-F(z)$}}}
\put(180,720){\line(-1, 0){ 60}}
\put(440,720){\vector( 1, 0){ 60}}
\put(340,700){\framebox(100,40){}}
\put(280,720){\line( 1, 0){ 60}}
\put(180,700){\framebox(100,40){}}
\put(340,420){\makebox(0,0)[b]{\smash{\SetFigFont{12}{14.4}{rm}(c) $\gamma=0$}}}
\put(340,540){\makebox(0,0)[b]{\smash{\SetFigFont{12}{14.4}{rm}(b) $\gamma=-1$}}}
\put(340,660){\makebox(0,0)[b]{\smash{\SetFigFont{12}{14.4}{rm}(a) $-1\leq \gamma \leq 0$}}}
\put(390,715){\makebox(0,0)[b]{\smash{\SetFigFont{12}{14.4}{rm}$1+\gamma F(z)$}}}
\put(310,730){\makebox(0,0)[b]{\smash{\SetFigFont{12}{14.4}{rm}$e_{\gamma}(n)$}}}
\put(140,730){\makebox(0,0)[b]{\smash{\SetFigFont{12}{14.4}{rm}$x(n)$}}}
\put(480,730){\makebox(0,0)[b]{\smash{\SetFigFont{12}{14.4}{rm}$e(n)$}}}
\put(230,715){\makebox(0,0)[b]{\smash{\SetFigFont{12}{14.4}{rm}$(1+\gamma F(z))^{-\frac{1}{\gamma}-1}$}}}
\end{picture}
\caption{Adaptive generalized cepstral analysis system}
\label{fig:agcep_block}
\end{center}
\end{figure}

\end{qsection}

\begin{options}
	\argm{m}{M}{order of generalized cepstrum}{25}
	\argm{g}{G}{power parameter $\gamma=-1/G$ for generalized cepstrum}{1}
	\argm{l}{L}{leakage factor $\lambda$}{0.98}
	\argm{t}{T}{momentum constant $\tau$}{0.9}
	\argm{k}{K}{step size $k$}{0.1}
	\argm{p}{P}{output period of generalized cepstrum}{1}
	\argm{s}{}{output smoothed generalized cepstrum}{FALSE}
	\argm{n}{}{output normalized generalized cepstrum}{FALSE}
	\argm{e}{E}{minimum value for $\varepsilon^{(n)}$}{0.0}
\end{options}

\begin{qsection}{EXAMPLE}
	In this example, the speech data is in the file {\em data.f}
        in float format, the cepstral coefficients are written to
        the file {\em data.agcep},
        and the prediction error can be found in {\em data.er}.
\begin{quote}
 \verb!agcep -m 15 data.er < data.f > data.agcep!
\end{quote} 
\end{qsection}

\begin{qsection}{SEE ALSO}
\hyperlink{acep}{acep},
\hyperlink{amcep}{amcep},
\hyperlink{glsadf}{glsadf}
\end{qsection}
