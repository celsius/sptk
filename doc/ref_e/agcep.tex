\name[ref:agcep-IEICEtaikai90s]{agcep}{adaptive generalized cepstral analysis}%
{$B2;@<J,@O(B}

\begin{synopsis}
\item [agcep] [ --m $M$ ] [ --g $G$ ] [ --l $L$ ] [ --t $T$] [ --k $K$ ]
	      [ --p $P$ ]
\item [\ ~~~~~~]  [ --s ] [ --n ] [ --e $E$ ] [ {\em pefile} ] $<$ {\em infile}
\end{synopsis}

\begin{qsection}{DESCRIPTION}
        This command undertakes adaptive generalized cepstrum analysis,
        sends the cepstrum coefficients $c_\gamma(m)$ to standard output,
        and the prediction error to {\em pefile}, whenever
        {\em pefile} is included in the command line.
 \par
	The format for input and output data is float.
 \par
	The algorithm to calculate recursively the
        adaptive generalized cepstrum coefficients is 
 %
\begin{eqnarray*}
  \mbox{\boldmath $c$}_\gamma^{(n+1)} &=& \mbox{\boldmath $c$}_\gamma^{(n)} 
     - \mu^{(n)} \hat{\nabla} \varepsilon_{\tau}^{(n)} \\
  \hat{\nabla} \varepsilon_{0}^{(n)} &=& -2 e_\gamma (n) \mbox{\boldmath $e$}_\gamma^{(n)} \hspace{3.5cm} ( \tau = 0 )\\
  \hat{\nabla} \varepsilon_{\tau}^{(n)} &=& -2 (1 - \tau) \sum_{i=-\infty}^{n} \tau^{n-i} e_\gamma (i) \mbox{\boldmath $e$}_\gamma^{(i)} ~~~ ( 0 \le \tau < 1 )\\
  \hat{\nabla} \varepsilon_{\tau}^{(n)} &=& \tau \hat{\nabla} \varepsilon_{\tau}^{(n-1)} - 2 (1 - \tau) e_\gamma (n) \mbox{\boldmath $e$}_\gamma^{(n)} \\
  \mu^{(n)} &=& \frac{k}{M \varepsilon^{(n)}} \\
  \varepsilon^{(n)} &=& \lambda \varepsilon^{(n-1)}
     + (1-\lambda)e_\gamma^2(n)
\end{eqnarray*}	

where
$\mbox{\boldmath$c$}_\gamma = [c_\gamma(1),\ldots,c_\gamma(M)]^T$,
$\mbox{\boldmath$e$}_\gamma = [e_\gamma(n-1),\ldots,e_\gamma(n-M)]^T$.
The signal $e_\gamma(n)$ is obtained passing the input signal
 $x(n)$ throught the filter $(1+\gamma F(z))^{-\frac{1}{\gamma}-1}$,
where 
\begin{displaymath}
F(z) = \sum_{m=1}^{M}c_\gamma(m)z^{-m}.
\end{displaymath}
\par
In the case $\gamma = -1/n$ , where $n$ is a natural number,
the adaptive generalized cepstrum analysis system is shown in 
Figure \ref{fig:agcep_block}.
In the case $n=1$, the adaptive generalized cepstrum
analysis is equivalent to the LMS linear predictor.
In the case $n \rightarrow \infty$,
the adaptive generalized cepstrum
analysis is equivalent to the 
adaptive cepstrum analysis.

%\def\topfraction{.8}
%\def\textfraction{.1}
%\def\floatpagefraction{.8}
\setcounter{figure}{0}
\newpage
\begin{figure}[t]
\begin{center}
\setlength{\unitlength}{0.01200in}%
%
\begingroup\makeatletter\ifx\SetFigFont\undefined
% extract first six characters in \fmtname
\def\x#1#2#3#4#5#6#7\relax{\def\x{#1#2#3#4#5#6}}%
\expandafter\x\fmtname xxxxxx\relax \def\y{splain}%
\ifx\x\y   % LaTeX or SliTeX?
\gdef\SetFigFont#1#2#3{%
  \ifnum #1<17\tiny\else \ifnum #1<20\small\else
  \ifnum #1<24\normalsize\else \ifnum #1<29\large\else
  \ifnum #1<34\Large\else \ifnum #1<41\LARGE\else
     \huge\fi\fi\fi\fi\fi\fi
  \csname #3\endcsname}%
\else
\gdef\SetFigFont#1#2#3{\begingroup
  \count@#1\relax \ifnum 25<\count@\count@25\fi
  \def\x{\endgroup\@setsize\SetFigFont{#2pt}}%
  \expandafter\x
    \csname \romannumeral\the\count@ pt\expandafter\endcsname
    \csname @\romannumeral\the\count@ pt\endcsname
  \csname #3\endcsname}%
\fi
\fi\endgroup
\begin{picture}(440,323)(114,420)
\thinlines
\put(440,475){\vector( 1, 0){ 60}}
\put(340,455){\framebox(100,40){}}
\put(280,475){\line( 1, 0){ 60}}
\put(390,470){\makebox(0,0)[b]{\smash{\SetFigFont{12}{14.4}{rm}$\exp F(z)$}}}
\put(300,485){\makebox(0,0)[b]{\smash{\SetFigFont{12}{14.4}{rm}$x(n)$}}}
\put(500,485){\makebox(0,0)[b]{\smash{\SetFigFont{12}{14.4}{rm}$e(n)=e_\gamma(n)$}}}
\put(280,600){\line( 1, 0){ 60}}
\put(340,580){\framebox(100,40){}}
\put(440,600){\vector( 1, 0){ 60}}
\put(480,610){\makebox(0,0)[b]{\smash{\SetFigFont{12}{14.4}{rm}$e(n)$}}}
\put(280,610){\makebox(0,0)[b]{\smash{\SetFigFont{12}{14.4}{rm}$x(n)=e_\gamma(n)$}}}
\put(390,595){\makebox(0,0)[b]{\smash{\SetFigFont{12}{14.4}{rm}$1-F(z)$}}}
\put(180,720){\line(-1, 0){ 60}}
\put(440,720){\vector( 1, 0){ 60}}
\put(340,700){\framebox(100,40){}}
\put(280,720){\line( 1, 0){ 60}}
\put(180,700){\framebox(100,40){}}
\put(340,420){\makebox(0,0)[b]{\smash{\SetFigFont{12}{14.4}{rm}(c) $\gamma=0$}}}
\put(340,540){\makebox(0,0)[b]{\smash{\SetFigFont{12}{14.4}{rm}(b) $\gamma=-1$}}}
\put(340,660){\makebox(0,0)[b]{\smash{\SetFigFont{12}{14.4}{rm}(a) $-1\leq \gamma \leq 0$}}}
\put(390,715){\makebox(0,0)[b]{\smash{\SetFigFont{12}{14.4}{rm}$1+\gamma F(z)$}}}
\put(310,730){\makebox(0,0)[b]{\smash{\SetFigFont{12}{14.4}{rm}$e_{\gamma}(n)$}}}
\put(140,730){\makebox(0,0)[b]{\smash{\SetFigFont{12}{14.4}{rm}$x(n)$}}}
\put(480,730){\makebox(0,0)[b]{\smash{\SetFigFont{12}{14.4}{rm}$e(n)$}}}
\put(230,715){\makebox(0,0)[b]{\smash{\SetFigFont{12}{14.4}{rm}$(1+\gamma F(z))^{-\frac{1}{\gamma}-1}$}}}
\end{picture}
\caption{Adaptive generalized cepstrum analysis system}
\label{fig:agcep_block}
\end{center}
\end{figure}

\end{qsection}

\begin{options}
	\argm{m}{M}{order of generalized cepstrum}{25}
	\argm{g}{G}{power parameter $\gamma$ of generalized cepstrum, \\
		 where if $G>1.0$ then $\gamma=-1/G$}{1}
	\argm{l}{L}{leakage factor $\lambda$}{0.98}
	\argm{t}{T}{momentum constant $\tau$}{0.9}
	\argm{k}{K}{step size $k$}{0.1}
	\argm{p}{P}{output period of generalized cepstrum}{1}
	\argm{s}{}{output smoothed generalized cepstrum}{FALSE}
	\argm{n}{}{output normalized generalized cepstrum}{FALSE}
	\argm{e}{E}{minimum value for $\varepsilon^{(n)}$}{0.0}
\end{options}

\begin{qsection}{EXAMPLE}
	In this example, the speech data is in the file {\em data.f}
        in float format, and the cepstrum coefficients are written in
        the file {\em data.agcep},
        and prediction error can be found in {\em data.er}.
\begin{quote}
 \verb!agcep -m 15 data.er < data.f > data.agcep!
\end{quote} 
\end{qsection}

\begin{qsection}{SEE ALSO}
acep, amcep, glsadf
\end{qsection}
