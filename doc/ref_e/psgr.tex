% ----------------------------------------------------------------
%       Speech Signal Processing Toolkit (SPTK): version 3.0
%                      SPTK Working Group
% 
%                Department of Computer Science
%                Nagoya Institute of Technology
%                             and
%   Interdisciplinary Graduate School of Science and Engineering
%                Tokyo Institute of Technology
%                   Copyright (c) 1984-2000
%                     All Rights Reserved.
% 
% Permission is hereby granted, free of charge, to use and
% distribute this software and its documentation without
% restriction, including without limitation the rights to use,
% copy, modify, merge, publish, distribute, sublicense, and/or
% sell copies of this work, and to permit persons to whom this
% work is furnished to do so, subject to the following conditions:
% 
%   1. The code must retain the above copyright notice, this list
%      of conditions and the following disclaimer.
% 
%   2. Any modifications must be clearly marked as such.
%                                                                        
% NAGOYA INSTITUTE OF TECHNOLOGY, TOKYO INSITITUTE OF TECHNOLOGY,
% SPTK WORKING GROUP, AND THE CONTRIBUTORS TO THIS WORK DISCLAIM
% ALL WARRANTIES WITH REGARD TO THIS SOFTWARE, INCLUDING ALL
% IMPLIED WARRANTIES OF MERCHANTABILITY AND FITNESS, IN NO EVENT
% SHALL NAGOYA INSTITUTE OF TECHNOLOGY, TOKYO INSITITUTE OF
% TECHNOLOGY, SPTK WORKING GROUP, NOR THE CONTRIBUTORS BE LIABLE
% FOR ANY SPECIAL, INDIRECT OR CONSEQUENTIAL DAMAGES OR ANY
% DAMAGES WHATSOEVER RESULTING FROM LOSS OF USE, DATA OR PROFITS,
% WHETHER IN AN ACTION OF CONTRACT, NEGLIGENCE OR OTHER TORTIOUS
% ACTION, ARISING OUT OF OR IN CONNECTION WITH THE USE OR
% PERFORMANCE OF THIS SOFTWARE.
% ----------------------------------------------------------------
%
\name{psgr}{XY-plotter simulator for EPSF}{plotting graphs}

\begin{synopsis}
 \item[psgr] [ --t {\em title} ] [ --s $S$ ] [ --c $C$ ] [ --x $X$ ]
[ --y $Y$ ] [ --p P ] [ --r $R$ ] [ --b ] 
\item[\ ~~~~~][ --T $T$ ] [ --B $B$ ]
[ --L $L$ ] [ --R $R$ ] [ --P ] [ {\em infile} ]
\end{synopsis}

\begin{qsection}{DESCRIPTION}
{\em psgr} converts FP5301 plotter commands 
from {\em infile} (or standard input) to PostScript (EPSF or PS), 
sending the result to standard output.
\end{qsection}

\begin{options}
	\argm{t}{title}{title of figure}{NULL}
	\argm{s}{S}{shrink}{1.0}
	\argm{c}{C}{number of copy}{1}
	\argm{x}{X}{x offset(mm)}{0}
	\argm{y}{Y}{y offset(mm)}{0}
	\argm{p}{P}{paper(Letter, A3, A4, A5, B4, B5)}{A4}
	\argm{l}{}{landscape}{FALSE}
	\argm{r}{R}{resolution(dpi)}{600}
	\argm{b}{}{bold font mode}{FALSE}
	\argm{T}{T}{top margin(mm)}{0}
	\argm{B}{B}{bottom margin(mm)}{0}
	\argm{L}{L}{left margin(mm)}{0}
	\argm{R}{R}{right margin(mm)}{0}
	\argm{P}{}{output Postscript code}{FALSE}
\end{options}

\begin{qsection}{EXAMPLE}
This example sends to a printer a figure file {\em data.fig}
written through the fig command:
\begin{quote}
 \verb!fig data.fig | psgr | lpr!
\end{quote}
\end{qsection}

\begin{qsection}{BUGS}
\begin{itemize}
\item There is possibility that a part of the Y axis label
is not properly outputed. In this case the user can 
change the margin to solve this problem.

\item In the case that the size of the figure is modified,
and included in a \TeX file, there is possibility that
it dose not appear correctly.
To solve this problem, please use \TeX options for including 
pictures and corresponding sizes.
\end{itemize}
\end{qsection}

\begin{qsection}{SEE ALSO}
 fig, fdrw, xgr
\end{qsection}
