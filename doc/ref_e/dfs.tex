% ----------------------------------------------------------------
%       Speech Signal Processing Toolkit (SPTK): version 3.0
%                      SPTK Working Group
% 
%                Department of Computer Science
%                Nagoya Institute of Technology
%                             and
%   Interdisciplinary Graduate School of Science and Engineering
%                Tokyo Institute of Technology
%                   Copyright (c) 1984-2000
%                     All Rights Reserved.
% 
% Permission is hereby granted, free of charge, to use and
% distribute this software and its documentation without
% restriction, including without limitation the rights to use,
% copy, modify, merge, publish, distribute, sublicense, and/or
% sell copies of this work, and to permit persons to whom this
% work is furnished to do so, subject to the following conditions:
% 
%   1. The code must retain the above copyright notice, this list
%      of conditions and the following disclaimer.
% 
%   2. Any modifications must be clearly marked as such.
%                                                                        
% NAGOYA INSTITUTE OF TECHNOLOGY, TOKYO INSITITUTE OF TECHNOLOGY,
% SPTK WORKING GROUP, AND THE CONTRIBUTORS TO THIS WORK DISCLAIM
% ALL WARRANTIES WITH REGARD TO THIS SOFTWARE, INCLUDING ALL
% IMPLIED WARRANTIES OF MERCHANTABILITY AND FITNESS, IN NO EVENT
% SHALL NAGOYA INSTITUTE OF TECHNOLOGY, TOKYO INSITITUTE OF
% TECHNOLOGY, SPTK WORKING GROUP, NOR THE CONTRIBUTORS BE LIABLE
% FOR ANY SPECIAL, INDIRECT OR CONSEQUENTIAL DAMAGES OR ANY
% DAMAGES WHATSOEVER RESULTING FROM LOSS OF USE, DATA OR PROFITS,
% WHETHER IN AN ACTION OF CONTRACT, NEGLIGENCE OR OTHER TORTIOUS
% ACTION, ARISING OUT OF OR IN CONNECTION WITH THE USE OR
% PERFORMANCE OF THIS SOFTWARE.
% ----------------------------------------------------------------
%
\name{dfs}{digital filter in standard form}{digital filter}

\begin{synopsis}
\item[dfs] [ --a $K$ $a(1)$ $\dots$ $a(M)$ ] 
	   [ --b $b(0)$ $b(1)$ $\dots$ $b(N)$ ] 
	   [ --p {\em pfile} ] [ --z {\em zfile} ]
\item[\ ~~~] [ {\em infile} ]
\end{synopsis}

\begin{qsection}{DESCRIPTION}
{\em dfs} filters data from {\em infile} (or standard output) 
with a digital filter in standard form, 
sending the result to standard output.
  The filter transfer function is
\begin{displaymath}
  H(z) 
  = 
  K\frac{\displaystyle{\sum_{n=0}^{N}{b(n)z^{-n}}}}{1+\displaystyle{\sum_{m=1}^{M}{a(m)z^{-m}}}}
\end{displaymath}
\par
The format of input and output data is float.
\end{qsection}

\begin{options}
	\argm{a}{K \; a(1) \dots a(M)}{denominator coefficients,
                       where $K$ is the gain of the transfer function.}{N/A}
	\argm{b}{b(0) \; b(1) \dots b(N)}{numerator coefficients}
		{N/A}
	\argm{p}{pfile}{denominator coefficients file in float format
                as follows\\
		\hspace*{2ex}$K, a(1), \ldots, a(M)$\\[-1ex]}{NULL}
	\argm{z}{zfile}{numerator coefficients file in float format
                as follows\\
		\hspace*{2ex}$b(0), b(1), \ldots, b(N)$\\[-1ex]}{NULL}
	\desc{If {\bf --a}, {\bf --p} options are not assigned,
              the denominator and $K$ are made equal to 1.
              If {\bf --b}, {\bf --z} options are not assigned,
              the numerator is made equal to 1.}
\end{options}

\begin{qsection}{EXAMPLE}
If we want to see the impulse response of the following transfer
 function
\begin{displaymath}
  H(z)=\frac{1+2z^{-1}+z^{-2}}{1+0.9z^{-1}}
\end{displaymath}
the command below can be used
\begin{quote}
 \verb!impulse | dfs -a 1 0.9 -b 1 2 1 | dmp!
\end{quote}
\par
If we want to see the frequency response plot of the digital filter
whose coefficients are defined in float form by the files
{\em data.p, data.z}, then we can use the following:
\begin{quote}
 \verb!impulse | dfs -p data.p -z data.z | spec | fdrw | xgr!
\end{quote}
The files {\em data.p} and {\em data.z} can be constructed
using the command {\em x2x}.
\end{qsection}
