\name{lspdf}{LSP speech synthesis digital filter}{digital filter}

\begin{synopsis}
\item [lspdf] [ --m $M$ ] [ --p $P$ ] [ --i $I$ ] [ --s $S$ ] [ --o $O$ ] 
              [ --k ] {\em lspfile} [ {\em infile} ] 
\end{synopsis}

\begin{qsection}{DESCRIPTION}
The {\em lspdf} command reads excitation information from
{\em infile},  passes it through an LSP digital filter
obtained from the coefficients read from {\em lspfile},
and sends the results to the standard output.
\par
Input and output are in float format.
\end{qsection}

\begin{options}
	\argm{m}{M}{order of coefficients}{25}
	\argm{p}{P}{frame period}{100}
	\argm{i}{I}{interpolation period}{1}
	\argm{k}{}{filtering without gain}{FALSE}
\end{options}

\begin{qsection}{EXAMPLE}
In the example below, excitation is generated from
pitch information given in {\em data.pitch} in float format,
this excitation is passed through the LSP synthesis filter
constructed from the LSP file {\em data.lsp},
and the synthesized speech is written to {\em data.syn}:
\begin{quote}
\verb! excite < data.pitch | lspdf data.lsp > data.syn!
\end{quote}
\end{qsection}

\begin{qsection}{SEE ALSO}
 lps, lpc2lsp
\end{qsection}
