% ----------------------------------------------------------------- %
%             The Speech Signal Processing Toolkit (SPTK)           %
%             developed by SPTK Working Group                       %
%             http://sp-tk.sourceforge.net/                         %
% ----------------------------------------------------------------- %
%                                                                   %
%  Copyright (c) 1984-2007  Tokyo Institute of Technology           %
%                           Interdisciplinary Graduate School of    %
%                           Science and Engineering                 %
%                                                                   %
%                1996-2008  Nagoya Institute of Technology          %
%                           Department of Computer Science          %
%                                                                   %
% All rights reserved.                                              %
%                                                                   %
% Redistribution and use in source and binary forms, with or        %
% without modification, are permitted provided that the following   %
% conditions are met:                                               %
%                                                                   %
% - Redistributions of source code must retain the above copyright  %
%   notice, this list of conditions and the following disclaimer.   %
% - Redistributions in binary form must reproduce the above         %
%   copyright notice, this list of conditions and the following     %
%   disclaimer in the documentation and/or other materials provided %
%   with the distribution.                                          %
% - Neither the name of the SPTK working group nor the names of its %
%   contributors may be used to endorse or promote products derived %
%   from this software without specific prior written permission.   %
%                                                                   %
% THIS SOFTWARE IS PROVIDED BY THE COPYRIGHT HOLDERS AND            %
% CONTRIBUTORS "AS IS" AND ANY EXPRESS OR IMPLIED WARRANTIES,       %
% INCLUDING, BUT NOT LIMITED TO, THE IMPLIED WARRANTIES OF          %
% MERCHANTABILITY AND FITNESS FOR A PARTICULAR PURPOSE ARE          %
% DISCLAIMED. IN NO EVENT SHALL THE COPYRIGHT OWNER OR CONTRIBUTORS %
% BE LIABLE FOR ANY DIRECT, INDIRECT, INCIDENTAL, SPECIAL,          %
% EXEMPLARY, OR CONSEQUENTIAL DAMAGES (INCLUDING, BUT NOT LIMITED   %
% TO, PROCUREMENT OF SUBSTITUTE GOODS OR SERVICES; LOSS OF USE,     %
% DATA, OR PROFITS; OR BUSINESS INTERRUPTION) HOWEVER CAUSED AND ON %
% ANY THEORY OF LIABILITY, WHETHER IN CONTRACT, STRICT LIABILITY,   %
% OR TORT (INCLUDING NEGLIGENCE OR OTHERWISE) ARISING IN ANY WAY    %
% OUT OF THE USE OF THIS SOFTWARE, EVEN IF ADVISED OF THE           %
% POSSIBILITY OF SUCH DAMAGE.                                       %
% ----------------------------------------------------------------- %
\hypertarget{lspdf}{}
\name{lspdf}{LSP speech synthesis digital filter}{filters for speech synthesis}

\begin{synopsis}
\item [lspdf] [ --m $M$ ] [ --p $P$ ] [ --i $I$ ] [ --s $S$ ] [ --o $O$ ] 
              [ --k ] [ --l ] {\em lspfile} [ {\em infile} ] 
\end{synopsis}

\begin{qsection}{DESCRIPTION}
{\em lspdf} derives an LSP digital filter 
from line spectral pair (LSP) coefficients in {\em lspfile} 
and uses it to filter an excitation sequence
from {\em infile} (or standard input) to synthesize speech data, 
sending the result to standard output.

Input and output are in float format.
\end{qsection}

\begin{options}
	\argm{m}{M}{order of coefficients}{25}
	\argm{p}{P}{frame period}{100}
	\argm{i}{I}{interpolation period}{1}
	\argm{k}{}{filtering without gain}{FALSE}
	\argm{l}{}{regard input as log gain}{FALSE}
\end{options}

\begin{qsection}{EXAMPLE}
In the example below, excitation is generated from
pitch information given in {\em data.pitch} in float format,
this excitation is passed through the LSP synthesis filter
constructed from the LSP file {\em data.lsp},
and the synthesized speech is written to {\em data.syn}:
\begin{quote}
\verb! excite < data.pitch | lspdf data.lsp > data.syn!
\end{quote}
\end{qsection}

\begin{qsection}{SEE ALSO}
\hyperlink{lsp}{lsp},
\hyperlink{lpc2lsp}{lpc2lsp}
\end{qsection}
