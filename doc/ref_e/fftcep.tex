% ----------------------------------------------------------------- %
%             The Speech Signal Processing Toolkit (SPTK)           %
%             developed by SPTK Working Group                       %
%             http://sp-tk.sourceforge.net/                         %
% ----------------------------------------------------------------- %
%                                                                   %
%  Copyright (c) 1984-2007  Tokyo Institute of Technology           %
%                           Interdisciplinary Graduate School of    %
%                           Science and Engineering                 %
%                                                                   %
%                1996-2016  Nagoya Institute of Technology          %
%                           Department of Computer Science          %
%                                                                   %
% All rights reserved.                                              %
%                                                                   %
% Redistribution and use in source and binary forms, with or        %
% without modification, are permitted provided that the following   %
% conditions are met:                                               %
%                                                                   %
% - Redistributions of source code must retain the above copyright  %
%   notice, this list of conditions and the following disclaimer.   %
% - Redistributions in binary form must reproduce the above         %
%   copyright notice, this list of conditions and the following     %
%   disclaimer in the documentation and/or other materials provided %
%   with the distribution.                                          %
% - Neither the name of the SPTK working group nor the names of its %
%   contributors may be used to endorse or promote products derived %
%   from this software without specific prior written permission.   %
%                                                                   %
% THIS SOFTWARE IS PROVIDED BY THE COPYRIGHT HOLDERS AND            %
% CONTRIBUTORS "AS IS" AND ANY EXPRESS OR IMPLIED WARRANTIES,       %
% INCLUDING, BUT NOT LIMITED TO, THE IMPLIED WARRANTIES OF          %
% MERCHANTABILITY AND FITNESS FOR A PARTICULAR PURPOSE ARE          %
% DISCLAIMED. IN NO EVENT SHALL THE COPYRIGHT OWNER OR CONTRIBUTORS %
% BE LIABLE FOR ANY DIRECT, INDIRECT, INCIDENTAL, SPECIAL,          %
% EXEMPLARY, OR CONSEQUENTIAL DAMAGES (INCLUDING, BUT NOT LIMITED   %
% TO, PROCUREMENT OF SUBSTITUTE GOODS OR SERVICES; LOSS OF USE,     %
% DATA, OR PROFITS; OR BUSINESS INTERRUPTION) HOWEVER CAUSED AND ON %
% ANY THEORY OF LIABILITY, WHETHER IN CONTRACT, STRICT LIABILITY,   %
% OR TORT (INCLUDING NEGLIGENCE OR OTHERWISE) ARISING IN ANY WAY    %
% OUT OF THE USE OF THIS SOFTWARE, EVEN IF ADVISED OF THE           %
% POSSIBILITY OF SUCH DAMAGE.                                       %
% ----------------------------------------------------------------- %
\hypertarget{fftcep}{}
\name{fftcep}{FFT cepstral analysis}{signal processing}

\begin{synopsis}
\item[fftcep] [ --m $M$ ] [ --l $L$ ] [ --j $J$ ] [ --k $K$ ] 
	    [ --e $E$ ] [ {\em infile} ] 
\end{synopsis}

\begin{qsection}{DESCRIPTION}
{\em fftcep} uses FFT cepstral analysis to calculate the cepstrum 
from windowed framed input data in {\em infile} (or standard input), 
sending the result to standard output.
The windowed input time domain sequence of length $L$ is of the form:
\begin{displaymath}
  x(0),x(1),\dots,x(L-1)
\end{displaymath}
\par
Input and output data are in float format.
\par
Also, the improved cepstral analysis method \cite{ref:icep-IECE} may be used if the
number of iterations $J$ and the acceleration factor $K$ are given.
\end{qsection}

\begin{options}
	\argm{m}{M}{order of cepstrum}{25}
	\argm{l}{L}{frame length}{256}
	\argm{j}{J}{number of iteration}{0}
	\argm{k}{K}{acceleration factor}{0.0}
	\argm{e}{E}{epsilon}{0.0}
\end{options}

\begin{qsection}{EXAMPLE}
In the example below, speech data in float format is read from {\em data.f}.  The
frame length and frame period are of 400 and 80, respectively.
FFT with 512 points is then performed and
the resultant cepstral coefficients are output to {\em data.cep}:
\begin{quote}
  \verb! frame -p 80 -l 400 < data.f | window -l 400 -L 512 | \! \\
  \verb! fftcep -l 512 > data.cep!
\end{quote}
\end{qsection}

\begin{qsection}{NOTICE}
When --j and --k options are specified, improved cepstral analysis is performed.
\end{qsection}

\begin{qsection}{SEE ALSO}
\hyperlink{uels}{uels}
\end{qsection}
