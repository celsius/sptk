% ----------------------------------------------------------------
%       Speech Signal Processing Toolkit (SPTK): version 3.0
%                      SPTK Working Group
% 
%                Department of Computer Science
%                Nagoya Institute of Technology
%                             and
%   Interdisciplinary Graduate School of Science and Engineering
%                Tokyo Institute of Technology
%                   Copyright (c) 1984-2000
%                     All Rights Reserved.
% 
% Permission is hereby granted, free of charge, to use and
% distribute this software and its documentation without
% restriction, including without limitation the rights to use,
% copy, modify, merge, publish, distribute, sublicense, and/or
% sell copies of this work, and to permit persons to whom this
% work is furnished to do so, subject to the following conditions:
% 
%   1. The code must retain the above copyright notice, this list
%      of conditions and the following disclaimer.
% 
%   2. Any modifications must be clearly marked as such.
%                                                                        
% NAGOYA INSTITUTE OF TECHNOLOGY, TOKYO INSITITUTE OF TECHNOLOGY,
% SPTK WORKING GROUP, AND THE CONTRIBUTORS TO THIS WORK DISCLAIM
% ALL WARRANTIES WITH REGARD TO THIS SOFTWARE, INCLUDING ALL
% IMPLIED WARRANTIES OF MERCHANTABILITY AND FITNESS, IN NO EVENT
% SHALL NAGOYA INSTITUTE OF TECHNOLOGY, TOKYO INSITITUTE OF
% TECHNOLOGY, SPTK WORKING GROUP, NOR THE CONTRIBUTORS BE LIABLE
% FOR ANY SPECIAL, INDIRECT OR CONSEQUENTIAL DAMAGES OR ANY
% DAMAGES WHATSOEVER RESULTING FROM LOSS OF USE, DATA OR PROFITS,
% WHETHER IN AN ACTION OF CONTRACT, NEGLIGENCE OR OTHER TORTIOUS
% ACTION, ARISING OUT OF OR IN CONNECTION WITH THE USE OR
% PERFORMANCE OF THIS SOFTWARE.
% ----------------------------------------------------------------
%
\hypertarget{fftcep}{}
\name{fftcep}{FFT cepstral analysis}{signal processing}

\begin{synopsis}
\item[fftcep] [ --m $M$ ] [ --l $L$ ] [ --j $J$ ] [ --k $K$ ] 
	    [ --e $E$ ] [ {\em infile} ] 
\end{synopsis}

\begin{qsection}{DESCRIPTION}
{\em fftcep} uses FFT cepstral analysis to calculate the cepstrum 
from windowed framed input data from {\em infile} (or standard input), 
sending the result to standard output.
The windowed input time domain sequence of length $L$ is
\begin{displaymath}
  x(0),x(1),\dots,x(L-1)
\end{displaymath}
\par
Input and output data are in float format.
\par
Assignment of the number of iterations $J$ and
the acceleration factor $K$, allows the use of the improved cepstral
analysis method \cite{ref:icep-IECE}.
\end{qsection}

\begin{options}
	\argm{m}{M}{order of cepstrum}{25}
	\argm{l}{L}{frame length}{256}
	\argm{j}{J}{number of iteration}{0}
	\argm{k}{K}{acceleration factor}{0.0}
	\argm{e}{E}{epsilon}{0.0}
\end{options}

\begin{qsection}{EXAMPLE}
In the example below, speech data in float format is read from
{\em data.f} and the cepstral coefficients are outputed to {\em data.cep}:
\begin{quote}
  \verb!frame < data.f | window | fftcep > data.cep !
\end{quote}
\end{qsection}

\begin{qsection}{SEE ALSO}
\hyperlink{uels}{uels}
\end{qsection}
