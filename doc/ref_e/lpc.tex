%  ---------------------------------------------------------------  %
%            Speech Signal Processing Toolkit (SPTK)                %
%                      SPTK Working Group                           %
%                                                                   %
%                  Department of Computer Science                   %
%                  Nagoya Institute of Technology                   %
%                               and                                 %
%   Interdisciplinary Graduate School of Science and Engineering    %
%                  Tokyo Institute of Technology                    %
%                                                                   %
%                     Copyright (c) 1984-2007                       %
%                       All Rights Reserved.                        %
%                                                                   %
%  Permission is hereby granted, free of charge, to use and         %
%  distribute this software and its documentation without           %
%  restriction, including without limitation the rights to use,     %
%  copy, modify, merge, publish, distribute, sublicense, and/or     %
%  sell copies of this work, and to permit persons to whom this     %
%  work is furnished to do so, subject to the following conditions: %
%                                                                   %
%    1. The source code must retain the above copyright notice,     %
%       this list of conditions and the following disclaimer.       %
%                                                                   %
%    2. Any modifications to the source code must be clearly        %
%       marked as such.                                             %
%                                                                   %
%    3. Redistributions in binary form must reproduce the above     %
%       copyright notice, this list of conditions and the           %
%       following disclaimer in the documentation and/or other      %
%       materials provided with the distribution.  Otherwise, one   %
%       must contact the SPTK working group.                        %
%                                                                   %
%  NAGOYA INSTITUTE OF TECHNOLOGY, TOKYO INSTITUTE OF TECHNOLOGY,   %
%  SPTK WORKING GROUP, AND THE CONTRIBUTORS TO THIS WORK DISCLAIM   %
%  ALL WARRANTIES WITH REGARD TO THIS SOFTWARE, INCLUDING ALL       %
%  IMPLIED WARRANTIES OF MERCHANTABILITY AND FITNESS, IN NO EVENT   %
%  SHALL NAGOYA INSTITUTE OF TECHNOLOGY, TOKYO INSTITUTE OF         %
%  TECHNOLOGY, SPTK WORKING GROUP, NOR THE CONTRIBUTORS BE LIABLE   %
%  FOR ANY SPECIAL, INDIRECT OR CONSEQUENTIAL DAMAGES OR ANY        %
%  DAMAGES WHATSOEVER RESULTING FROM LOSS OF USE, DATA OR PROFITS,  %
%  WHETHER IN AN ACTION OF CONTRACT, NEGLIGENCE OR OTHER TORTUOUS   %
%  ACTION, ARISING OUT OF OR IN CONNECTION WITH THE USE OR          %
%  PERFORMANCE OF THIS SOFTWARE.                                    %
%                                                                   %
%  ---------------------------------------------------------------  %
%
\hypertarget{lpc}{}
\name{lpc}{LPC analysis using Levinson-Durbin method}{signal processing}

\begin{synopsis}
\item [lpc] [ --l $L$ ] [ --m $M$ ] [ {\em infile} ] 
\end{synopsis}

\begin{qsection}{DESCRIPTION}
{\em lpc} calculates linear prediction coefficients (LPC) 
from $L$-length framed windowed data from {\em infile} (or standard input), 
sending the result to standard output.

For each $L$-length input vector
\begin{displaymath}
  x(0),x(1),\ldots,x(L-1), 
\end{displaymath}
the autocorrelation function is calculated (see \hyperlink{acorr}{acorr}),
then the gain $K$ and the linear prediction coefficients 
$a(k) (1 \leq k \leq M)$ 
\begin{displaymath}
  K, a(1), \ldots, a(M)
\end{displaymath}
are calculated using the Levinson-Durbin algorithm. 

Input and output data are in float format.
\end{qsection}

\begin{options}
	\argm{l}{L}{frame length}{256}
	\argm{m}{M}{order of LPC}{25}
\end{options}

\begin{qsection}{EXAMPLE}
In this example, the 20-th order linear prediction analysis is applied
to input read from {\em data.f} in float format,
and the linear prediction coefficients are written to
{\em data.lpc}:
\begin{quote}
 \verb!frame < data.f | window | lpc -m 20 > data.lpc!
\end{quote} 
\end{qsection}

\begin{qsection}{SEE ALSO}
\hyperlink{acorr}{acorr},
\hyperlink{levdur}{levdur},
\hyperlink{lpc2par}{lpc2par},
\hyperlink{par2lpc}{par2lpc},
\hyperlink{lpc2c}{lpc2c},
\hyperlink{lpc2lsp}{lpc2lsp},
\hyperlink{lsp2lpc}{lsp2lpc}
\hyperlink{ltcdf}{ltcdf},
\hyperlink{lspdf}{lspdf}
\end{qsection}
