% ----------------------------------------------------------------
%       Speech Signal Processing Toolkit (SPTK): version 3.0
%                      SPTK Working Group
% 
%                Department of Computer Science
%                Nagoya Institute of Technology
%                             and
%   Interdisciplinary Graduate School of Science and Engineering
%                Tokyo Institute of Technology
%                   Copyright (c) 1984-2000
%                     All Rights Reserved.
% 
% Permission is hereby granted, free of charge, to use and
% distribute this software and its documentation without
% restriction, including without limitation the rights to use,
% copy, modify, merge, publish, distribute, sublicense, and/or
% sell copies of this work, and to permit persons to whom this
% work is furnished to do so, subject to the following conditions:
% 
%   1. The code must retain the above copyright notice, this list
%      of conditions and the following disclaimer.
% 
%   2. Any modifications must be clearly marked as such.
%                                                                        
% NAGOYA INSTITUTE OF TECHNOLOGY, TOKYO INSITITUTE OF TECHNOLOGY,
% SPTK WORKING GROUP, AND THE CONTRIBUTORS TO THIS WORK DISCLAIM
% ALL WARRANTIES WITH REGARD TO THIS SOFTWARE, INCLUDING ALL
% IMPLIED WARRANTIES OF MERCHANTABILITY AND FITNESS, IN NO EVENT
% SHALL NAGOYA INSTITUTE OF TECHNOLOGY, TOKYO INSITITUTE OF
% TECHNOLOGY, SPTK WORKING GROUP, NOR THE CONTRIBUTORS BE LIABLE
% FOR ANY SPECIAL, INDIRECT OR CONSEQUENTIAL DAMAGES OR ANY
% DAMAGES WHATSOEVER RESULTING FROM LOSS OF USE, DATA OR PROFITS,
% WHETHER IN AN ACTION OF CONTRACT, NEGLIGENCE OR OTHER TORTIOUS
% ACTION, ARISING OUT OF OR IN CONNECTION WITH THE USE OR
% PERFORMANCE OF THIS SOFTWARE.
% ----------------------------------------------------------------
%
\name{lpc}{LPC analysis using Levinson-Durbin method}{signal processing}

\begin{synopsis}
\item [lpc] [ --l $L$ ] [ --m $M$ ] [ {\em infile} ] 
\end{synopsis}

\begin{qsection}{DESCRIPTION}
This command undertakes linear prediction analysis.
A window of length $L$ is passed through the assigned
data file generating the following sequence.
\begin{displaymath}
  x(0),x(1),\ldots,x(L-1)
\end{displaymath}
First of all, the autocorrelation function is
calculated (please refer to ``accor'').
Afterwards, the linear prediction coefficients
\begin{displaymath}
  K, a(1), \ldots, a(M)
\end{displaymath}
are evaluated from the Levinson-Durbin algorithm,
and the results is sent to the standard output.
Also, the gain $K$ is computed.
\par
Input and output data are in float format.
\end{qsection}

\begin{options}
	\argm{l}{L}{frame length}{256}
	\argm{m}{M}{order of LPC}{25}
\end{options}

\begin{qsection}{EXAMPLE}
In this example, the 20 order linear prediction analysis is applied
to input read from {\em data.f} in float format,
and the linear prediction coefficients are written to
{\em data.lpc}:
\begin{quote}
 \verb!frame < data.f | window | lpc -m 20 > data.lpc!
\end{quote} 
\end{qsection}

\begin{qsection}{SEE ALSO}
 acorr, levdur, lpc2par, par2lpc, lpc2c, lpc2lsp, lsp2lpc, 
 ltcdf, lspdf
\end{qsection}
