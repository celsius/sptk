% ----------------------------------------------------------------- %
%             The Speech Signal Processing Toolkit (SPTK)           %
%             developed by SPTK Working Group                       %
%             http://sp-tk.sourceforge.net/                         %
% ----------------------------------------------------------------- %
%                                                                   %
%  Copyright (c) 1984-2007  Tokyo Institute of Technology           %
%                           Interdisciplinary Graduate School of    %
%                           Science and Engineering                 %
%                                                                   %
%                1996-2008  Nagoya Institute of Technology          %
%                           Department of Computer Science          %
%                                                                   %
% All rights reserved.                                              %
%                                                                   %
% Redistribution and use in source and binary forms, with or        %
% without modification, are permitted provided that the following   %
% conditions are met:                                               %
%                                                                   %
% - Redistributions of source code must retain the above copyright  %
%   notice, this list of conditions and the following disclaimer.   %
% - Redistributions in binary form must reproduce the above         %
%   copyright notice, this list of conditions and the following     %
%   disclaimer in the documentation and/or other materials provided %
%   with the distribution.                                          %
% - Neither the name of the SPTK working group nor the names of its %
%   contributors may be used to endorse or promote products derived %
%   from this software without specific prior written permission.   %
%                                                                   %
% THIS SOFTWARE IS PROVIDED BY THE COPYRIGHT HOLDERS AND            %
% CONTRIBUTORS "AS IS" AND ANY EXPRESS OR IMPLIED WARRANTIES,       %
% INCLUDING, BUT NOT LIMITED TO, THE IMPLIED WARRANTIES OF          %
% MERCHANTABILITY AND FITNESS FOR A PARTICULAR PURPOSE ARE          %
% DISCLAIMED. IN NO EVENT SHALL THE COPYRIGHT OWNER OR CONTRIBUTORS %
% BE LIABLE FOR ANY DIRECT, INDIRECT, INCIDENTAL, SPECIAL,          %
% EXEMPLARY, OR CONSEQUENTIAL DAMAGES (INCLUDING, BUT NOT LIMITED   %
% TO, PROCUREMENT OF SUBSTITUTE GOODS OR SERVICES; LOSS OF USE,     %
% DATA, OR PROFITS; OR BUSINESS INTERRUPTION) HOWEVER CAUSED AND ON %
% ANY THEORY OF LIABILITY, WHETHER IN CONTRACT, STRICT LIABILITY,   %
% OR TORT (INCLUDING NEGLIGENCE OR OTHERWISE) ARISING IN ANY WAY    %
% OUT OF THE USE OF THIS SOFTWARE, EVEN IF ADVISED OF THE           %
% POSSIBILITY OF SUCH DAMAGE.                                       %
% ----------------------------------------------------------------- %
\hypertarget{lpc}{}
\name{lpc}{LPC analysis using Levinson-Durbin method}{signal processing}

\begin{synopsis}
\item [lpc] [ --l $L$ ] [ --m $M$ ] [ --f $F$ ] [ {\em infile} ] 
\end{synopsis}

\begin{qsection}{DESCRIPTION}
{\em lpc} calculates linear prediction coefficients (LPC) 
from $L$-length framed windowed data from {\em infile} (or standard input), 
sending the result to standard output.

For each $L$-length input vector
\begin{displaymath}
  x(0),x(1),\ldots,x(L-1), 
\end{displaymath}
the autocorrelation function is calculated (see \hyperlink{acorr}{acorr}),
then the gain $K$ and the linear prediction coefficients 
$a(k) (1 \leq k \leq M)$ 
\begin{displaymath}
  K, a(1), \ldots, a(M)
\end{displaymath}
are calculated using the Levinson-Durbin algorithm. 

Input and output data are in float format.
\end{qsection}

\begin{options}
	\argm{l}{L}{frame length}{256}
	\argm{m}{M}{order of LPC}{25}
	\argm{f}{F}{mimimum value of the determinant of the normal matrix}{0.000001}
\end{options}

\begin{qsection}{EXAMPLE}
In this example, the 20-th order linear prediction analysis is applied
to input read from {\em data.f} in float format,
and the linear prediction coefficients are written to
{\em data.lpc}:
\begin{quote}
 \verb!frame +f < data.f | window | lpc -m 20 > data.lpc!
\end{quote} 
\end{qsection}

\begin{qsection}{SEE ALSO}
\hyperlink{acorr}{acorr},
\hyperlink{levdur}{levdur},
\hyperlink{lpc2par}{lpc2par},
\hyperlink{par2lpc}{par2lpc},
\hyperlink{lpc2c}{lpc2c},
\hyperlink{lpc2lsp}{lpc2lsp},
\hyperlink{lsp2lpc}{lsp2lpc}
\hyperlink{ltcdf}{ltcdf},
\hyperlink{lspdf}{lspdf}
\end{qsection}
