% ----------------------------------------------------------------
%       Speech Signal Processing Toolkit (SPTK): version 3.0
%                      SPTK Working Group
% 
%                Department of Computer Science
%                Nagoya Institute of Technology
%                             and
%   Interdisciplinary Graduate School of Science and Engineering
%                Tokyo Institute of Technology
%                   Copyright (c) 1984-2000
%                     All Rights Reserved.
% 
% Permission is hereby granted, free of charge, to use and
% distribute this software and its documentation without
% restriction, including without limitation the rights to use,
% copy, modify, merge, publish, distribute, sublicense, and/or
% sell copies of this work, and to permit persons to whom this
% work is furnished to do so, subject to the following conditions:
% 
%   1. The code must retain the above copyright notice, this list
%      of conditions and the following disclaimer.
% 
%   2. Any modifications must be clearly marked as such.
%                                                                        
% NAGOYA INSTITUTE OF TECHNOLOGY, TOKYO INSITITUTE OF TECHNOLOGY,
% SPTK WORKING GROUP, AND THE CONTRIBUTORS TO THIS WORK DISCLAIM
% ALL WARRANTIES WITH REGARD TO THIS SOFTWARE, INCLUDING ALL
% IMPLIED WARRANTIES OF MERCHANTABILITY AND FITNESS, IN NO EVENT
% SHALL NAGOYA INSTITUTE OF TECHNOLOGY, TOKYO INSITITUTE OF
% TECHNOLOGY, SPTK WORKING GROUP, NOR THE CONTRIBUTORS BE LIABLE
% FOR ANY SPECIAL, INDIRECT OR CONSEQUENTIAL DAMAGES OR ANY
% DAMAGES WHATSOEVER RESULTING FROM LOSS OF USE, DATA OR PROFITS,
% WHETHER IN AN ACTION OF CONTRACT, NEGLIGENCE OR OTHER TORTIOUS
% ACTION, ARISING OUT OF OR IN CONNECTION WITH THE USE OR
% PERFORMANCE OF THIS SOFTWARE.
% ----------------------------------------------------------------
%
\hypertarget{train}{}
\name{train}{generate pulse sequence}{signal generation}

\begin{synopsis}
\item[train] [ --l $L$ ] [ --p $P$ ]
\end{synopsis}

\begin{qsection}{DESCRIPTION}
{\em train} generates a normalized pulse train sequence 
or a sequence with values $\pm 1$, 
sending the result to standard output.
Output data is in float format.
\end{qsection}

\begin{options}
	\argm{l}{L}{frame length}{256}
	\argm{p}{P}{frame period\\
                    if $P=0$ then a sequence with values
                    $\pm 1$ is generated.}{0}
	\argm{n}{N}{type of normalization\\
                    When $x(n)$ is impulse sequence\\
			\begin{tabular}{ll}\\ [-1ex]
			 0 & no-normalization\\
			 1 & normalization as 
                             $\displaystyle \sum_{n=0}^{L-1} x^2(n) = 1$\\
			 2 & normalization  as 
                             $\displaystyle \sum_{n=0}^{L-1} x(n) = 1$\\
			 \end{tabular}\\\hspace*{\fill}}{1}
\end{options}

\begin{qsection}{EXAMPLE}
The following example displays the spectrum of
the signal obtained from passing a train pulse sequence through
digital filter:
\begin{quote}
\verb!train | dfs -b 1 0.9 | window | spec | fdrw | xgr!
\end{quote}
\end{qsection}

\begin{qsection}{SEE ALSO}
\hyperlink{impulse}{impulse},
\hyperlink{sin}{sin},
\hyperlink{step}{step},
\hyperlink{ramp}{ramp}
\end{qsection}


