\name{train}{generate pulse sequence}{signal generation}

\begin{synopsis}
\item[train] [ --l $L$ ] [ --p $P$ ]
\end{synopsis}

\begin{qsection}{DESCRIPTION}
This command generates a pulse train sequence or a sequence with
values $\pm 1$.
Output data is in float format.
\end{qsection}

\begin{options}
	\argm{l}{L}{frame length}{256}
	\argm{p}{P}{frame period\\
                    if $P=0$ then a sequence with values
                    $\pm 1$ is generated.}{0}
	\argm{n}{N}{type of normalization\\
                    When $x(n)$ is impulse sequence\\
			\begin{tabular}{ll}\\ [-1zh]
			 0 & no-normalization\\
			 1 & normalization as 
                             $\displaystyle \sum_{n=0}^{L-1} x^2(n) = 1$\\
			 2 & normalization  as 
                             $\displaystyle \sum_{n=0}^{L-1} x(n) = 1$\\
			 \end{tabular}\\\hspace*{\fill}}{1}
\end{options}

\begin{qsection}{EXAMPLE}
The following example displays the spectrum of
the signal obtained from passing a train pulse sequence through
digital filter:
\begin{quote}
\verb!train | dfs -b 1 0.9 | window | spec | fdrw | xgr!
\end{quote}
\end{qsection}

\begin{qsection}{SEE ALSO}
  impulse, sin, step, ramp
\end{qsection}


