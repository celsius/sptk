%  ---------------------------------------------------------------  %
%            Speech Signal Processing Toolkit (SPTK)                %
%                      SPTK Working Group                           %
%                                                                   %
%                  Department of Computer Science                   %
%                  Nagoya Institute of Technology                   %
%                               and                                 %
%   Interdisciplinary Graduate School of Science and Engineering    %
%                  Tokyo Institute of Technology                    %
%                                                                   %
%                     Copyright (c) 1984-2007                       %
%                       All Rights Reserved.                        %
%                                                                   %
%  Permission is hereby granted, free of charge, to use and         %
%  distribute this software and its documentation without           %
%  restriction, including without limitation the rights to use,     %
%  copy, modify, merge, publish, distribute, sublicense, and/or     %
%  sell copies of this work, and to permit persons to whom this     %
%  work is furnished to do so, subject to the following conditions: %
%                                                                   %
%    1. The source code must retain the above copyright notice,     %
%       this list of conditions and the following disclaimer.       %
%                                                                   %
%    2. Any modifications to the source code must be clearly        %
%       marked as such.                                             %
%                                                                   %
%    3. Redistributions in binary form must reproduce the above     %
%       copyright notice, this list of conditions and the           %
%       following disclaimer in the documentation and/or other      %
%       materials provided with the distribution.  Otherwise, one   %
%       must contact the SPTK working group.                        %
%                                                                   %
%  NAGOYA INSTITUTE OF TECHNOLOGY, TOKYO INSTITUTE OF TECHNOLOGY,   %
%  SPTK WORKING GROUP, AND THE CONTRIBUTORS TO THIS WORK DISCLAIM   %
%  ALL WARRANTIES WITH REGARD TO THIS SOFTWARE, INCLUDING ALL       %
%  IMPLIED WARRANTIES OF MERCHANTABILITY AND FITNESS, IN NO EVENT   %
%  SHALL NAGOYA INSTITUTE OF TECHNOLOGY, TOKYO INSTITUTE OF         %
%  TECHNOLOGY, SPTK WORKING GROUP, NOR THE CONTRIBUTORS BE LIABLE   %
%  FOR ANY SPECIAL, INDIRECT OR CONSEQUENTIAL DAMAGES OR ANY        %
%  DAMAGES WHATSOEVER RESULTING FROM LOSS OF USE, DATA OR PROFITS,  %
%  WHETHER IN AN ACTION OF CONTRACT, NEGLIGENCE OR OTHER TORTUOUS   %
%  ACTION, ARISING OUT OF OR IN CONNECTION WITH THE USE OR          %
%  PERFORMANCE OF THIS SOFTWARE.                                    %
%                                                                   %
%  ---------------------------------------------------------------  %
%
\hypertarget{nan}{}
\name{nan}{data check}{data processing}

\begin{synopsis}
\item[nan] [ --d ] [ infile ]
\end{synopsis}

\begin{qsection}{DESCRIPTION}

{\em nan} checks whether input data contains NaN, Infty, or Indefinite Value, 
showing the positions where these values occurred. 
If the --d option is not give, 
the input format is float. 

\end{qsection}

\begin{options}
        \argm{d}{}{regard input data as double}{FALSE}
\end{options}

\begin{qsection}{EXAMPLE}
This example reads input data {\em data.f} in float format and checks it:
\begin{quote}
  \verb!nan data.f!
\end{quote}
\end{qsection}

