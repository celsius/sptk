\name{b2mc}{transform MLSA digital filter coefficients to mel-cepstrum}%
{speech parameter transformation}

\begin{synopsis}
\item [b2mc] [ --m $M$ ] [ --a $A$ ] [ {\em infile} ]
\end{synopsis}

\begin{qsection}{DESCRIPTION}
This command evaluates mel-cepstrum coefficients $c_\alpha(m)$ from
MLSA filter coefficients $b(m)$, and sends the results to the standard
output.
\par
Input and output data are in float format.
\par
The transformation from $b(m)$ coefficients to mel-cepstrum coefficients
$c_\alpha(m)$ is as follows:
\begin{displaymath}
c_\alpha(m) = \left\{
	\begin{array}{ll}
	  b(M), & m=M \\
	  b(m) + \alpha b(m+1), & 0 \leq m < M \\
	\end{array} \right.
\end{displaymath}
This transformation is the inverse transformation which is undertaken
by the command mc2b.

\end{qsection}

\begin{options}
	\argm{m}{M}{order of mel cepstrum}{25}
	\argm{a}{A}{all-pass constant $\alpha$}{0.35}
\end{options}

\begin{qsection}{EXAMPLE}
The example below converts the coefficients of a MLSA filter,
which are in float format in file {\em data.b},
into mel-cepstrum coefficients
in file {\em data.mcep} with $M=15$ and $\alpha=0.35$.
\begin{quote}
 \verb!b2mc -m 15 < data.b > data.mcep!
\end{quote} 
\end{qsection}

\begin{qsection}{SEE ALSO}
b2mc, mcep, mlsadf
\end{qsection}
