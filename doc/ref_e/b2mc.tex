% ----------------------------------------------------------------- %
%             The Speech Signal Processing Toolkit (SPTK)           %
%             developed by SPTK Working Group                       %
%             http://sp-tk.sourceforge.net/                         %
% ----------------------------------------------------------------- %
%                                                                   %
%  Copyright (c) 1984-2007  Tokyo Institute of Technology           %
%                           Interdisciplinary Graduate School of    %
%                           Science and Engineering                 %
%                                                                   %
%                1996-2011  Nagoya Institute of Technology          %
%                           Department of Computer Science          %
%                                                                   %
% All rights reserved.                                              %
%                                                                   %
% Redistribution and use in source and binary forms, with or        %
% without modification, are permitted provided that the following   %
% conditions are met:                                               %
%                                                                   %
% - Redistributions of source code must retain the above copyright  %
%   notice, this list of conditions and the following disclaimer.   %
% - Redistributions in binary form must reproduce the above         %
%   copyright notice, this list of conditions and the following     %
%   disclaimer in the documentation and/or other materials provided %
%   with the distribution.                                          %
% - Neither the name of the SPTK working group nor the names of its %
%   contributors may be used to endorse or promote products derived %
%   from this software without specific prior written permission.   %
%                                                                   %
% THIS SOFTWARE IS PROVIDED BY THE COPYRIGHT HOLDERS AND            %
% CONTRIBUTORS "AS IS" AND ANY EXPRESS OR IMPLIED WARRANTIES,       %
% INCLUDING, BUT NOT LIMITED TO, THE IMPLIED WARRANTIES OF          %
% MERCHANTABILITY AND FITNESS FOR A PARTICULAR PURPOSE ARE          %
% DISCLAIMED. IN NO EVENT SHALL THE COPYRIGHT OWNER OR CONTRIBUTORS %
% BE LIABLE FOR ANY DIRECT, INDIRECT, INCIDENTAL, SPECIAL,          %
% EXEMPLARY, OR CONSEQUENTIAL DAMAGES (INCLUDING, BUT NOT LIMITED   %
% TO, PROCUREMENT OF SUBSTITUTE GOODS OR SERVICES; LOSS OF USE,     %
% DATA, OR PROFITS; OR BUSINESS INTERRUPTION) HOWEVER CAUSED AND ON %
% ANY THEORY OF LIABILITY, WHETHER IN CONTRACT, STRICT LIABILITY,   %
% OR TORT (INCLUDING NEGLIGENCE OR OTHERWISE) ARISING IN ANY WAY    %
% OUT OF THE USE OF THIS SOFTWARE, EVEN IF ADVISED OF THE           %
% POSSIBILITY OF SUCH DAMAGE.                                       %
% ----------------------------------------------------------------- %
\hypertarget{b2mc}{}
\name{b2mc}{transform MLSA digital filter coefficients to mel-cepstrum}%
{speech parameter transformation}

\begin{synopsis}
\item [b2mc] [ --m $M$ ] [ --a $A$ ] [ {\em infile} ]
\end{synopsis}

\begin{qsection}{DESCRIPTION}
{\em b2mc} calculates mel-cepstral coefficients $c_\alpha(m)$ 
from MLSA filter coefficients $b(m)$ in the {\em infile} (or standard input), 
sending the result to standard output.

Input and output data are in float format.

The transformation from $b(m)$ coefficients to mel-cepstral coefficients
$c_\alpha(m)$ is as follows:
\begin{displaymath}
c_\alpha(m) = \begin{cases}
	  \;\; b(M) & m=M \\
	  \;\; b(m) + \alpha b(m+1) & 0 \leq m < M \\
	\end{cases}
\end{displaymath}
The command {\em b2mc} and {\em mc2b} are in inverse conversion
 relationship to each other.

\end{qsection}

\begin{options}
	\argm{m}{M}{order of mel cepstrum}{25}
	\argm{a}{A}{all-pass constant $\alpha$}{0.35}
\end{options}

\begin{qsection}{EXAMPLE}
The example below converts the coefficients of an MLSA filter,
which are in file {\em data.b} in float format,
into mel-cepstral coefficients
in file {\em data.mcep}, with $M=15$ and $\alpha=0.35$.
\begin{quote}
 \verb!b2mc -m 15 < data.b > data.mcep!
\end{quote} 
\end{qsection}

\begin{qsection}{SEE ALSO}
\hyperlink{mc2b}{mc2b},
\hyperlink{mcep}{mcep},
\hyperlink{mlsadf}{mlsadf}
\end{qsection}
