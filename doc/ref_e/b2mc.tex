% ----------------------------------------------------------------
%       Speech Signal Processing Toolkit (SPTK): version 3.0
%                      SPTK Working Group
% 
%                Department of Computer Science
%                Nagoya Institute of Technology
%                             and
%   Interdisciplinary Graduate School of Science and Engineering
%                Tokyo Institute of Technology
%                   Copyright (c) 1984-2000
%                     All Rights Reserved.
% 
% Permission is hereby granted, free of charge, to use and
% distribute this software and its documentation without
% restriction, including without limitation the rights to use,
% copy, modify, merge, publish, distribute, sublicense, and/or
% sell copies of this work, and to permit persons to whom this
% work is furnished to do so, subject to the following conditions:
% 
%   1. The code must retain the above copyright notice, this list
%      of conditions and the following disclaimer.
% 
%   2. Any modifications must be clearly marked as such.
%                                                                        
% NAGOYA INSTITUTE OF TECHNOLOGY, TOKYO INSITITUTE OF TECHNOLOGY,
% SPTK WORKING GROUP, AND THE CONTRIBUTORS TO THIS WORK DISCLAIM
% ALL WARRANTIES WITH REGARD TO THIS SOFTWARE, INCLUDING ALL
% IMPLIED WARRANTIES OF MERCHANTABILITY AND FITNESS, IN NO EVENT
% SHALL NAGOYA INSTITUTE OF TECHNOLOGY, TOKYO INSITITUTE OF
% TECHNOLOGY, SPTK WORKING GROUP, NOR THE CONTRIBUTORS BE LIABLE
% FOR ANY SPECIAL, INDIRECT OR CONSEQUENTIAL DAMAGES OR ANY
% DAMAGES WHATSOEVER RESULTING FROM LOSS OF USE, DATA OR PROFITS,
% WHETHER IN AN ACTION OF CONTRACT, NEGLIGENCE OR OTHER TORTIOUS
% ACTION, ARISING OUT OF OR IN CONNECTION WITH THE USE OR
% PERFORMANCE OF THIS SOFTWARE.
% ----------------------------------------------------------------
%
\hypertarget{b2mc}{}
\name{b2mc}{transform MLSA digital filter coefficients to mel-cepstrum}%
{speech parameter transformation}

\begin{synopsis}
\item [b2mc] [ --m $M$ ] [ --a $A$ ] [ {\em infile} ]
\end{synopsis}

\begin{qsection}{DESCRIPTION}
{\em b2mc} calculates mel-cepstral coefficients $c_\alpha(m)$ 
from MLSA filter coefficients $b(m)$ from {\em infile} (or standard input), 
sending the result to standard output.

Input and output data are in float format.

The transformation from $b(m)$ coefficients to mel-cepstral coefficients
$c_\alpha(m)$ is as follows:
\begin{displaymath}
c_\alpha(m) = \begin{cases}
	  \;\; b(M) & m=M \\
	  \;\; b(m) + \alpha b(m+1) & 0 \leq m < M \\
	\end{cases}
\end{displaymath}
This transformation is the inverse transformation which is undertaken
by the command mc2b.

\end{qsection}

\begin{options}
	\argm{m}{M}{order of mel cepstrum}{25}
	\argm{a}{A}{all-pass constant $\alpha$}{0.35}
\end{options}

\begin{qsection}{EXAMPLE}
The example below converts the coefficients of an MLSA filter,
which are in float format in file {\em data.b},
into mel-cepstral coefficients
in file {\em data.mcep} with $M=15$ and $\alpha=0.35$.
\begin{quote}
 \verb!b2mc -m 15 < data.b > data.mcep!
\end{quote} 
\end{qsection}

\begin{qsection}{SEE ALSO}
\hyperlink{mc2b}{mc2b},
\hyperlink{mcep}{mcep},
\hyperlink{mlsadf}{mlsadf}
\end{qsection}
