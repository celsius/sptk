\name{ltcdf}{all-pole lattice digital filter for speech synthesis}%
{filters for speech synthesis}

\begin{synopsis}
\item [ltcdf] [ --m $M$ ] [ --p $P$ ] [ --i $I$ ] [ --k ] {\em rcfile} 
	      [ {\em infile} ] 
\end{synopsis}

\begin{qsection}{DESCRIPTION}
The {\em ltcdf} command reads excitation information from
{\em infile},  passes it through a lattice filter
obtained from the PARCOR coefficients read from {\em rcfile},
and sends the results to the standard output.
\par
Input and output data are in float format.
\end{qsection}

\begin{options}
	\argm{m}{M}{order of coefficients}{25}
	\argm{p}{P}{frame period}{100}
	\argm{i}{I}{interpolation period}{1}
	\argm{k}{}{filtering without gain}{FALSE}
\end{options}

\begin{qsection}{EXAMPLE}
In the example below, excitation is generated from
pitch information given in {\em data.pitch} in float format,
this excitation is passed through the lattice filter
constructed from the LPC file {\em data.rc},
and the synthesized speech is written to {\em data.syn}:
\begin{quote}
 \verb!excite < data.pitch | ltcdf data.k > data.syn!
\end{quote} 
\end{qsection}

\begin{qsection}{SEE ALSO}
 lpc, acorr, levdur, lpc2par, par2lpc, poledf, zerodf, lspdf
\end{qsection}
