% ----------------------------------------------------------------- %
%             The Speech Signal Processing Toolkit (SPTK)           %
%             developed by SPTK Working Group                       %
%             http://sp-tk.sourceforge.net/                         %
% ----------------------------------------------------------------- %
%                                                                   %
%  Copyright (c) 1984-2007  Tokyo Institute of Technology           %
%                           Interdisciplinary Graduate School of    %
%                           Science and Engineering                 %
%                                                                   %
%                1996-2011  Nagoya Institute of Technology          %
%                           Department of Computer Science          %
%                                                                   %
% All rights reserved.                                              %
%                                                                   %
% Redistribution and use in source and binary forms, with or        %
% without modification, are permitted provided that the following   %
% conditions are met:                                               %
%                                                                   %
% - Redistributions of source code must retain the above copyright  %
%   notice, this list of conditions and the following disclaimer.   %
% - Redistributions in binary form must reproduce the above         %
%   copyright notice, this list of conditions and the following     %
%   disclaimer in the documentation and/or other materials provided %
%   with the distribution.                                          %
% - Neither the name of the SPTK working group nor the names of its %
%   contributors may be used to endorse or promote products derived %
%   from this software without specific prior written permission.   %
%                                                                   %
% THIS SOFTWARE IS PROVIDED BY THE COPYRIGHT HOLDERS AND            %
% CONTRIBUTORS "AS IS" AND ANY EXPRESS OR IMPLIED WARRANTIES,       %
% INCLUDING, BUT NOT LIMITED TO, THE IMPLIED WARRANTIES OF          %
% MERCHANTABILITY AND FITNESS FOR A PARTICULAR PURPOSE ARE          %
% DISCLAIMED. IN NO EVENT SHALL THE COPYRIGHT OWNER OR CONTRIBUTORS %
% BE LIABLE FOR ANY DIRECT, INDIRECT, INCIDENTAL, SPECIAL,          %
% EXEMPLARY, OR CONSEQUENTIAL DAMAGES (INCLUDING, BUT NOT LIMITED   %
% TO, PROCUREMENT OF SUBSTITUTE GOODS OR SERVICES; LOSS OF USE,     %
% DATA, OR PROFITS; OR BUSINESS INTERRUPTION) HOWEVER CAUSED AND ON %
% ANY THEORY OF LIABILITY, WHETHER IN CONTRACT, STRICT LIABILITY,   %
% OR TORT (INCLUDING NEGLIGENCE OR OTHERWISE) ARISING IN ANY WAY    %
% OUT OF THE USE OF THIS SOFTWARE, EVEN IF ADVISED OF THE           %
% POSSIBILITY OF SUCH DAMAGE.                                       %
% ----------------------------------------------------------------- %
\hypertarget{mfcc}{}
\name{mfcc}{mel-frequency cepstral analysis}{speech analysis}

\begin{synopsis}
\item[mfcc] [ --a $A$ ] [ --e $E$ ] [ --l $L_1$ ] [ --L $L_2$ ] [ --f $F$ ] [ --m $M$ ]
\item[\ ~~~][ --n $N$ ] [ --f $F$ ] [ --w $W$] [ --d ] [ -- E ] [ --0 ][ {\em infile} ] 
\end{synopsis}

\begin{qsection}{DESCRIPTION}
{\em mfcc} uses mel-frequency cepstral analysis to calculate 
mel-frequency cepstrum from  $L_1$-length framed data from {\em infile} (or
standard input), sending the result to standard output.Since {\em
  mfcc} can apply a window function to input data in the function, it is
not necessary to use windowed data as input. The input time domain 
sequence of length $L_1$ is of the form:
\begin{displaymath}
  x(0),x(1),\dots,x(L_1-1)
\end{displaymath}
Also, note that the input and output data are in float format, and
that the output data cannot be used for speech synthesis through
the MLSA filter.
\end{qsection}

\begin{options}
	\argm{a}{A}{preemphasise coefficient}{0.97}
	\argm{c}{C}{liftering coefficient}{22}
	\argm{e}{E}{small value for calculating log()}{0.0}
	\argm{l}{L_1}{frame length of input}{256}
	\argm{L}{L_2}{frame length for fft. default value $2^n$
          satisfies $L_1 \leq 2^n$ }{$2^n$}
	\argm{m}{M}{order of mfcc}{12}
	\argm{n}{N}{order of channnel for mel-filter bank}{20}
	\argm{f}{F}{sampling rate}{16000}
	\argm{w}{W}{type of window\\
			\begin{tabular}{ll}\\ [-1ex]
			 0 & Hamming \\
			 1 & Do not use a window function\\
			\end{tabular}\\\hspace*{\fill}}{0}
	\argm{d}{}{use dft (without using fft) for dct}{FALSE}
	\argm{E}{}{output energy}{FALSE}
	\argm{0}{}{output $0$'th static coefficient}{FALSE}
        \desc{if the -E or -0 option is given, energy $E$ or $0$'th
          static coefficient $C0$ is outputted as follows.
        \begin{displaymath}
          mc(0),mc(1),\dots,mc(m-1),E (C0)
        \end{displaymath}
          Also, if both -E and -0 option are given, the output is as folows.
        \begin{displaymath}
          mc(0),mc(1),\dots,mc(m-1),C0,E 
        \end{displaymath}
}
\end{options}

\begin{qsection}{EXAMPLE}
In the example below, speech data in float format is read from
{\em data.f}. 
Here, we specify tge frame length, frame shift and sampling rate as
40ms, 10ms and 16kHz, respectivelly. The 12 order mel-frequency
cepstral coefficients, together with the energy component, are
outputted to {\em data.mfc}.
\begin{quote}
  \verb!frame +f -l 640 -p 160  data.f |\                        ! \\
  \verb!mfcc -l 640 -m 12 -f 16000 -E > data.mfc                  ! \\
\end{quote}

Also, in case we want to calculate the coefficients the same way as in
HTK, following the conditions:
\begin{quote}
  \verb!SOURCEFORMAT = NOHEAD! \\
  \verb!SOURCEKIND = WAVEFORM ! \\
  \verb!SOURCERATE = 625      # Sampling rate (1 / 16000 * 10^7)!\\
  \verb!TARGETKIND = MFCC_D_A_E ! \\
  \verb!TARGETRATE = 100000   # Frame shift (ns)! \\
  \verb!WINDOWSIZE = 400000   # Frame length (ns)! \\
  \verb!DELTAWINDOW = 1       # Delta widndow size! \\
  \verb!ACCWINDOW = 1         # Accelaration widndow size! \\
  \verb!ENORMALISE = FALSE ! \\
\end{quote}
We have to use the following command in SPTK. Below, because of the difference of the
calcuration method of regression coefficients between SPTK and HTK,
differencial coefficients are specified directly using -d option in
{\em delta} command. 
\begin{quote}
  \verb!frame +f -l 640 -p 160  data.f |\                        ! \\
  \verb!mfcc -l 640 -m 12 -f 16000 -E > data.mfc                  ! \\
  \verb!delta -m 12 -d -0.5 0 0.5 |\ ! \\
  \verb!-d 0.25 0 -0.5 0 0.25 data.mfc > data.mfc.diff! \\
\end{quote}
Here, because of the difference in the calculation method of
regression coefficients between SPTK and HTK, differencial
coefficients are specified directly using the --d option in {\em
  delta} dommand.
The correspondence between the option of SPTK's command option and the
HTK's configuration for extracting mel-frequency cepstrum is shown in Table
\ref{tbl:mfcc_config}. Please, refer to the HTKBook for more
information on extracting mel-frequency cepstrum with HTK.

\setcounter{table}{1}
\begin{table}
        \caption{Configuration for extracting MFCC}
        \label{tbl:mfcc_config}
        \setlength{\arrayrulewidth}{0.5pt}
        \renewcommand{\arraystretch}{1.2}
        \begin{center}
        \begin{tabular}{|c||c|c|} \hline
        Settings                          & SPTK  & HTK \\ \hline\hline
        pre-emphasis coefficient             & -a (at {\em mfcc} command)& PREEMCOEF \\ \hline
        liftering coefficient                & -c (at {\em mfcc} command) & CEPLIFTER \\ \hline
        small value for calculating log()    & -e (at {\em mfcc} command)& N/A \\ \hline
        sampling rate                        & -f (at {\em mfcc} command)& SOURCERATE \\ \hline
        frame shift                          & -p (at {\em frame} command) & TARGETRATE \\ \hline
        frame length of input                & -l (at {\em frame} command) & WINDOWSIZE \\ 
                                             & -l (at {\em mfcc} command)&  \\ \hline
        frame length for fft                 & -L (at {\em mfcc} command)& N/A \\
                                             &    & (automatically calculated) \\ \hline
        order of cepstrum                    & -m (at {\em mfcc} command)& NUMCEPS \\ \hline
        order of channel for mel-filter bank & -n (at {\em mfcc} command)& NUMCHANS \\ \hline
        use hamming window                   & -w (at {\em mfcc} command)& USEHAMMING \\ \hline
        use dft                              & -d (at {\em mfcc} command)& N/A \\ \hline
        output energy                        & -E (at {\em mfcc} command)& TARGETKIND \\ \hline
        output $0$'th static coefficient     & -0 (at {\em mfcc} command)& TARGETKIND \\ \hline 
        delta window size                    & -r (at {\em delta} command)& DELTAWINDOW \\ \hline
        acceleration window size             & -r (at {\em delta} command)& ACCWINDOW \\ \hline
        Normalize log energy                 & N/A & ENORMALISE \\  
        \hline
        \end{tabular}
        \end{center}
\label{tbl:mfcc_config}
\end{table}
\end{qsection}

\begin{qsection}{SEE ALSO}
\hyperlink{frame}{frame},
\hyperlink{gcep}{gcep},
\hyperlink{mcep}{mcep},
\hyperlink{mgcep}{mgcep},
\hyperlink{spec}{spec}
\end{qsection}
