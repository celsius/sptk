% ----------------------------------------------------------------- %
%             The Speech Signal Processing Toolkit (SPTK)           %
%             developed by SPTK Working Group                       %
%             http://sp-tk.sourceforge.net/                         %
% ----------------------------------------------------------------- %
%                                                                   %
%  Copyright (c) 1984-2007  Tokyo Institute of Technology           %
%                           Interdisciplinary Graduate School of    %
%                           Science and Engineering                 %
%                                                                   %
%                1996-2009  Nagoya Institute of Technology          %
%                           Department of Computer Science          %
%                                                                   %
% All rights reserved.                                              %
%                                                                   %
% Redistribution and use in source and binary forms, with or        %
% without modification, are permitted provided that the following   %
% conditions are met:                                               %
%                                                                   %
% - Redistributions of source code must retain the above copyright  %
%   notice, this list of conditions and the following disclaimer.   %
% - Redistributions in binary form must reproduce the above         %
%   copyright notice, this list of conditions and the following     %
%   disclaimer in the documentation and/or other materials provided %
%   with the distribution.                                          %
% - Neither the name of the SPTK working group nor the names of its %
%   contributors may be used to endorse or promote products derived %
%   from this software without specific prior written permission.   %
%                                                                   %
% THIS SOFTWARE IS PROVIDED BY THE COPYRIGHT HOLDERS AND            %
% CONTRIBUTORS "AS IS" AND ANY EXPRESS OR IMPLIED WARRANTIES,       %
% INCLUDING, BUT NOT LIMITED TO, THE IMPLIED WARRANTIES OF          %
% MERCHANTABILITY AND FITNESS FOR A PARTICULAR PURPOSE ARE          %
% DISCLAIMED. IN NO EVENT SHALL THE COPYRIGHT OWNER OR CONTRIBUTORS %
% BE LIABLE FOR ANY DIRECT, INDIRECT, INCIDENTAL, SPECIAL,          %
% EXEMPLARY, OR CONSEQUENTIAL DAMAGES (INCLUDING, BUT NOT LIMITED   %
% TO, PROCUREMENT OF SUBSTITUTE GOODS OR SERVICES; LOSS OF USE,     %
% DATA, OR PROFITS; OR BUSINESS INTERRUPTION) HOWEVER CAUSED AND ON %
% ANY THEORY OF LIABILITY, WHETHER IN CONTRACT, STRICT LIABILITY,   %
% OR TORT (INCLUDING NEGLIGENCE OR OTHERWISE) ARISING IN ANY WAY    %
% OUT OF THE USE OF THIS SOFTWARE, EVEN IF ADVISED OF THE           %
% POSSIBILITY OF SUCH DAMAGE.                                       %
% ----------------------------------------------------------------- %
\hypertarget{sopr}{}
\name{sopr}{execute scalar operations}{number operation}

\begin{synopsis}
\item[sopr] [ --a $A$ ] [ --s $S$ ] [ --m $M$ ] [ --d $D$ ] [ --ABS ] [ --INV ]
\item[\ ~~~~] [ --P ] [ --R ] [ --SQRT ] [ --LN ] [ --LOG2 ] [ --LOG10 ] [ --EXP ]
\item[\ ~~~~] [ --POW2 ] [ --POW10 ] [ --FIX ] [ --UNIT ] [ --CLIP ] [ --SIN ]
\item[\ ~~~~] [ --COS ] [ --TAN ] [ --ATAN ] [--magic $magic$]
  [--MAGIC $MAGIC$]
\item[\ ~~~~] [ --r m$n$ ] [ --w m$n$ ] [ {\em infile} ]
\end{synopsis}

\begin{qsection}{DESCRIPTION}
{\em sopr} performs a sequence of scalar operations on float data 
from {\em infile} (or standard input), 
sending the float output data to standard output.

The sequence of operations is specified by command line options
and is performed in the given order.
\end{qsection}

\begin{options}
	\argm{a}{A}{addition $y=x+A$}{FALSE}
	\argm{s}{S}{subtraction $y=x-S$}{FALSE}
	\argm{m}{M}{multiplication $y=x\ast M$}{FALSE}
	\argm{d}{D}{division $y=x/D$}{FALSE}
	\desc[1ex]{If the argument of the operation option is
                ``{\em dB}'' then the value $20/\log_e10$
                is assigned.
		Also similarly, if ``{\em pi}'' is written after
                the operation option, then its value will be used.
                Actuary expression such as ``{\em ln2}'',
                ``{\em exp10}'', ``{\em sqrt30}'' can also be
                used as arguments.}
	\argm[1ex]{ABS}{}{absolute $y=|x|$}{FALSE}
	\argm{INV}{}{inverse $y=1/x$}{FALSE}
	\argm{P}{}{square $y=x^2$}{FALSE}
	\argm{R}{}{square root $y=\sqrt{x}$}{FALSE}
	\argm{SQRT}{}{square root $y=\sqrt{x}$}{FALSE}
	\argm{LN}{}{logarithm $y=\log{x}$}{FALSE}
	\argm{LOG2}{}{logarithm $y=\log_{2}{x}$}{FALSE}
	\argm{LOG10}{}{logarithm $y=\log_{10}{x}$}{FALSE}
	\argm{EXP}{}{exponential $y=\exp{x}$}{FALSE}
	\argm{POW2}{}{power of 2 $y=2^x$}{FALSE}
	\argm{POW10}{}{power of 10 $y=10^x$}{FALSE}
	\argm{FIX}{}{round $(int)x$}{FALSE}
	\argm{UNIT}{}{unit step $u(x)$}{FALSE}
	\argm{CLIP}{}{clipping $x \ast u(x)$}{FALSE}
	\argm{SIN}{}{sin $y=\sin(x)$}{FALSE}
	\argm{COS}{}{cos $y=\cos(x)$}{FALSE}
	\argm{TAN}{}{tan $y=\tan(x)$}{FALSE}
	\argm{ATAN}{}{atan $y=\atan(x)$}{FALSE}

	\argm{magic}{magic}{remove magic number}{FALSE}
	\argm{MAGIC}{MAGIC}{replace magic number by $MAGIC$ \\
                            if -magic option is not given, return error}{FALSE}

        \argm{r}{\mbox{m}n}{read from memory register m$n$ $(n=0..9)$}{}
        \argm{w}{\mbox{m}n}{write from memory register m$n$ $(n=0..9)$}{}
\end{options}

\begin{qsection}{EXAMPLE}
In the following example, a ramp function ($0,1,2,\ldots$)
is multiplied by 2 ($0,2,4,\ldots$)
and then 1 is added ($1,3,5,\ldots$):
\begin{quote}
  \verb!ramp | sopr -m 2 -a 1 | dmp +f!
\end{quote}
\par
The output file {\em data.avrg} contains the average taken from
data in files {\em data.f1} and {\em data.f2} read in float format:
\begin{quote}
  \verb!vopr -a data.f1 data.f2 | sopr -d 2 > data.avrg!
\end{quote}
\par
Data is read in float format from {\em data.f},
and the results in dB is written to the output:
\begin{quote}
  \verb!sopr data.f -LN -m dB | dmp +f!
\end{quote}
The following example gives the same results:
\begin{quote}
  \verb!sopr data.f -LOG10 -m 20 | dmp +f!
\end{quote}
\par
If we want to evaluate the following equation,
\[
y = (1 + 3x + 4x^2) / (1 + 2x + 5x^2) 
\]
then memory registers can be used as follows.
\begin{quote}
\verb!sopr data.f -w m0 -m 5 -a 2 -m m0 -a 1 -w m1 \!\\
\verb!       -r m0 -m 4 -a 3 -m m0 -a 1 -d m1 | dmp +f!
\end{quote}
In the example above, m0 and m1 are memory registers.
Registers from m0 to m9 can be used.
The --w option is used to write into memory register,
while the --r option is used to read from a register.
\end{qsection}

\begin{qsection}{SEE ALSO}
\hyperlink{vopr}{vopr},
\hyperlink{vsum}{vsum}
\end{qsection}
