% ----------------------------------------------------------------- %
%             The Speech Signal Processing Toolkit (SPTK)           %
%             developed by SPTK Working Group                       %
%             http://sp-tk.sourceforge.net/                         %
% ----------------------------------------------------------------- %
%                                                                   %
%  Copyright (c) 1984-2007  Tokyo Institute of Technology           %
%                           Interdisciplinary Graduate School of    %
%                           Science and Engineering                 %
%                                                                   %
%                1996-2014  Nagoya Institute of Technology          %
%                           Department of Computer Science          %
%                                                                   %
% All rights reserved.                                              %
%                                                                   %
% Redistribution and use in source and binary forms, with or        %
% without modification, are permitted provided that the following   %
% conditions are met:                                               %
%                                                                   %
% - Redistributions of source code must retain the above copyright  %
%   notice, this list of conditions and the following disclaimer.   %
% - Redistributions in binary form must reproduce the above         %
%   copyright notice, this list of conditions and the following     %
%   disclaimer in the documentation and/or other materials provided %
%   with the distribution.                                          %
% - Neither the name of the SPTK working group nor the names of its %
%   contributors may be used to endorse or promote products derived %
%   from this software without specific prior written permission.   %
%                                                                   %
% THIS SOFTWARE IS PROVIDED BY THE COPYRIGHT HOLDERS AND            %
% CONTRIBUTORS "AS IS" AND ANY EXPRESS OR IMPLIED WARRANTIES,       %
% INCLUDING, BUT NOT LIMITED TO, THE IMPLIED WARRANTIES OF          %
% MERCHANTABILITY AND FITNESS FOR A PARTICULAR PURPOSE ARE          %
% DISCLAIMED. IN NO EVENT SHALL THE COPYRIGHT OWNER OR CONTRIBUTORS %
% BE LIABLE FOR ANY DIRECT, INDIRECT, INCIDENTAL, SPECIAL,          %
% EXEMPLARY, OR CONSEQUENTIAL DAMAGES (INCLUDING, BUT NOT LIMITED   %
% TO, PROCUREMENT OF SUBSTITUTE GOODS OR SERVICES; LOSS OF USE,     %
% DATA, OR PROFITS; OR BUSINESS INTERRUPTION) HOWEVER CAUSED AND ON %
% ANY THEORY OF LIABILITY, WHETHER IN CONTRACT, STRICT LIABILITY,   %
% OR TORT (INCLUDING NEGLIGENCE OR OTHERWISE) ARISING IN ANY WAY    %
% OUT OF THE USE OF THIS SOFTWARE, EVEN IF ADVISED OF THE           %
% POSSIBILITY OF SUCH DAMAGE.                                       %
% ----------------------------------------------------------------- %
\hypertarget{delta}{}
\name{delta}{delta calculation}{data processing}

\begin{synopsis}
	\item [delta] [ --m $M$ ] [ --l $L$ ] [ --t $T$ ] 
		[ --d ($fn$ $|$ $d_0$ [$d_1$ $\dots$]) ]
		[ --r $N_R$ $W_1$ [$W_2$] ]
 \item [\ ~~~~~~~]
 [ --R $N_R$ ${W_F}_{1}$ ${W_B}_{1}$ [${W_F}_{2}$ ${W_B}_{2}$]]
 [ --M $magic$ ] [ --n $N$ ] [ --e $e$ ][ {\em infile} ]
\end{synopsis}

\begin{qsection}{DESCRIPTION}
	{\em delta} calculates dynamic features from {\em infile} (or standard
	input), sending the result (static and dynamic features) to
        the standard output. Input and output are of the form:
    \begin{align}
	\mathrm{input}  & \dots, x_t(0), \dots, x_t(M), \dots \notag \\
	\mathrm{output} & \dots, x_t(0), \dots, x_t(M), \Delta^{(1)} x_t(0), \dots, \Delta^{(1)} x_t(M), \dots,
	\Delta^{(n)} x_t(0), \dots, \Delta^{(n)} x_t(M), \dots \notag
 \end{align}
Also, input and output data are in float format.
The dynamic feature vector $\Delta^{(n)}\bx_t$ can be
obtained from the static feature vector as follows.
 \begin{displaymath}
	\Delta^{(n)}\bx_t 
	= \sum_{\tau=-L^{(n)}}^{L^{(n)}} w^{(n)}(\tau)\bx_{t+\tau}
 \end{displaymath}
where $n$ is the order of the dynamic feature vector. For
example, when we evaluate the $\Delta^2$ parameter, $n=2$.
\end{qsection}

\begin{options}
	\argm{m}{M}{order of vector}{25}
	\argm{l}{L}{length of vector }{$M+1$}
	\argm{d}{(fn~|~d_0~[d_1~\dots])}{$fn$ is 
                the file name of the parameters $w^{(n)}(\tau)$
               	used when evaluating the dynamic feature vector.
                It is assumed that the number of coefficients
                to the left and to the right are the same. In case
                this is not true, then zeros are added to the
                shortest side.
                For example, if the coefficients are given by:
	 \begin{displaymath}
		w(-1), w(0), w(1), w(2), w(3)
	 \end{displaymath}
		then zeros must be added to the left as follows.
	 \begin{displaymath}
		0, 0, w(-1), w(0), w(1), w(2), w(3)
	 \end{displaymath}
		Instead of entering the filename $fn$,
                the coefficients(which compose the file $fn$)
                can be directly inputted from the command line.
                When the order of the dynamic feature vector is higher
                than one, then the sets of coefficients can be inputted
                one after the other as shown in the example below.
		This option cannot be used with the --r nor --R options.}{N/A}
	\argm{r}{N_R~W_1~[W_2]}{
		This option is used when $N_R$-th order dynamic parameters
                are used and the weighting coefficients $w^{(n)}(\tau)$
                are evaluated by regression.
                $N_R$ can be made equal to 1 or 2.
		The variables $W_1$ and $W_2$ represent the
                widths of the first and second order regression
                coefficients, respectively.
		The first order regression coefficients for
                $\Delta\bx_t$ at frame $t$ are evaluated as follows.
	 \begin{displaymath}
		\Delta\bx_t
		= \frac{\sum_{\tau=-W_1}^{W_1}\tau \bc_{t+\tau}}%
			{\sum_{\tau=-W_1}^{W_1}\tau^2}
	 \end{displaymath}
		For the second order regression coefficients,
		$a_2 = \sum_{\tau=-W_2}^{W_2} \tau^4$,
		$a_1 = \sum_{\tau=-W_2}^{W_2} \tau^2$,
		$a_0 = \sum_{\tau=-W_2}^{W_2} 1$
                and 
	 \begin{displaymath}
		\Delta^2\bx_t
		= \frac{2\sum_{\tau=-W_2}^{W_2}
				(a_0\tau^2 - a_1) \bx_{t+\tau}}
			{a_2a_0-a_1^2}
	 \end{displaymath}
 This option cannot be used with the --d nor --R options.}{N/A}
 \argm{R}{N_R~{W_F}_1~{W_B}_1[{W_F}_2~{W_B}_2]}{
Similarly to the --r option, by using this option, we can obtain
$N_{R}$-th order dynamic feature parameters  and the weighting
coefficients will be evaluated by regression. $N_{R}$ can be made equal
 to 1 or 2. The variables $W_{Fi}$ and $W_{Bi}$ represent the width of the
 $i$-th order regression
 coefficients in the forward and backward direction, respectively.
 Combining this option with the --M option, the regression coefficients can be evaluated skipping the magic
 number from the input.
 This option cannot be used with the --d nor --r options.
 }{N/A}
 \argm{M}{magic}{
 The magic number $magic$ can be skipped from the input during the calculation of
 the dynamic features. This option is valid
 only when the --R option is also specified.
 }{N/A}
 \argm{n}{N}{$N$ is the order of regression polynomial.
 Note that $N$ must be less than or equal to $\mathop{\rm max}\limits_{i=1,2}
 \left(W_{Fi} + W_{Bi}\right)$.}{N/A}
 \argm{e}{e}{
 small value added to diagonal component for calculating inverse matrix
 }{0.0}
\end{options}

\begin{qsection}{EXAMPLE}
In the example below, the first and second order dynamic features are calculated from 15-dimensional
coefficient vectors from {\em data.static} using windows whose width
are 1. The resultant
static and dynamic features are sent to {\em data.delta}:
 \begin{quote}
	\verb!delta -m 15 -r 2 1 1 data.static > data.delta!
 \end{quote}
	or
 \begin{quote}
	\verb!echo "-0.5 0 0.5" | x2x +af > delta! \\
	\verb!echo "1.0 -2.0 1.0" | x2x +af > accel! \\
	\verb!delta -m 15 -d delta -d accel data.static > data.delta!
 \end{quote}
Another example is presented bellow, where the first and second order
dynamic features are calculated from the scalar sequence in {\em
  data.f0}, sending windows with 2 units width and skipping the magic
number -1.0E15.
 \begin{quote}
	\verb!delta -l 1 -R 2 2 2 2 2 -M -1.0E15 data.f0 > data.delta!
 \end{quote}
\end{qsection}

\begin{qsection}{SEE ALSO}
\hyperlink{mlpg}{mlpg}
\end{qsection}
