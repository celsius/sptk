\name[ref:MGLSA-IECE]{imglsadf}{inverse MGLSA digital filter}
{filters for speech synthesis}

\begin{synopsis}
\item [imglsadf] [ --m $M$ ] [ --a $A$ ] [ --g $G$ ] [ --p $P$ ]
		 [ --i $I$ ]  [ --t ]  [ --k ]
\item [\ ~~~~~~~~~] {\em mgcfile} [ {\em infile} ]
\end{synopsis}

\begin{qsection}{DESCRIPTION}
This commands reads the mel-generalized cepstrum coefficients
 $c_{\alpha,\gamma}(m)$, evaluates the inverse filter,
passes the speech input data file through the calculated
inverse filter, and sends the results to the standard output.
\par
Input and output data are in float format.
\end{qsection}

\begin{options}
	\argm{m}{M}{order of generalized cepstrum}{25}
	\argm{a}{A}{$\alpha$}{0.35}
	\argm{g}{G}{power parameter $\gamma$ of generalized cepstrum\\
			if $G>1.0$ then $\gamma=-1/G$.}{1}
	\argm{p}{P}{frame period}{100}
	\argm{i}{I}{interpolation period}{1}
	\argm{t}{}{transpose filter}{FALSE}
	\argm{k}{}{filtering without gain}{FALSE}
\end{options}

\begin{qsection}{EXAMPLE}
In the first line of the example below, 
speech is read in float format from {\em data.f},
the mel-generalized cepstrum coefficients $(M=15,\alpha=0.35,\gamma=-1/2)$
are evaluated, and written in {\em data.mgc}.
In the second line,
speech data and cepstral coefficients are
inputed to {\em imglsadf} command and the resulting excitation file
is written to {\em data.e}.
\begin{quote}
 \verb!frame < data.f | window | \!\\
 \verb!    mgcep -m 15 -a 0.35 -g -0.5 > data.mgc!\\
 \verb!imglsadf -m 15 -a 0.35 -g -0.5 data.mgc < data.f \!\\
 \verb!    > data.e!
\end{quote} 
\end{qsection}

\begin{qsection}{BUGS}
Correct response of this command is obtained only when
$\gamma = -1/n$, with $n$ a natural number.
\end{qsection}

\begin{qsection}{SEE ALSO}
 mgcep, amgcep, ltcdf, lmadf, mlsadf, glsadf, lspdf
\end{qsection}
