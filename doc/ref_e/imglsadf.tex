% ----------------------------------------------------------------- %
%             The Speech Signal Processing Toolkit (SPTK)           %
%             developed by SPTK Working Group                       %
%             http://sp-tk.sourceforge.net/                         %
% ----------------------------------------------------------------- %
%                                                                   %
%  Copyright (c) 1984-2007  Tokyo Institute of Technology           %
%                           Interdisciplinary Graduate School of    %
%                           Science and Engineering                 %
%                                                                   %
%                1996-2008  Nagoya Institute of Technology          %
%                           Department of Computer Science          %
%                                                                   %
% All rights reserved.                                              %
%                                                                   %
% Redistribution and use in source and binary forms, with or        %
% without modification, are permitted provided that the following   %
% conditions are met:                                               %
%                                                                   %
% - Redistributions of source code must retain the above copyright  %
%   notice, this list of conditions and the following disclaimer.   %
% - Redistributions in binary form must reproduce the above         %
%   copyright notice, this list of conditions and the following     %
%   disclaimer in the documentation and/or other materials provided %
%   with the distribution.                                          %
% - Neither the name of the SPTK working group nor the names of its %
%   contributors may be used to endorse or promote products derived %
%   from this software without specific prior written permission.   %
%                                                                   %
% THIS SOFTWARE IS PROVIDED BY THE COPYRIGHT HOLDERS AND            %
% CONTRIBUTORS "AS IS" AND ANY EXPRESS OR IMPLIED WARRANTIES,       %
% INCLUDING, BUT NOT LIMITED TO, THE IMPLIED WARRANTIES OF          %
% MERCHANTABILITY AND FITNESS FOR A PARTICULAR PURPOSE ARE          %
% DISCLAIMED. IN NO EVENT SHALL THE COPYRIGHT OWNER OR CONTRIBUTORS %
% BE LIABLE FOR ANY DIRECT, INDIRECT, INCIDENTAL, SPECIAL,          %
% EXEMPLARY, OR CONSEQUENTIAL DAMAGES (INCLUDING, BUT NOT LIMITED   %
% TO, PROCUREMENT OF SUBSTITUTE GOODS OR SERVICES; LOSS OF USE,     %
% DATA, OR PROFITS; OR BUSINESS INTERRUPTION) HOWEVER CAUSED AND ON %
% ANY THEORY OF LIABILITY, WHETHER IN CONTRACT, STRICT LIABILITY,   %
% OR TORT (INCLUDING NEGLIGENCE OR OTHERWISE) ARISING IN ANY WAY    %
% OUT OF THE USE OF THIS SOFTWARE, EVEN IF ADVISED OF THE           %
% POSSIBILITY OF SUCH DAMAGE.                                       %
% ----------------------------------------------------------------- %
\hypertarget{imglsadf}{}
\name[ref:MGLSA-IECE,ref:MGLSA-IEICE]{imglsadf}{inverse MGLSA digital filter}
{filters for speech synthesis}

\begin{synopsis}
\item [imglsadf] [ --m $M$ ] [ --a $A$ ] [ --g $G$ ] [ --p $P$ ]
		 [ --i $I$ ]  [ --t ]  [ --k ]
\item [\ ~~~~~~~~~] {\em mgcfile} [ {\em infile} ]
\end{synopsis}

\begin{qsection}{DESCRIPTION}
{\em imglsadf} derives an Inverse Mel-Generalized 
Log Spectral Approximation filter from 
mel-generalized cepstral coefficients $c_{\alpha,\gamma}(m)$ in {\em gcfile} 
and uses it to filter speech data from {\em infile} (or standard input) 
to generate an excitation sequence,
sending the result to standard output.

Input and output data are in float format.
\end{qsection}

\begin{options} 
	\argm{m}{M}{order of generalized cepstrum}{25}
	\argm{a}{A}{$\alpha$}{0.35}
	\argm{g}{G}{power parameter $\gamma=-1/G$ for generalized cepstrum}{1}
	\argm{p}{P}{frame period}{100}
	\argm{i}{I}{interpolation period}{1}
	\argm{t}{}{transpose filter}{FALSE}
	\argm{k}{}{filtering without gain}{FALSE}
\end{options}

\begin{qsection}{EXAMPLE}
In the first line of the example below, 
speech is read in float format from {\em data.f},
the mel-generalized cepstral coefficients $(M=15,\alpha=0.35,\gamma=-1/2)$
are evaluated, and written in {\em data.mgc}.
In the second line,
speech data and cepstral coefficients are
inputed to {\em imglsadf} command and the resulting excitation file
is written to {\em data.e}.
\begin{quote}
 \verb!frame +f < data.f | window | \!\\
 \verb!    mgcep -m 15 -a 0.35 -g -0.5 > data.mgc!\\
 \verb!imglsadf -m 15 -a 0.35 -g -0.5 data.mgc < data.f \!\\
 \verb!    > data.e!
\end{quote} 
\end{qsection}

\begin{qsection}{SEE ALSO}
\hyperlink{mgcep}{mgcep},
\hyperlink{ltcdf}{ltcdf},
\hyperlink{lmadf}{lmadf},
\hyperlink{mlsadf}{mlsadf},
\hyperlink{glsadf}{glsadf},
\hyperlink{lspdf}{lspdf}
\end{qsection}
