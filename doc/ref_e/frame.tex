% ----------------------------------------------------------------- %
%             The Speech Signal Processing Toolkit (SPTK)           %
%             developed by SPTK Working Group                       %
%             http://sp-tk.sourceforge.net/                         %
% ----------------------------------------------------------------- %
%                                                                   %
%  Copyright (c) 1984-2007  Tokyo Institute of Technology           %
%                           Interdisciplinary Graduate School of    %
%                           Science and Engineering                 %
%                                                                   %
%                1996-2011  Nagoya Institute of Technology          %
%                           Department of Computer Science          %
%                                                                   %
% All rights reserved.                                              %
%                                                                   %
% Redistribution and use in source and binary forms, with or        %
% without modification, are permitted provided that the following   %
% conditions are met:                                               %
%                                                                   %
% - Redistributions of source code must retain the above copyright  %
%   notice, this list of conditions and the following disclaimer.   %
% - Redistributions in binary form must reproduce the above         %
%   copyright notice, this list of conditions and the following     %
%   disclaimer in the documentation and/or other materials provided %
%   with the distribution.                                          %
% - Neither the name of the SPTK working group nor the names of its %
%   contributors may be used to endorse or promote products derived %
%   from this software without specific prior written permission.   %
%                                                                   %
% THIS SOFTWARE IS PROVIDED BY THE COPYRIGHT HOLDERS AND            %
% CONTRIBUTORS "AS IS" AND ANY EXPRESS OR IMPLIED WARRANTIES,       %
% INCLUDING, BUT NOT LIMITED TO, THE IMPLIED WARRANTIES OF          %
% MERCHANTABILITY AND FITNESS FOR A PARTICULAR PURPOSE ARE          %
% DISCLAIMED. IN NO EVENT SHALL THE COPYRIGHT OWNER OR CONTRIBUTORS %
% BE LIABLE FOR ANY DIRECT, INDIRECT, INCIDENTAL, SPECIAL,          %
% EXEMPLARY, OR CONSEQUENTIAL DAMAGES (INCLUDING, BUT NOT LIMITED   %
% TO, PROCUREMENT OF SUBSTITUTE GOODS OR SERVICES; LOSS OF USE,     %
% DATA, OR PROFITS; OR BUSINESS INTERRUPTION) HOWEVER CAUSED AND ON %
% ANY THEORY OF LIABILITY, WHETHER IN CONTRACT, STRICT LIABILITY,   %
% OR TORT (INCLUDING NEGLIGENCE OR OTHERWISE) ARISING IN ANY WAY    %
% OUT OF THE USE OF THIS SOFTWARE, EVEN IF ADVISED OF THE           %
% POSSIBILITY OF SUCH DAMAGE.                                       %
% ----------------------------------------------------------------- %
\hypertarget{frame}{}
\name{frame}{extract frame from data sequence}{signal processing,speech analysis and synthesis}

\begin{synopsis}
 \item [frame] [ --l $L$ ] [ --n ] [ --p $P$ ] [ {\em infile} ]
\end{synopsis}

\begin{qsection}{DESCRIPTION}
{\em frame} converts a sequence of input data 
from {\em infile} (or standard input) 
to a series of possibly-overlapping frames with period $P$ and length $L$, 
sending the result to standard output.
If the input data is $x(0),x(1),\ldots,x(T)$, then the output data is
\begin{center}
\begin{tabular}{ccccccccccc}
$0$&$,$&$0$&$,$&$\ldots$&$,$&$x(0)$&$,$&$\ldots$&$,$&$x(L/2)$\\
$x(P-L/2)$&$,$&$x(P-L/2+1)$&$,$&$\ldots$&$,$&$x(P)$&$,$&$\ldots$&$,$&$x(P+L/2)$\\
$x(2P-L/2)$&$,$&$x(2P-L/2+1)$&$,$&$\ldots$&$,$&$x(2P)$&$,$&$\ldots$&$,$&$x(2P+L/2)$\\
&&&&&&$\vdots$&&&&
\end{tabular}
\end{center}

\end{qsection}

\begin{options}
	\argm{l}{L}{frame length}{256}
	\argm{p}{P}{frame period}{100}
	\argm{n}{}{This option is used when instead of having x(0) is
                   center point in the first frame we want to make x(0)
                   as the first point of the first frame}{FALSE}
\end{options}


\begin{qsection}{EXAMPLE}
In the example below, data is read from {\em data.f} file,
frames of period 80 are applied, a Blackman window is passed, and
a linear prediction analysis is undertaken. The output is written in
{\em data.lpc} file:
\begin{quote}
 \verb!frame -p 80 < data.f | window | lpc > data.lpc!
\end{quote} 
\end{qsection}

\begin{qsection}{SEE ALSO}
\hyperlink{bcp}{bcp},
\hyperlink{x2x}{x2x},
\hyperlink{bcut}{bcut},
\hyperlink{window}{window}
\end{qsection}
