% ----------------------------------------------------------------
%       Speech Signal Processing Toolkit (SPTK): version 3.0
%                      SPTK Working Group
% 
%                Department of Computer Science
%                Nagoya Institute of Technology
%                             and
%   Interdisciplinary Graduate School of Science and Engineering
%                Tokyo Institute of Technology
%                   Copyright (c) 1984-2000
%                     All Rights Reserved.
% 
% Permission is hereby granted, free of charge, to use and
% distribute this software and its documentation without
% restriction, including without limitation the rights to use,
% copy, modify, merge, publish, distribute, sublicense, and/or
% sell copies of this work, and to permit persons to whom this
% work is furnished to do so, subject to the following conditions:
% 
%   1. The code must retain the above copyright notice, this list
%      of conditions and the following disclaimer.
% 
%   2. Any modifications must be clearly marked as such.
%                                                                        
% NAGOYA INSTITUTE OF TECHNOLOGY, TOKYO INSITITUTE OF TECHNOLOGY,
% SPTK WORKING GROUP, AND THE CONTRIBUTORS TO THIS WORK DISCLAIM
% ALL WARRANTIES WITH REGARD TO THIS SOFTWARE, INCLUDING ALL
% IMPLIED WARRANTIES OF MERCHANTABILITY AND FITNESS, IN NO EVENT
% SHALL NAGOYA INSTITUTE OF TECHNOLOGY, TOKYO INSITITUTE OF
% TECHNOLOGY, SPTK WORKING GROUP, NOR THE CONTRIBUTORS BE LIABLE
% FOR ANY SPECIAL, INDIRECT OR CONSEQUENTIAL DAMAGES OR ANY
% DAMAGES WHATSOEVER RESULTING FROM LOSS OF USE, DATA OR PROFITS,
% WHETHER IN AN ACTION OF CONTRACT, NEGLIGENCE OR OTHER TORTIOUS
% ACTION, ARISING OUT OF OR IN CONNECTION WITH THE USE OR
% PERFORMANCE OF THIS SOFTWARE.
% ----------------------------------------------------------------
%
\name{frame}{extract frame from data sequence}{signal processing,speech analysis and synthesis}

\begin{synopsis}
 \item [frame] [ --l $L$ ] [ --n ] [ --p $P$ ] [ +{\em type} ] [ {\em infile} ]
\end{synopsis}

\begin{qsection}{DESCRIPTION}
{\em frame} converts a sequence of input data 
from {\em infile} (or standard input) 
to a series of possibly-overlapping frames with period $P$ and length $L$, 
sending the result to standard output.
If the input data is $x(0),x(1),\ldots,x(T)$, then the output data is
\begin{center}
\begin{tabular}{ccccccccccc}
$0$&$,$&$0$&$,$&$\ldots$&$,$&$x(0)$&$,$&$\ldots$&$,$&$x(L/2)$\\
$x(P-L/2)$&$,$&$x(P-L/2+1)$&$,$&$\ldots$&$,$&$x(P)$&$,$&$\ldots$&$,$&$x(P+L/2)$\\
$x(2P-L/2)$&$,$&$x(2P-L/2+1)$&$,$&$\ldots$&$,$&$x(2P)$&$,$&$\ldots$&$,$&$x(2P+L/2)$\\
&&&&&&$\vdots$&&&&
\end{tabular}
\end{center}

\end{qsection}

\begin{options}
	\argm{l}{L}{frame length}{256}
	\argm{p}{P}{frame period}{100}
	\argm{n}{}{This option is used when instead of having x(0) is
                   center point in the first frame we want to make x(0)
                   as the first point of the first frame}{FALSE}
	\argp{t}{data type\\ 
		\begin{tabular}{llcll} \\[-1ex]
			c & char (1byte) & \quad &
			s & short (2bytes) \\
			i & int (4bytes) & \quad &
			l & long (4bytes) \\
			f & float (4bytes) & \quad &
			d & double (8bytes) 
		\end{tabular}\\\hspace*{\fill}}{f}
\end{options}


\begin{qsection}{EXAMPLE}
In the example below, data in float format is read from {\em data.f} file,
frames of period 80 are applied, a Blackman window is passed, and
a linear prediction analysis is undertaken. The output is written in
{\em data.lpc} file:
\begin{quote}
 \verb!frame -p 80 < data.f | window | lpc > data.lpc!
\end{quote} 
\end{qsection}

\begin{qsection}{SEE ALSO}
 bcp, x2x, bcut, window
\end{qsection}
