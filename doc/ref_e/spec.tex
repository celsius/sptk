\name{spec}{transform real sequence to log spectrum}{signal processing}

\begin{synopsis}
 \item[spec]   [ --l $L$ ] [ --m $M$ ] [ --n $N$ ] [ --z {\em zfile} ]
               [ --p {\em pfile} ]
 \item[\ ~~~~] [ --y $Y$ ] [ {\em infile} ]
\end{synopsis}

\begin{qsection}{DESCRIPTION}
This command reads a real sequence and computes the log spectrum
magnitude.
If the input sequence is
\begin{displaymath}
  x(0), x(1), \ldots, x(L-1)
\end{displaymath}
and the FFT algorithm is used to evaluate
\begin{eqnarray*}
  X_k &=& X(e^{j\omega}) \left|
	\begin{array}{c}
	\\
        \omega=\frac{2\pi k}{L}
	\end{array}
    \right. \nonumber \\
          &=& \sum_{m=0}^{L-1}x(m)e^{-j\omega m} \left|
	\begin{array}{c}
	\\
        \omega=\frac{2\pi\, k}{L}
	\end{array}
    \right.,~~~~ k=0,1,\ldots,L-1
\end{eqnarray*}
then if the {\bf --y} option is applied, then
output is
\begin{displaymath}
  Y_k=20\,\log_{10}\,|X_k|,~~~~ k=0,1,\ldots,L/2
\end{displaymath}
The output data corresponds to angular frequencies varying from $0\sim \pi$.
Input and output data are in float format.
\par
If the {\bf --p, --z} options are assigned
then the phase of the corresponding filter related to
the assigned coefficients is calculated
\footnote{
In this case the phase is not evaluated from the filter
impulse response, the phase is evaluated from
the difference between the numerator and denominator phases}.
\end{qsection}

\begin{options}
	\argm{l}{L}{FFT window length\\
                    $L$ must be power of 2}{256}
	\argm{m}{M}{order of MA part\\
			In the case where the number of input data
                        values is less then $M+1$, then $M$ is made
                        equal to the number of input data values $-1$.
                        If the input data should not be analyzed in blocks of
                        size $M+1$, then it is not necessary to assign
                        a value to $M$.}{$L-1$}
	\argm{n}{N}{order of AR part\\
                        Similarly to the --m option,
			in the case where the number of input data
                        values is less then $N+1$, then $N$ is made
                        equal to the number of input data values $-1$.
                        If the input data should not be analyzed in blocks of
                        size $N+1$, then it is not necessary to assign
                        a value to $N$.}{$L-1$}
	\argm{z}{zfile}{MA coefficients filename\\
			The {\em zfile} should follow this structure in
                        float format:\\
			\hspace*{2zw}$b(0), b(1), \ldots, b(N)$\\[-1zh]}{NULL}
	\argm{p}{pfile}{AR coefficients filename\\
			The {\em pfile} should follow this structure in
                        float format:\\
			\hspace*{2zw}$K, a(1), \ldots, a(M)$\\[-1zh]}{NULL}
	\argm{y}{Y}{output format\\
			\begin{tabular}{lll} \\[-1zh]
			$Y=0$ & ~~~$20\times\log |X_k|$ & $k=0,1,\ldots,L/2$\\
			$Y=1$ & ~~~$\ln |X_k|$ & $k=0,1,\ldots,L/2$\\
			$Y=2$ & ~~~$|X_k|$ & $k=0,1,\ldots,L/2$\\
			\end{tabular} \\\hspace*{\fill}}{0}
	\desc{The contents of {\em pfile} and {\em zfile}
                        should be in a similar form to that used in
                        command {\em dfs}.
			When only the {\bf --p} option is assigned
                        then the denominator is made equal to 1.
                        When only the {\bf --z} option is assigned
                        then the numerator and the gain $K$ are made
                        equal to 1.
                        If neither {\bf --p} nor {\bf --z} are
                        assigned, data is read from the standard input.}
\end{options}

\begin{qsection}{EXAMPLE}
In the example below, a pulse train excitation is passed
through digital filter and Blackman window,
then the log spectrum magnitude is evaluated and plotted
on the screen:
\begin{quote}
  \verb!train -t 50 | dfs -a 1 0.9 | window | spec | fdrw | xgr !
\end{quote}
\par
This example evaluates the frequency response of
digital filter with coefficients assigned by {\em data.p, data.z}
in float format:
\begin{quote}
  \verb!spec -p data.p -z data.z | fdrw | xgr !
\end{quote}
A similar results can be obtained with the following command
when filter is stable:
\begin{quote}
  \verb!impulse | dfs -p data.p -z data.z | spec | fdrw | xgr !
\end{quote}
\end{qsection}

\begin{qsection}{SEE ALSO}
  phase, fft, fftr, dfs
\end{qsection}
