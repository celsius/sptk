\name{reverse}{reverse the order of data in each block}{data}

\begin{synopsis}
\item[reverse] [ --l $L$ ] [ --n $N$ ] [ {\em infile} ]
\end{synopsis}

\begin{qsection}{DESCRIPTION}

The reverse command reads data sequence in float format from
the standard input, cuts it into blocks of size assigned by an option,
reverses the order inside each block and sends the results
to the standard output.
If the block size is not assigned, then the whole file is assigned
as the default.
If the whole file can not be devided exactly into blocks of the
assigned length, then the remainder values are disregarded as
shown in the example below.

\end{qsection}

\begin{options}
	\argm{l}{L}{length of block}{EOF}
	\argm{n}{N}{order of block}{EOF-1}
\end{options}

\begin{qsection}{EXAMPLE}
Let's assmue that the following data
is read from {\em data.in} file in float format.
\begin{displaymath}
 \underbrace{0.0, ~1.0, ~2.0}, ~
 \underbrace{3.0, ~4.0, ~5.0}, ~
 \underbrace{6.0, ~7.0, ~8.0}, ~9.0
\end{displaymath}
The command
\begin{quote}
\verb!reverse -l 3 data.in > data.out!
\end{quote}
will write the output below to {\em data.out}.
\begin{displaymath}
 \underbrace{2.0, ~1.0, ~0.0}, ~
 \underbrace{5.0, ~4.0, ~3.0}, ~
 \underbrace{8.0, ~7.0, ~6.0}
\end{displaymath}
\end{qsection}
