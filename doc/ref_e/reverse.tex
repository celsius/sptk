%  ---------------------------------------------------------------  %
%            Speech Signal Processing Toolkit (SPTK)                %
%                      SPTK Working Group                           %
%                                                                   %
%                  Department of Computer Science                   %
%                  Nagoya Institute of Technology                   %
%                               and                                 %
%   Interdisciplinary Graduate School of Science and Engineering    %
%                  Tokyo Institute of Technology                    %
%                                                                   %
%                     Copyright (c) 1984-2007                       %
%                       All Rights Reserved.                        %
%                                                                   %
%  Permission is hereby granted, free of charge, to use and         %
%  distribute this software and its documentation without           %
%  restriction, including without limitation the rights to use,     %
%  copy, modify, merge, publish, distribute, sublicense, and/or     %
%  sell copies of this work, and to permit persons to whom this     %
%  work is furnished to do so, subject to the following conditions: %
%                                                                   %
%    1. The source code must retain the above copyright notice,     %
%       this list of conditions and the following disclaimer.       %
%                                                                   %
%    2. Any modifications to the source code must be clearly        %
%       marked as such.                                             %
%                                                                   %
%    3. Redistributions in binary form must reproduce the above     %
%       copyright notice, this list of conditions and the           %
%       following disclaimer in the documentation and/or other      %
%       materials provided with the distribution.  Otherwise, one   %
%       must contact the SPTK working group.                        %
%                                                                   %
%  NAGOYA INSTITUTE OF TECHNOLOGY, TOKYO INSTITUTE OF TECHNOLOGY,   %
%  SPTK WORKING GROUP, AND THE CONTRIBUTORS TO THIS WORK DISCLAIM   %
%  ALL WARRANTIES WITH REGARD TO THIS SOFTWARE, INCLUDING ALL       %
%  IMPLIED WARRANTIES OF MERCHANTABILITY AND FITNESS, IN NO EVENT   %
%  SHALL NAGOYA INSTITUTE OF TECHNOLOGY, TOKYO INSTITUTE OF         %
%  TECHNOLOGY, SPTK WORKING GROUP, NOR THE CONTRIBUTORS BE LIABLE   %
%  FOR ANY SPECIAL, INDIRECT OR CONSEQUENTIAL DAMAGES OR ANY        %
%  DAMAGES WHATSOEVER RESULTING FROM LOSS OF USE, DATA OR PROFITS,  %
%  WHETHER IN AN ACTION OF CONTRACT, NEGLIGENCE OR OTHER TORTUOUS   %
%  ACTION, ARISING OUT OF OR IN CONNECTION WITH THE USE OR          %
%  PERFORMANCE OF THIS SOFTWARE.                                    %
%                                                                   %
%  ---------------------------------------------------------------  %
%
\hypertarget{reverse}{}
\name{reverse}{reverse the order of data in each block}{data operation}

\begin{synopsis}
\item[reverse] [ --l $L$ ] [ --n $N$ ] [ {\em infile} ]
\end{synopsis}

\begin{qsection}{DESCRIPTION}
{\em reverse} reverses the order of data within $L$-length blocks 
of input data from {\em infile} (or standard input), 
sending the result to standard output. 
The default value for $L$ is the entire file. 
If $L$ is given but the file length is not a multiple of $L$, 
leftover values are discarded as shown in the example below.
\end{qsection}

\begin{options}
	\argm{l}{L}{length of block}{EOF}
	\argm{n}{N}{order of block}{EOF-1}
\end{options}

\begin{qsection}{EXAMPLE}
Let's assume that the following data
is read from {\em data.in} file in float format.
\begin{displaymath}
 \underbrace{0.0, ~1.0, ~2.0}, ~
 \underbrace{3.0, ~4.0, ~5.0}, ~
 \underbrace{6.0, ~7.0, ~8.0}, ~9.0
\end{displaymath}
The command
\begin{quote}
\verb!reverse -l 3 data.in > data.out!
\end{quote}
will write the output below to {\em data.out}.
\begin{displaymath}
 \underbrace{2.0, ~1.0, ~0.0}, ~
 \underbrace{5.0, ~4.0, ~3.0}, ~
 \underbrace{8.0, ~7.0, ~6.0}
\end{displaymath}
\end{qsection}
