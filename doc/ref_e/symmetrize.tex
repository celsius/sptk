% ----------------------------------------------------------------- %
%             The Speech Signal Processing Toolkit (SPTK)           %
%             developed by SPTK Working Group                       %
%             http://sp-tk.sourceforge.net/                         %
% ----------------------------------------------------------------- %
%                                                                   %
%  Copyright (c) 1984-2007  Tokyo Institute of Technology           %
%                           Interdisciplinary Graduate School of    %
%                           Science and Engineering                 %
%                                                                   %
%                1996-2017  Nagoya Institute of Technology          %
%                           Department of Computer Science          %
%                                                                   %
% All rights reserved.                                              %
%                                                                   %
% Redistribution and use in source and binary forms, with or        %
% without modification, are permitted provided that the following   %
% conditions are met:                                               %
%                                                                   %
% - Redistributions of source code must retain the above copyright  %
%   notice, this list of conditions and the following disclaimer.   %
% - Redistributions in binary form must reproduce the above         %
%   copyright notice, this list of conditions and the following     %
%   disclaimer in the documentation and/or other materials provided %
%   with the distribution.                                          %
% - Neither the name of the SPTK working group nor the names of its %
%   contributors may be used to endorse or promote products derived %
%   from this software without specific prior written permission.   %
%                                                                   %
% THIS SOFTWARE IS PROVIDED BY THE COPYRIGHT HOLDERS AND            %
% CONTRIBUTORS "AS IS" AND ANY EXPRESS OR IMPLIED WARRANTIES,       %
% INCLUDING, BUT NOT LIMITED TO, THE IMPLIED WARRANTIES OF          %
% MERCHANTABILITY AND FITNESS FOR A PARTICULAR PURPOSE ARE          %
% DISCLAIMED. IN NO EVENT SHALL THE COPYRIGHT OWNER OR CONTRIBUTORS %
% BE LIABLE FOR ANY DIRECT, INDIRECT, INCIDENTAL, SPECIAL,          %
% EXEMPLARY, OR CONSEQUENTIAL DAMAGES (INCLUDING, BUT NOT LIMITED   %
% TO, PROCUREMENT OF SUBSTITUTE GOODS OR SERVICES; LOSS OF USE,     %
% DATA, OR PROFITS; OR BUSINESS INTERRUPTION) HOWEVER CAUSED AND ON %
% ANY THEORY OF LIABILITY, WHETHER IN CONTRACT, STRICT LIABILITY,   %
% OR TORT (INCLUDING NEGLIGENCE OR OTHERWISE) ARISING IN ANY WAY    %
% OUT OF THE USE OF THIS SOFTWARE, EVEN IF ADVISED OF THE           %
% POSSIBILITY OF SUCH DAMAGE.                                       %
% ----------------------------------------------------------------- %
\hypertarget{symmetrize}{}
\name{symmetrize}{symmetrize the sequence of data}{data operation}

\begin{synopsis}
\item[symmetrize] [ --l $L$ ] [ --o $o$ ] [ {\em infile} ]
\end{synopsis}

\begin{qsection}{DESCRIPTION}
{\em symmetrize} symmetrizes the sequence of $L/2$-length
of input data from {\em infile} (or standard input) and 
sends the result to standard output.
The value of $L$ must be even number.
The output format is specified by the -o option.
If the file length is not a multiple of $L/2$, 
leftover values are discarded as shown in the example below.
\begin{displaymath}
\begin{array}{ll}
\mbox{Input sequence} & x(0),~x(1),~\ldots,~x(L/2-1)
\end{array}
\end{displaymath}
\end{qsection}

\begin{options}
	\argm{l}{L}{frame length}{256}
	\argm{o}{o}{output format\\
 \begin{tabular}{ll} \\[-1ex]
                        $o=0$ & $x(0),~x(1), ~\ldots, ~x(L/2-1), ~x(L/2-2), ~\ldots, ~x(2), ~x(1)$ \\
                        $o=1$ & $x(L/2-1), ~x(L/2-2), ~\ldots, ~x(1), ~x(0), ~x(1), ~\ldots, ~x(L/2-1)$ \\
                        $o=2$ & $x(L/2-1)/2, ~x(L/2-2), ~\ldots, ~x(1), ~x(0), ~x(1), ~\ldots, ~x(L/2-1)/2$ \\[1ex]
                \end{tabular} \\\hspace*{\fill}}{0}
\end{options}
\begin{qsection}{EXAMPLE}
Let's assume that the following data
is read from {\em data.in} file in float format.
\begin{displaymath}
 \underbrace{0.0, ~1.0, ~2.0, 3.0}, ~4.0
\end{displaymath}
The command
\begin{quote}
\verb!symmetrize -l 8 -o 1 data.in > data.out!
\end{quote}
will write the following output to {\em data.out}.
\begin{displaymath}
 \underbrace{3.0, ~2.0, ~1.0, ~0.0, ~1.0, ~2.0, ~3.0}
\end{displaymath}
\end{qsection}

\begin{qsection}{NOTICE}
\begin{itemize}
\item value of $L$ must be even number.
\item value of $L$ must be $L \geq 4$.
\item value of $L$ must be $L \geq 6$ (if $o==0$).
\end{itemize}
\end{qsection}
