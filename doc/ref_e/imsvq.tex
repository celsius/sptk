\name{imsvq}{decoder of multi stage vector quantization}{vector quantization}

\begin{synopsis}
\item [imsvq] [ --l $L$ ] [ --n $N$ ] [ --s $S \;$ {\em cbfile} ] [ {\em infile} ]
\end{synopsis}

\begin{qsection}{DESCRIPTION}
The {\em imsvq} command reads the codebook indexes from {\em infile}
and sends the resulting multi stage output to the standard output.
The order of multi stage decoder is equal to the number of --s options
used.
\par
Input data is in int format, and output data is in float format.
\end{qsection}

\begin{options}
	\argm{l}{L}{length of vector}{26}
	\argm{n}{N}{order of vector}{L-1}
	\argm{s}{S \; cbfile}{codebook\\
		\begin{tabular}{ll}\\[-1zh]
		$S$ & codebook size\\
		$cbfile$ & codebook file \\
		\end{tabular}\\\hspace*{\fill}}{N/A N/A}
\end{options}

\begin{qsection}{EXAMPLE}
In the example below,
the decoded vector {\em data.ivq} is obtained
from the first stage codebook {\em cbfile1}
and the second stage codebook {\em cbfile2},
both of size 256, as well as from the index file {\em data.vq}.
\begin{quote}
\verb! imsvq -s 256 cbfile1 -s 256 cbfile2 < data.vq > data.ivq!
\end{quote}
\end{qsection}

\begin{qsection}{SEE ALSO}
msvq, ivq, vq
\end{qsection}
