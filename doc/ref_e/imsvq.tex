% ----------------------------------------------------------------- %
%             The Speech Signal Processing Toolkit (SPTK)           %
%             developed by SPTK Working Group                       %
%             http://sp-tk.sourceforge.net/                         %
% ----------------------------------------------------------------- %
%                                                                   %
%  Copyright (c) 1984-2007  Tokyo Institute of Technology           %
%                           Interdisciplinary Graduate School of    %
%                           Science and Engineering                 %
%                                                                   %
%                1996-2013  Nagoya Institute of Technology          %
%                           Department of Computer Science          %
%                                                                   %
% All rights reserved.                                              %
%                                                                   %
% Redistribution and use in source and binary forms, with or        %
% without modification, are permitted provided that the following   %
% conditions are met:                                               %
%                                                                   %
% - Redistributions of source code must retain the above copyright  %
%   notice, this list of conditions and the following disclaimer.   %
% - Redistributions in binary form must reproduce the above         %
%   copyright notice, this list of conditions and the following     %
%   disclaimer in the documentation and/or other materials provided %
%   with the distribution.                                          %
% - Neither the name of the SPTK working group nor the names of its %
%   contributors may be used to endorse or promote products derived %
%   from this software without specific prior written permission.   %
%                                                                   %
% THIS SOFTWARE IS PROVIDED BY THE COPYRIGHT HOLDERS AND            %
% CONTRIBUTORS "AS IS" AND ANY EXPRESS OR IMPLIED WARRANTIES,       %
% INCLUDING, BUT NOT LIMITED TO, THE IMPLIED WARRANTIES OF          %
% MERCHANTABILITY AND FITNESS FOR A PARTICULAR PURPOSE ARE          %
% DISCLAIMED. IN NO EVENT SHALL THE COPYRIGHT OWNER OR CONTRIBUTORS %
% BE LIABLE FOR ANY DIRECT, INDIRECT, INCIDENTAL, SPECIAL,          %
% EXEMPLARY, OR CONSEQUENTIAL DAMAGES (INCLUDING, BUT NOT LIMITED   %
% TO, PROCUREMENT OF SUBSTITUTE GOODS OR SERVICES; LOSS OF USE,     %
% DATA, OR PROFITS; OR BUSINESS INTERRUPTION) HOWEVER CAUSED AND ON %
% ANY THEORY OF LIABILITY, WHETHER IN CONTRACT, STRICT LIABILITY,   %
% OR TORT (INCLUDING NEGLIGENCE OR OTHERWISE) ARISING IN ANY WAY    %
% OUT OF THE USE OF THIS SOFTWARE, EVEN IF ADVISED OF THE           %
% POSSIBILITY OF SUCH DAMAGE.                                       %
% ----------------------------------------------------------------- %
\hypertarget{imsvq}{}
\name{imsvq}{decoder of multi stage vector quantization}{vector quantization}

\begin{synopsis}
\item [imsvq] [ --l $L$ ] [ --n $N$ ] [ --s $S \;$ {\em cbfile} ] [ {\em infile} ]
\end{synopsis}

\begin{qsection}{DESCRIPTION}
{\em imsvq} decodes multi-stage vector-quantized data 
from a sequence of codebook indexes from {\em infile} (or standard input), 
using codebooks specified by multiple --s options, 
sending the result to standard output. 
The number of decoder stages is equal to the number of --s options.

Input data is in int format, and output data is in float format.
\end{qsection}

\begin{options}
	\argm{l}{L}{length of vector}{26}
	\argm{n}{N}{order of vector}{L-1}
	\argm{s}{S \; cbfile}{codebook\\
		\begin{tabular}{ll}\\[-1ex]
		$S$ & codebook size\\
		$cbfile$ & codebook file \\
		\end{tabular}\\\hspace*{\fill}}{N/A N/A}
\end{options}

\begin{qsection}{EXAMPLE}
In the example below,
the decoded vector {\em data.ivq} is obtained
from the first stage codebook {\em cbfile1}
and the second stage codebook {\em cbfile2},
both of size 256, as well as from the index file {\em data.vq}.
\begin{quote}
\verb! imsvq -s 256 cbfile1 -s 256 cbfile2 < data.vq > data.ivq!
\end{quote}
\end{qsection}

\begin{qsection}{NOTICE}
The --s option is specified number of stages.
\end{qsection}

\begin{qsection}{SEE ALSO}
\hyperlink{msvq}{msvq},
\hyperlink{ivq}{ivq},
\hyperlink{vq}{vq}
\end{qsection}
