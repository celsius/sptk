\name{poledf}{all pole digital filter for speech synthesis}%
{digital filter}

\begin{synopsis}
\item[poledf] [ --m $M$ ] [ --p $P$ ] [ --i $I$ ] [ --t ] [ --k ]
              {\em afile} [ {\em infile} ]
\end{synopsis}

\begin{qsection}{DESCRIPTION}
This command reads excitation information from
{\em infile} and the linear prediction coefficients
$K,a(1),\ldots,a(M)$ from {\em afile},
passes the excitation through the corresponding all pole standard form
filter and sends the results to the standard output.
\par
Input and output data are in float format.
\par
The transfer function $H(z)$ of an all pole standard form
filter is
\begin{displaymath}
H(z) = \frac{K}{\displaystyle 1+\sum_{m=1}^M a(m) z^{-m}}
\end{displaymath}
\end{qsection}

\begin{options}
	\argm{m}{M}{order of coefficients}{25}
	\argm{p}{P}{frame period}{100}
	\argm{i}{I}{interpolation period}{1}
	\argm{t}{}{transpose filter}{FALSE}
	\argm{k}{}{filtering without gain}{FALSE}
\end{options}

\begin{qsection}{EXAMPLE}
In the example below, the excitation is generated
from pitch information read from {\em data.pitch} in
float format, then it is passed through the standard form synthesis
filter built from the linear prediction coefficients file
{\em data.lpc}, and synthesized speech is outputed to
{\em data.syn}:
\begin{quote}
  \verb!excite < data.pitch | poledf data.lpc > data.syn!
\end{quote}
\end{qsection}

\begin{qsection}{SEE ALSO}
  lpc, acorr, ltcdf, lmadf, zerodf
\end{qsection}
