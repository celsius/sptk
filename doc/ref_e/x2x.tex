% ----------------------------------------------------------------- %
%             The Speech Signal Processing Toolkit (SPTK)           %
%             developed by SPTK Working Group                       %
%             http://sp-tk.sourceforge.net/                         %
% ----------------------------------------------------------------- %
%                                                                   %
%  Copyright (c) 1984-2007  Tokyo Institute of Technology           %
%                           Interdisciplinary Graduate School of    %
%                           Science and Engineering                 %
%                                                                   %
%                1996-2009  Nagoya Institute of Technology          %
%                           Department of Computer Science          %
%                                                                   %
% All rights reserved.                                              %
%                                                                   %
% Redistribution and use in source and binary forms, with or        %
% without modification, are permitted provided that the following   %
% conditions are met:                                               %
%                                                                   %
% - Redistributions of source code must retain the above copyright  %
%   notice, this list of conditions and the following disclaimer.   %
% - Redistributions in binary form must reproduce the above         %
%   copyright notice, this list of conditions and the following     %
%   disclaimer in the documentation and/or other materials provided %
%   with the distribution.                                          %
% - Neither the name of the SPTK working group nor the names of its %
%   contributors may be used to endorse or promote products derived %
%   from this software without specific prior written permission.   %
%                                                                   %
% THIS SOFTWARE IS PROVIDED BY THE COPYRIGHT HOLDERS AND            %
% CONTRIBUTORS "AS IS" AND ANY EXPRESS OR IMPLIED WARRANTIES,       %
% INCLUDING, BUT NOT LIMITED TO, THE IMPLIED WARRANTIES OF          %
% MERCHANTABILITY AND FITNESS FOR A PARTICULAR PURPOSE ARE          %
% DISCLAIMED. IN NO EVENT SHALL THE COPYRIGHT OWNER OR CONTRIBUTORS %
% BE LIABLE FOR ANY DIRECT, INDIRECT, INCIDENTAL, SPECIAL,          %
% EXEMPLARY, OR CONSEQUENTIAL DAMAGES (INCLUDING, BUT NOT LIMITED   %
% TO, PROCUREMENT OF SUBSTITUTE GOODS OR SERVICES; LOSS OF USE,     %
% DATA, OR PROFITS; OR BUSINESS INTERRUPTION) HOWEVER CAUSED AND ON %
% ANY THEORY OF LIABILITY, WHETHER IN CONTRACT, STRICT LIABILITY,   %
% OR TORT (INCLUDING NEGLIGENCE OR OTHERWISE) ARISING IN ANY WAY    %
% OUT OF THE USE OF THIS SOFTWARE, EVEN IF ADVISED OF THE           %
% POSSIBILITY OF SUCH DAMAGE.                                       %
% ----------------------------------------------------------------- %
\hypertarget{x2x}{}
\name{x2x}{data type transformation}{data operation}

\begin{synopsis}
\item[x2x] [ +$type1$ ] [ +$type2$ ] [ $\%format$ ] [ +$a$ A ] [ --r ]
\end{synopsis}

\begin{qsection}{DESCRIPTION}
{\em x2x} converts data from standard input to a different data type,
sending the result to standard output.

The input and output data type are specified by command line options 
as described below.
\end{qsection}

\begin{options}
	\argp{type1}{input data type}{f}
	\argp{type2}{output data type\\
	both options $type1, type2$ can be assigned,
        one of the options below.\\
                \begin{tabular}{llcll} \\[-1ex]
                        c & char (1 byte) & \quad &
                        C & unsigned char (1 byte) \\
                        s & short (2 bytes) & \quad &
                        S & unsigned short (2 bytes) \\
                        i3 & int (3 bytes) & \quad &
                        I3 & unsigned int (3 bytes) \\		 
                        i & int (4 bytes) & \quad &
                        I & unsigned int (4 bytes) \\
                        l & long (4 bytes) & \quad &
                        L & unsigned long (4 bytes) \\
                        f & float (4 bytes) & \quad &
                        d & double (8 bytes) \\
                        a & ASCII \\ [1ex] 
                \end{tabular} \\
                data type is converted from $t_1(type_1)$ to $t_2(type_2)$.
                if $t_2$ is not assigned then no operation is
                undertaken, and the output file is equal to the input
                file.}{type1}
	\argp{{\bf a} \qquad A}{column number}{1}
	\argm{r}{}{specify rounding off when a real number
                 is substituted for a integer}{FALSE}
	\argh{format}{}{specify output format similar to 'printf()', 
                        if $type2$ is ASCII.}{$\%g$}
\end{options}

\begin{qsection}{EXAMPLE}
The following example converts data in ASCII format
read from {\em data.asc}, converts to float format,
and writes the output to {\em data.f}:
\begin{quote}
  \verb!x2x +af < data.asc > data.f!
\end{quote}
\par
This example reads data in float format from {\em data.f}
converts to ASCII format, and sends the output to the screen:
\begin{quote}
  \verb!x2x +fa < data.f!
\end{quote}
For example, if contents of {\em data.f} in float format are
\begin{displaymath}
  1, 2, 3, 4, 5, 6, 7
\end{displaymath}
then the following output is printed to the screen.
\begin{quote}
  \verb!1! \\
  \verb!2! \\
  \verb!3! \\
  \verb!4! \\
  \verb!5! \\
  \verb!6! \\
  \verb!7!
\end{quote}
\par
If for the same data in the example above
the number of column is assigned:
column
\begin{quote}
  \verb!x2x +fa 3 < data.f!
\end{quote}
the output is
\begin{quote}
  \verb!1       2        3! \\
  \verb!4       5        6! \\
  \verb!7!
\end{quote}
\par
The output uses the printf command \%e format:
\begin{quote}
  \verb!x2x +fa %9.4e < data.f!
\end{quote}
In this example the total number of characters for each number
is 11, and the number of decimal points assigned to 4.
\begin{quote}
  \verb!1.0000e+000! \\
  \verb!2.0000e+000! \\
  \mbox{\hspace{2em}}$\vdots$ \\
  \verb!7.0000e+000!
\end{quote}
\end{qsection}

\begin{qsection}{SEE ALSO}
\hyperlink{dmp}{dmp}
\end{qsection}
