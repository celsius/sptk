%  ---------------------------------------------------------------  %
%            Speech Signal Processing Toolkit (SPTK)                %
%                      SPTK Working Group                           %
%                                                                   %
%                  Department of Computer Science                   %
%                  Nagoya Institute of Technology                   %
%                               and                                 %
%   Interdisciplinary Graduate School of Science and Engineering    %
%                  Tokyo Institute of Technology                    %
%                                                                   %
%                     Copyright (c) 1984-2007                       %
%                       All Rights Reserved.                        %
%                                                                   %
%  Permission is hereby granted, free of charge, to use and         %
%  distribute this software and its documentation without           %
%  restriction, including without limitation the rights to use,     %
%  copy, modify, merge, publish, distribute, sublicense, and/or     %
%  sell copies of this work, and to permit persons to whom this     %
%  work is furnished to do so, subject to the following conditions: %
%                                                                   %
%    1. The source code must retain the above copyright notice,     %
%       this list of conditions and the following disclaimer.       %
%                                                                   %
%    2. Any modifications to the source code must be clearly        %
%       marked as such.                                             %
%                                                                   %
%    3. Redistributions in binary form must reproduce the above     %
%       copyright notice, this list of conditions and the           %
%       following disclaimer in the documentation and/or other      %
%       materials provided with the distribution.  Otherwise, one   %
%       must contact the SPTK working group.                        %
%                                                                   %
%  NAGOYA INSTITUTE OF TECHNOLOGY, TOKYO INSTITUTE OF TECHNOLOGY,   %
%  SPTK WORKING GROUP, AND THE CONTRIBUTORS TO THIS WORK DISCLAIM   %
%  ALL WARRANTIES WITH REGARD TO THIS SOFTWARE, INCLUDING ALL       %
%  IMPLIED WARRANTIES OF MERCHANTABILITY AND FITNESS, IN NO EVENT   %
%  SHALL NAGOYA INSTITUTE OF TECHNOLOGY, TOKYO INSTITUTE OF         %
%  TECHNOLOGY, SPTK WORKING GROUP, NOR THE CONTRIBUTORS BE LIABLE   %
%  FOR ANY SPECIAL, INDIRECT OR CONSEQUENTIAL DAMAGES OR ANY        %
%  DAMAGES WHATSOEVER RESULTING FROM LOSS OF USE, DATA OR PROFITS,  %
%  WHETHER IN AN ACTION OF CONTRACT, NEGLIGENCE OR OTHER TORTUOUS   %
%  ACTION, ARISING OUT OF OR IN CONNECTION WITH THE USE OR          %
%  PERFORMANCE OF THIS SOFTWARE.                                    %
%                                                                   %
%  ---------------------------------------------------------------  %
%
\hypertarget{x2x}{}
\name{x2x}{data type transformation}{data operation}

\begin{synopsis}
\item[x2x] [ +$type1$ ] [ +$type2$ ] [ $\%format$ ] [ +$a$ A ] [ --r ]
\end{synopsis}

\begin{qsection}{DESCRIPTION}
{\em x2x} converts data from standard input to a different data type,
sending the result to standard output.

The input and output data type are specified by command line options 
as described below.
\end{qsection}

\begin{options}
	\argp{type1}{input data type}{f}
	\argp{type2}{output data type\\
	both options $type1, type2$ can be assigned,
        one of the options below.\\
                \begin{tabular}{llcll} \\[-1ex]
                        c & char (1 byte) & \quad &
                        C & unsigned char (1 byte) \\
                        s & short (2 bytes) & \quad &
                        S & unsigned short (2 bytes) \\
                        i & int (4 bytes) & \quad &
                        I & unsigned int (4 bytes) \\
                        l & long (4 bytes) & \quad &
                        L & unsigned long (4 bytes) \\
                        f & float (4 bytes) & \quad &
                        d & double (8 bytes) \\
                        a & ASCII \\ [1ex] 
                \end{tabular} \\
                data type is converted from $t_1(type_1)$ to $t_2(type_2)$.
                if $t_2$ is not assigned then no operation is
                undertaken, and the output file is equal to the input
                file.}{type1}
	\argp{{\bf a} \qquad A}{column number}{1}
	\argm{r}{}{specify rounding off when a real number
                 is substituted for a integer}{FALSE}
	\argh{format}{}{specify output format similar to 'printf()', 
                        if $type2$ is ASCII.}{$\%g$}
\end{options}

\begin{qsection}{EXAMPLE}
The following example converts data in ASCII format
read from {\em data.asc}, converts to float format,
and writes the output to {\em data.f}:
\begin{quote}
  \verb!x2x +af < data.asc > data.f!
\end{quote}
\par
This example reads data in float format from {\em data.f}
converts to ASCII format, and sends the output to the screen:
\begin{quote}
  \verb!x2x +fa < data.f!
\end{quote}
For example, if contents of {\em data.f} in float format are
\begin{displaymath}
  1, 2, 3, 4, 5, 6, 7
\end{displaymath}
then the following output is printed to the screen.
\begin{quote}
  \verb!1! \\
  \verb!2! \\
  \verb!3! \\
  \verb!4! \\
  \verb!5! \\
  \verb!6! \\
  \verb!7!
\end{quote}
\par
If for the same data in the example above
the number of column is assigned:
column
\begin{quote}
  \verb!x2x +fa 3 < data.f!
\end{quote}
the output is
\begin{quote}
  \verb!1       2        3! \\
  \verb!4       5        6! \\
  \verb!7!
\end{quote}
\par
The output uses the printf command \%e format:
\begin{quote}
  \verb!x2x +fa %9.4e < data.f!
\end{quote}
In this example the total number of characters for each number
is 11, and the number of decimal points assigned to 4.
\begin{quote}
  \verb!1.0000e+000! \\
  \verb!2.0000e+000! \\
  \mbox{\hspace{2em}}$\vdots$ \\
  \verb!7.0000e+000!
\end{quote}
\end{qsection}

\begin{qsection}{SEE ALSO}
\hyperlink{dmp}{dmp}
\end{qsection}
