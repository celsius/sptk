%  ---------------------------------------------------------------  %
%            Speech Signal Processing Toolkit (SPTK)                %
%                      SPTK Working Group                           %
%                                                                   %
%                  Department of Computer Science                   %
%                  Nagoya Institute of Technology                   %
%                               and                                 %
%   Interdisciplinary Graduate School of Science and Engineering    %
%                  Tokyo Institute of Technology                    %
%                                                                   %
%                     Copyright (c) 1984-2007                       %
%                       All Rights Reserved.                        %
%                                                                   %
%  Permission is hereby granted, free of charge, to use and         %
%  distribute this software and its documentation without           %
%  restriction, including without limitation the rights to use,     %
%  copy, modify, merge, publish, distribute, sublicense, and/or     %
%  sell copies of this work, and to permit persons to whom this     %
%  work is furnished to do so, subject to the following conditions: %
%                                                                   %
%    1. The source code must retain the above copyright notice,     %
%       this list of conditions and the following disclaimer.       %
%                                                                   %
%    2. Any modifications to the source code must be clearly        %
%       marked as such.                                             %
%                                                                   %
%    3. Redistributions in binary form must reproduce the above     %
%       copyright notice, this list of conditions and the           %
%       following disclaimer in the documentation and/or other      %
%       materials provided with the distribution.  Otherwise, one   %
%       must contact the SPTK working group.                        %
%                                                                   %
%  NAGOYA INSTITUTE OF TECHNOLOGY, TOKYO INSTITUTE OF TECHNOLOGY,   %
%  SPTK WORKING GROUP, AND THE CONTRIBUTORS TO THIS WORK DISCLAIM   %
%  ALL WARRANTIES WITH REGARD TO THIS SOFTWARE, INCLUDING ALL       %
%  IMPLIED WARRANTIES OF MERCHANTABILITY AND FITNESS, IN NO EVENT   %
%  SHALL NAGOYA INSTITUTE OF TECHNOLOGY, TOKYO INSTITUTE OF         %
%  TECHNOLOGY, SPTK WORKING GROUP, NOR THE CONTRIBUTORS BE LIABLE   %
%  FOR ANY SPECIAL, INDIRECT OR CONSEQUENTIAL DAMAGES OR ANY        %
%  DAMAGES WHATSOEVER RESULTING FROM LOSS OF USE, DATA OR PROFITS,  %
%  WHETHER IN AN ACTION OF CONTRACT, NEGLIGENCE OR OTHER TORTUOUS   %
%  ACTION, ARISING OUT OF OR IN CONNECTION WITH THE USE OR          %
%  PERFORMANCE OF THIS SOFTWARE.                                    %
%                                                                   %
%  ---------------------------------------------------------------  %
%
\hypertarget{lspcheck}{}
\name{lspcheck}{check stability and rearrange LSP}{speech parameter transformation}

\begin{synopsis}
\item [lspcheck] [ --m $M$ ] [ --s $S$ ] [ --k ] [ --i $I$ ] [ --o $O$ ]
		[ --r ] [ {\em infile} ] 
\end{synopsis}

\begin{qsection}{DESCRIPTION}
{\em lspcheck} tests the stability of the filter 
corresponding to the line spectral pair (LSP) coefficients 
from {\em infile} (or standard input), 
sending the result to standard output.

By default, the output is an ASCII report of the unstable frames.
However, if the --r option is given, 
the output is frames of coefficients 
that have been rearranged so the filter is stable.
\end{qsection}

\begin{options}
	\argm{m}{M}{order of LPC}{25}
	\argm{s}{S}{sampling frequency(kHz)}{10}
	\argm{k}{}{input \& output gain}{TRUE}
	\argm{i}{I}{input format}{0}
	\argm{o}{O}{output format \\
		\begin{tabular}{ll} \\[-1ex]
			$0$ & normalized frequency $(0 \ldots \pi)$ \\
			$1$ & normalized frequency $(0 \ldots 0.5)$ \\
			$2$ & frequency (kHz) \\
			$3$ & frequency (Hz)  \\
		\end{tabular}\\\hspace*{\fill}}{0}
	\argm{r}{}{rearrange LSP}{FALSE}
\end{options}

\begin{qsection}{EXAMPLE}
In the following example, 10-th order LSP coefficients are
read from {\em data.lsp} in float format,
stability is checked, the unstable coefficients are rearranged
so that they become stable, and the rearranged LSP coefficients
are written to {\em data.lspr}:
\begin{quote}
\verb!lspcheck -r < data.lsp > data.lspr!
\end{quote}
\end{qsection}

\begin{qsection}{SEE ALSO}
\hyperlink{lpc}{lpc},
\hyperlink{lpc2lsp}{lpc2lsp},
\hyperlink{lsp2lpc}{lsp2lpc}
\end{qsection}
