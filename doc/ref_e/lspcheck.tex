% ----------------------------------------------------------------- %
%             The Speech Signal Processing Toolkit (SPTK)           %
%             developed by SPTK Working Group                       %
%             http://sp-tk.sourceforge.net/                         %
% ----------------------------------------------------------------- %
%                                                                   %
%  Copyright (c) 1984-2007  Tokyo Institute of Technology           %
%                           Interdisciplinary Graduate School of    %
%                           Science and Engineering                 %
%                                                                   %
%                1996-2013  Nagoya Institute of Technology          %
%                           Department of Computer Science          %
%                                                                   %
% All rights reserved.                                              %
%                                                                   %
% Redistribution and use in source and binary forms, with or        %
% without modification, are permitted provided that the following   %
% conditions are met:                                               %
%                                                                   %
% - Redistributions of source code must retain the above copyright  %
%   notice, this list of conditions and the following disclaimer.   %
% - Redistributions in binary form must reproduce the above         %
%   copyright notice, this list of conditions and the following     %
%   disclaimer in the documentation and/or other materials provided %
%   with the distribution.                                          %
% - Neither the name of the SPTK working group nor the names of its %
%   contributors may be used to endorse or promote products derived %
%   from this software without specific prior written permission.   %
%                                                                   %
% THIS SOFTWARE IS PROVIDED BY THE COPYRIGHT HOLDERS AND            %
% CONTRIBUTORS "AS IS" AND ANY EXPRESS OR IMPLIED WARRANTIES,       %
% INCLUDING, BUT NOT LIMITED TO, THE IMPLIED WARRANTIES OF          %
% MERCHANTABILITY AND FITNESS FOR A PARTICULAR PURPOSE ARE          %
% DISCLAIMED. IN NO EVENT SHALL THE COPYRIGHT OWNER OR CONTRIBUTORS %
% BE LIABLE FOR ANY DIRECT, INDIRECT, INCIDENTAL, SPECIAL,          %
% EXEMPLARY, OR CONSEQUENTIAL DAMAGES (INCLUDING, BUT NOT LIMITED   %
% TO, PROCUREMENT OF SUBSTITUTE GOODS OR SERVICES; LOSS OF USE,     %
% DATA, OR PROFITS; OR BUSINESS INTERRUPTION) HOWEVER CAUSED AND ON %
% ANY THEORY OF LIABILITY, WHETHER IN CONTRACT, STRICT LIABILITY,   %
% OR TORT (INCLUDING NEGLIGENCE OR OTHERWISE) ARISING IN ANY WAY    %
% OUT OF THE USE OF THIS SOFTWARE, EVEN IF ADVISED OF THE           %
% POSSIBILITY OF SUCH DAMAGE.                                       %
% ----------------------------------------------------------------- %
\hypertarget{lspcheck}{}
\name{lspcheck}{check stability and rearrange LSP}{speech parameter transformation}

\begin{synopsis}
\item [lspcheck] [ --m $M$ ] [ --s $S$ ] [ --k ] [ --L ] [ --i $I$ ] [ --o $O$ ]
		[ --r $R$] [ --G $G$ ] [ --g ] [ {\em infile} ]
\end{synopsis}

\begin{qsection}{DESCRIPTION}
{\em lspcheck} tests the stability of the filter 
corresponding to the line spectral pair (LSP) coefficients 
from {\em infile} (or standard input), 
sending the result to standard output.

By default, the output is the same as the input.
When the --c option is given,
the output is LSP coefficients
that have been rearranged so the filter is stable.
If an frame is unstable, an ASCII report of the number of the frame
is sent to standard error.
\end{qsection}

\begin{options}
	\argm{m}{M}{order of LPC}{25}
	\argm{s}{S}{sampling frequency (kHz)}{10.0}
	\argm{k}{}{input \& output gain}{TRUE}
    \argm{L}{}{regard input as log gain}{FALSE}
	\argm{i}{I}{input format}{0}
	\argm{o}{O}{output format \\
		\begin{tabular}{ll} \\[-1ex]
			$0$ & normalized frequency $(0 \ldots \pi)$ \\
			$1$ & normalized frequency $(0 \ldots 0.5)$ \\
			$2$ & frequency (kHz) \\
			$3$ & frequency (Hz)  \\
		\end{tabular}\\\hspace*{\fill}}{$I$}
	\argm{c}{}{rearrange LSP\\
check the distance between two consecutive LSPs\\
and extend the distance (if it is smaller than $R\times \pi/M$)}{N/A}
	\argm{r}{R}{threshold of rearrangement of LSP\\
    $s.t.\hspace{0.5cm} 0 \leq R \leq 1$}{0.0}
    \argm{G}{G}{minimum value of gain\\
                $G$ must be greater than $0$.}{1e-10}
    \argm{g}{}{modify gain value if gain is less than $G$.}{FALSE}
\end{options}

\begin{qsection}{EXAMPLE}
In the following example, 10-th order LSP coefficients are
read from {\em data.lsp} in float format,
stability is checked, the unstable coefficients are rearranged
so that they become stable, and the distance between two
consecutive LSPs are extended to $\pi /1000$ if
 it is smaller than $\pi /1000$, and the
 rearranged LSP coefficients are written to {\em data.lspr}:
\begin{quote}
\verb!lspcheck -m 10 -c -r 0.01 < data.lsp > data.lspr!
\end{quote}
\end{qsection}

\begin{qsection}{SEE ALSO}
\hyperlink{lpc}{lpc},
\hyperlink{lpc2lsp}{lpc2lsp},
\hyperlink{lsp2lpc}{lsp2lpc}
\end{qsection}
