\name{lbg}{LBG algorithm for vector quantizer design}{vector quantization}

\begin{synopsis}
\item [lbg] [ --l $L$ ] [ --n $N$ ] [ --t $T$ ] [ --s $S$ ] [ --e $E$ ]
        [ --f $F$ ] [ --d $D$ ] [ --r $R$ ] 
\item [\ ~~~~] [ {\em indexfile} ] $<$ {\em infile}
\end{synopsis}

\begin{qsection}{DESCRIPTION}
The {\em lbg} command undertakes the lbg algorithm for training the
codebook.
Input and output data are in float format.

The lbg algorithm reads input sequence vectors of size $L$
\begin{displaymath} 
\mbox{\boldmath x}(0), \mbox{\boldmath x}(1), \ldots, \mbox{\boldmath x}(T-1)
\end{displaymath}
generates the following codebook,
\begin{displaymath}
\mbox{\boldmath C}_E =\{ \mbox{\boldmath c}_E(0), \mbox{\boldmath c}_E(1), 
\ldots, \mbox{\boldmath c}_E(E-1) \}
\end{displaymath}
and sends the results to the standard output.
The generation of the codebook is undertaken by the following algorithm.

\begin{description}
\item[\bf step.0~~~]
When a initial codebook $\mbox{\boldmath C}_S$ is not assigned,
the initial codebook is obtained from the whole collection of
training data as follows,
\begin{displaymath}
\mbox{\boldmath c}_1(0) = \frac{1}{T} \sum_{n=0}^{T-1} \mbox{\boldmath x}(n)
\end{displaymath}
and the initial codebook with $S = 1$ is $\mbox{\boldmath C}_1 = \{ \mbox{\boldmath c}_1(0) \}$.

\item[\bf step.1~~~]
From codebook $\mbox{\boldmath C}_{S}$ obtain $\mbox{\boldmath C}_{2S}$.
For this step, normalized random vector of size $L$ and splitting factor
$R$ are used as follows,
\begin{displaymath}
\mbox{\boldmath c}_{2S}(n)=\left\{ \begin{array}{ll}
\mbox{\boldmath c}_S(n) + R \cdot \mbox{\boldmath rnd} & ( 0 \le n \le S-1 ) \\
\mbox{\boldmath c}_S(n) - R \cdot \mbox{\boldmath rnd} & ( S \le n \le 2S-1 )
\end{array}\right.
\end{displaymath}
and we make $D_0 = \infty$ and $k = 0$.

\item[\bf step.2~~~]
The present codebook $\mbox{\boldmath C}_{2S}$ is now applied
to the training vectors.
After that the mean Euclidean distance $D_k$ is evaluated
from every training vector and the corresponding code vector.
If the following condition 
\begin{displaymath}
|\frac{D_{k-1}-D_{k}}{D_{k}}| < D
\end{displaymath}
is valid then go to {\bf step.4}.
If it is not valid then go to {\bf step.3}.

\item[\bf step.3~~~]
Centroids are evaluated from the results obtained in {\bf step.2}.
The codebook $\mbox{\boldmath C}_{2S}$ is updated.
Also, if a cell has no training vector, then the corresponding
code vector is erased from codebook,
and a new code vector is generated from the code vector
which corresponds to the cell with more training vectors 
$\mbox{\boldmath c}_{2S}(j)$, as follows.
\begin{displaymath}
\mbox{\boldmath c}_{2S}(i)=\mbox{\boldmath c}_{2S}(j) + R \cdot \mbox{\boldmath rnd}
\end{displaymath}
After that, we assigned $k=k+1$ and go back {\bf step.2}.

\item[\bf step.4~~~]
If $2S = E$ then end.
If it is not then we make $S$ = $2S$ and go back {\bf step.1}.

\end{description}
\end{qsection}

\begin{options}
        \argm{l}{L}{length of vector}{26}
        \argm{n}{N}{order of vector}{L-1}
        \argm{t}{T}{number of training vector}{N/A}
        \argm{s}{S}{initial codebook size}{1}
        \argm{e}{E}{final codebook size}{256}
        \argm{f}{F}{initial codebook filename}{NULL}
        \desc[1zh]{Usually, the options below do not need to be assigned.}
        \argm{d}{D}{end condition}{0.0001}
        \argm{r}{R}{splitting factor}{0.0001}
\end{options}

\begin{qsection}{EXAMPLE}
In the following example, a codebook of size 256 is generated from
the 25 order training vector {\em data.f} in float format,
and the output is written to {\em cbfile}.
\begin{quote}
\verb! lbg < data.f > cbfile!
\end{quote}
\end{qsection}

\begin{qsection}{SEE ALSO}
vq, ivq, msvq
\end{qsection}
