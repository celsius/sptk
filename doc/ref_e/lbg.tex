% ----------------------------------------------------------------- %
%             The Speech Signal Processing Toolkit (SPTK)           %
%             developed by SPTK Working Group                       %
%             http://sp-tk.sourceforge.net/                         %
% ----------------------------------------------------------------- %
%                                                                   %
%  Copyright (c) 1984-2007  Tokyo Institute of Technology           %
%                           Interdisciplinary Graduate School of    %
%                           Science and Engineering                 %
%                                                                   %
%                1996-2010  Nagoya Institute of Technology          %
%                           Department of Computer Science          %
%                                                                   %
% All rights reserved.                                              %
%                                                                   %
% Redistribution and use in source and binary forms, with or        %
% without modification, are permitted provided that the following   %
% conditions are met:                                               %
%                                                                   %
% - Redistributions of source code must retain the above copyright  %
%   notice, this list of conditions and the following disclaimer.   %
% - Redistributions in binary form must reproduce the above         %
%   copyright notice, this list of conditions and the following     %
%   disclaimer in the documentation and/or other materials provided %
%   with the distribution.                                          %
% - Neither the name of the SPTK working group nor the names of its %
%   contributors may be used to endorse or promote products derived %
%   from this software without specific prior written permission.   %
%                                                                   %
% THIS SOFTWARE IS PROVIDED BY THE COPYRIGHT HOLDERS AND            %
% CONTRIBUTORS "AS IS" AND ANY EXPRESS OR IMPLIED WARRANTIES,       %
% INCLUDING, BUT NOT LIMITED TO, THE IMPLIED WARRANTIES OF          %
% MERCHANTABILITY AND FITNESS FOR A PARTICULAR PURPOSE ARE          %
% DISCLAIMED. IN NO EVENT SHALL THE COPYRIGHT OWNER OR CONTRIBUTORS %
% BE LIABLE FOR ANY DIRECT, INDIRECT, INCIDENTAL, SPECIAL,          %
% EXEMPLARY, OR CONSEQUENTIAL DAMAGES (INCLUDING, BUT NOT LIMITED   %
% TO, PROCUREMENT OF SUBSTITUTE GOODS OR SERVICES; LOSS OF USE,     %
% DATA, OR PROFITS; OR BUSINESS INTERRUPTION) HOWEVER CAUSED AND ON %
% ANY THEORY OF LIABILITY, WHETHER IN CONTRACT, STRICT LIABILITY,   %
% OR TORT (INCLUDING NEGLIGENCE OR OTHERWISE) ARISING IN ANY WAY    %
% OUT OF THE USE OF THIS SOFTWARE, EVEN IF ADVISED OF THE           %
% POSSIBILITY OF SUCH DAMAGE.                                       %
% ----------------------------------------------------------------- %
\hypertarget{lbg}{}
\name{lbg}{LBG algorithm for vector quantizer design}{vector quantization}

\begin{synopsis}
\item [lbg] [ --l $L$ ] [ --n $N$ ] [ --t $T$ ] [ --s $S$ ] [ --e $E$ ]
        [ --f $F$ ] [ --i $I$ ] [ --m $M$ ] [ --S $s$ ] 
\item [\ ~~~~~] [ --c $C$ ] [ --d $D$ ] [ --r $R$ ] [ {\em indexfile} ] $<$ {\em infile}
\end{synopsis}

\begin{qsection}{DESCRIPTION}
{\em lbg} uses the LBG algorithm to train a codebook 
from a sequence of vectors from {\em infile} (or standard input), 
sending the result to standard output.

The input sequence consists of $T$ float vectors $\bx$, 
each of size $L$
\begin{displaymath} 
\bx(0), \bx(1), \dots, \bx(T-1). 
\end{displaymath}
The result is a codebook consisting of $E$ float vectors, 
each of length $L$,
\begin{displaymath}
\bC_E =\{ \bc_E(0), \bc_E(1), \dots, \bc_E(E-1) \}, 
\end{displaymath}
generated by the following algorithm.

\begin{description}
\item[\bf step.0~~~]
When a initial codebook $\bC_S$ is not assigned,
the initial codebook is obtained from the whole collection of
training data as follows,
\begin{displaymath}
\bc_1(0) = \frac{1}{T} \sum_{n=0}^{T-1} \bx(n)
\end{displaymath}
and the initial codebook with $S = 1$ is $\bC_1 = \{ \bc_1(0) \}$.

\item[\bf step.1~~~]
From codebook $\bC_{S}$ obtain $\bC_{2S}$.
For this step, normalized random vector of size $L$ and splitting factor
$R$ are used as follows,
\begin{displaymath}
\bc_{2S}(n)= \begin{cases}
\;\;\bc_S(n) + R \cdot \bm{\mathrm{rnd}} & ( 0 \le n \le S-1 ) \\
\;\;\bc_S(n) - R \cdot \bm{\mathrm{rnd}} & ( S \le n \le 2S-1 )
\end{cases}
\end{displaymath}
and we make $D_0 = \infty$ , $k = 0$.

\item[\bf step.2~~~]
First, check that $k \le I$ where $I$ is the maximum number of the
iteration specified by --i option.
If it is true, proceed the following steps.
If not, then go to {\bf step.4}.
The present codebook $\bC_{2S}$ is now applied
to the training vectors.
After that the mean Euclidean distance $D_k$ is evaluated
from every training vector and the corresponding code vector.
If the following condition 
\begin{displaymath}
\left|\frac{D_{k-1}-D_{k}}{D_{k}}\right| < D
\end{displaymath}
is valid then go to {\bf step.4}.
If it is not valid then go to {\bf step.3}.

\item[\bf step.3~~~]
Centroids are evaluated from the results obtained in {\bf step.2}.
The codebook $\bC_{2S}$ is updated.
Also, if a cell has training vectors less than $M$, then the corresponding
code vector is erased from codebook,
and a new code vector is generated from
 1) the code vector $\bc_{2S}(j)$  corresponding to the cell with more training vectors
as follows.
\begin{displaymath}
\bc_{2S}(i) = \bc_{2S}(j) + R \cdot \bm{\mathrm{rnd}}
\end{displaymath}
Also , $\bc_{2S}(j)$ is modified as follows.
\begin{displaymath}
\bc_{2S}(j) = \bc_{2S}(j) - R \cdot \bm{\mathrm{rnd}}
\end{displaymath}
 2) the vector $\bp$ which internally divide
two centroids in proportion to the number of training vectors for the cell.
They were split from the same parent centroid.
The vector $\bp$ can be written as follows, 
\begin{displaymath}
\bp= \frac{n_{j}\bc_{2S}(i) + n_{i}\bc_{2S}(j)}{n_{i}+n_{j}},
\end{displaymath}
where $n_{i}$ and $n_{j}$ are the number of trainging vectors for the cell.
The update method is as follows.
\begin{displaymath}
\bc_{2S}(i) = \bp + R \cdot \bm{\mathrm{rnd}},
\end{displaymath}
\begin{displaymath}
\bc_{2S}(j) = \bp- R \cdot \bm{\mathrm{rnd}}.
\end{displaymath}
The type of split can be specified by --c option.
After that, we assign $k=k+1$ then go back to {\bf step.2}

\item[\bf step.4~~~]
If $2S = E$ then end.
If it is not then we make $S$ = $2S$ and go back {\bf step.1}.

\end{description}
\end{qsection}

\begin{options}
        \argm{l}{L}{length of vector}{26}
        \argm{n}{N}{order of vector}{L$-$1}
        \argm{t}{T}{number of training vector}{N/A}
        \argm{s}{S}{initial codebook size}{1}
        \argm{e}{E}{final codebook size}{256}
        \argm{f}{F}{initial codebook filename}{NULL}
        \argm{i}{I}{maximum number of iteration for centroid update}{1000} 
        \argm{m}{M}{minimum number of training vectors for each cell}{1}
        \argm{S}{s}{seed for normalized random vector}{1}
        \argm{c}{C}{type of exception procedure for centroid update \\
                    when the number of training vectors for the cell is less than $M$ \\
                    \begin{tabular}{ll} \\[-1ex]
                      $C=1$ & split the centroid with most training vectors\\
                      $C=2$ & split the vector which internally divide\\
                            & two centroids sharing the same parent centroid,\\
                            & in proportion to the number of training vectors for the cell.
                    \end{tabular}\\\hspace*{\fill}
                   }{1}
        \desc[1ex]{Usually, the options below do not need to be assigned.}
        \argm{d}{D}{end condition}{0.0001}
        \argm{r}{R}{splitting factor}{0.0001}
\end{options}

\begin{qsection}{EXAMPLE}
In the following example, a codebook of size 256 is generated from
the 25-th order training vector {\em data.f} in float format,
and the output is written to {\em cbfile}.
\begin{quote}
\verb! lbg < data.f > cbfile!
\end{quote}
\end{qsection}

\begin{qsection}{SEE ALSO}
\hyperlink{vq}{vq},
\hyperlink{ivq}{ivq},
\hyperlink{msvq}{msvq}
\end{qsection}
