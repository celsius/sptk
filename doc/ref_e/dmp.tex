%  ---------------------------------------------------------------  %
%            Speech Signal Processing Toolkit (SPTK)                %
%                      SPTK Working Group                           %
%                                                                   %
%                  Department of Computer Science                   %
%                  Nagoya Institute of Technology                   %
%                               and                                 %
%   Interdisciplinary Graduate School of Science and Engineering    %
%                  Tokyo Institute of Technology                    %
%                                                                   %
%                     Copyright (c) 1984-2007                       %
%                       All Rights Reserved.                        %
%                                                                   %
%  Permission is hereby granted, free of charge, to use and         %
%  distribute this software and its documentation without           %
%  restriction, including without limitation the rights to use,     %
%  copy, modify, merge, publish, distribute, sublicense, and/or     %
%  sell copies of this work, and to permit persons to whom this     %
%  work is furnished to do so, subject to the following conditions: %
%                                                                   %
%    1. The source code must retain the above copyright notice,     %
%       this list of conditions and the following disclaimer.       %
%                                                                   %
%    2. Any modifications to the source code must be clearly        %
%       marked as such.                                             %
%                                                                   %
%    3. Redistributions in binary form must reproduce the above     %
%       copyright notice, this list of conditions and the           %
%       following disclaimer in the documentation and/or other      %
%       materials provided with the distribution.  Otherwise, one   %
%       must contact the SPTK working group.                        %
%                                                                   %
%  NAGOYA INSTITUTE OF TECHNOLOGY, TOKYO INSTITUTE OF TECHNOLOGY,   %
%  SPTK WORKING GROUP, AND THE CONTRIBUTORS TO THIS WORK DISCLAIM   %
%  ALL WARRANTIES WITH REGARD TO THIS SOFTWARE, INCLUDING ALL       %
%  IMPLIED WARRANTIES OF MERCHANTABILITY AND FITNESS, IN NO EVENT   %
%  SHALL NAGOYA INSTITUTE OF TECHNOLOGY, TOKYO INSTITUTE OF         %
%  TECHNOLOGY, SPTK WORKING GROUP, NOR THE CONTRIBUTORS BE LIABLE   %
%  FOR ANY SPECIAL, INDIRECT OR CONSEQUENTIAL DAMAGES OR ANY        %
%  DAMAGES WHATSOEVER RESULTING FROM LOSS OF USE, DATA OR PROFITS,  %
%  WHETHER IN AN ACTION OF CONTRACT, NEGLIGENCE OR OTHER TORTUOUS   %
%  ACTION, ARISING OUT OF OR IN CONNECTION WITH THE USE OR          %
%  PERFORMANCE OF THIS SOFTWARE.                                    %
%                                                                   %
%  ---------------------------------------------------------------  %
%
\hypertarget{dmp}{}
\name{dmp}{binary file dump}{data operation}

\begin{synopsis}
\item[dmp] [ --n $N$ ] [ --l $L$ ] [ +{\em type} ] [ $\%${\em form} ] [ {\em infile} ]
\end{synopsis}

\begin{qsection}{DESCRIPTION}
{\em dmp} converts data from {\em infile} (or standard input) 
to human readable form, 
one sample per line with line numbers, 
sending the result to standard output.
\end{qsection}

\begin{options}
	\argm{n}{N}{block order (0,...,n)}{EOD}
	\argm{l}{L}{block length  (1,...,l)}{EOD}
	\argp{t}{input data format\\
		\begin{tabular}{llcll} \\[-1ex]
			c & char (1 byte) & \quad &
			s & short (2 bytes) \\
			i & int (4 bytes) & \quad &
			l & long (4 bytes) \\
			f & float (4 bytes) & \quad &
			d & double (8 bytes)
		\end{tabular}\\\hspace*{\fill}}{f}
        \argh{form}{}{print format(printf style)}{N/A}

\end{options}

\begin{qsection}{EXAMPLE}
In this example, data is read from the input file
{\em data.f} in float format, and the enumerated data is sent
to the screen:
\begin{quote}
 \verb!dmp +f data.f!
\end{quote}
For example, if the {\em data.f} file has the following values
in float format
\begin{displaymath}
  1, 2, 3, 4, 5, 6, 7
\end{displaymath}
then the following output will be displayed on the screen:
\begin{quote}
  \verb!0       1! \\
  \verb!1       2! \\
  \verb!2       3! \\
  \verb!3       4! \\
  \verb!4       5! \\
  \verb!5       6! \\
  \verb!6       7!
\end{quote}
\par
In case we want to assign a block length:
\begin{quote}
 \verb!dmp -n 2 +f data.f!
\end{quote}
Then the output would be
\begin{quote}
  \verb!0       1! \\
  \verb!1       2! \\
  \verb!2       3! \\
  \verb!0       4! \\
  \verb!1       5! \\
  \verb!2       6! \\
  \verb!0       7!
\end{quote}
\par
If we want to print on the screen the unit impulse response of a digital
filter:
\begin{quote}
  \verb!impulse | dfs -a 1 0.9 | dmp!
\end{quote}
\par
If we want to print a sine wave then we can use the \%e option of
{\em printf} as follows:
\begin{quote}
  \verb!sin -p 30 | dmp %e!
\end{quote}
\par
If we want to represent the sine wave with three decimal points:
\begin{quote}
  \verb!sin -p 30 | dmp %.3e!
\end{quote}
\end{qsection}

\begin{qsection}{SEE ALSO}
\hyperlink{x2x}{x2x},
\hyperlink{fd}{fd}
\end{qsection}
