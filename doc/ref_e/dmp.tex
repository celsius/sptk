\name{dmp}{binary file dump}{data operation}

\begin{synopsis}
\item[dmp] [ --n $N$ ] [ --l $L$ ] [ +{\em type} ] [ $\%${\em form} ] [ {\em infile} ]
\end{synopsis}

\begin{qsection}{DESCRIPTION}
This command reads data from an assigned file,
numerates it and prints the numerated data on the screen.
If the input file is not assigned, then data is read from the stardard
input.
\end{qsection}

\begin{options}
	\argm{n}{N}{block order (0,...,n)}{EOD}
	\argm{l}{L}{block length  (1,...,l)}{EOD}
	\argp{t}{input data format\\
		\begin{tabular}{llcll} \\[-1zh]
			c & char (1byte) & \quad &
			s & short (2bytes) \\
			i & int (4bytes) & \quad &
			l & long (4bytes) \\
			f & float (4bytes) & \quad &
			d & double (8bytes)
		\end{tabular}\\\hspace*{\fill}}{f}
        \argh{form}{}{print format(printf style)}{N/A}

\end{options}

\begin{qsection}{EXAMPLE}
In this example, data is read from the input file
{\em data.f} in float format, and the enumerated data is sent
to the screen:
\begin{quote}
 \verb!dmp +f data.f!
\end{quote}
For example, if the {\em data.f} file has the following values
in float format
\begin{displaymath}
  1, 2, 3, 4, 5, 6, 7
\end{displaymath}
then the following output will be displayed on the screen:
\begin{quote}
  \verb!0       1! \\
  \verb!1       2! \\
  \verb!2       3! \\
  \verb!3       4! \\
  \verb!4       5! \\
  \verb!5       6! \\
  \verb!6       7!
\end{quote}
\par
In case we want to assign a block length:
\begin{quote}
 \verb!dmp -n 2 +f data.f!
\end{quote}
Then the output would be
\begin{quote}
  \verb!0       1! \\
  \verb!1       2! \\
  \verb!2       3! \\
  \verb!0       4! \\
  \verb!1       5! \\
  \verb!2       6! \\
  \verb!0       7!
\end{quote}
\par
If we want to print on the screen the unit impulse response of a digital
filter:
\begin{quote}
  \verb!impulse | dfs -a 1 0.9 | dmp!
\end{quote}
\par
If we want to print a sine wave then we can use the \%e option of
{\em printf} as follows:
\begin{quote}
  \verb!sin -p 30 | dmp %e!
\end{quote}
\par
If we want to represent the sine wave with three decimal points:
\begin{quote}
  \verb!sin -p 30 | dmp %.3e!
\end{quote}
\end{qsection}

\begin{qsection}{SEE ALSO}
x2x, fd
\end{qsection}
