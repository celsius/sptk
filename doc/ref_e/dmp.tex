% ----------------------------------------------------------------- %
%             The Speech Signal Processing Toolkit (SPTK)           %
%             developed by SPTK Working Group                       %
%             http://sp-tk.sourceforge.net/                         %
% ----------------------------------------------------------------- %
%                                                                   %
%  Copyright (c) 1984-2007  Tokyo Institute of Technology           %
%                           Interdisciplinary Graduate School of    %
%                           Science and Engineering                 %
%                                                                   %
%                1996-2011  Nagoya Institute of Technology          %
%                           Department of Computer Science          %
%                                                                   %
% All rights reserved.                                              %
%                                                                   %
% Redistribution and use in source and binary forms, with or        %
% without modification, are permitted provided that the following   %
% conditions are met:                                               %
%                                                                   %
% - Redistributions of source code must retain the above copyright  %
%   notice, this list of conditions and the following disclaimer.   %
% - Redistributions in binary form must reproduce the above         %
%   copyright notice, this list of conditions and the following     %
%   disclaimer in the documentation and/or other materials provided %
%   with the distribution.                                          %
% - Neither the name of the SPTK working group nor the names of its %
%   contributors may be used to endorse or promote products derived %
%   from this software without specific prior written permission.   %
%                                                                   %
% THIS SOFTWARE IS PROVIDED BY THE COPYRIGHT HOLDERS AND            %
% CONTRIBUTORS "AS IS" AND ANY EXPRESS OR IMPLIED WARRANTIES,       %
% INCLUDING, BUT NOT LIMITED TO, THE IMPLIED WARRANTIES OF          %
% MERCHANTABILITY AND FITNESS FOR A PARTICULAR PURPOSE ARE          %
% DISCLAIMED. IN NO EVENT SHALL THE COPYRIGHT OWNER OR CONTRIBUTORS %
% BE LIABLE FOR ANY DIRECT, INDIRECT, INCIDENTAL, SPECIAL,          %
% EXEMPLARY, OR CONSEQUENTIAL DAMAGES (INCLUDING, BUT NOT LIMITED   %
% TO, PROCUREMENT OF SUBSTITUTE GOODS OR SERVICES; LOSS OF USE,     %
% DATA, OR PROFITS; OR BUSINESS INTERRUPTION) HOWEVER CAUSED AND ON %
% ANY THEORY OF LIABILITY, WHETHER IN CONTRACT, STRICT LIABILITY,   %
% OR TORT (INCLUDING NEGLIGENCE OR OTHERWISE) ARISING IN ANY WAY    %
% OUT OF THE USE OF THIS SOFTWARE, EVEN IF ADVISED OF THE           %
% POSSIBILITY OF SUCH DAMAGE.                                       %
% ----------------------------------------------------------------- %
\hypertarget{dmp}{}
\name{dmp}{binary file dump}{data operation}

\begin{synopsis}
\item[dmp] [ --n $N$ ] [ --l $L$ ] [ +{\em type} ] [ $\%${\em form} ] [ {\em infile} ]
\end{synopsis}

\begin{qsection}{DESCRIPTION}
{\em dmp} converts data from {\em infile} (or standard input) 
to a human readable form, 
(one sample per line, with line numbers) 
and sends the result to standard output.
\end{qsection}

\begin{options}
	\argm{n}{N}{block order (0,...,n)}{EOD}
	\argm{l}{L}{block length  (1,...,l)}{EOD}
	\argp{t}{input data format\\
		\begin{tabular}{llcll} \\[-1ex]
         c & char (1 byte) & \quad &
         C & unsigned char (1 byte) \\
         s & short (2 bytes) & \quad &
                     S & unsigned short (2 bytes) \\
         i3 & int (3 bytes) & \quad &
                     I3 & unsigned int (3 bytes) \\
         i & int (4 bytes) & \quad &
                     I & unsigned int (4 bytes) \\
         l & long (4 bytes) & \quad &
                     L & unsigned long (4 bytes) \\
         le & long long (8 bytes) & \quad &
                     LE & unsigned long long (8 bytes) \\
         f & float (4 bytes) & \quad &
                     d & double (8 bytes) \\
		\end{tabular}\\\hspace*{\fill}}{f}
        \argh{form}{}{print format(printf style)}{N/A}

\end{options}

\begin{qsection}{EXAMPLE}
In this example, data is read from the input file
{\em data.f} in float format, and the enumerated data is shown on the screen:
\begin{quote}
 \verb!dmp +f data.f!
\end{quote}
For example, if the {\em data.f} file has the following values
in float format
\begin{displaymath}
  1, 2, 3, 4, 5, 6, 7
\end{displaymath}
then the following output will be displayed on the screen:
\begin{quote}
  \verb!0       1! \\
  \verb!1       2! \\
  \verb!2       3! \\
  \verb!3       4! \\
  \verb!4       5! \\
  \verb!5       6! \\
  \verb!6       7!
\end{quote}
\par
In case one wants to assign a block length the following command can
be used.
\begin{quote}
 \verb!dmp +f -n 2 data.f!
\end{quote}
And the output will be given by:
\begin{quote}
  \verb!0       1! \\
  \verb!1       2! \\
  \verb!2       3! \\
  \verb!0       4! \\
  \verb!1       5! \\
  \verb!2       6! \\
  \verb!0       7!
\end{quote}
Some other examples are provided bellow:
\par
Print the unit impulse response of a digital filter on the screen:
\begin{quote}
  \verb!impulse | dfs -a 1 0.9 | dmp +f!
\end{quote}
\par
Print a sine wave using the \%e option of {\em printf}:
\begin{quote}
  \verb!sin -p 30 | dmp +f %e!
\end{quote}
\par
Print the same sine wave represented by three decimal points:
\begin{quote}
  \verb!sin -p 30 | dmp +f %.3e!
\end{quote}
\end{qsection}

\begin{qsection}{SEE ALSO}
\hyperlink{x2x}{x2x},
\hyperlink{fd}{fd}
\end{qsection}
