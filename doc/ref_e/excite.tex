% ----------------------------------------------------------------
%       Speech Signal Processing Toolkit (SPTK): version 3.0
%                      SPTK Working Group
% 
%                Department of Computer Science
%                Nagoya Institute of Technology
%                             and
%   Interdisciplinary Graduate School of Science and Engineering
%                Tokyo Institute of Technology
%                   Copyright (c) 1984-2000
%                     All Rights Reserved.
% 
% Permission is hereby granted, free of charge, to use and
% distribute this software and its documentation without
% restriction, including without limitation the rights to use,
% copy, modify, merge, publish, distribute, sublicense, and/or
% sell copies of this work, and to permit persons to whom this
% work is furnished to do so, subject to the following conditions:
% 
%   1. The code must retain the above copyright notice, this list
%      of conditions and the following disclaimer.
% 
%   2. Any modifications must be clearly marked as such.
%                                                                        
% NAGOYA INSTITUTE OF TECHNOLOGY, TOKYO INSITITUTE OF TECHNOLOGY,
% SPTK WORKING GROUP, AND THE CONTRIBUTORS TO THIS WORK DISCLAIM
% ALL WARRANTIES WITH REGARD TO THIS SOFTWARE, INCLUDING ALL
% IMPLIED WARRANTIES OF MERCHANTABILITY AND FITNESS, IN NO EVENT
% SHALL NAGOYA INSTITUTE OF TECHNOLOGY, TOKYO INSITITUTE OF
% TECHNOLOGY, SPTK WORKING GROUP, NOR THE CONTRIBUTORS BE LIABLE
% FOR ANY SPECIAL, INDIRECT OR CONSEQUENTIAL DAMAGES OR ANY
% DAMAGES WHATSOEVER RESULTING FROM LOSS OF USE, DATA OR PROFITS,
% WHETHER IN AN ACTION OF CONTRACT, NEGLIGENCE OR OTHER TORTIOUS
% ACTION, ARISING OUT OF OR IN CONNECTION WITH THE USE OR
% PERFORMANCE OF THIS SOFTWARE.
% ----------------------------------------------------------------
%
\hypertarget{excite}{}
\name{excite}{generate excitation}{signal generation,speech analysis and synthesis}

\begin{synopsis}
\item [excite] [ --p $P$ ] [ --i $I$ ] [ --n ] [ --s $S$ ] [ {\em infile} ]
\end{synopsis}

\begin{qsection}{DESCRIPTION}
{\em excite} generates an excitation sequence 
from pitch period information from {\em infile} (or standard input), 
sending the result to standard output. 
When the pitch period is nonzero (i.e. voiced), 
the excitation sequence consists of a pulse train at that pitch. 
When the pitch period is zero (i.e. unvoiced),
the excitation sequence consists of Gaussian or M-sequence noise.

Input and output data are in float format.
\end{qsection}

\begin{options}
	\argm{p}{P}{frame period}{100}
	\argm{i}{I}{interpolation period}{1}
	\argm{n}{}{gauss/M-sequence for unvoiced\\
                   default is M-sequence}{FALSE}
	\argm{s}{S}{seed for nrand for Gaussian noise}{1}
\end{options}

\begin{qsection}{EXAMPLE}
In the example below, the excitation is generated from the
{\em data.p} file and passed through a LPC synthesis filter
whose coefficients are in the {\em data.lpc} file.
The speech signal is outputed to the {\em data.syn} file.
\begin{quote}
 \verb!excite < data.p | poledf data.lpc > data.syn!
\end{quote} 
In case we use Gaussian noise to generate are an unvoiced sound:
\begin{quote}
 \verb!excite -n < data.p | poledf data.lpc > data.syn!
\end{quote}
\end{qsection}

\begin{qsection}{SEE ALSO}
\hyperlink{poledf}{poledf}
\end{qsection}
