% ----------------------------------------------------------------- %
%             The Speech Signal Processing Toolkit (SPTK)           %
%             developed by SPTK Working Group                       %
%             http://sp-tk.sourceforge.net/                         %
% ----------------------------------------------------------------- %
%                                                                   %
%  Copyright (c) 1984-2007  Tokyo Institute of Technology           %
%                           Interdisciplinary Graduate School of    %
%                           Science and Engineering                 %
%                                                                   %
%                1996-2011  Nagoya Institute of Technology          %
%                           Department of Computer Science          %
%                                                                   %
% All rights reserved.                                              %
%                                                                   %
% Redistribution and use in source and binary forms, with or        %
% without modification, are permitted provided that the following   %
% conditions are met:                                               %
%                                                                   %
% - Redistributions of source code must retain the above copyright  %
%   notice, this list of conditions and the following disclaimer.   %
% - Redistributions in binary form must reproduce the above         %
%   copyright notice, this list of conditions and the following     %
%   disclaimer in the documentation and/or other materials provided %
%   with the distribution.                                          %
% - Neither the name of the SPTK working group nor the names of its %
%   contributors may be used to endorse or promote products derived %
%   from this software without specific prior written permission.   %
%                                                                   %
% THIS SOFTWARE IS PROVIDED BY THE COPYRIGHT HOLDERS AND            %
% CONTRIBUTORS "AS IS" AND ANY EXPRESS OR IMPLIED WARRANTIES,       %
% INCLUDING, BUT NOT LIMITED TO, THE IMPLIED WARRANTIES OF          %
% MERCHANTABILITY AND FITNESS FOR A PARTICULAR PURPOSE ARE          %
% DISCLAIMED. IN NO EVENT SHALL THE COPYRIGHT OWNER OR CONTRIBUTORS %
% BE LIABLE FOR ANY DIRECT, INDIRECT, INCIDENTAL, SPECIAL,          %
% EXEMPLARY, OR CONSEQUENTIAL DAMAGES (INCLUDING, BUT NOT LIMITED   %
% TO, PROCUREMENT OF SUBSTITUTE GOODS OR SERVICES; LOSS OF USE,     %
% DATA, OR PROFITS; OR BUSINESS INTERRUPTION) HOWEVER CAUSED AND ON %
% ANY THEORY OF LIABILITY, WHETHER IN CONTRACT, STRICT LIABILITY,   %
% OR TORT (INCLUDING NEGLIGENCE OR OTHERWISE) ARISING IN ANY WAY    %
% OUT OF THE USE OF THIS SOFTWARE, EVEN IF ADVISED OF THE           %
% POSSIBILITY OF SUCH DAMAGE.                                       %
% ----------------------------------------------------------------- %
\hypertarget{excite}{}
\name{excite}{generate excitation}{signal generation,speech analysis and synthesis}

\begin{synopsis}
\item [excite] [ --p $P$ ] [ --i $I$ ] [ --n ] [ --s $S$ ] [ {\em infile} ]
\end{synopsis}

\begin{qsection}{DESCRIPTION}
{\em excite} generates an excitation sequence 
from pitch period information from {\em infile} (or standard input), 
sending the result to standard output. 
When the pitch period is nonzero (i.e. voiced), 
the excitation sequence consists of a pulse train at that pitch. 
When the pitch period is zero (i.e. unvoiced),
the excitation sequence consists of Gaussian or M-sequence noise.

Input and output data are in float format.
\end{qsection}

\begin{options}
	\argm{p}{P}{frame period}{100}
	\argm{i}{I}{interpolation period}{1}
	\argm{n}{}{gauss/M-sequence for unvoiced\\
                   default is M-sequence}{FALSE}
	\argm{s}{S}{seed for nrand for Gaussian noise}{1}
\end{options}

\begin{qsection}{EXAMPLE}
In the example below, the excitation is generated from the
{\em data.p} file and passed through a LPC synthesis filter
whose coefficients are in the {\em data.lpc} file.
The speech signal is output to the {\em data.syn} file.
\begin{quote}
 \verb!excite < data.p | poledf data.lpc > data.syn!
\end{quote} 
In case we use Gaussian noise to generate are an unvoiced sound:
\begin{quote}
 \verb!excite -n < data.p | poledf data.lpc > data.syn!
\end{quote}
\end{qsection}

\begin{qsection}{SEE ALSO}
\hyperlink{poledf}{poledf}
\end{qsection}
