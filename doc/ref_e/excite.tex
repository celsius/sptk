\name{excite}{generate excitation}{signal generation,speech analysis and synthesis}

\begin{synopsis}
\item [excite] [ --p $P$ ] [ --i $I$ ] [ --n ] [ --s $S$ ] [ {\em infile} ]
\end{synopsis}

\begin{qsection}{DESCRIPTION}
This command generates from a pitch period file a pulse train for voiced
or a gaussian noise sequence for unvoiced signals.
\par
Input and output data are in float format.
\end{qsection}

\begin{options}
	\argm{p}{P}{frame period}{100}
	\argm{i}{I}{interpolation period}{1}
	\argm{n}{}{gauss/M-sequence for unvoiced\\
                   default is M-sequence}{FALSE}
	\argm{s}{S}{seed for nrand for gaussian noise}{1}
\end{options}

\begin{qsection}{EXAMPLE}
In the example below, the excitation is generated from the
{\em data.p} file and passed through a LPC synthesis filter
whose coefficients are in the {\em data.lpc} file.
The speech signal is outputed to the {\em data.syn} file.
\begin{quote}
 \verb!excite < data.p | poledf data.lpc > data.syn!
\end{quote} 
In case we use gaussian noise to generate are an unvoiced sound:
\begin{quote}
 \verb!excite -n < data.p | poledf data.lpc > data.syn!
\end{quote}
\end{qsection}

\begin{qsection}{SEE ALSO}
 poledf
\end{qsection}
