% ----------------------------------------------------------------- %
%             The Speech Signal Processing Toolkit (SPTK)           %
%             developed by SPTK Working Group                       %
%             http://sp-tk.sourceforge.net/                         %
% ----------------------------------------------------------------- %
%                                                                   %
%  Copyright (c) 1984-2007  Tokyo Institute of Technology           %
%                           Interdisciplinary Graduate School of    %
%                           Science and Engineering                 %
%                                                                   %
%                1996-2011  Nagoya Institute of Technology          %
%                           Department of Computer Science          %
%                                                                   %
% All rights reserved.                                              %
%                                                                   %
% Redistribution and use in source and binary forms, with or        %
% without modification, are permitted provided that the following   %
% conditions are met:                                               %
%                                                                   %
% - Redistributions of source code must retain the above copyright  %
%   notice, this list of conditions and the following disclaimer.   %
% - Redistributions in binary form must reproduce the above         %
%   copyright notice, this list of conditions and the following     %
%   disclaimer in the documentation and/or other materials provided %
%   with the distribution.                                          %
% - Neither the name of the SPTK working group nor the names of its %
%   contributors may be used to endorse or promote products derived %
%   from this software without specific prior written permission.   %
%                                                                   %
% THIS SOFTWARE IS PROVIDED BY THE COPYRIGHT HOLDERS AND            %
% CONTRIBUTORS "AS IS" AND ANY EXPRESS OR IMPLIED WARRANTIES,       %
% INCLUDING, BUT NOT LIMITED TO, THE IMPLIED WARRANTIES OF          %
% MERCHANTABILITY AND FITNESS FOR A PARTICULAR PURPOSE ARE          %
% DISCLAIMED. IN NO EVENT SHALL THE COPYRIGHT OWNER OR CONTRIBUTORS %
% BE LIABLE FOR ANY DIRECT, INDIRECT, INCIDENTAL, SPECIAL,          %
% EXEMPLARY, OR CONSEQUENTIAL DAMAGES (INCLUDING, BUT NOT LIMITED   %
% TO, PROCUREMENT OF SUBSTITUTE GOODS OR SERVICES; LOSS OF USE,     %
% DATA, OR PROFITS; OR BUSINESS INTERRUPTION) HOWEVER CAUSED AND ON %
% ANY THEORY OF LIABILITY, WHETHER IN CONTRACT, STRICT LIABILITY,   %
% OR TORT (INCLUDING NEGLIGENCE OR OTHERWISE) ARISING IN ANY WAY    %
% OUT OF THE USE OF THIS SOFTWARE, EVEN IF ADVISED OF THE           %
% POSSIBILITY OF SUCH DAMAGE.                                       %
% ----------------------------------------------------------------- %
\hypertarget{snr}{}
\name{snr}{evaluate SNR and segmental SNR}{data processing}

\begin{synopsis}
\item [snr] [ --l $L$ ] [ --o $O$ ] {\em file1} [ {\em infile} ] 
\end{synopsis}

\begin{qsection}{DESCRIPTION}
{\em srn} calculates the SNR (Signal to Noise Ratio) 
and the $\mathrm{SNR}_{\mathrm{seg}}$ (segmental SNR) 
between corresponding $L$-length frames
of {\em file1} and {\em infile} (or standard input), 
sending the result to standard output.
The output format is specified by the --o option.

The SNR and $\mathrm{SNR}_{\mathrm{seg}}$ are calculated
through the following equations.
\begin{displaymath}
\mathrm{SNR} = 10~\log \frac{\displaystyle\sum_{n} \left\{ x(n) \right\}^{2}}
{\displaystyle\sum_{n} \left\{ e(n) \right\}^{2}} \quad \mathrm{[dB]}
\end{displaymath}
\begin{displaymath}
\mathrm{SNR}_{\mathrm{seg}} = \frac{1}{N_{i}} \sum_{i = 1}^{N_{i}}
\mathrm{SNR}_{i} \quad \mathrm{[dB]}
\end{displaymath}
where
\begin{displaymath}
e(n) = x_1(n) - x_2(n)
\end{displaymath}
The number of frames is represented by $N_i$.
For signals with small amplitudes, such as consonant sounds,
the segmental SNR represents a better subjective measure than the SNR.
\end{qsection}

\newpage
\begin{options}
        \argm{l}{L}{frame length}{256}
        \argm{o}{O}{output data format\\
                        \begin{tabular}{ll} \\[-1ex]
                          0 & SNR and SNRseg\\
                          1 & SNR and SNRseg in detail\\
                          2 & SNR\\
                          3 & SNRseg
                        \end{tabular}\\\hspace*{\fill}\\
                        if 0 or 1 are assigned\\
                        the output data is written in ASCII format.\\
                        if 2 or 3 are assigned\\
                        the output data is written in float format}{0}
\end{options}

\begin{qsection}{EXAMPLE}
The following command reads the input files {\em data.f1}
and {\em data.f2}, evaluates the SNR and
segmental SNR, and sends the results to the standard output:
\begin{quote}
 \verb!snr data.f1 data.f2!
\end{quote} 
\end{qsection}

\begin{qsection}{SEE ALSO}
\hyperlink{histogram}{histogram},
\hyperlink{average}{average},
\hyperlink{rmse}{rmse}
\end{qsection}
