\name{snr}{evaluate SNR and segmental SNR}{data}

\begin{synopsis}
\item [snr] [ --l $L$ ] [ +$t_1 t_2$ ] [ --o $O$ ] {\em file1} [ {\em infile} ] 
\end{synopsis}

\begin{qsection}{DESCRIPTION}
This command evaluates the SNR(Signal Noise Ratio) and
the $\mbox{SNR}_{\mbox{seg}}$(segmental SNR)
between two assigned input files as follows.
\begin{displaymath}
  x_1(0),x_1(1),\ldots,x_1(L-1)
\end{displaymath}
\begin{displaymath}
  x_2(0),x_2(1),\ldots,x_2(L-1)
\end{displaymath}
\par
The SNR and $\mbox{SNR}_{\mbox{seg}}$ can be calculated
through the following equation.
\begin{displaymath}
\mbox{SNR} = 10~\log \frac{\displaystyle\sum_{n} ( x(n) )^{2}}
{\displaystyle\sum_{n} (e(n))^{2}}~~~\mbox{[dB]}
\end{displaymath}
\begin{displaymath}
\mbox{SNR}_{\mbox{seg}} = \frac{1}{N_{i}} \sum_{i = 1}^{N_{i}}
\mbox{SNR}_{i} ~~~\mbox{[dB]}
\end{displaymath}
where
\begin{displaymath}
e(n) = x_1(n) - x_2(n)
\end{displaymath}
The number of frame is represented by $N_i$.
The segmental SNR has the characteristic that
for signals with small amplitude such as consonant sounds
it gives rise to a better
subjective measure then the SNR.
\end{qsection}

\newpage
\begin{options}
        \argm{l}{L}{frame length}{256}
        \argp{t_1 t_2}{$t_1$ and $t_2$ are represented data formats
                      of {\em file1} and {\em infile} respectively\\
                \begin{tabular}{llcll} \\[-1zh]
                        s & short (2bytes) & \quad &
                        f & float (4bytes) \\
                \end{tabular}\\\hspace*{\fill}}{sf}
        \argm{o}{O}{output data format\\
                        \begin{tabular}{ll} \\[-1zh]
                          0 & SNR and SNRseg\\
                          1 & SNR and SNRseg in detail\\
                          2 & SNR\\
                          3 & SNRseg
                        \end{tabular}\\\hspace*{\fill}\\
                        if 0 or 1 are assigned\\
                        then output data is written in ascii format.\\
                        if 2 or 3 are assigned\\
                        then output data is written in float format}{0}
\end{options}

\begin{qsection}{EXAMPLE}
The following command reads the input files {\em data.s} in short format
and {\em data.f} in float format, evaluates the SNR and
segmantal SNR, and sends the results to the standard output:
\begin{quote}
 \verb!snr +sf data.s data.f!
\end{quote} 
\end{qsection}

\begin{qsection}{SEE ALSO}
histogram, average, rmse
\end{qsection}
