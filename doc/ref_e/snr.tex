% ----------------------------------------------------------------
%       Speech Signal Processing Toolkit (SPTK): version 3.0
%                      SPTK Working Group
% 
%                Department of Computer Science
%                Nagoya Institute of Technology
%                             and
%   Interdisciplinary Graduate School of Science and Engineering
%                Tokyo Institute of Technology
%                   Copyright (c) 1984-2000
%                     All Rights Reserved.
% 
% Permission is hereby granted, free of charge, to use and
% distribute this software and its documentation without
% restriction, including without limitation the rights to use,
% copy, modify, merge, publish, distribute, sublicense, and/or
% sell copies of this work, and to permit persons to whom this
% work is furnished to do so, subject to the following conditions:
% 
%   1. The code must retain the above copyright notice, this list
%      of conditions and the following disclaimer.
% 
%   2. Any modifications must be clearly marked as such.
%                                                                        
% NAGOYA INSTITUTE OF TECHNOLOGY, TOKYO INSITITUTE OF TECHNOLOGY,
% SPTK WORKING GROUP, AND THE CONTRIBUTORS TO THIS WORK DISCLAIM
% ALL WARRANTIES WITH REGARD TO THIS SOFTWARE, INCLUDING ALL
% IMPLIED WARRANTIES OF MERCHANTABILITY AND FITNESS, IN NO EVENT
% SHALL NAGOYA INSTITUTE OF TECHNOLOGY, TOKYO INSITITUTE OF
% TECHNOLOGY, SPTK WORKING GROUP, NOR THE CONTRIBUTORS BE LIABLE
% FOR ANY SPECIAL, INDIRECT OR CONSEQUENTIAL DAMAGES OR ANY
% DAMAGES WHATSOEVER RESULTING FROM LOSS OF USE, DATA OR PROFITS,
% WHETHER IN AN ACTION OF CONTRACT, NEGLIGENCE OR OTHER TORTIOUS
% ACTION, ARISING OUT OF OR IN CONNECTION WITH THE USE OR
% PERFORMANCE OF THIS SOFTWARE.
% ----------------------------------------------------------------
%
\hypertarget{snr}{}
\name{snr}{evaluate SNR and segmental SNR}{data processing}

\begin{synopsis}
\item [snr] [ --l $L$ ] [ +$t_1 t_2$ ] [ --o $O$ ] {\em file1} [ {\em infile} ] 
\end{synopsis}

\begin{qsection}{DESCRIPTION}
{\em srn} calculates the SNR (Signal to Noise Ratio) 
and the $\mathrm{SNR}_{\mathrm{seg}}$ (segmental SNR) 
between corresponding $L$-length frames
of {\em file1} and {\em infile} (or standard input), 
sending the result to standard output.
The input format is specified by the + option.
The output format is specified by the --o option.

The SNR and $\mathrm{SNR}_{\mathrm{seg}}$ can be calculated
through the following equation.
\begin{displaymath}
\mathrm{SNR} = 10~\log \frac{\displaystyle\sum_{n} \left\{ x(n) \right\}^{2}}
{\displaystyle\sum_{n} \left\{ e(n) \right\}^{2}} \quad \mathrm{[dB]}
\end{displaymath}
\begin{displaymath}
\mathrm{SNR}_{\mathrm{seg}} = \frac{1}{N_{i}} \sum_{i = 1}^{N_{i}}
\mathrm{SNR}_{i} \quad \mathrm{[dB]}
\end{displaymath}
where
\begin{displaymath}
e(n) = x_1(n) - x_2(n)
\end{displaymath}
The number of frame is represented by $N_i$.
The segmental SNR has the characteristic that
for signals with small amplitude such as consonant sounds
it gives rise to a better
subjective measure then the SNR.
\end{qsection}

\newpage
\begin{options}
        \argm{l}{L}{frame length}{256}
        \argp{t_1 t_2}{$t_1$ and $t_2$ are represented data formats
                      of {\em file1} and {\em infile} respectively\\
                \begin{tabular}{llcll} \\[-1ex]
                        s & short (2 bytes) & \quad &
                        f & float (4 bytes) \\
                \end{tabular}\\\hspace*{\fill}}{sf}
        \argm{o}{O}{output data format\\
                        \begin{tabular}{ll} \\[-1ex]
                          0 & SNR and SNRseg\\
                          1 & SNR and SNRseg in detail\\
                          2 & SNR\\
                          3 & SNRseg
                        \end{tabular}\\\hspace*{\fill}\\
                        if 0 or 1 are assigned\\
                        then output data is written in ASCII format.\\
                        if 2 or 3 are assigned\\
                        then output data is written in float format}{0}
\end{options}

\begin{qsection}{EXAMPLE}
The following command reads the input files {\em data.s} in short format
and {\em data.f} in float format, evaluates the SNR and
segmental SNR, and sends the results to the standard output:
\begin{quote}
 \verb!snr +sf data.s data.f!
\end{quote} 
\end{qsection}

\begin{qsection}{SEE ALSO}
\hyperlink{histogram}{histogram},
\hyperlink{average}{average},
\hyperlink{rmse}{rmse}
\end{qsection}
