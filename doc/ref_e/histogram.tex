\name{histogram}{histogram}{data processing}

\begin{synopsis}
\item [histogram] [ --l $L$ ] [ --i $I$ ] [ --j $J$ ] [ --s $S$ ] [ --n ] 
                  [ {\em infile} ] 
\end{synopsis}

\begin{qsection}{DESCRIPTION}
This command evaluates the histogram from the assigned file
and sends the results to the standard output.
\par
Input and output data are in float format.
If a graph is wanted, please send the output of ``histogram'' to ``fdrw''.
\par
In case a data value is outside of the assigned interval,
the histogram will evaluate its output but the return value of
the command will be different from $0$.
\end{qsection}

\begin{options}
        \argm{l}{L}{frame size\\
          \begin{tabular}{ll}\\ [-1zh]
            $L>0$ & evaluate the histogram for every frame\\
            $L=0$ & evaluate the histogram for the whole file\\
          \end{tabular}\\\hspace*{\fill}}{0}
        \argm{i}{I}{infimum}{0.0}
        \argm{j}{J}{supremum}{1.0}
        \argm{s}{S}{step size}{0.1}
        \argm{n}{}{normalization}{FALSE}
\end{options}

\begin{qsection}{EXAMPLE}
The example below plots the histogram of the speech waveform file
{\em data.f} in float format.
\begin{quote}
 \verb!histogram -i -16000 -j 16000 -s 100 data.f | fdrw | xgr!
\end{quote} 
\end{qsection}

\begin{qsection}{SEE ALSO}
 average
\end{qsection}
