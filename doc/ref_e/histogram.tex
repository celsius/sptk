% ----------------------------------------------------------------
%       Speech Signal Processing Toolkit (SPTK): version 3.0
%                      SPTK Working Group
% 
%                Department of Computer Science
%                Nagoya Institute of Technology
%                             and
%   Interdisciplinary Graduate School of Science and Engineering
%                Tokyo Institute of Technology
%                   Copyright (c) 1984-2000
%                     All Rights Reserved.
% 
% Permission is hereby granted, free of charge, to use and
% distribute this software and its documentation without
% restriction, including without limitation the rights to use,
% copy, modify, merge, publish, distribute, sublicense, and/or
% sell copies of this work, and to permit persons to whom this
% work is furnished to do so, subject to the following conditions:
% 
%   1. The code must retain the above copyright notice, this list
%      of conditions and the following disclaimer.
% 
%   2. Any modifications must be clearly marked as such.
%                                                                        
% NAGOYA INSTITUTE OF TECHNOLOGY, TOKYO INSITITUTE OF TECHNOLOGY,
% SPTK WORKING GROUP, AND THE CONTRIBUTORS TO THIS WORK DISCLAIM
% ALL WARRANTIES WITH REGARD TO THIS SOFTWARE, INCLUDING ALL
% IMPLIED WARRANTIES OF MERCHANTABILITY AND FITNESS, IN NO EVENT
% SHALL NAGOYA INSTITUTE OF TECHNOLOGY, TOKYO INSITITUTE OF
% TECHNOLOGY, SPTK WORKING GROUP, NOR THE CONTRIBUTORS BE LIABLE
% FOR ANY SPECIAL, INDIRECT OR CONSEQUENTIAL DAMAGES OR ANY
% DAMAGES WHATSOEVER RESULTING FROM LOSS OF USE, DATA OR PROFITS,
% WHETHER IN AN ACTION OF CONTRACT, NEGLIGENCE OR OTHER TORTIOUS
% ACTION, ARISING OUT OF OR IN CONNECTION WITH THE USE OR
% PERFORMANCE OF THIS SOFTWARE.
% ----------------------------------------------------------------
%
\name{histogram}{histogram}{data processing}

\begin{synopsis}
\item [histogram] [ --l $L$ ] [ --i $I$ ] [ --j $J$ ] [ --s $S$ ] [ --n ] 
                  [ {\em infile} ] 
\end{synopsis}

\begin{qsection}{DESCRIPTION}
This command evaluates the histogram from the assigned file
and sends the results to the standard output.
\par
Input and output data are in float format.
If a graph is wanted, please send the output of ``histogram'' to ``fdrw''.
\par
In case a data value is outside of the assigned interval,
the histogram will evaluate its output but the return value of
the command will be different from $0$.
\end{qsection}

\begin{options}
        \argm{l}{L}{frame size\\
          \begin{tabular}{ll}\\ [-1zh]
            $L>0$ & evaluate the histogram for every frame\\
            $L=0$ & evaluate the histogram for the whole file\\
          \end{tabular}\\\hspace*{\fill}}{0}
        \argm{i}{I}{infimum}{0.0}
        \argm{j}{J}{supremum}{1.0}
        \argm{s}{S}{step size}{0.1}
        \argm{n}{}{normalization}{FALSE}
\end{options}

\begin{qsection}{EXAMPLE}
The example below plots the histogram of the speech waveform file
{\em data.f} in float format.
\begin{quote}
 \verb!histogram -i -16000 -j 16000 -s 100 data.f | fdrw | xgr!
\end{quote} 
\end{qsection}

\begin{qsection}{SEE ALSO}
 average
\end{qsection}
