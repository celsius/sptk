% ----------------------------------------------------------------
%       Speech Signal Processing Toolkit (SPTK): version 3.0
%                      SPTK Working Group
% 
%                Department of Computer Science
%                Nagoya Institute of Technology
%                             and
%   Interdisciplinary Graduate School of Science and Engineering
%                Tokyo Institute of Technology
%                   Copyright (c) 1984-2000
%                     All Rights Reserved.
% 
% Permission is hereby granted, free of charge, to use and
% distribute this software and its documentation without
% restriction, including without limitation the rights to use,
% copy, modify, merge, publish, distribute, sublicense, and/or
% sell copies of this work, and to permit persons to whom this
% work is furnished to do so, subject to the following conditions:
% 
%   1. The code must retain the above copyright notice, this list
%      of conditions and the following disclaimer.
% 
%   2. Any modifications must be clearly marked as such.
%                                                                        
% NAGOYA INSTITUTE OF TECHNOLOGY, TOKYO INSITITUTE OF TECHNOLOGY,
% SPTK WORKING GROUP, AND THE CONTRIBUTORS TO THIS WORK DISCLAIM
% ALL WARRANTIES WITH REGARD TO THIS SOFTWARE, INCLUDING ALL
% IMPLIED WARRANTIES OF MERCHANTABILITY AND FITNESS, IN NO EVENT
% SHALL NAGOYA INSTITUTE OF TECHNOLOGY, TOKYO INSITITUTE OF
% TECHNOLOGY, SPTK WORKING GROUP, NOR THE CONTRIBUTORS BE LIABLE
% FOR ANY SPECIAL, INDIRECT OR CONSEQUENTIAL DAMAGES OR ANY
% DAMAGES WHATSOEVER RESULTING FROM LOSS OF USE, DATA OR PROFITS,
% WHETHER IN AN ACTION OF CONTRACT, NEGLIGENCE OR OTHER TORTIOUS
% ACTION, ARISING OUT OF OR IN CONNECTION WITH THE USE OR
% PERFORMANCE OF THIS SOFTWARE.
% ----------------------------------------------------------------
%
\name{fd}{file dump}{data operation}

\begin{synopsis}
 \item [fd] [ --a $A$ ] [ --n $N$ ] [ --m $M$ ] [ --{\em ent} ] 
	    [ +{\em type} ] [ $\%${\em form} ] [ {\em infile} ]
\end{synopsis}

\begin{qsection}{DESCRIPTION}
This command writes to the output file data read from the input file
with an assigned format.
If the input file is omitted, data is read from the standard input.
\end{qsection}

\begin{options}
	\argm{a}{A}{address}{0}
	\argm{n}{N}{initial value for numbering}{0}
	\argm{m}{M}{modulo for numbering}{EOF}
	\argm{{\em ent}}{}{number of data in each line}{0}
	\argp{t}{data type\\
		\begin{tabular}{llcll} \\[-1zh]
			c & char (1byte) & \quad &
			s & short (2bytes) \\
			i & int (4bytes) & \quad &
			l & long (4bytes) \\
			f & float (4bytes) & \quad &
			d & double (8bytes) 
		\end{tabular}\\\hspace*{\fill}}{c}
        \argh{form}{}{print format(printf style)}{N/A}
\end{options}

\begin{qsection}{EXAMPLE}
 This example displays the speech data in ``sample.wav'' with
 the corresponding addresses:
\begin{quote}
 \verb!fd -a 0 sample.wav!
\end{quote}
 Results:\\
\verb!000000  52 49 46 46 9a 15 00 00 57 41 56 45 66 6d 74 20 |RIFF....WAVEfmt |!\\
\verb!000010  10 00 00 00 01 00 01 00 40 1f 00 00 40 1f 00 00 |........@...@...|!\\
\verb!000020  01 00 08 00 64 61 74 61 76 15 00 00 8a 8a 8f 99 |....datav.......|!

\begin{center}
 $\vdots$\\
\end{center}
\end{qsection}

\begin{qsection}{SEE ALSO}
 dmp
\end{qsection}
