% ----------------------------------------------------------------- %
%             The Speech Signal Processing Toolkit (SPTK)           %
%             developed by SPTK Working Group                       %
%             http://sp-tk.sourceforge.net/                         %
% ----------------------------------------------------------------- %
%                                                                   %
%  Copyright (c) 1984-2007  Tokyo Institute of Technology           %
%                           Interdisciplinary Graduate School of    %
%                           Science and Engineering                 %
%                                                                   %
%                1996-2011  Nagoya Institute of Technology          %
%                           Department of Computer Science          %
%                                                                   %
% All rights reserved.                                              %
%                                                                   %
% Redistribution and use in source and binary forms, with or        %
% without modification, are permitted provided that the following   %
% conditions are met:                                               %
%                                                                   %
% - Redistributions of source code must retain the above copyright  %
%   notice, this list of conditions and the following disclaimer.   %
% - Redistributions in binary form must reproduce the above         %
%   copyright notice, this list of conditions and the following     %
%   disclaimer in the documentation and/or other materials provided %
%   with the distribution.                                          %
% - Neither the name of the SPTK working group nor the names of its %
%   contributors may be used to endorse or promote products derived %
%   from this software without specific prior written permission.   %
%                                                                   %
% THIS SOFTWARE IS PROVIDED BY THE COPYRIGHT HOLDERS AND            %
% CONTRIBUTORS "AS IS" AND ANY EXPRESS OR IMPLIED WARRANTIES,       %
% INCLUDING, BUT NOT LIMITED TO, THE IMPLIED WARRANTIES OF          %
% MERCHANTABILITY AND FITNESS FOR A PARTICULAR PURPOSE ARE          %
% DISCLAIMED. IN NO EVENT SHALL THE COPYRIGHT OWNER OR CONTRIBUTORS %
% BE LIABLE FOR ANY DIRECT, INDIRECT, INCIDENTAL, SPECIAL,          %
% EXEMPLARY, OR CONSEQUENTIAL DAMAGES (INCLUDING, BUT NOT LIMITED   %
% TO, PROCUREMENT OF SUBSTITUTE GOODS OR SERVICES; LOSS OF USE,     %
% DATA, OR PROFITS; OR BUSINESS INTERRUPTION) HOWEVER CAUSED AND ON %
% ANY THEORY OF LIABILITY, WHETHER IN CONTRACT, STRICT LIABILITY,   %
% OR TORT (INCLUDING NEGLIGENCE OR OTHERWISE) ARISING IN ANY WAY    %
% OUT OF THE USE OF THIS SOFTWARE, EVEN IF ADVISED OF THE           %
% POSSIBILITY OF SUCH DAMAGE.                                       %
% ----------------------------------------------------------------- %
\hypertarget{us16}{}
\name{us16}{up-sampling from 10 or 12 kHz to 16 kHz}%
{sampling rate transformation}

\begin{synopsis}
\item [us16] [ --s $S$ ] [ {\em infile} ] [ {\em outfile} ]
\item [us16] [ --s $S$ ] {\em infile1} $\dots$ [ {\em infileN}] {\em outdir} 
\end{synopsis}

\begin{qsection}{DESCRIPTION}
{\em us16} up-samples data from 10 kHz or 12 kHz to 16 kHz. 
If {\em infile} and {\em outfile} arguments are not given, 
standard input and standard output are used. 
If several input files are given, 
the last argument is taken to be a directory name 
and multiple output files are created in that directory, 
with names similar to the input file names 
but the suffixes are changed to ``.16''.
\end{qsection}

\begin{options}
	\argm{s}{S}{input sampling frequency 10|12 kHz}{$10$}
\end{options}

\begin{qsection}{EXAMPLE}
In the example below, speech data sampled at 10 kHz
is read from {\em data.10}, up sampling to 16 kHz is undertaken,
and the results are written to {data.16}:
\begin{quote}
\verb!us16 -s 10 < data.10 > data.16!
\end{quote}
\end{qsection}

%\begin{qsection}{BUGS}
%none
%\end{qsection}

\begin{qsection}{SEE ALSO}
 \hyperlink{ds}{ds},
 \hyperlink{us}{us},
 \hyperlink{uscd}{uscd}
\end{qsection}
