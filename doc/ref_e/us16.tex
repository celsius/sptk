% ----------------------------------------------------------------
%       Speech Signal Processing Toolkit (SPTK): version 3.0
%                      SPTK Working Group
% 
%                Department of Computer Science
%                Nagoya Institute of Technology
%                             and
%   Interdisciplinary Graduate School of Science and Engineering
%                Tokyo Institute of Technology
%                   Copyright (c) 1984-2000
%                     All Rights Reserved.
% 
% Permission is hereby granted, free of charge, to use and
% distribute this software and its documentation without
% restriction, including without limitation the rights to use,
% copy, modify, merge, publish, distribute, sublicense, and/or
% sell copies of this work, and to permit persons to whom this
% work is furnished to do so, subject to the following conditions:
% 
%   1. The code must retain the above copyright notice, this list
%      of conditions and the following disclaimer.
% 
%   2. Any modifications must be clearly marked as such.
%                                                                        
% NAGOYA INSTITUTE OF TECHNOLOGY, TOKYO INSITITUTE OF TECHNOLOGY,
% SPTK WORKING GROUP, AND THE CONTRIBUTORS TO THIS WORK DISCLAIM
% ALL WARRANTIES WITH REGARD TO THIS SOFTWARE, INCLUDING ALL
% IMPLIED WARRANTIES OF MERCHANTABILITY AND FITNESS, IN NO EVENT
% SHALL NAGOYA INSTITUTE OF TECHNOLOGY, TOKYO INSITITUTE OF
% TECHNOLOGY, SPTK WORKING GROUP, NOR THE CONTRIBUTORS BE LIABLE
% FOR ANY SPECIAL, INDIRECT OR CONSEQUENTIAL DAMAGES OR ANY
% DAMAGES WHATSOEVER RESULTING FROM LOSS OF USE, DATA OR PROFITS,
% WHETHER IN AN ACTION OF CONTRACT, NEGLIGENCE OR OTHER TORTIOUS
% ACTION, ARISING OUT OF OR IN CONNECTION WITH THE USE OR
% PERFORMANCE OF THIS SOFTWARE.
% ----------------------------------------------------------------
%
\name{us16}{sampling rate conversion from 10|12kHz to 16kHz}%
{sampling rate transformation}

\begin{synopsis}
\item [us16] [ --s $S$ ] [ +{\em type} ] [ {\em infile} ] [ {\em outfile} ]
\item [us16] [ --s $S$ ] [ +{\em type} ] {\em infile1} $\cdots$ [ {\em infileN}] {\em outdir} 
\end{synopsis}

\begin{qsection}{DESCRIPTION}
{\em us16} upsamples data from 10kHz or 12kHz to 16kHz. 
If {\em infile} and {\em outfile} arguments are not given, 
standard input and standard output are used. 
If several input files are given, 
the last argument is taken to be a directory name 
and multiple output files are created in that directory, 
with names similar to the input file names 
but the suffixes are changed to ``.16''.
\end{qsection}

\begin{options}
	\argm{s}{S}{input sampling frequency 10|12kHz}{$10$}
	\argp{t}{input data format\\
		\begin{tabular}{ll} \\[-1ex]
			s & short (2bytes) \\
			f & float (4bytes)
		\end{tabular}\\}{s}
\end{options}

\begin{qsection}{EXAMPLE}
In the example below, speech data sampled at 10kHz in short format
is read from {\em data.10}, up sampling to 16kHz is undertaken,
and the results are written to {data.16}:
\begin{quote}
\verb!us16 -s 10 < data.10 > data.16!
\end{quote}
\end{qsection}

%\begin{qsection}{BUGS}
%none
%\end{qsection}

%\begin{qsection}{SEE ALSO}
%none
%\end{qsection}
