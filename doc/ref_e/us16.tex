%  ---------------------------------------------------------------  %
%            Speech Signal Processing Toolkit (SPTK)                %
%                      SPTK Working Group                           %
%                                                                   %
%                  Department of Computer Science                   %
%                  Nagoya Institute of Technology                   %
%                               and                                 %
%   Interdisciplinary Graduate School of Science and Engineering    %
%                  Tokyo Institute of Technology                    %
%                                                                   %
%                     Copyright (c) 1984-2007                       %
%                       All Rights Reserved.                        %
%                                                                   %
%  Permission is hereby granted, free of charge, to use and         %
%  distribute this software and its documentation without           %
%  restriction, including without limitation the rights to use,     %
%  copy, modify, merge, publish, distribute, sublicense, and/or     %
%  sell copies of this work, and to permit persons to whom this     %
%  work is furnished to do so, subject to the following conditions: %
%                                                                   %
%    1. The source code must retain the above copyright notice,     %
%       this list of conditions and the following disclaimer.       %
%                                                                   %
%    2. Any modifications to the source code must be clearly        %
%       marked as such.                                             %
%                                                                   %
%    3. Redistributions in binary form must reproduce the above     %
%       copyright notice, this list of conditions and the           %
%       following disclaimer in the documentation and/or other      %
%       materials provided with the distribution.  Otherwise, one   %
%       must contact the SPTK working group.                        %
%                                                                   %
%  NAGOYA INSTITUTE OF TECHNOLOGY, TOKYO INSTITUTE OF TECHNOLOGY,   %
%  SPTK WORKING GROUP, AND THE CONTRIBUTORS TO THIS WORK DISCLAIM   %
%  ALL WARRANTIES WITH REGARD TO THIS SOFTWARE, INCLUDING ALL       %
%  IMPLIED WARRANTIES OF MERCHANTABILITY AND FITNESS, IN NO EVENT   %
%  SHALL NAGOYA INSTITUTE OF TECHNOLOGY, TOKYO INSTITUTE OF         %
%  TECHNOLOGY, SPTK WORKING GROUP, NOR THE CONTRIBUTORS BE LIABLE   %
%  FOR ANY SPECIAL, INDIRECT OR CONSEQUENTIAL DAMAGES OR ANY        %
%  DAMAGES WHATSOEVER RESULTING FROM LOSS OF USE, DATA OR PROFITS,  %
%  WHETHER IN AN ACTION OF CONTRACT, NEGLIGENCE OR OTHER TORTUOUS   %
%  ACTION, ARISING OUT OF OR IN CONNECTION WITH THE USE OR          %
%  PERFORMANCE OF THIS SOFTWARE.                                    %
%                                                                   %
%  ---------------------------------------------------------------  %
%
\hypertarget{us16}{}
\name{us16}{sampling rate conversion from 10|12kHz to 16kHz}%
{sampling rate transformation}

\begin{synopsis}
\item [us16] [ --s $S$ ] [ +{\em type} ] [ {\em infile} ] [ {\em outfile} ]
\item [us16] [ --s $S$ ] [ +{\em type} ] {\em infile1} $\dots$ [ {\em infileN}] {\em outdir} 
\end{synopsis}

\begin{qsection}{DESCRIPTION}
{\em us16} up-samples data from 10 kHz or 12 kHz to 16 kHz. 
If {\em infile} and {\em outfile} arguments are not given, 
standard input and standard output are used. 
If several input files are given, 
the last argument is taken to be a directory name 
and multiple output files are created in that directory, 
with names similar to the input file names 
but the suffixes are changed to ``.16''.
\end{qsection}

\begin{options}
	\argm{s}{S}{input sampling frequency 10|12 kHz}{$10$}
	\argp{t}{input and output data format\\
		\begin{tabular}{ll} \\[-1ex]
			s & short (2 bytes) \\
			f & float (4 bytes)
		\end{tabular}\\}{f}
\end{options}

\begin{qsection}{EXAMPLE}
In the example below, speech data sampled at 10 kHz in short format
is read from {\em data.10}, up sampling to 16 kHz is undertaken,
and the results are written to {data.16}:
\begin{quote}
\verb!us16 +s -s 10 < data.10 > data.16!
\end{quote}
\end{qsection}

%\begin{qsection}{BUGS}
%none
%\end{qsection}

%\begin{qsection}{SEE ALSO}
%none
%\end{qsection}
