\name{us16}{sampling rate conversion from 10|12kHz to 16kHz}%
{sampling rate conversion}

\begin{synopsis}
\item [us16] [ --s $S$ ] [ +{\em type} ] [ {\em infile} ] [ {\em outfile} ]
\item [us16] [ --s $S$ ] [ +{\em type} ] {\em infile1} $\cdots$ [ {\em infileN}] {\em outdir} 
\end{synopsis}

\begin{qsection}{DESCRIPTION}
The {\em us16} command converts sampling rate of 10, 12kHz
into 16kHz.
If input and/or output files are not assigned,
then the standard input and/or output are used, respectively.
Also, several files can be assigned at same time.
In this case the output is automatically written in the assigned
{\em outdir} directory for every input file
changing the corresponding suffix.
\par
\end{qsection}

\begin{options}
	\argm{s}{S}{input sampling frequency 10|12kHz}{$10$}
	\argp{t}{input data format\\
		\begin{tabular}{ll} \\[-1zh]
			s & short (2bytes) \\
			f & float (4bytes)
		\end{tabular}\\}{s}
\end{options}

\begin{qsection}{EXAMPLE}
In the example below, speech data sampled at 10kHz in short format
is read from {\em data.10}, up sampling to 16kHz is undertaken,
and the results are written to {data.16}:
\begin{quote}
\verb!us16 -s 10 < data.10 > data.16!
\end{quote}
\end{qsection}

%\begin{qsection}{BUGS}
%none
%\end{qsection}

%\begin{qsection}{SEE ALSO}
%none
%\end{qsection}
