% ----------------------------------------------------------------- %
%             The Speech Signal Processing Toolkit (SPTK)           %
%             developed by SPTK Working Group                       %
%             http://sp-tk.sourceforge.net/                         %
% ----------------------------------------------------------------- %
%                                                                   %
%  Copyright (c) 1984-2007  Tokyo Institute of Technology           %
%                           Interdisciplinary Graduate School of    %
%                           Science and Engineering                 %
%                                                                   %
%                1996-2014  Nagoya Institute of Technology          %
%                           Department of Computer Science          %
%                                                                   %
% All rights reserved.                                              %
%                                                                   %
% Redistribution and use in source and binary forms, with or        %
% without modification, are permitted provided that the following   %
% conditions are met:                                               %
%                                                                   %
% - Redistributions of source code must retain the above copyright  %
%   notice, this list of conditions and the following disclaimer.   %
% - Redistributions in binary form must reproduce the above         %
%   copyright notice, this list of conditions and the following     %
%   disclaimer in the documentation and/or other materials provided %
%   with the distribution.                                          %
% - Neither the name of the SPTK working group nor the names of its %
%   contributors may be used to endorse or promote products derived %
%   from this software without specific prior written permission.   %
%                                                                   %
% THIS SOFTWARE IS PROVIDED BY THE COPYRIGHT HOLDERS AND            %
% CONTRIBUTORS "AS IS" AND ANY EXPRESS OR IMPLIED WARRANTIES,       %
% INCLUDING, BUT NOT LIMITED TO, THE IMPLIED WARRANTIES OF          %
% MERCHANTABILITY AND FITNESS FOR A PARTICULAR PURPOSE ARE          %
% DISCLAIMED. IN NO EVENT SHALL THE COPYRIGHT OWNER OR CONTRIBUTORS %
% BE LIABLE FOR ANY DIRECT, INDIRECT, INCIDENTAL, SPECIAL,          %
% EXEMPLARY, OR CONSEQUENTIAL DAMAGES (INCLUDING, BUT NOT LIMITED   %
% TO, PROCUREMENT OF SUBSTITUTE GOODS OR SERVICES; LOSS OF USE,     %
% DATA, OR PROFITS; OR BUSINESS INTERRUPTION) HOWEVER CAUSED AND ON %
% ANY THEORY OF LIABILITY, WHETHER IN CONTRACT, STRICT LIABILITY,   %
% OR TORT (INCLUDING NEGLIGENCE OR OTHERWISE) ARISING IN ANY WAY    %
% OUT OF THE USE OF THIS SOFTWARE, EVEN IF ADVISED OF THE           %
% POSSIBILITY OF SUCH DAMAGE.                                       %
% ----------------------------------------------------------------- %
\hypertarget{gmmp}{}
\name{gmmp}{calculation of GMM log-probability}{probability calculation}

\begin{synopsis}
\item [gmmp] [ --l $L$ ] [ --m $M$ ] [ --a ] {\em gmmfile} [ {\em infile} ]
\end{synopsis}

\begin{qsection}{DESCRIPTION}
{\em gmmp} calculates GMM log-probabilities of input vectors from {\em
infile} (or standard input). 
The {\em gmmfile} has the same file format as the one generated by the {\em gmm} command,
i.e., {\em gmmfile} consists of $M$ mixture weights
$\bw$ and $M$ Gaussians with mean vector $\bmu$ and diagonal variance vector
$\bv$, each of length $L$:
\begin{align}
 \lambda =
 \left[\bw,\right.&\left.\bmu(0),\bv(0), \bmu(1), \bv(1),
 \ldots, \bmu(M-1), \bv(M-1)\right],\notag\\[2mm]
 \bw &=\left[ w(0), w(1), \ldots, w(M-1) \right],\notag\\
 \bmu(m) &=\left[\mu_m(0), \mu_m(1), \ldots, \mu_m(L-1)\right],\notag\\
 \bv(m) &=\left[\sigma_m^2(0), \sigma_m^2(1), \ldots,
 \sigma_m^2(L-1)\right].\notag
\end{align}


The input sequence consists of $T$ float vectors $\bx$, each of
size $L$:
\begin{displaymath}
 \bx(0), \bx(1), \dots, \bx(T-1).
\end{displaymath}
The result is a sequence of log-probabilities of input vectors:
\begin{displaymath}
 \log b(\bx(0)), \log b(\bx(1)), \ldots, \log b(\bx(T-1)),
\end{displaymath}
or an average log-probability (if -a option is used):
\begin{displaymath}
 \log P(\bX) = \frac{1}{T}\sum_{t=0}^{T-1}\log b(\bx(t)),
\end{displaymath}
where
\begin{align}
 &b(\bx(t)) =\sum_{m=0}^{M-1}
 w(m){\cal N}(\bx(t) \; ; \; \bmu(m),\bv(m)),\notag\\
 &{\cal N}(\bx(t) \; ; \; \bmu(m),\bv(m))%
  =\frac{1}{(2\pi)^{L/2}\prod_{l=0}^{L-1}\sigma_m(l)}%
  \exp{\left\{-\frac{1}{2}%
    \sum_{l=0}^{L-1}
    \frac{\left(x_t(l)-\mu_m(l)\right)^2}%
    {\sigma_m^2(l)}\right\}}.\notag
\end{align}

\end{qsection}

\begin{options}
 \argm{l}{L}{length of vector}{26}
 \argm{m}{M}{number of Gaussian components}{16}
 \argm{f}{}{full covariance}{FALSE}
 \argm{a}{}{print average log-probability}{FALSE}
\end{options}

\begin{qsection}{EXAMPLE}
In the following example, frame log-probabilities of input data {\em
data.f} for GMM with 8 Gaussians {\em gmm.f} are written to {\em
probs.f}.

\begin{quote}
\verb! gmmp -m 8 gmm.f data.f > probs.f!
\end{quote}
\end{qsection}

\begin{qsection}{SEE ALSO}
\hyperlink{gmm}{gmm}
\end{qsection}
