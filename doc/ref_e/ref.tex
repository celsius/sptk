% ----------------------------------------------------------------- %
%             The Speech Signal Processing Toolkit (SPTK)           %
%             developed by SPTK Working Group                       %
%             http://sp-tk.sourceforge.net/                         %
% ----------------------------------------------------------------- %
%                                                                   %
%  Copyright (c) 1984-2007  Tokyo Institute of Technology           %
%                           Interdisciplinary Graduate School of    %
%                           Science and Engineering                 %
%                                                                   %
%                1996-2009  Nagoya Institute of Technology          %
%                           Department of Computer Science          %
%                                                                   %
% All rights reserved.                                              %
%                                                                   %
% Redistribution and use in source and binary forms, with or        %
% without modification, are permitted provided that the following   %
% conditions are met:                                               %
%                                                                   %
% - Redistributions of source code must retain the above copyright  %
%   notice, this list of conditions and the following disclaimer.   %
% - Redistributions in binary form must reproduce the above         %
%   copyright notice, this list of conditions and the following     %
%   disclaimer in the documentation and/or other materials provided %
%   with the distribution.                                          %
% - Neither the name of the SPTK working group nor the names of its %
%   contributors may be used to endorse or promote products derived %
%   from this software without specific prior written permission.   %
%                                                                   %
% THIS SOFTWARE IS PROVIDED BY THE COPYRIGHT HOLDERS AND            %
% CONTRIBUTORS "AS IS" AND ANY EXPRESS OR IMPLIED WARRANTIES,       %
% INCLUDING, BUT NOT LIMITED TO, THE IMPLIED WARRANTIES OF          %
% MERCHANTABILITY AND FITNESS FOR A PARTICULAR PURPOSE ARE          %
% DISCLAIMED. IN NO EVENT SHALL THE COPYRIGHT OWNER OR CONTRIBUTORS %
% BE LIABLE FOR ANY DIRECT, INDIRECT, INCIDENTAL, SPECIAL,          %
% EXEMPLARY, OR CONSEQUENTIAL DAMAGES (INCLUDING, BUT NOT LIMITED   %
% TO, PROCUREMENT OF SUBSTITUTE GOODS OR SERVICES; LOSS OF USE,     %
% DATA, OR PROFITS; OR BUSINESS INTERRUPTION) HOWEVER CAUSED AND ON %
% ANY THEORY OF LIABILITY, WHETHER IN CONTRACT, STRICT LIABILITY,   %
% OR TORT (INCLUDING NEGLIGENCE OR OTHERWISE) ARISING IN ANY WAY    %
% OUT OF THE USE OF THIS SOFTWARE, EVEN IF ADVISED OF THE           %
% POSSIBILITY OF SUCH DAMAGE.                                       %
% ----------------------------------------------------------------- %
\begin{thebibliography}{99}

% improved cepstral analysis

\bibitem{ref:icep-IECE}
S. Imai and Y. Abe,
``Spectral envelope extraction by improved cepstral method,''
{\sl Journal of IEICE},
 Vol.J62-A, No.4, pp.217--223, Apr. 1987. ({\sl in Japanese})


% unbiased estimation of log spectrum

\bibitem{ref:UELS-IEICE}
S. Imai and C. Furuichi,
``Unbiased estimation of log spectrum,''
{\sl Journal of IEICE}, 
Vol.J70-A, No.3, pp.471--480, Mar. 1987. ({\sl in Japanese})

\bibitem{ref:UELS-SignalProcessingIV}
S. Imai and C. Furuichi,
``Unbiased estimator of log spectrum
  and its application to speech signal processing,''
{\sl Signal Processing IV: Theory and Applications},
  Vol.1, pp.203--206, Elsevier, North-Holland, 1988.

\bibitem{ref:acep-IEICE}
K. Tokuda, T. Kobayashi, S. Shiomoto, and S. Imai,
``Adaptive cepstral analysis --- Adaptive filtering based on cepstral representation ---,''
{\sl Journal of IEICE},
Vol.J73-A, No.7, pp.1207--1215, July 1990. ({\sl in Japanese})

\bibitem{ref:acep-IEEESP}
K. Tokuda, T. Kobayashi, and S. Imai,
``Adaptive cepstral analysis of speech,''
{\sl IEEE Trans. Speech and Audio Process.}, 
Vol.3, No.6, pp.481--488, Nov. 1995.


% generalized cepstral analysis

\bibitem{ref:gcep-IEICE}
K. Tokuda, T. Kobayashi, R. Yamamoto, and S. Imai,
``Spectral estimation of speech based on generalized cepstral representation,''
{\sl Journal of IEICE},
Vol.J72-A, No.3, pp.457--465, Mar. 1989. ({\sl in Japanese})

\bibitem{ref:gcep-IEEEASSP}
T. Kobayashi and S. Imai,
``Spectral analysis using generalized cepstrum,''
{\sl IEEE Trans. Acoust., Speech, Signal Process.},
  Vol.ASSP-32, No.5, pp.1087--1089, Oct. 1984.

\bibitem{ref:gcep-ICSLP90}
K. Tokuda, T. Kobayashi, and S. Imai,
``Generalized cepstral analysis of speech --- a unified approach to LPC and cepstral method,''
{\sl Proc. ICSLP-90}, % Kobe, Japan,
  pp.37--40, 
  Nov. 1990.

\bibitem{ref:agcep-IEICEtaikai90s}
T. Fukada, K. Tokuda, T. Kobayashi, and S. Imai,
``A study on adaptive generalized cepstral analysis,''
{\sl IEICE Spring National Convention},
A-150, p.150, Mar. 1990. ({\sl in Japanese})


% mel-cepstral analysis

\bibitem{ref:mcep-IEICE}
K. Tokuda, T. Kobayashi, T. Fukada, H. Saito, and S. Imai,
``Spectral estimation of speech based on mel-cepstral representation,''
{\sl Journal of IEICE},
Vol.J74-A, No.8, pp.1240--1248, Aug. 1991. ({\sl in Japanese})

\bibitem{ref:amcep-IEICE}
K. Tokuda, T. Kobayashi, T. Fukada, and S. Imai,
``Adaptive mel-cepstral analysis of speech,''
{\sl Journal of IEICE},
Vol.J74-A, No.8, pp.1249--1256, Aug. 1991. ({\sl in Japanese})

\bibitem{ref:amcep-ICASSP92}
T. Fukada, K. Tokuda, T. Kobayashi, and S. Imai, 
``An adaptive algorithm for mel-cepstral analysis of speech,''
{\sl Proc. ICASSP-92}, % San Francisco, USA, 
  pp.137--140, %
  Mar. 1992.


% mel-generalized cepstral analysis

\bibitem{ref:mgcep-IEICE}
K. Tokuda, T. Kobayashi, K. Chiba, and S. Imai,
``Spectral estimation of speech by mel-generalized cepstral analysis,''
{\sl Journal of IEICE},
Vol.J75-A, No.7, pp.1124--1134, July 1992. ({\sl in Japanese})

\bibitem{ref:mgcep-ICSLP94}
K. Tokuda, T. Kobayashi, T. Masuko, and S. Imai,
``Mel-generalized cepstral analysis --- a unified approach to speech spectral estimation,''
{\sl Proc. ICSLP-94}, 
pp.1043--1046, Sep. 1994.


% mel-cepstral analysis using a 2nd-order all-pass function

\bibitem{ref:smcep-IEICE}
T. Wakako, K. Tokuda, T. Masuko, T. Kobayashi, and T. Kitamura,
``Speech spectral estimation based on expansion of log spectrum by arbitrary basis functions,'',
{\sl Journal of IEICE},
Vol.J82-D-II, No.12, pp.2203--2211, Dec. 1999. ({\sl in Japanese})

\bibitem{ref:smcep-SPCOM}
C. Miyajima C, H. Watanabe, K. Tokuda, T. Kitamura, and S. Katagiri,
``A new approach to designing a feature extractor in speaker identification based on discriminative feature extraction,''
{\sl Speech Communication},
Vol.35, No.3, pp.203--218, Oct. 2001.


% LMA, GLSA, MLSA, MGLSA filters

\bibitem{ref:LMA-IECE}
S. Imai,
``Log magnitude approximation (LMA) filter,''
{\sl Journal of IEICE},
Vol.J63-A, No.12, pp.886--893, Dec. 1987. ({\sl in Japanese})

\bibitem{ref:GLSA-IEICEtaikai90s}
T. Chiba, K. Tokuda, T. Kobayashi, and S. Imai,
``Speech synthesis based on mel-generalized cepstral representation,''
{\sl IEICE Spring National Convention},
A-243, p.243, Mar. 1988. ({\sl in Japanese})

\bibitem{ref:MLSA-ICASSP}
S. Imai, 
``Cepstral analysis synthesis on the mel frequency scale,''
{\sl Proc. ICASSP-83},
pp.93--96, Apr. 1983.

\bibitem{ref:MLSA-IECE}
S. Imai, K. Sumita, and C. Furuichi,
``Mel log spectrum approximation (MLSA) filter for speech synthesis,''
{\sl Journal of IEICE},
Vol.J66-A, No.2, pp.122--129, Feb. 1983. ({\sl in Japanese})

\bibitem{ref:MGLSA-IECE}
T. Kobayashi, S. Imai, and Y. Fukuda, 
``Mel generalized-log spectrum approximation (MGLSA) filter,''
{\sl Journal of IEICE},
Vol.J68-A, No.6, pp.610--611, June 1985. ({\sl in Japanese})

\bibitem{ref:MGLSA-IEICE}
K. Koishida, G. Hirabayashi, K. Tokuda, and T. Kobayashi, 
``A 16kbit/s wideband CELP-based speech coder using mel-generalized cepstral analysis,'' 
{\sl IEICE Trans. Inf. and Syst.},
vol.E83-D, no.4, pp.876--883, Apr. 2000.

% parameter generation algorithm

\bibitem{ref:synHMM-EUROSPEECH95}
K. Tokuda, T. Masuko, T. Yamada, T. Kobayashi, and S. Imai,
``An algorithm for speech parameter generation
  from continuous mixture HMMs with dynamic features,''
{\sl Proc. EUROSPEECH-95}, pp.757--760, Sep. 1995.

% pitch extraction algorithm
\bibitem{ref:pitch-RAPT}
D. Talkin, ``A Robust Algorithm for Pitch Tracking (RAPT),''
in {\sl Speech Coding \& Synthesis}, W. B. Kleijn and K. K. Pailwal
(Eds.), Elsevier, pp.495--518, 1995.

\bibitem{ref:pitch-SWIPE}
A. Camacho, ``SWIPE: A Sawtooth Waveform Inspired Pitch Estimator for Speech And Music,''
Ph.D. Thesis, University of Florida, 116p., 2007.

% GMM-based voice conversion
\bibitem{ref:vc-IEEETASLP}
T. Toda, Alan W. Black, and K. Tokuda, ``Voice Conversion Based on Maximum-Likelihood
Estimation of Spectral Parameter Trajectory,''
{\sl IEEE Trans. Audio, Speech, Language Process.},
Vol.15, No. 8, pp.2222--2235, Nov.
\end{thebibliography}
