% ----------------------------------------------------------------
%       Speech Signal Processing Toolkit (SPTK): version 3.0
%                      SPTK Working Group
% 
%                Department of Computer Science
%                Nagoya Institute of Technology
%                             and
%   Interdisciplinary Graduate School of Science and Engineering
%                Tokyo Institute of Technology
%                   Copyright (c) 1984-2000
%                     All Rights Reserved.
% 
% Permission is hereby granted, free of charge, to use and
% distribute this software and its documentation without
% restriction, including without limitation the rights to use,
% copy, modify, merge, publish, distribute, sublicense, and/or
% sell copies of this work, and to permit persons to whom this
% work is furnished to do so, subject to the following conditions:
% 
%   1. The code must retain the above copyright notice, this list
%      of conditions and the following disclaimer.
% 
%   2. Any modifications must be clearly marked as such.
%                                                                        
% NAGOYA INSTITUTE OF TECHNOLOGY, TOKYO INSITITUTE OF TECHNOLOGY,
% SPTK WORKING GROUP, AND THE CONTRIBUTORS TO THIS WORK DISCLAIM
% ALL WARRANTIES WITH REGARD TO THIS SOFTWARE, INCLUDING ALL
% IMPLIED WARRANTIES OF MERCHANTABILITY AND FITNESS, IN NO EVENT
% SHALL NAGOYA INSTITUTE OF TECHNOLOGY, TOKYO INSITITUTE OF
% TECHNOLOGY, SPTK WORKING GROUP, NOR THE CONTRIBUTORS BE LIABLE
% FOR ANY SPECIAL, INDIRECT OR CONSEQUENTIAL DAMAGES OR ANY
% DAMAGES WHATSOEVER RESULTING FROM LOSS OF USE, DATA OR PROFITS,
% WHETHER IN AN ACTION OF CONTRACT, NEGLIGENCE OR OTHER TORTIOUS
% ACTION, ARISING OUT OF OR IN CONNECTION WITH THE USE OR
% PERFORMANCE OF THIS SOFTWARE.
% ----------------------------------------------------------------
%
\hypertarget{root_pol}{}
\name{root\_pol}{calculate roots of a polynomial equation}{signal processing}

\begin{synopsis}
\item [root\_pol] [ --m $M$ ] [ --n $N$ ] [ --e $E$ ] [ --i ]
 [ --s ] [ --r ] [ {\em infile} ]
\end{synopsis}

\begin{qsection}{DESCRIPTION}
{\em root\_pol} finds root values of a polynomial equation
from {\em infile} (or standard input), 
sending the result to standard output.

For given input file, read coefficients
\begin{displaymath}
  a_0, a_1, \dots, a_n
\end{displaymath}
of an $n$-th order polynomial equation
\begin{displaymath}
  P(x) = a_0x^n + a_1x^{n-1} + \dots + a_{n-1}x + a_n
\end{displaymath}
calculate root values by Durand-Kerner-Aberth method.
\par
If roots of $P(x)$ are $z_i$, 
the result is sent to standard output 
in complex form as
\begin{displaymath}
   \begin{matrix}
   \mathrm{Re}[z_0], & \mathrm{Im}[z_0] \\
   \mathrm{Re}[z_1], & \mathrm{Im}[z_1] \\
   \vdots            &                  \\
   \mathrm{Re}[z_{n-1}], & \mathrm{Im}[z_{n-1}] \\
   \end{matrix}
\end{displaymath}
or polar form as
\begin{displaymath}
   \begin{matrix}
   |z_0|, & \arg[z_0] \\
   |z_1|, & \arg[z_1] \\
   \vdots &           \\
   |z_{n-1}|, & \arg[z_{n-1}] \\
   \end{matrix}
\end{displaymath}
\par
Both input and output data are in float format.
\end{qsection}

\begin{options}
        \argm{m}{M}{order of polynomial equation}{32}
        \argm{n}{N}{maximum iteration to search roots}{1000}
        \argm{e}{E}{error margin for roots $\varepsilon$}{$10^{-14}$}
        \argm{i}{}{set $a_0 = 1$}{FALSE}
        \argm{s}{}{reverse order of coefficients}{FALSE}
        \argm{r}{}{output results in polar form}{complex form}
\end{options}

\begin{qsection}{EXAMPLE}
This example calculates roots of a polynomial equation from file {\em data.z}
and output its results in polar form:
\begin{quote}
 \verb!root -r < data.z | x2x +a 2!
\end{quote}
\end{qsection}

%\begin{qsection}{SEE ALSO}
% 
%\end{qsection}
