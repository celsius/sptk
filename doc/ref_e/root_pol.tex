% ----------------------------------------------------------------- %
%             The Speech Signal Processing Toolkit (SPTK)           %
%             developed by SPTK Working Group                       %
%             http://sp-tk.sourceforge.net/                         %
% ----------------------------------------------------------------- %
%                                                                   %
%  Copyright (c) 1984-2007  Tokyo Institute of Technology           %
%                           Interdisciplinary Graduate School of    %
%                           Science and Engineering                 %
%                                                                   %
%                1996-2008  Nagoya Institute of Technology          %
%                           Department of Computer Science          %
%                                                                   %
% All rights reserved.                                              %
%                                                                   %
% Redistribution and use in source and binary forms, with or        %
% without modification, are permitted provided that the following   %
% conditions are met:                                               %
%                                                                   %
% - Redistributions of source code must retain the above copyright  %
%   notice, this list of conditions and the following disclaimer.   %
% - Redistributions in binary form must reproduce the above         %
%   copyright notice, this list of conditions and the following     %
%   disclaimer in the documentation and/or other materials provided %
%   with the distribution.                                          %
% - Neither the name of the SPTK working group nor the names of its %
%   contributors may be used to endorse or promote products derived %
%   from this software without specific prior written permission.   %
%                                                                   %
% THIS SOFTWARE IS PROVIDED BY THE COPYRIGHT HOLDERS AND            %
% CONTRIBUTORS "AS IS" AND ANY EXPRESS OR IMPLIED WARRANTIES,       %
% INCLUDING, BUT NOT LIMITED TO, THE IMPLIED WARRANTIES OF          %
% MERCHANTABILITY AND FITNESS FOR A PARTICULAR PURPOSE ARE          %
% DISCLAIMED. IN NO EVENT SHALL THE COPYRIGHT OWNER OR CONTRIBUTORS %
% BE LIABLE FOR ANY DIRECT, INDIRECT, INCIDENTAL, SPECIAL,          %
% EXEMPLARY, OR CONSEQUENTIAL DAMAGES (INCLUDING, BUT NOT LIMITED   %
% TO, PROCUREMENT OF SUBSTITUTE GOODS OR SERVICES; LOSS OF USE,     %
% DATA, OR PROFITS; OR BUSINESS INTERRUPTION) HOWEVER CAUSED AND ON %
% ANY THEORY OF LIABILITY, WHETHER IN CONTRACT, STRICT LIABILITY,   %
% OR TORT (INCLUDING NEGLIGENCE OR OTHERWISE) ARISING IN ANY WAY    %
% OUT OF THE USE OF THIS SOFTWARE, EVEN IF ADVISED OF THE           %
% POSSIBILITY OF SUCH DAMAGE.                                       %
% ----------------------------------------------------------------- %
\hypertarget{root_pol}{}
\name{root\_pol}{calculate roots of a polynomial equation}{signal processing}

\begin{synopsis}
\item [root\_pol] [ --m $M$ ] [ --n $N$ ] [ --e $E$ ] [ --i ]
 [ --s ] [ --r ] [ {\em infile} ]
\end{synopsis}

\begin{qsection}{DESCRIPTION}
{\em root\_pol} finds root values of a polynomial equation
from {\em infile} (or standard input), 
sending the result to standard output.

For given input file, read coefficients
\begin{displaymath}
  a_0, a_1, \dots, a_n
\end{displaymath}
of an $n$-th order polynomial equation
\begin{displaymath}
  P(x) = a_0x^n + a_1x^{n-1} + \dots + a_{n-1}x + a_n
\end{displaymath}
calculate root values by Durand-Kerner-Aberth method.
\par
If roots of $P(x)$ are $z_i$, 
the result is sent to standard output 
in complex form as
\begin{displaymath}
   \begin{matrix}
   \mathrm{Re}[z_0], & \mathrm{Im}[z_0] \\
   \mathrm{Re}[z_1], & \mathrm{Im}[z_1] \\
   \vdots            &                  \\
   \mathrm{Re}[z_{n-1}], & \mathrm{Im}[z_{n-1}] \\
   \end{matrix}
\end{displaymath}
or polar form as
\begin{displaymath}
   \begin{matrix}
   |z_0|, & \arg[z_0] \\
   |z_1|, & \arg[z_1] \\
   \vdots &           \\
   |z_{n-1}|, & \arg[z_{n-1}] \\
   \end{matrix}
\end{displaymath}
\par
Both input and output data are in float format.
\end{qsection}

\begin{options}
        \argm{m}{M}{order of polynomial equation}{32}
        \argm{n}{N}{maximum iteration to search roots}{1000}
        \argm{e}{E}{error margin for roots $\varepsilon$}{$10^{-14}$}
        \argm{i}{}{set $a_0 = 1$}{FALSE}
        \argm{s}{}{reverse order of coefficients}{FALSE}
        \argm{r}{}{output results in polar form}{complex form}
\end{options}

\begin{qsection}{EXAMPLE}
This example calculates roots of a polynomial equation from file {\em data.z}
and output its results in polar form:
\begin{quote}
 \verb!root_pol -r < data.z | x2x +a 2!
\end{quote}
\end{qsection}

%\begin{qsection}{SEE ALSO}
% 
%\end{qsection}
