% ----------------------------------------------------------------
%       Speech Signal Processing Toolkit (SPTK): version 3.0
%                      SPTK Working Group
% 
%                Department of Computer Science
%                Nagoya Institute of Technology
%                             and
%   Interdisciplinary Graduate School of Science and Engineering
%                Tokyo Institute of Technology
%                   Copyright (c) 1984-2000
%                     All Rights Reserved.
% 
% Permission is hereby granted, free of charge, to use and
% distribute this software and its documentation without
% restriction, including without limitation the rights to use,
% copy, modify, merge, publish, distribute, sublicense, and/or
% sell copies of this work, and to permit persons to whom this
% work is furnished to do so, subject to the following conditions:
% 
%   1. The code must retain the above copyright notice, this list
%      of conditions and the following disclaimer.
% 
%   2. Any modifications must be clearly marked as such.
%                                                                        
% NAGOYA INSTITUTE OF TECHNOLOGY, TOKYO INSITITUTE OF TECHNOLOGY,
% SPTK WORKING GROUP, AND THE CONTRIBUTORS TO THIS WORK DISCLAIM
% ALL WARRANTIES WITH REGARD TO THIS SOFTWARE, INCLUDING ALL
% IMPLIED WARRANTIES OF MERCHANTABILITY AND FITNESS, IN NO EVENT
% SHALL NAGOYA INSTITUTE OF TECHNOLOGY, TOKYO INSITITUTE OF
% TECHNOLOGY, SPTK WORKING GROUP, NOR THE CONTRIBUTORS BE LIABLE
% FOR ANY SPECIAL, INDIRECT OR CONSEQUENTIAL DAMAGES OR ANY
% DAMAGES WHATSOEVER RESULTING FROM LOSS OF USE, DATA OR PROFITS,
% WHETHER IN AN ACTION OF CONTRACT, NEGLIGENCE OR OTHER TORTIOUS
% ACTION, ARISING OUT OF OR IN CONNECTION WITH THE USE OR
% PERFORMANCE OF THIS SOFTWARE.
% ----------------------------------------------------------------
%
\name{par2lpc}{transform PARCOR to LPC}{speech parameter transformation}

\begin{synopsis}
\item [par2lpc] [ --m $M$ ] [ {\em infile} ] 
\end{synopsis}

\begin{qsection}{DESCRIPTION}
{\em par2lpc} calculates linear prediction (LPC) coefficients 
from $M$-order PARCOR coefficients from {\em infile} (or standard input), 
sending the result to standard output.

The PARCOR input format is
\begin{displaymath}
  K, k(1),\ldots, k(M), 
\end{displaymath}
and the LPC output format is
\begin{displaymath}
  K, a(1),\ldots, a(M).
\end{displaymath}

Input and output data are in float format.

The transformation of PARCOR coefficients into 
linear prediction coefficients is undertaken 
by a part of Durbin algorithm as follows.
\begin{eqnarray*} 
a^{(m)}(m) &=& k(m) \\
a^{(m)}(i) &=& a^{(m-1)}(i) + k(m) a^{(m-1)}(m-i), ~~~~~1\leq i \leq m
\end{eqnarray*}
where  $m=1, 2, \ldots, p$.
The initial condition is 
\begin{displaymath}
a^{(M)}(m) = a(m), ~~~~~1 \leq m \leq M.
\end{displaymath}
\end{qsection}

\begin{options}
	\argm{m}{M}{order of LPC}{25}
\end{options}

\begin{qsection}{EXAMPLE}
PARCOR coefficients are read in float format from {\em data.rc}
and converted into the corresponding linear prediction coefficients.
The output is written to {\em data.lpc}:
\begin{quote}
 \verb!par2lpc < data.rc > data.lpc!
\end{quote} 
\end{qsection}

\begin{qsection}{SEE ALSO}
acorr, levdur, lpc, lpc2par
\end{qsection}
