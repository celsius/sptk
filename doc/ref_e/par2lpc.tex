% ----------------------------------------------------------------- %
%             The Speech Signal Processing Toolkit (SPTK)           %
%             developed by SPTK Working Group                       %
%             http://sp-tk.sourceforge.net/                         %
% ----------------------------------------------------------------- %
%                                                                   %
%  Copyright (c) 1984-2007  Tokyo Institute of Technology           %
%                           Interdisciplinary Graduate School of    %
%                           Science and Engineering                 %
%                                                                   %
%                1996-2011  Nagoya Institute of Technology          %
%                           Department of Computer Science          %
%                                                                   %
% All rights reserved.                                              %
%                                                                   %
% Redistribution and use in source and binary forms, with or        %
% without modification, are permitted provided that the following   %
% conditions are met:                                               %
%                                                                   %
% - Redistributions of source code must retain the above copyright  %
%   notice, this list of conditions and the following disclaimer.   %
% - Redistributions in binary form must reproduce the above         %
%   copyright notice, this list of conditions and the following     %
%   disclaimer in the documentation and/or other materials provided %
%   with the distribution.                                          %
% - Neither the name of the SPTK working group nor the names of its %
%   contributors may be used to endorse or promote products derived %
%   from this software without specific prior written permission.   %
%                                                                   %
% THIS SOFTWARE IS PROVIDED BY THE COPYRIGHT HOLDERS AND            %
% CONTRIBUTORS "AS IS" AND ANY EXPRESS OR IMPLIED WARRANTIES,       %
% INCLUDING, BUT NOT LIMITED TO, THE IMPLIED WARRANTIES OF          %
% MERCHANTABILITY AND FITNESS FOR A PARTICULAR PURPOSE ARE          %
% DISCLAIMED. IN NO EVENT SHALL THE COPYRIGHT OWNER OR CONTRIBUTORS %
% BE LIABLE FOR ANY DIRECT, INDIRECT, INCIDENTAL, SPECIAL,          %
% EXEMPLARY, OR CONSEQUENTIAL DAMAGES (INCLUDING, BUT NOT LIMITED   %
% TO, PROCUREMENT OF SUBSTITUTE GOODS OR SERVICES; LOSS OF USE,     %
% DATA, OR PROFITS; OR BUSINESS INTERRUPTION) HOWEVER CAUSED AND ON %
% ANY THEORY OF LIABILITY, WHETHER IN CONTRACT, STRICT LIABILITY,   %
% OR TORT (INCLUDING NEGLIGENCE OR OTHERWISE) ARISING IN ANY WAY    %
% OUT OF THE USE OF THIS SOFTWARE, EVEN IF ADVISED OF THE           %
% POSSIBILITY OF SUCH DAMAGE.                                       %
% ----------------------------------------------------------------- %
\hypertarget{par2lpc}{}
\name{par2lpc}{transform PARCOR to LPC}{speech parameter transformation}

\begin{synopsis}
\item [par2lpc] [ --m $M$ ] [ {\em infile} ] 
\end{synopsis}

\begin{qsection}{DESCRIPTION}
{\em par2lpc} calculates linear prediction (LPC) coefficients 
from $M$-th order PARCOR coefficients from {\em infile} (or standard input), 
sending the result to standard output.

The PARCOR input format is
\begin{displaymath}
  K, k(1),\dots, k(M), 
\end{displaymath}
and the LPC output format is
\begin{displaymath}
  K, a(1),\dots, a(M).
\end{displaymath}

Input and output data are in float format.

The transformation of PARCOR coefficients into 
linear prediction coefficients is undertaken 
by a part of Durbin algorithm as follows.
\begin{align} 
a^{(m)}(m) &= k(m) \notag \\
a^{(m)}(i) &= a^{(m-1)}(i) + k(m) a^{(m-1)}(m-i), \qquad 1\leq i \leq m \notag
\end{align}
where  $m=1, 2, \dots, p$.
The initial condition is 
\begin{displaymath}
a^{(M)}(m) = a(m), \qquad 1 \leq m \leq M.
\end{displaymath}
\end{qsection}

\begin{options}
	\argm{m}{M}{order of LPC}{25}
\end{options}

\begin{qsection}{EXAMPLE}
PARCOR coefficients are read in float format from {\em data.rc}
and converted into the corresponding linear prediction coefficients.
The output is written to {\em data.lpc}:
\begin{quote}
 \verb!par2lpc < data.rc > data.lpc!
\end{quote} 
\end{qsection}

\begin{qsection}{SEE ALSO}
\hyperlink{acorr}{acorr},
\hyperlink{levdur}{levdur},
\hyperlink{lpc}{lpc},
\hyperlink{lpc2par}{lpc2par}
\end{qsection}
