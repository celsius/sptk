\name{par2lpc}{transform PARCOR to LPC}{speech parameter transformation}

\begin{synopsis}
\item [par2lpc] [ --m $M$ ] [ {\em infile} ] 
\end{synopsis}

\begin{qsection}{DESCRIPTION}
This command evaluates linear prediction coefficients
from PARCOR coefficients.
It is reads $M$ order PARCOR coefficients
\begin{displaymath}
  K, k(1),\ldots, k(M)
\end{displaymath}
evaluates the corresponding linear predicition coefficients
\begin{displaymath}
  K, a(1),\ldots, a(M)
\end{displaymath}
and send the reaults to the standard output.
\par
Input and output data are in float format.
\par
The transformation of PARCOR coefficients into 
linear prediction coefficients is undertaken 
by a part of Durbin algorithm as follows.
\begin{eqnarray*} 
a^{(m)}(m) &=& k(m) \\
a^{(m)}(i) &=& a^{(m-1)}(i) + k(m) a^{(m-1)}(m-i), ~~~~~1\leq i \leq m
\end{eqnarray*}
where  $m=1, 2, \ldots, p$.
The initial condition is 
\begin{displaymath}
a^{(M)}(m) = a(m), ~~~~~1 \leq m \leq M.
\end{displaymath}
\end{qsection}

\begin{options}
	\argm{m}{M}{order of LPC}{25}
\end{options}

\begin{qsection}{EXAMPLE}
PARCOR coefficients are read in float format from {\em data.rc}
and converted into the corresponding linear predition coefficients.
The output is written to {\em data.lpc}:
\begin{quote}
 \verb!par2lpc < data.rc > data.lpc!
\end{quote} 
\end{qsection}

\begin{qsection}{SEE ALSO}
acorr, levdur, lpc, lpc2par
\end{qsection}
