% ----------------------------------------------------------------
%       Speech Signal Processing Toolkit (SPTK): version 3.0
%                      SPTK Working Group
% 
%                Department of Computer Science
%                Nagoya Institute of Technology
%                             and
%   Interdisciplinary Graduate School of Science and Engineering
%                Tokyo Institute of Technology
%                   Copyright (c) 1984-2000
%                     All Rights Reserved.
% 
% Permission is hereby granted, free of charge, to use and
% distribute this software and its documentation without
% restriction, including without limitation the rights to use,
% copy, modify, merge, publish, distribute, sublicense, and/or
% sell copies of this work, and to permit persons to whom this
% work is furnished to do so, subject to the following conditions:
% 
%   1. The code must retain the above copyright notice, this list
%      of conditions and the following disclaimer.
% 
%   2. Any modifications must be clearly marked as such.
%                                                                        
% NAGOYA INSTITUTE OF TECHNOLOGY, TOKYO INSITITUTE OF TECHNOLOGY,
% SPTK WORKING GROUP, AND THE CONTRIBUTORS TO THIS WORK DISCLAIM
% ALL WARRANTIES WITH REGARD TO THIS SOFTWARE, INCLUDING ALL
% IMPLIED WARRANTIES OF MERCHANTABILITY AND FITNESS, IN NO EVENT
% SHALL NAGOYA INSTITUTE OF TECHNOLOGY, TOKYO INSITITUTE OF
% TECHNOLOGY, SPTK WORKING GROUP, NOR THE CONTRIBUTORS BE LIABLE
% FOR ANY SPECIAL, INDIRECT OR CONSEQUENTIAL DAMAGES OR ANY
% DAMAGES WHATSOEVER RESULTING FROM LOSS OF USE, DATA OR PROFITS,
% WHETHER IN AN ACTION OF CONTRACT, NEGLIGENCE OR OTHER TORTIOUS
% ACTION, ARISING OUT OF OR IN CONNECTION WITH THE USE OR
% PERFORMANCE OF THIS SOFTWARE.
% ----------------------------------------------------------------
%
\hypertarget{vopr}{}
\name{vopr}{execute vector operations}{number operation}

\begin{synopsis}
\item[vopr] [ --l $L$ ] [ --n $N$ ] [ --i ] [ --a ] [ --s ] [ --m ] [ --d ] 
\item[\ ~~~~] [ --ATAN2 ][ {\em file1} ] [ {\em infile} ]
\end{synopsis}

\begin{qsection}{DESCRIPTION}
This command undertakes vector operations in input files.
In other words
\begin{description}
\itemb{\em file1}
first vector file (if it is not assigned then stdin)
\itemb{\em infile}
second vector file (if it is not assigned then stdin)
\end{description}
the first file gives the operation vectors {\bf a}
and the second file gives the operation vectors {\bf b}.
The assigned operation is undertaken and the results
are sent to the standard output.
\par
Input and output data are in float format.
\par
The undertaken action depends on the number of assigned files
as well as the vector lengths as exemplified in the following.
\par
If two files are assigned (when only one file is assigned
then it is assumes that it corresponds to {\em infile}) then
depending on the values of vector sizes, the following actions
are undertaken.
\begin{description}
\item{when $L=1$}~\\
\begin{tabular}{l|c|c|c|c|c} \cline{2-6}
{\em file1} (stdin)	& {$a_1$} & {$a_2$} & {\dots}
			& {$a_i$} & {\dots} \\ \cline{2-6}
\multicolumn{6}{c}{}	\\[-10pt] \cline{2-6}
{\em infile}		& {$b_1$} & {$b_2$} & {\dots}
			& {$b_i$} & {\dots} \\ \cline{2-6}
\multicolumn{6}{c}{}	\\[-10pt] \cline{2-6}
{\em Output} (stdout)	& {$y_1$} & {$y_2$} & {\dots}
			& {$y_i$} & {\dots} \\ \cline{2-6}
\end{tabular}
\par
One data from one file corresponds to one data on the other file.
\item{when $L\geq 2$}~\\
\begin{tabular}{l|c|c|c|l} \cline{2-5}
{\em file1} (stdin)	& {$a_{11}$,\dots,$a_{1L}$}
			& {$a_{21}$,\dots,$a_{2L}$}
			& {$a_{31}$,\dots,$a_{3L}$}
			& {$a_{41}$,\dots} \\ \cline{2-5}
\multicolumn{5}{c}{}	\\[-10pt]
			\cline{2-2}
{\em infile}		& {$b_{1}$,\dots,$b_{L}$}
			& \multicolumn{3}{c}{} \\ \cline{2-2}
\multicolumn{5}{c}{}	\\[-10pt]
			\cline{2-5}			
{\em Output} (stdout)	& {$y_{11}$,\dots,$y_{1L}$}
			& {$y_{21}$,\dots,$y_{2L}$}
			& {$y_{31}$,\dots,$y_{3L}$}
			& {$y_{41}$,\dots} \\ \cline{2-5}
\end{tabular}
\par
In this case the operation vector is read only once
{\em infile}, and the operations recursively undertaken
in vectors from {\em file1}.
\end{description}
\par
When the information related to {\bf a} and {\bf b} is contained
in a single file,
(that is, if only one file is assigned,
or if no file assignment is made),
then the --i option should be used
and the action does not depend on the vector length.
\begin{description}
\item{when $L\geq 1$}~\\
\begin{tabular}{l|c|c|c|c|l} \cline{2-6}
{\em file} (stdin)	& {$a_{11}$,\dots,$a_{1L}$}
			& {$b_{11}$,\dots,$b_{1L}$}
			& {$a_{21}$,\dots,$a_{2L}$}
			& {$b_{21}$,\dots,$b_{2L}$}
			& ~~~ \\ \cline{2-6}
%			& {$a_{31}$,\dots} \\ \cline{2-6}
\multicolumn{6}{c}{}	\\[-10pt]
			\cline{2-2} \cline{4-4} \cline{6-6}
{\em Output} (stdout)	& {$y_{11}$,\dots,$y_{1L}$} &
			& {$y_{21}$,\dots,$y_{2L}$} &
			& ~~~ \\
%			& {$y_{31}$,\dots} \\
			\cline{2-2} \cline{4-4} \cline{6-6}
\end{tabular}
\par
Input vectors are read from a single file.
\end{description}
\end{qsection}

\begin{options}
	\argm{l}{L}{length of vector}{1}
	\argm{n}{N}{order of vector}{L-1}
	\argm{i}{}{when only a file is specified, 
                   the file contains a and b.}{FALSE}
	\argm{a}{}{addition $y_i=a_i+b_i$}{FALSE}
	\argm{s}{}{subtraction $y_i=a_i-b_i$}{FALSE}
	\argm{m}{}{multiplication $y_i=a_i*b_i$}{FALSE}
	\argm{d}{}{division $y_i=a_i/b_i$}{FALSE}
	\argm{ATAN2}{}{atan2 $y_i=\atan2(b_i,a_i)$}{FALSE}
\end{options}

\begin{qsection}{EXAMPLE}
The output file {\em data.c} contains addition of
vectors in float format read from {\em data.a} and {\em data.b}:
\begin{quote}
  \verb!vopr -a data.a data.b > data.c !
\end{quote}
\par
In the following example, a sin wave is passed through
a window with length 256 and coefficients given from
{\em data.w}:
\begin{quote}
  \verb!sin -t 30 -l 1000 | vopr data.w -l 256 -m | fdrw | xgr!
\end{quote}
The above example can be undertaken in similar way
through the example below if the contents of {\em data.w} corresponds
to Blackman window:
\begin{quote}
  \verb!sin -t 30 -l 1000 | window | fdrw | xgr!
\end{quote}
\end{qsection}

\begin{qsection}{SEE ALSO}
\hyperlink{sopr}{sopr},
\hyperlink{vsum}{vsum}
\end{qsection}
