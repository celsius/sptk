\name{msvq}{multi stage vector quantization}{vector quantization}

\begin{synopsis}
\item [msvq] [ --l $L$ ] [ --n $N$ ][ --s $S \;$ {\em cbfile} ] [ --q ] [ {\em infile} ]
\end{synopsis}

\begin{qsection}{DESCRIPTION}
This command undertakes multi-stage vactor quantization in
assinged file. 
The -s option assigns the number of stage levels.
\par
Input data is in float format and output data is in int format.
\end{qsection}

\begin{options}
	\argm{l}{L}{length of vector}{26}
        \argm{n}{N}{order of vector}{$L-1$}
	\argm{s}{S \; cbfile}{codebook\\
		\begin{tabular}{ll}\\ [-1zh]
		$S$ & codebook size\\
		$cbfile$ & codebook file\\
		\end{tabular}\\\hspace*{\fill}}{N/A N/A}
	\argm{q}{}{output quantized vector}{FALSE}
\end{options}

\begin{qsection}{EXAMPLE}
In the example below, a two level vq is undertaken in input {\em data.f}
file. the codebook sizes of 
{\em cbfile1} and {\em cbfile2} are 256 and the output is written
to {\em data.vq}:
\begin{quote}
\verb! msvq -s 256 cbfile1 -s 256 cbfile2 < data.f > data.vq!
\end{quote} 
\end{qsection}

\begin{qsection}{SEE ALSO}
 imsvq, vq, ivq, lbg
\end{qsection}
