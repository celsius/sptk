% ----------------------------------------------------------------
%       Speech Signal Processing Toolkit (SPTK): version 3.0
%                      SPTK Working Group
% 
%                Department of Computer Science
%                Nagoya Institute of Technology
%                             and
%   Interdisciplinary Graduate School of Science and Engineering
%                Tokyo Institute of Technology
%                   Copyright (c) 1984-2000
%                     All Rights Reserved.
% 
% Permission is hereby granted, free of charge, to use and
% distribute this software and its documentation without
% restriction, including without limitation the rights to use,
% copy, modify, merge, publish, distribute, sublicense, and/or
% sell copies of this work, and to permit persons to whom this
% work is furnished to do so, subject to the following conditions:
% 
%   1. The code must retain the above copyright notice, this list
%      of conditions and the following disclaimer.
% 
%   2. Any modifications must be clearly marked as such.
%                                                                        
% NAGOYA INSTITUTE OF TECHNOLOGY, TOKYO INSITITUTE OF TECHNOLOGY,
% SPTK WORKING GROUP, AND THE CONTRIBUTORS TO THIS WORK DISCLAIM
% ALL WARRANTIES WITH REGARD TO THIS SOFTWARE, INCLUDING ALL
% IMPLIED WARRANTIES OF MERCHANTABILITY AND FITNESS, IN NO EVENT
% SHALL NAGOYA INSTITUTE OF TECHNOLOGY, TOKYO INSITITUTE OF
% TECHNOLOGY, SPTK WORKING GROUP, NOR THE CONTRIBUTORS BE LIABLE
% FOR ANY SPECIAL, INDIRECT OR CONSEQUENTIAL DAMAGES OR ANY
% DAMAGES WHATSOEVER RESULTING FROM LOSS OF USE, DATA OR PROFITS,
% WHETHER IN AN ACTION OF CONTRACT, NEGLIGENCE OR OTHER TORTIOUS
% ACTION, ARISING OUT OF OR IN CONNECTION WITH THE USE OR
% PERFORMANCE OF THIS SOFTWARE.
% ----------------------------------------------------------------
%
\name{msvq}{multi stage vector quantization}{vector quantization}

\begin{synopsis}
\item [msvq] [ --l $L$ ] [ --n $N$ ][ --s $S \;$ {\em cbfile} ] [ --q ] [ {\em infile} ]
\end{synopsis}

\begin{qsection}{DESCRIPTION}
{\em msvq} encodes the data from {\em infile} (or standard input) 
using multi-stage vector quantization 
with codebooks specified by multiple --s options,
sending the result to standard output.

Input data is in float format and output data is in int format.
\end{qsection}

\begin{options}
	\argm{l}{L}{length of vector}{26}
        \argm{n}{N}{order of vector}{$L-1$}
	\argm{s}{S \; cbfile}{codebook\\
		\begin{tabular}{ll}\\ [-1ex]
		$S$ & codebook size\\
		$cbfile$ & codebook file\\
		\end{tabular}\\\hspace*{\fill}}{N/A N/A}
	\argm{q}{}{output quantized vector}{FALSE}
\end{options}

\begin{qsection}{EXAMPLE}
In the example below, a two level vq is undertaken in input {\em data.f}
file. the codebook sizes of 
{\em cbfile1} and {\em cbfile2} are 256 and the output is written
to {\em data.vq}:
\begin{quote}
\verb! msvq -s 256 cbfile1 -s 256 cbfile2 < data.f > data.vq!
\end{quote} 
\end{qsection}

\begin{qsection}{SEE ALSO}
 imsvq, vq, ivq, lbg
\end{qsection}
