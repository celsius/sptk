% ----------------------------------------------------------------- %
%             The Speech Signal Processing Toolkit (SPTK)           %
%             developed by SPTK Working Group                       %
%             http://sp-tk.sourceforge.net/                         %
% ----------------------------------------------------------------- %
%                                                                   %
%  Copyright (c) 1984-2007  Tokyo Institute of Technology           %
%                           Interdisciplinary Graduate School of    %
%                           Science and Engineering                 %
%                                                                   %
%                1996-2014  Nagoya Institute of Technology          %
%                           Department of Computer Science          %
%                                                                   %
% All rights reserved.                                              %
%                                                                   %
% Redistribution and use in source and binary forms, with or        %
% without modification, are permitted provided that the following   %
% conditions are met:                                               %
%                                                                   %
% - Redistributions of source code must retain the above copyright  %
%   notice, this list of conditions and the following disclaimer.   %
% - Redistributions in binary form must reproduce the above         %
%   copyright notice, this list of conditions and the following     %
%   disclaimer in the documentation and/or other materials provided %
%   with the distribution.                                          %
% - Neither the name of the SPTK working group nor the names of its %
%   contributors may be used to endorse or promote products derived %
%   from this software without specific prior written permission.   %
%                                                                   %
% THIS SOFTWARE IS PROVIDED BY THE COPYRIGHT HOLDERS AND            %
% CONTRIBUTORS "AS IS" AND ANY EXPRESS OR IMPLIED WARRANTIES,       %
% INCLUDING, BUT NOT LIMITED TO, THE IMPLIED WARRANTIES OF          %
% MERCHANTABILITY AND FITNESS FOR A PARTICULAR PURPOSE ARE          %
% DISCLAIMED. IN NO EVENT SHALL THE COPYRIGHT OWNER OR CONTRIBUTORS %
% BE LIABLE FOR ANY DIRECT, INDIRECT, INCIDENTAL, SPECIAL,          %
% EXEMPLARY, OR CONSEQUENTIAL DAMAGES (INCLUDING, BUT NOT LIMITED   %
% TO, PROCUREMENT OF SUBSTITUTE GOODS OR SERVICES; LOSS OF USE,     %
% DATA, OR PROFITS; OR BUSINESS INTERRUPTION) HOWEVER CAUSED AND ON %
% ANY THEORY OF LIABILITY, WHETHER IN CONTRACT, STRICT LIABILITY,   %
% OR TORT (INCLUDING NEGLIGENCE OR OTHERWISE) ARISING IN ANY WAY    %
% OUT OF THE USE OF THIS SOFTWARE, EVEN IF ADVISED OF THE           %
% POSSIBILITY OF SUCH DAMAGE.                                       %
% ----------------------------------------------------------------- %
\hypertarget{minmax}{}
\name{minmax}{find minimum and maximum values}{data operation}

\begin{synopsis}
 \item [minmax] [ --l $L$ ] [ --n $N$ ] [ --b $B$ ] [ --o $O$ ] [ --d ] [ {\em infile} ]
\end{synopsis}

\begin{qsection}{DESCRIPTION}
{\em minmax} determines the $B$ (default 1) minimum and maximum values, 
on a frame-by-frame basis, 
of the data from {\em infile} (or standard input), 
sending the result to standard output.
If the frame length $L$ is 1, 
each input number is considered to be both
the minimum and maximum value for its length-1 frame.

The input format is float by default.
If the --d option is not given, 
the output format will also be float,
consisting of the minimum and maximum values.
If the --d option is given,
the output format will be ASCII,
showing the positions within the frame 
where the minimum and maximum values occurred, as follows:
\begin{displaymath}
value:position_0,position_1,\dots
\end{displaymath}
Also, when specifying --o 0, --o 1, and --o 2,
{\em minmax} output minimum and maximum values, only minimum values, and
only maximum values, respectively.
\end{qsection}

\begin{options}
        \argm{l}{L}{length of vector}{1}
        \argm{n}{N}{order of vector}{L-1}
        \argm{b}{B}{find n-best values}{1}
        \argm{o}{O}{output format \\
                   \begin{tabular}{lll} \\[-1ex]
                    $O=0$ & ~~~minimum and maximum\\
                    $O=1$ & ~~~minimum\\
                    $O=2$ & ~~~maximum\\
                   \end{tabular} \\\hspace*{\fill}}{0}
        \argm{d}{}{output data number}{FALSE}
\end{options}

\begin{qsection}{EXAMPLE}
If, for example, the input data in {\em data.f} in float format
is given as
\[1,1,2,3,4,5,6,7,8,9,9,10\]
then the output of the following command
\begin{quote}
 \verb!minmax data.f -l 6 > data.m!
\end{quote}
is written to {\em data.m} as
\[1,5,6,10\]
Also, if the following command is applied
\begin{quote}
 \verb!minmax -n 2 -d data.f!
\end{quote}
then the result will be
\begin{quote}
 \verb!1:0!\\
 \verb!2:2!\\
 \verb!3:0!\\
 \verb!5:2!\\
 \verb!6:0!\\
 \verb!8:2!\\
 \verb!9:0,1!\\
 \verb!10:2!
\end{quote}
\end{qsection}
