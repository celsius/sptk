% ----------------------------------------------------------------
%       Speech Signal Processing Toolkit (SPTK): version 3.0
%                      SPTK Working Group
% 
%                Department of Computer Science
%                Nagoya Institute of Technology
%                             and
%   Interdisciplinary Graduate School of Science and Engineering
%                Tokyo Institute of Technology
%                   Copyright (c) 1984-2000
%                     All Rights Reserved.
% 
% Permission is hereby granted, free of charge, to use and
% distribute this software and its documentation without
% restriction, including without limitation the rights to use,
% copy, modify, merge, publish, distribute, sublicense, and/or
% sell copies of this work, and to permit persons to whom this
% work is furnished to do so, subject to the following conditions:
% 
%   1. The code must retain the above copyright notice, this list
%      of conditions and the following disclaimer.
% 
%   2. Any modifications must be clearly marked as such.
%                                                                        
% NAGOYA INSTITUTE OF TECHNOLOGY, TOKYO INSITITUTE OF TECHNOLOGY,
% SPTK WORKING GROUP, AND THE CONTRIBUTORS TO THIS WORK DISCLAIM
% ALL WARRANTIES WITH REGARD TO THIS SOFTWARE, INCLUDING ALL
% IMPLIED WARRANTIES OF MERCHANTABILITY AND FITNESS, IN NO EVENT
% SHALL NAGOYA INSTITUTE OF TECHNOLOGY, TOKYO INSITITUTE OF
% TECHNOLOGY, SPTK WORKING GROUP, NOR THE CONTRIBUTORS BE LIABLE
% FOR ANY SPECIAL, INDIRECT OR CONSEQUENTIAL DAMAGES OR ANY
% DAMAGES WHATSOEVER RESULTING FROM LOSS OF USE, DATA OR PROFITS,
% WHETHER IN AN ACTION OF CONTRACT, NEGLIGENCE OR OTHER TORTIOUS
% ACTION, ARISING OUT OF OR IN CONNECTION WITH THE USE OR
% PERFORMANCE OF THIS SOFTWARE.
% ----------------------------------------------------------------
%
\name{minmax}{find minimum and maximum values}{data operation}

\begin{synopsis}
 \item [minmax] [ --l $L$ ] [ --n $N$ ] [ --b $B$ ] [ --d ] [ {\em infile} ]
\end{synopsis}

\begin{qsection}{DESCRIPTION}
{\em minmax} determines the $B$ (default 1) minimum and maximum values, 
on a frame-by-frame basis, 
of the data from {\em infile} (or standard input), 
sending the result to standard output.

If the frame length $L$ is 1, 
each input number is considered to be both
the minimum and maximum value for its length-1 frame.

The input format is float. 
If the --d option is not given, 
the output format is float, 
consisting of the minimum and maximum values.
If the --d option is give, 
the output format is ASCII, 
showing the positions within the frame 
where the minimum and maximum values occurred, as follows:
\begin{displaymath}
value:position_0,position_1,\dots
\end{displaymath}
\end{qsection}

\begin{options}
	\argm{l}{L}{length of vector}{1}
	\argm{n}{N}{order of vector}{L-1}
	\argm{b}{B}{find n-best values}{1}
	\argm{d}{}{output data number}{FALSE}
\end{options}

\begin{qsection}{EXAMPLE}
If, for example, the input data in {\em data.f} in float format
is as follows
\[1,1,2,3,4,5,6,7,8,9,9,10\],
then the output of the following command
\begin{quote}
 \verb!minmax data.f -l 6 > data.m!
\end{quote}
is written to {\em data.m} as follows.
\[1,5,6,10\]
Also if the following command is applied
\begin{quote}
 \verb!minmax -n 2 -d data.f!
\end{quote}
then the result is 
\begin{quote}
 \verb!1:0,17!\\
 \verb!2:2!\\
 \verb!3:0!\\
 \verb!5:2!\\
 \verb!6:0!\\
 \verb!8:2!\\
 \verb!9:0,1!\\
 \verb!10:2!
\end{quote}
\end{qsection}
