\name{minmax}{find minimum and maximum values}{data operation}

\begin{synopsis}
 \item [minmax] [ --l $L$ ] [ --n $N$ ] [ --b $B$ ] [ --d ] [ {\em infile} ]
\end{synopsis}

\begin{qsection}{DESCRIPTION}
This command reads data from {\em infile}(or the standard input if
{\em infile is omitted}), and for every frame minimum and maximum
values are sent to the standard output.
It can output also the $B$ maximum and/or minimum values.

If frame length $L$ is equal to 1, then the whole data in file
is outputed as maximum and minimum.

The input data is read in float format,
and the output data is written in ASCII format
if the position number is desired or in float format
if the value is desired.
When the data position number is desired then
the output is written following the representation 
with the $n$ minimum values and
$n$ maximum values, as follows.
\begin{displaymath}
value:position_0,position_1,\ldots
\end{displaymath}
\end{qsection}

\begin{options}
	\argm{l}{L}{length of vector}{1}
	\argm{n}{N}{order of vector}{L-1}
	\argm{b}{B}{find n-best values}{1}
	\argm{d}{}{output data number}{FALSE}
\end{options}

\begin{qsection}{EXAMPLE}
If, for example, the input data in {\em data.f} in float format
is as follows
\[1,1,2,3,4,5,6,7,8,9,9,10\],
then the output of the following command
\begin{quote}
 \verb!minmax data.f -l 6 > data.m!
\end{quote}
is written to {\em data.m} as follows.
\[1,5,6,10\]
Also if the following command is applied
\begin{quote}
 \verb!minmax -n 2 -d data.f!
\end{quote}
then the result is 
\begin{quote}
 \verb!1:0,17!\\
 \verb!2:2!\\
 \verb!3:0!\\
 \verb!5:2!\\
 \verb!6:0!\\
 \verb!8:2!\\
 \verb!9:0,1!\\
 \verb!10:2!
\end{quote}
\end{qsection}
