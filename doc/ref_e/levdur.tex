%  ---------------------------------------------------------------  %
%            Speech Signal Processing Toolkit (SPTK)                %
%                      SPTK Working Group                           %
%                                                                   %
%                  Department of Computer Science                   %
%                  Nagoya Institute of Technology                   %
%                               and                                 %
%   Interdisciplinary Graduate School of Science and Engineering    %
%                  Tokyo Institute of Technology                    %
%                                                                   %
%                     Copyright (c) 1984-2007                       %
%                       All Rights Reserved.                        %
%                                                                   %
%  Permission is hereby granted, free of charge, to use and         %
%  distribute this software and its documentation without           %
%  restriction, including without limitation the rights to use,     %
%  copy, modify, merge, publish, distribute, sublicense, and/or     %
%  sell copies of this work, and to permit persons to whom this     %
%  work is furnished to do so, subject to the following conditions: %
%                                                                   %
%    1. The source code must retain the above copyright notice,     %
%       this list of conditions and the following disclaimer.       %
%                                                                   %
%    2. Any modifications to the source code must be clearly        %
%       marked as such.                                             %
%                                                                   %
%    3. Redistributions in binary form must reproduce the above     %
%       copyright notice, this list of conditions and the           %
%       following disclaimer in the documentation and/or other      %
%       materials provided with the distribution.  Otherwise, one   %
%       must contact the SPTK working group.                        %
%                                                                   %
%  NAGOYA INSTITUTE OF TECHNOLOGY, TOKYO INSTITUTE OF TECHNOLOGY,   %
%  SPTK WORKING GROUP, AND THE CONTRIBUTORS TO THIS WORK DISCLAIM   %
%  ALL WARRANTIES WITH REGARD TO THIS SOFTWARE, INCLUDING ALL       %
%  IMPLIED WARRANTIES OF MERCHANTABILITY AND FITNESS, IN NO EVENT   %
%  SHALL NAGOYA INSTITUTE OF TECHNOLOGY, TOKYO INSTITUTE OF         %
%  TECHNOLOGY, SPTK WORKING GROUP, NOR THE CONTRIBUTORS BE LIABLE   %
%  FOR ANY SPECIAL, INDIRECT OR CONSEQUENTIAL DAMAGES OR ANY        %
%  DAMAGES WHATSOEVER RESULTING FROM LOSS OF USE, DATA OR PROFITS,  %
%  WHETHER IN AN ACTION OF CONTRACT, NEGLIGENCE OR OTHER TORTUOUS   %
%  ACTION, ARISING OUT OF OR IN CONNECTION WITH THE USE OR          %
%  PERFORMANCE OF THIS SOFTWARE.                                    %
%                                                                   %
%  ---------------------------------------------------------------  %
%
\hypertarget{levdur}{}
\name{levdur}{solve an autocorrelation normal equation 
                    using Levinson-Durbin method}{signal processing}


\begin{synopsis}
 \item [levdur] [ --m $M$ ] [ {\em infile} ] 
\end{synopsis}

\begin{qsection}{DESCRIPTION}
{\em levdur} calculates linear prediction coefficients (LPC) 
from the autocorrelation matrix from {\em infile} (or standard input), 
sending the result to standard output.

The input is the $M$-th order autocorrelation matrix
\begin{displaymath}
  r(0),r(1),\dots,r(M).
\end{displaymath}
{\em levdur} uses the Levinson-Durbin algorithm 
to solve a system of linear equations
obtained from the autocorrelation matrix.

Input and output data are in float format.
\par
The linear prediction coefficients are the set of coefficients
$K, a(1), \dots, a(M)$ of the all-pole digital filter
\begin{displaymath}
H(z) = \frac{K}{\displaystyle{1+\sum_{i=1}^{M}a(k)z^{-i}}}.
\end{displaymath}
The linear prediction coefficients are evaluated by solving
the following set of linear equations, which were obtained
through the autocorrelation method,
\begin{displaymath}
\begin{pmatrix}
        r(0) & r(1) & \dots & r(M-1) \\
        r(1) & r(0) &        & \vdots \\
        \vdots &    & \ddots &         \\
        r(M-1) &    & \dots & r(0)   \\
\end{pmatrix}
\begin{pmatrix}
	a(1) \\
	a(2) \\
	\vdots \\
	a(M) \\
\end{pmatrix}
= - 
\begin{pmatrix}
	r(1) \\
	r(2) \\
	\vdots \\
	r(M) \\
\end{pmatrix}
\end{displaymath}
The Durbin iterative and efficient algorithm is used
in the following taking advantage of the Toeplitz characteristic
of the autocorrelation matrix:
\begin{align}
E^{(0)}    &= r(0) \notag \\
k(i)       &= \frac{\displaystyle{-r(i)-\sum_{j=1}^i a^{(i-1)}(j)r(i-j)}}
		{E^{(i-1)}} \label{eqn:lev_dur_k}\notag\\
a^{(i)}(i) &= k(i) \notag\\
a^{(i)}(j) &=  a^{(i-1)}(j) + k(i) a^{(i-1)}(i-j), 
		\qquad 1\leq j \leq i-1\\
E^{(i)}    &= (1-k^2(i)) E^{(i-1)} \label{eqn:lev_dur_E}
\end{align}
Also, for $i=1,2,\ldots,M$, equations (\ref{eqn:lev_dur_k}) to
 (\ref{eqn:lev_dur_E}) are applied recursively,
and the gain $K$ is calculated as follows.
\begin{displaymath}
K = \sqrt{E^{(M)}}
\end{displaymath}
\end{qsection}

\begin{options}
	\argm{m}{M}{order of correlation}{25}
\end{options}

\begin{qsection}{EXAMPLE}
In this example, input data is read in float format from
{\em data.f} and linear prediction coefficients are written
to {\em data.lpc}:
\begin{quote}
 \verb!frame < data.f | window | acorr -m 25 | levdur > data.lpc!
\end{quote} 
\end{qsection}

\begin{qsection}{SEE ALSO}
\hyperlink{acorr}{acorr},
\hyperlink{lpc}{lpc}
\end{qsection}
