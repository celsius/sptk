% ----------------------------------------------------------------- %
%             The Speech Signal Processing Toolkit (SPTK)           %
%             developed by SPTK Working Group                       %
%             http://sp-tk.sourceforge.net/                         %
% ----------------------------------------------------------------- %
%                                                                   %
%  Copyright (c) 1984-2007  Tokyo Institute of Technology           %
%                           Interdisciplinary Graduate School of    %
%                           Science and Engineering                 %
%                                                                   %
%                1996-2016  Nagoya Institute of Technology          %
%                           Department of Computer Science          %
%                                                                   %
% All rights reserved.                                              %
%                                                                   %
% Redistribution and use in source and binary forms, with or        %
% without modification, are permitted provided that the following   %
% conditions are met:                                               %
%                                                                   %
% - Redistributions of source code must retain the above copyright  %
%   notice, this list of conditions and the following disclaimer.   %
% - Redistributions in binary form must reproduce the above         %
%   copyright notice, this list of conditions and the following     %
%   disclaimer in the documentation and/or other materials provided %
%   with the distribution.                                          %
% - Neither the name of the SPTK working group nor the names of its %
%   contributors may be used to endorse or promote products derived %
%   from this software without specific prior written permission.   %
%                                                                   %
% THIS SOFTWARE IS PROVIDED BY THE COPYRIGHT HOLDERS AND            %
% CONTRIBUTORS "AS IS" AND ANY EXPRESS OR IMPLIED WARRANTIES,       %
% INCLUDING, BUT NOT LIMITED TO, THE IMPLIED WARRANTIES OF          %
% MERCHANTABILITY AND FITNESS FOR A PARTICULAR PURPOSE ARE          %
% DISCLAIMED. IN NO EVENT SHALL THE COPYRIGHT OWNER OR CONTRIBUTORS %
% BE LIABLE FOR ANY DIRECT, INDIRECT, INCIDENTAL, SPECIAL,          %
% EXEMPLARY, OR CONSEQUENTIAL DAMAGES (INCLUDING, BUT NOT LIMITED   %
% TO, PROCUREMENT OF SUBSTITUTE GOODS OR SERVICES; LOSS OF USE,     %
% DATA, OR PROFITS; OR BUSINESS INTERRUPTION) HOWEVER CAUSED AND ON %
% ANY THEORY OF LIABILITY, WHETHER IN CONTRACT, STRICT LIABILITY,   %
% OR TORT (INCLUDING NEGLIGENCE OR OTHERWISE) ARISING IN ANY WAY    %
% OUT OF THE USE OF THIS SOFTWARE, EVEN IF ADVISED OF THE           %
% POSSIBILITY OF SUCH DAMAGE.                                       %
% ----------------------------------------------------------------- %
\hypertarget{levdur}{}
\name{levdur}{solve an autocorrelation normal equation 
                    using Levinson-Durbin method}{signal processing}


\begin{synopsis}
 \item [levdur] [ --m $M$ ] [ --f $F$ ] [ {\em infile} ] 
\end{synopsis}

\begin{qsection}{DESCRIPTION}
{\em levdur} calculates linear prediction coefficients (LPC) 
from the autocorrelation matrix from {\em infile} (or standard input), 
sending the result to standard output.

The input is the $M$-th order autocorrelation matrix
\begin{displaymath}
  r(0),r(1),\dots,r(M).
\end{displaymath}
{\em levdur} uses the Levinson-Durbin algorithm 
to solve a system of linear equations
obtained from the autocorrelation matrix.

Input and output data are in float format.
\par
The linear prediction coefficients are the set of coefficients
$K, a(1), \dots, a(M)$ of an all-pole digital filter
\begin{displaymath}
H(z) = \frac{K}{\displaystyle{1+\sum_{i=1}^{M}a(k)z^{-i}}}.
\end{displaymath}
The linear prediction coefficients are evaluated by solving
the following set of linear equations, which were obtained
through the autocorrelation method,
\begin{displaymath}
\begin{pmatrix}
        r(0) & r(1) & \dots & r(M-1) \\
        r(1) & r(0) &        & \vdots \\
        \vdots &    & \ddots &         \\
        r(M-1) &    & \dots & r(0)   \\
\end{pmatrix}
\begin{pmatrix}
	a(1) \\
	a(2) \\
	\vdots \\
	a(M) \\
\end{pmatrix}
= - 
\begin{pmatrix}
	r(1) \\
	r(2) \\
	\vdots \\
	r(M) \\
\end{pmatrix}
\end{displaymath}
The Durbin iterative and efficient algorithm is used
to solve the system above. It takes advantage of the Toeplitz characteristic
of the autocorrelation matrix:
\begin{align}
E^{(0)}    &= r(0) \notag \\
k(i)       &= \frac{\displaystyle{-r(i)-\sum_{j=1}^i a^{(i-1)}(j)r(i-j)}}
		{E^{(i-1)}} \label{eqn:lev_dur_k}\notag\\
a^{(i)}(i) &= k(i) \notag\\
a^{(i)}(j) &=  a^{(i-1)}(j) + k(i) a^{(i-1)}(i-j), 
		\qquad 1\leq j \leq i-1\\
E^{(i)}    &= (1-k^2(i)) E^{(i-1)} \label{eqn:lev_dur_E}
\end{align}
Also, for $i=1,2,\ldots,M$, equations (\ref{eqn:lev_dur_k}) and
 (\ref{eqn:lev_dur_E}) are applied recursively,
and the gain $K$ is calculated as follows.
\begin{displaymath}
K = \sqrt{E^{(M)}}
\end{displaymath}
\end{qsection}

\begin{options}
	\argm{m}{M}{order of correlation}{25}
	\argm{f}{F}{mimimum value of the determinant of the normal matrix}{0.0}
\end{options}

\begin{qsection}{EXAMPLE}
In this example, input data is read in float format from
{\em data.f} and linear prediction coefficients are written
to {\em data.lpc}:
\begin{quote}
 \verb!frame < data.f | window | acorr -m 25 | levdur > data.lpc!
\end{quote} 
\end{qsection}

\begin{qsection}{NOTICE}
The default value for -f option is zero as a trial. In the previous version, the 
default value is 1.0E-6.
\end{qsection}

\begin{qsection}{SEE ALSO}
\hyperlink{acorr}{acorr},
\hyperlink{lpc}{lpc}
\end{qsection}
