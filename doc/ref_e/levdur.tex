\name{levdur}{solve normal equation 
                    using Levinson-Durbin method}{signal processing}


\begin{synopsis}
 \item [levdur] [ --m $M$ ] [ {\em infile} ] 
\end{synopsis}

\begin{qsection}{DESCRIPTION}
This command evaluates the linear prediction coefficients
by solving a set of linear equations obtained from
the autocorrelation matrix.
From the assinged file, $M$ order autocorrelation matrix
\begin{displaymath}
  r(0),r(1),\ldots,r(M)
\end{displaymath}
is read and the linear system is solved by the Levinson-Durbin
algorithm.
\par
Input and output data are in float format.
\par
The linear prediction coefficients are the set of coefficients
$K, a(1), \ldots, a(M)$ of the all-pole digital filter
\begin{displaymath}
H(z) = \frac{K}{\displaystyle{1+\sum_{i=1}^{M}a(k)z^{-i}}}.
\end{displaymath}
The linear prediction coefficients are evaluated by solving
the following set of linear equations, which were obtained
through the autocorrelation method,
\begin{displaymath}
\left( \begin{array}{cccc}
        r(0) & r(1) & \cdots & r(M-1) \\
        r(1) & r(0) &        & \vdots \\
        \vdots &    & \ddots &         \\
        r(M-1) &    & \cdots & r(0)   \\
        \end{array} \right)
\left( \begin{array}{c}
	a(1) \\
	a(2) \\
	\vdots \\
	a(M) \\
	\end{array} \right)
= - \left( \begin{array}{c}
	r(1) \\
	r(2) \\
	\vdots \\
	r(M) \\
	\end{array} \right)
\end{displaymath}
The Durbin iterative and efficient algorithm is used
in the following taking advantage of the Toeplitz characteristic
of the autocorrelation matrix:
\begin{eqnarray}
E^{(0)}    &=& r(0) \\
k(i)       &=& \frac{\displaystyle{-r(i)-\sum_{j=1}^i a^{(i-1)}(j)r(i-j)}}
		{E^{(i-1)}} \label{eqn:lev_dur_k}\\
a^{(i)}(i) &=& k(i) \\
a^{(i)}(j) &=&  a^{(i-1)}(j) + k(i) a^{(i-1)}(i-j), 
		~~~~~1\leq j \leq i-1\\
E^{(i)}    &=& (1-k^2(i)) E^{(i-1)} \label{eqn:lev_dur_E}
\end{eqnarray}
Also, for $i=1,2,\ldots,M$, equations (\ref{eqn:lev_dur_k}) to
 (\ref{eqn:lev_dur_E}) are applied recursively,
and the gain $K$ is calculated as follows.
\begin{displaymath}
K = \sqrt{E^{(M)}}
\end{displaymath}
\end{qsection}

\begin{options}
	\argm{m}{M}{order of correlation}{25}
\end{options}

\begin{qsection}{EXAMPLE}
In this example, input data is read in float format from
{\em data.f} and linear prediction coefficients are written
to {\em data.lpc}:
\begin{quote}
 \verb!frame < data.f | window | acorr -m 25 | levdur > data.lpc!
\end{quote} 
\end{qsection}

\begin{qsection}{SEE ALSO}
 acorr, lpc
\end{qsection}
