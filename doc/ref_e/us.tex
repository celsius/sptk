% ----------------------------------------------------------------- %
%             The Speech Signal Processing Toolkit (SPTK)           %
%             developed by SPTK Working Group                       %
%             http://sp-tk.sourceforge.net/                         %
% ----------------------------------------------------------------- %
%                                                                   %
%  Copyright (c) 1984-2007  Tokyo Institute of Technology           %
%                           Interdisciplinary Graduate School of    %
%                           Science and Engineering                 %
%                                                                   %
%                1996-2017  Nagoya Institute of Technology          %
%                           Department of Computer Science          %
%                                                                   %
% All rights reserved.                                              %
%                                                                   %
% Redistribution and use in source and binary forms, with or        %
% without modification, are permitted provided that the following   %
% conditions are met:                                               %
%                                                                   %
% - Redistributions of source code must retain the above copyright  %
%   notice, this list of conditions and the following disclaimer.   %
% - Redistributions in binary form must reproduce the above         %
%   copyright notice, this list of conditions and the following     %
%   disclaimer in the documentation and/or other materials provided %
%   with the distribution.                                          %
% - Neither the name of the SPTK working group nor the names of its %
%   contributors may be used to endorse or promote products derived %
%   from this software without specific prior written permission.   %
%                                                                   %
% THIS SOFTWARE IS PROVIDED BY THE COPYRIGHT HOLDERS AND            %
% CONTRIBUTORS "AS IS" AND ANY EXPRESS OR IMPLIED WARRANTIES,       %
% INCLUDING, BUT NOT LIMITED TO, THE IMPLIED WARRANTIES OF          %
% MERCHANTABILITY AND FITNESS FOR A PARTICULAR PURPOSE ARE          %
% DISCLAIMED. IN NO EVENT SHALL THE COPYRIGHT OWNER OR CONTRIBUTORS %
% BE LIABLE FOR ANY DIRECT, INDIRECT, INCIDENTAL, SPECIAL,          %
% EXEMPLARY, OR CONSEQUENTIAL DAMAGES (INCLUDING, BUT NOT LIMITED   %
% TO, PROCUREMENT OF SUBSTITUTE GOODS OR SERVICES; LOSS OF USE,     %
% DATA, OR PROFITS; OR BUSINESS INTERRUPTION) HOWEVER CAUSED AND ON %
% ANY THEORY OF LIABILITY, WHETHER IN CONTRACT, STRICT LIABILITY,   %
% OR TORT (INCLUDING NEGLIGENCE OR OTHERWISE) ARISING IN ANY WAY    %
% OUT OF THE USE OF THIS SOFTWARE, EVEN IF ADVISED OF THE           %
% POSSIBILITY OF SUCH DAMAGE.                                       %
% ----------------------------------------------------------------- %
\hypertarget{us}{}
\name{us}{up-sampling}%
{sampling rate transformation}

\begin{synopsis}
\item [us] [ --s $S$ ] [ --c {\em file} ] [ --u $U$ ] [ --d $D$ ] [ {\em infile} ]
\end{synopsis}

\begin{qsection}{DESCRIPTION}
{\em us} up-samples data from {\em infile} (or standard input), 
sending the result to standard output.

The format of input and output data is float.
The following filter coefficients can be used.

\begin{tabular}{ll} \\[-1ex]
	$S=23F$ & \$SPTK/share/SPTK/lpfcoef.2to3f \\
	$S=23S$ & \$SPTK/share/SPTK/lpfcoef.2to3s \\
	$S=34$ & \$SPTK/share/SPTK/lpfcoef.3to4 \\
	$S=35$ & \$SPTK/share/SPTK/lpfcoef.3to5 \\
	$S=45$ & \$SPTK/share/SPTK/lpfcoef.4to5 \\
	$S=57$ & \$SPTK/share/SPTK/lpfcoef.5to7 \\
	$S=58$ & \$SPTK/share/SPTK/lpfcoef.5to8 \\
	$S=78$ & \$SPTK/share/SPTK/lpfcoef.7to8 \\
        &(\$SPTK is the directory where toolkit was installed.)
\end{tabular}

The ratio between up-sampling and down-sampling can be modified by
the {\bf --u} and {\bf --d} options respectively.
If you want to specify filter coefficients,
{\bf --c} should also be specified.

Filter coefficients are in ASCII format.

 For up-sampling from 10 or 12 to 16kHz,
 the \hyperlink{us16}{\em us16} command can be used.
 For up/down-sampling between 8, 10, 12 or and 11.025, 22.05 or 44.1 kHz,
 the \hyperlink{uscd}{\em uscd} command can be used.
 The \hyperlink{ds}{\em ds} command may also be used for down-sampling.
\end{qsection}

\begin{options}
	\argm{s}{S}{conversion type\\
		\begin{tabular}{ll} \\[-1ex]
			$S=23F$ & up-sampling by $2 : 3$\\
			$S=23S$ & up-sampling by $2 : 3$\\
			$S=34$ & up-sampling by $3 : 4$\\
			$S=34$ & up-sampling by $3 : 5$\\
			$S=45$ & up-sampling by $4 : 5$\\
			$S=57$ & up-sampling by $5 : 7$\\
			$S=58$ & up-sampling by $5 : 8$ \\
			$S=78$ & up-sampling by $7 : 8$
		\end{tabular}\\}{58}
	\argm{c}{\mbox{\em file}}{filename of low pass filter coefficients}{Default}
	\argm{u}{U}{up-sampling ratio}{N/A}
	\argm{d}{D}{down-sampling ratio}{N/A}
\end{options}

\begin{qsection}{EXAMPLE}
In this example, the speech data in the input file {\em data.16},
which was sampled at 16 kHz in short int format, is converted to
an 44.1 kHz sampling rate:
\begin{quote}
\verb!x2x +sf data.16 | us -s 23F | us -s 23S | us -s 57 | \! \\
\verb!us -c /usr/local/SPTK/lib/lpfcoef.5to7 -u 7 -d 8 | \! \\
\verb!x2x +fs > data.44! \\ [5mm]
Note:~~
$\displaystyle\frac{44100}{16000} = 
	\frac{3\times3\times7\times7\times100}{2\times2\times5\times8\times100}$
\end{quote}
\end{qsection}

%\begin{qsection}{BUGS}
%none
%\end{qsection}

\begin{qsection}{SEE ALSO}
 \hyperlink{ds}{ds},
 \hyperlink{uscd}{uscd},
 \hyperlink{us16}{us16}
\end{qsection}
