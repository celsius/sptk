%  ---------------------------------------------------------------  %
%            Speech Signal Processing Toolkit (SPTK)                %
%                      SPTK Working Group                           %
%                                                                   %
%                  Department of Computer Science                   %
%                  Nagoya Institute of Technology                   %
%                               and                                 %
%   Interdisciplinary Graduate School of Science and Engineering    %
%                  Tokyo Institute of Technology                    %
%                                                                   %
%                     Copyright (c) 1984-2007                       %
%                       All Rights Reserved.                        %
%                                                                   %
%  Permission is hereby granted, free of charge, to use and         %
%  distribute this software and its documentation without           %
%  restriction, including without limitation the rights to use,     %
%  copy, modify, merge, publish, distribute, sublicense, and/or     %
%  sell copies of this work, and to permit persons to whom this     %
%  work is furnished to do so, subject to the following conditions: %
%                                                                   %
%    1. The source code must retain the above copyright notice,     %
%       this list of conditions and the following disclaimer.       %
%                                                                   %
%    2. Any modifications to the source code must be clearly        %
%       marked as such.                                             %
%                                                                   %
%    3. Redistributions in binary form must reproduce the above     %
%       copyright notice, this list of conditions and the           %
%       following disclaimer in the documentation and/or other      %
%       materials provided with the distribution.  Otherwise, one   %
%       must contact the SPTK working group.                        %
%                                                                   %
%  NAGOYA INSTITUTE OF TECHNOLOGY, TOKYO INSTITUTE OF TECHNOLOGY,   %
%  SPTK WORKING GROUP, AND THE CONTRIBUTORS TO THIS WORK DISCLAIM   %
%  ALL WARRANTIES WITH REGARD TO THIS SOFTWARE, INCLUDING ALL       %
%  IMPLIED WARRANTIES OF MERCHANTABILITY AND FITNESS, IN NO EVENT   %
%  SHALL NAGOYA INSTITUTE OF TECHNOLOGY, TOKYO INSTITUTE OF         %
%  TECHNOLOGY, SPTK WORKING GROUP, NOR THE CONTRIBUTORS BE LIABLE   %
%  FOR ANY SPECIAL, INDIRECT OR CONSEQUENTIAL DAMAGES OR ANY        %
%  DAMAGES WHATSOEVER RESULTING FROM LOSS OF USE, DATA OR PROFITS,  %
%  WHETHER IN AN ACTION OF CONTRACT, NEGLIGENCE OR OTHER TORTUOUS   %
%  ACTION, ARISING OUT OF OR IN CONNECTION WITH THE USE OR          %
%  PERFORMANCE OF THIS SOFTWARE.                                    %
%                                                                   %
%  ---------------------------------------------------------------  %
%
\hypertarget{us}{}
\name{us}{up-sampling}%
{sampling rate transformation}

\begin{synopsis}
\item [us] [ --s $S$ ] [ --c {\em file} ] [ --u $U$ ] [ --d $D$ ] [ {\em infile} ]
\end{synopsis}

\begin{qsection}{DESCRIPTION}
{\em us} up-samples data from {\em infile} (or standard input), 
sending the result to standard output.

The format of input and output data is float.
The following filter coefficients can be used.

\begin{tabular}{ll} \\[-1ex]
	$S=23F$ & \$SPTK/lib/lpfcoef.2to3f \\
	$S=23S$ & \$SPTK/lib/lpfcoef.2to3s \\
	$S=34$ & \$SPTK/lib/lpfcoef.3to4 \\
	$S=45$ & \$SPTK/lib/lpfcoef.4to5 \\
	$S=57$ & \$SPTK/lib/lpfcoef.5to7 \\
	$S=58$ & \$SPTK/lib/lpfcoef.5to8 \\
        &(\$SPTK is the directory where toolkit was installed.)
\end{tabular}

The ratio between up-sampling and down-sampling can be modified by
{\bf --u��--d} options.
If you want to specify filter coefficients, 
{\bf --u��--d} should also be given.

Filter coefficients are in ASCII format.
\end{qsection}

\begin{options}
	\argm{s}{S}{conversion type\\
		\begin{tabular}{ll} \\[-1ex]
			$S=23F$ & up sampling by $2 : 3$\\
			$S=23S$ & up sampling by $2 : 3$\\
			$S=34$ & up sampling by $3 : 4$\\
			$S=45$ & up sampling by $4 : 5$\\
			$S=57$ & up sampling by $5 : 7$\\
			$S=58$ & up sampling by $5 : 8$
		\end{tabular}\\}{58}
	\argm{c}{\mbox{\em file}}{filename of low pass filter coefficients}{Default}
	\argm{u}{U}{up-sampling ratio}{N/A}
	\argm{d}{D}{down-sampling ratio}{N/A}
\end{options}

\begin{qsection}{EXAMPLE}
In this example, the speech data in the input file {\em data.16},
which was sampled at 16 kHz in float format, is converted to
an 44.1 kHz sampling rate:
\begin{quote}
\verb!us -s 23F data.16 | us -s 23S | us -s 57 | \! \\
\verb!us -c /usr/local/SPTK/lib/lpfcoef.5to7 -u 7 -d 8 > data.44! \\[5mm]
Note:~~
$\displaystyle\frac{44100}{16000} = 
	\frac{3\times3\times7\times7\times100}{2\times2\times5\times8\times100}$
\end{quote}
\end{qsection}

%\begin{qsection}{BUGS}
%none
%\end{qsection}

%\begin{qsection}{SEE ALSO}
%none
%\end{qsection}
