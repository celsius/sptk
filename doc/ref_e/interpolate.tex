% ----------------------------------------------------------------- %
%             The Speech Signal Processing Toolkit (SPTK)           %
%             developed by SPTK Working Group                       %
%             http://sp-tk.sourceforge.net/                         %
% ----------------------------------------------------------------- %
%                                                                   %
%  Copyright (c) 1984-2007  Tokyo Institute of Technology           %
%                           Interdisciplinary Graduate School of    %
%                           Science and Engineering                 %
%                                                                   %
%                1996-2008  Nagoya Institute of Technology          %
%                           Department of Computer Science          %
%                                                                   %
% All rights reserved.                                              %
%                                                                   %
% Redistribution and use in source and binary forms, with or        %
% without modification, are permitted provided that the following   %
% conditions are met:                                               %
%                                                                   %
% - Redistributions of source code must retain the above copyright  %
%   notice, this list of conditions and the following disclaimer.   %
% - Redistributions in binary form must reproduce the above         %
%   copyright notice, this list of conditions and the following     %
%   disclaimer in the documentation and/or other materials provided %
%   with the distribution.                                          %
% - Neither the name of the SPTK working group nor the names of its %
%   contributors may be used to endorse or promote products derived %
%   from this software without specific prior written permission.   %
%                                                                   %
% THIS SOFTWARE IS PROVIDED BY THE COPYRIGHT HOLDERS AND            %
% CONTRIBUTORS "AS IS" AND ANY EXPRESS OR IMPLIED WARRANTIES,       %
% INCLUDING, BUT NOT LIMITED TO, THE IMPLIED WARRANTIES OF          %
% MERCHANTABILITY AND FITNESS FOR A PARTICULAR PURPOSE ARE          %
% DISCLAIMED. IN NO EVENT SHALL THE COPYRIGHT OWNER OR CONTRIBUTORS %
% BE LIABLE FOR ANY DIRECT, INDIRECT, INCIDENTAL, SPECIAL,          %
% EXEMPLARY, OR CONSEQUENTIAL DAMAGES (INCLUDING, BUT NOT LIMITED   %
% TO, PROCUREMENT OF SUBSTITUTE GOODS OR SERVICES; LOSS OF USE,     %
% DATA, OR PROFITS; OR BUSINESS INTERRUPTION) HOWEVER CAUSED AND ON %
% ANY THEORY OF LIABILITY, WHETHER IN CONTRACT, STRICT LIABILITY,   %
% OR TORT (INCLUDING NEGLIGENCE OR OTHERWISE) ARISING IN ANY WAY    %
% OUT OF THE USE OF THIS SOFTWARE, EVEN IF ADVISED OF THE           %
% POSSIBILITY OF SUCH DAMAGE.                                       %
% ----------------------------------------------------------------- %
\hypertarget{interpolate}{}
\name{interpolate}{interpolation of data sequence}{signal processing}

\begin{synopsis}
\item[interpolate] [ --p $P$ ] [ --s $S$ ] [ --d ] [ {\em infile} ]
\end{synopsis}

\begin{qsection}{DESCRIPTION}
{\em interpolate} supplements a sequence of input data
from {\em infile} (or standard input)
by 0 or input data with interval $P$ and start number $S$,
sending the result to standard output.

If the input data is
\begin{displaymath}
 x(0), x(1), x(2), \dots
\end{displaymath}
then the output data is
\begin{displaymath}
\underbrace{0, 0, \dots, 0}_{S-1},\underbrace{x(0), 0, 0, \dots, 0}_{P},\underbrace{x(1), 0, 0, \dots, 0}_{P},x(2), \dots
\end{displaymath}
If the --d option is given, the output data is
\begin{displaymath}
\underbrace{0, 0, \dots, 0}_{S-1},\underbrace{x(0), x(0), x(0), \dots, x(0)}_{P},\underbrace{x(1), x(1), x(1), \dots, x(1)}_{P},x(2), \dots
\end{displaymath}
\par
Input and output data are in float format.
\end{qsection}

\begin{options}
        \argm{p}{P}{interpolation period}{10}
        \argm{s}{S}{start sample}{0}
        \argm{d}{}{pad input data rather than 0}{FALSE}
\end{options}

\begin{qsection}{EXAMPLE}
This example decimates input data from {\em data.f} file with interval 2,
interpolates 0 with interval 2, and then outputs it to {\em
data.di} file:
\begin{quote}
  \verb!decimate -p 2  < data.f | interpolate -p 2 > data.di!
\end{quote}
\end{qsection}

\begin{qsection}{SEE ALSO}
\hyperlink{decimate}{decimate}
\end{qsection}
