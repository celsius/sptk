% ----------------------------------------------------------------
%       Speech Signal Processing Toolkit (SPTK): version 3.0
%                      SPTK Working Group
% 
%                Department of Computer Science
%                Nagoya Institute of Technology
%                             and
%   Interdisciplinary Graduate School of Science and Engineering
%                Tokyo Institute of Technology
%                   Copyright (c) 1984-2000
%                     All Rights Reserved.
% 
% Permission is hereby granted, free of charge, to use and
% distribute this software and its documentation without
% restriction, including without limitation the rights to use,
% copy, modify, merge, publish, distribute, sublicense, and/or
% sell copies of this work, and to permit persons to whom this
% work is furnished to do so, subject to the following conditions:
% 
%   1. The code must retain the above copyright notice, this list
%      of conditions and the following disclaimer.
% 
%   2. Any modifications must be clearly marked as such.
%                                                                        
% NAGOYA INSTITUTE OF TECHNOLOGY, TOKYO INSITITUTE OF TECHNOLOGY,
% SPTK WORKING GROUP, AND THE CONTRIBUTORS TO THIS WORK DISCLAIM
% ALL WARRANTIES WITH REGARD TO THIS SOFTWARE, INCLUDING ALL
% IMPLIED WARRANTIES OF MERCHANTABILITY AND FITNESS, IN NO EVENT
% SHALL NAGOYA INSTITUTE OF TECHNOLOGY, TOKYO INSITITUTE OF
% TECHNOLOGY, SPTK WORKING GROUP, NOR THE CONTRIBUTORS BE LIABLE
% FOR ANY SPECIAL, INDIRECT OR CONSEQUENTIAL DAMAGES OR ANY
% DAMAGES WHATSOEVER RESULTING FROM LOSS OF USE, DATA OR PROFITS,
% WHETHER IN AN ACTION OF CONTRACT, NEGLIGENCE OR OTHER TORTIOUS
% ACTION, ARISING OUT OF OR IN CONNECTION WITH THE USE OR
% PERFORMANCE OF THIS SOFTWARE.
% ----------------------------------------------------------------
%
\name{vstat}{vector statistics calculation}{data processing}

\begin{synopsis}
\item[vstat] [ --l $L$ ] [ --n $N$ ] [ --t $T$ ] [ --d ] [ --o $O$] [ {\em infile} ]
\end{synopsis}

\begin{qsection}{DESCRIPTION}
This command reads an $L$ dimension vector from the assigned file,
evaluates the mean vector and the covariance matrix
for every $T$ vectors, and sends the results to the standard output.
That is, if the input data is
\begin{displaymath}
\overbrace{
  \overbrace{x_1(1),\ldots,x_1(L)}^{L},
  \overbrace{x_2(1),\ldots,x_2(L)}^{L},\ldots,
  \overbrace{x_N(1),\ldots,x_N(L)}^{L}
}^{T \times L},\ldots
\end{displaymath}
then the output will be 
\begin{displaymath}
  \overbrace{m(1),\ldots,m(L)}^L, 
  \overbrace{
    \overbrace{U(11),\ldots,U(1L)}^L, \ldots
    \overbrace{U(L1),\ldots,U(LL)}^L
  }^{L\times L}, \ldots
\end{displaymath}
evaluation of {\boldmath $m$}��{\boldmath $U$} is undertaken by
\begin{displaymath}
  \mbox{\boldmath $m$} = \frac{1}{N}\sum_{k=1}^{N} \mbox{\boldmath $x$}
\end{displaymath}
\begin{displaymath}
  \mbox{\boldmath $U$} = \frac{1}{N}\sum_{k=1}^{N}
	\mbox{\boldmath $x$} \mbox{\boldmath $x$}^T
	- \mbox{\boldmath $m$} \mbox{\boldmath $m$}^T
\end{displaymath}
Also, if the diagonal covariance is wanted(--d option),
then the values outside of the diagonal are not outputed.

If the input file is omitted, then data is read from the standard input.

Input and output data are in float format.
\end{qsection}

\begin{options}
	\argm{l}{L}{length of vector}{1}
	\argm{n}{N}{order of vector}{L-1}
	\argm{t}{T}{number of vector}{N/A}
	\argm{o}{O}{output format\\
		\begin{tabular}{ll} \\[-1zh]
                        $O=0$ & mean \& covariance\\
                        $O=1$ & mean\\
                        $O=2$ & covariance\\[1zh]
                \end{tabular}\\\hspace*{\fill}}{0}
	\argm{d}{}{diagonal covariance}{FALSE}
	\argm{r}{}{output correlation instead of covariance}{FALSE}
\end{options}

\begin{qsection}{EXAMPLE}
The output file {\em data.stat} contains the mean and covariance matrix
taken from the whole data in {\em data.f} read in float format.
\begin{quote}
  \verb!vstat data.f > data.stat!
\end{quote}

In the example below, the mean of 15 order coefficients vector is taken
for every group of 3 frames and sent to {\em data.av}:
\begin{quote}
  \verb!vstat -l 15 -t 3 -o 1 data.f > data.av!
\end{quote}
\end{qsection}

\begin{qsection}{SEE ALSO}
  average, vsum
\end{qsection}
