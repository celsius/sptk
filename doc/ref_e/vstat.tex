% ----------------------------------------------------------------- %
%             The Speech Signal Processing Toolkit (SPTK)           %
%             developed by SPTK Working Group                       %
%             http://sp-tk.sourceforge.net/                         %
% ----------------------------------------------------------------- %
%                                                                   %
%  Copyright (c) 1984-2007  Tokyo Institute of Technology           %
%                           Interdisciplinary Graduate School of    %
%                           Science and Engineering                 %
%                                                                   %
%                1996-2015  Nagoya Institute of Technology          %
%                           Department of Computer Science          %
%                                                                   %
% All rights reserved.                                              %
%                                                                   %
% Redistribution and use in source and binary forms, with or        %
% without modification, are permitted provided that the following   %
% conditions are met:                                               %
%                                                                   %
% - Redistributions of source code must retain the above copyright  %
%   notice, this list of conditions and the following disclaimer.   %
% - Redistributions in binary form must reproduce the above         %
%   copyright notice, this list of conditions and the following     %
%   disclaimer in the documentation and/or other materials provided %
%   with the distribution.                                          %
% - Neither the name of the SPTK working group nor the names of its %
%   contributors may be used to endorse or promote products derived %
%   from this software without specific prior written permission.   %
%                                                                   %
% THIS SOFTWARE IS PROVIDED BY THE COPYRIGHT HOLDERS AND            %
% CONTRIBUTORS "AS IS" AND ANY EXPRESS OR IMPLIED WARRANTIES,       %
% INCLUDING, BUT NOT LIMITED TO, THE IMPLIED WARRANTIES OF          %
% MERCHANTABILITY AND FITNESS FOR A PARTICULAR PURPOSE ARE          %
% DISCLAIMED. IN NO EVENT SHALL THE COPYRIGHT OWNER OR CONTRIBUTORS %
% BE LIABLE FOR ANY DIRECT, INDIRECT, INCIDENTAL, SPECIAL,          %
% EXEMPLARY, OR CONSEQUENTIAL DAMAGES (INCLUDING, BUT NOT LIMITED   %
% TO, PROCUREMENT OF SUBSTITUTE GOODS OR SERVICES; LOSS OF USE,     %
% DATA, OR PROFITS; OR BUSINESS INTERRUPTION) HOWEVER CAUSED AND ON %
% ANY THEORY OF LIABILITY, WHETHER IN CONTRACT, STRICT LIABILITY,   %
% OR TORT (INCLUDING NEGLIGENCE OR OTHERWISE) ARISING IN ANY WAY    %
% OUT OF THE USE OF THIS SOFTWARE, EVEN IF ADVISED OF THE           %
% POSSIBILITY OF SUCH DAMAGE.                                       %
% ----------------------------------------------------------------- %
\hypertarget{vstat}{}
\name{vstat}{vector statistics calculation}{data processing}

\begin{synopsis}
\item[vstat] [ --l $L$ ] [ --n $N$ ] [ --t $T$ ] [ --c $C$ ] [ --d ] [ --o $O$ ] [ {\em infile} ]
\end{synopsis}

\begin{qsection}{DESCRIPTION}
{\em vstat} calculates the mean and covariance of groups of vectors 
from {\em infile} (or standard input), 
sending the result to standard output.

For each group of $T$ input vectors of length $L$, 
{\em vstat} calculates the mean vector of length $L$ 
and the $L\times L$ covariance matrix. 
In other words, if the input data is:
\begin{displaymath}
\overbrace{
  \overbrace{x_1(1),\dots,x_1(L)}^{L},
  \overbrace{x_2(1),\dots,x_2(L)}^{L},\dots,
  \overbrace{x_N(1),\dots,x_N(L)}^{L}
}^{T \times L},\dots
\end{displaymath}
then the output will be given by:
\begin{displaymath}
  \overbrace{\mu(1),\dots,\mu(L)}^L, 
  \overbrace{
    \overbrace{\sigma(11),\dots,\sigma(1L)}^L, \dots
    \overbrace{\sigma(L1),\dots,\sigma(LL)}^L
  }^{L\times L}, \dots
\end{displaymath}
and the values of $\bmu$, $\bSigma$ can be obtained through the following:
\begin{displaymath}
  \bmu = \frac{1}{N}\sum_{k=1}^{N} \bx
\end{displaymath}
\begin{displaymath}
  \bSigma = \frac{1}{N}\sum_{k=1}^{N}
	\bx \bx'
	- \bmu \bmu'
\end{displaymath}
If the --d option is given, 
the length $L$ diagonal of the covariance matrix is outputted 
instead of the entire $L\times L$ matrix.

If the --o 3 option is specified, {\em vstat} also calculates the confidence
interval of the mean via Student's t-distribution for each dimension,
i.e. for each dimension, the confidence interval can be estimated
at the confidence level $\alpha$ (\%) satisfying the following condition:
 \begin{displaymath}
  t(\alpha, \phi) \geq
  \left| \frac{\mu(i) -m(i)}{\sqrt{\hat{\sigma(i)}^{2} / L} }\right|
  ,   \;\;\;\; i = 1,  2, \dots, L
 \end{displaymath}
where $t(\alpha, \phi)$ is the upper $0.5(100 - \alpha)$-th percentile
of the t-distribution with $\phi$ degrees of freedom,
$m(i)$ is the population mean, $\hat{\sigma(i)^{2}}$ is the unbiased variance.
The confidence level $\alpha$ can be specified by the --c option.
The upper and lower bounds $u(i)$ and $l(i)$ can be written as
 \begin{displaymath}
  u(i) = \mu(i) + t(\alpha, L - 1) \sqrt{\frac{\hat{\sigma(i)}^{2}}{L}},
 \end{displaymath}
 \begin{displaymath}
  l(i) = \mu(i) - t(\alpha, L - 1) \sqrt{\frac{\hat{\sigma(i)}^{2}}{L}}.
 \end{displaymath}
The order of the output is as follows.
 \begin{displaymath}
  \overbrace{\mu(1),\dots,\mu(L)}^L, \; \overbrace{u(1),\dots,u(L)}^L,
    \;\overbrace{l(1),\dots,l(L)}^L
 \end{displaymath}

If the --o 4 option is specified, {\em vstat} outputs the median of 
input vectors of length $L$.
If the number of vectors is even number, {\em vstat} outputs the 
arithmetic mean of two vectors of center.
 

Also, input and output data are in float format.
\end{qsection}

\begin{options}
	\argm{l}{L}{length of vector}{1}
	\argm{n}{N}{order of vector}{L-1}
	\argm{t}{T}{number of vector}{N/A}
	\argm{o}{O}{output format\\
		\begin{tabular}{ll} \\[-1ex]
                        $O=0$ & mean \& covariance\\
                        $O=1$ & mean\\
                        $O=2$ & covariance\\
                        $O=3$ & mean \& upper / lower bound of confidence interval\\
                                &via Student's t-distribution\\
                        $O=4$ & median\\[1ex]
                \end{tabular}\\\hspace*{\fill}}{0}
	\argm{c}{C}{confidence level of confidence interval (\%)}{95.00}
	\argm{d}{}{diagonal covariance}{FALSE}
	\argm{i}{}{output inverse covariance instead of covariance}{FALSE}
	\argm{r}{}{output correlation instead of covariance}{FALSE}
\end{options}

\begin{qsection}{EXAMPLE}
The output file {\em data.stat} contains the mean and covariance matrix
taken from the whole data in {\em data.f} read in float format.
\begin{quote}
  \verb!vstat data.f > data.stat!
\end{quote}

In the example below, the mean of 15-th order coefficients vector is taken
for every group of 3 frames and sent to {\em data.av}:
\begin{quote}
  \verb!vstat -l 15 -t 3 -o 1 data.f > data.av!
\end{quote}

The output file {\em data.stat} contains the mean and upper / lower
bound of the confidence interval (90\%) calculated via Student's t-distribution.
\begin{quote}
  \verb!vstat -C 90.0 -o 3 data.f > data.stat!
\end{quote}
\end{qsection}

\begin{qsection}{NOTICE}
If --d is specified, off-diagonal elements are suppressed.
\end{qsection}

\begin{qsection}{SEE ALSO}
\hyperlink{average}{average},
\hyperlink{vsum}{vsum}
\end{qsection}
