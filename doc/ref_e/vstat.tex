% ----------------------------------------------------------------
%       Speech Signal Processing Toolkit (SPTK): version 3.0
%                      SPTK Working Group
% 
%                Department of Computer Science
%                Nagoya Institute of Technology
%                             and
%   Interdisciplinary Graduate School of Science and Engineering
%                Tokyo Institute of Technology
%                   Copyright (c) 1984-2000
%                     All Rights Reserved.
% 
% Permission is hereby granted, free of charge, to use and
% distribute this software and its documentation without
% restriction, including without limitation the rights to use,
% copy, modify, merge, publish, distribute, sublicense, and/or
% sell copies of this work, and to permit persons to whom this
% work is furnished to do so, subject to the following conditions:
% 
%   1. The code must retain the above copyright notice, this list
%      of conditions and the following disclaimer.
% 
%   2. Any modifications must be clearly marked as such.
%                                                                        
% NAGOYA INSTITUTE OF TECHNOLOGY, TOKYO INSITITUTE OF TECHNOLOGY,
% SPTK WORKING GROUP, AND THE CONTRIBUTORS TO THIS WORK DISCLAIM
% ALL WARRANTIES WITH REGARD TO THIS SOFTWARE, INCLUDING ALL
% IMPLIED WARRANTIES OF MERCHANTABILITY AND FITNESS, IN NO EVENT
% SHALL NAGOYA INSTITUTE OF TECHNOLOGY, TOKYO INSITITUTE OF
% TECHNOLOGY, SPTK WORKING GROUP, NOR THE CONTRIBUTORS BE LIABLE
% FOR ANY SPECIAL, INDIRECT OR CONSEQUENTIAL DAMAGES OR ANY
% DAMAGES WHATSOEVER RESULTING FROM LOSS OF USE, DATA OR PROFITS,
% WHETHER IN AN ACTION OF CONTRACT, NEGLIGENCE OR OTHER TORTIOUS
% ACTION, ARISING OUT OF OR IN CONNECTION WITH THE USE OR
% PERFORMANCE OF THIS SOFTWARE.
% ----------------------------------------------------------------
%
\name{vstat}{vector statistics calculation}{data processing}

\begin{synopsis}
\item[vstat] [ --l $L$ ] [ --n $N$ ] [ --t $T$ ] [ --d ] [ --o $O$] [ {\em infile} ]
\end{synopsis}

\begin{qsection}{DESCRIPTION}
{\em vstat} calculates the mean and covariance of groups of vectors 
from {\em infile} (or standard input), 
sending the result to standard output.

For each group of $T$ input vectors of length $L$, 
{\em vstat} calculates the length $L$ mean vector 
and the $L\times L$ covariance matrix. 
That is, if the input data is
\begin{displaymath}
\overbrace{
  \overbrace{x_1(1),\dots,x_1(L)}^{L},
  \overbrace{x_2(1),\dots,x_2(L)}^{L},\dots,
  \overbrace{x_N(1),\dots,x_N(L)}^{L}
}^{T \times L},\dots
\end{displaymath}
then the output will be 
\begin{displaymath}
  \overbrace{\mu(1),\dots,\mu(L)}^L, 
  \overbrace{
    \overbrace{\sigma(11),\dots,\sigma(1L)}^L, \dots
    \overbrace{\sigma(L1),\dots,\sigma(LL)}^L
  }^{L\times L}, \dots
\end{displaymath}
evaluation of $\bmu$, $\bSigma$ is undertaken by
\begin{displaymath}
  \bmu = \frac{1}{N}\sum_{k=1}^{N} \bx
\end{displaymath}
\begin{displaymath}
  \bSigma = \frac{1}{N}\sum_{k=1}^{N}
	\bx \bx'
	- \bmu \bmu'
\end{displaymath}
If the --d option is given, 
the length $L$ diagonal of the covariance matrix is output 
instead of the entire $L\times L$ matrix.

Input and output data are in float format.
\end{qsection}

\begin{options}
	\argm{l}{L}{length of vector}{1}
	\argm{n}{N}{order of vector}{L-1}
	\argm{t}{T}{number of vector}{N/A}
	\argm{o}{O}{output format\\
		\begin{tabular}{ll} \\[-1ex]
                        $O=0$ & mean \& covariance\\
                        $O=1$ & mean\\
                        $O=2$ & covariance\\[1ex]
                \end{tabular}\\\hspace*{\fill}}{0}
	\argm{d}{}{diagonal covariance}{FALSE}
	\argm{i}{}{output inverse covariance instead of covariance}{FALSE}
	\argm{r}{}{output correlation instead of covariance}{FALSE}
\end{options}

\begin{qsection}{EXAMPLE}
The output file {\em data.stat} contains the mean and covariance matrix
taken from the whole data in {\em data.f} read in float format.
\begin{quote}
  \verb!vstat data.f > data.stat!
\end{quote}

In the example below, the mean of 15 order coefficients vector is taken
for every group of 3 frames and sent to {\em data.av}:
\begin{quote}
  \verb!vstat -l 15 -t 3 -o 1 data.f > data.av!
\end{quote}
\end{qsection}

\begin{qsection}{SEE ALSO}
  average, vsum
\end{qsection}
