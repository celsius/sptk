\name{ifft}{inverse FFT for complex sequence}{signal processing}

\begin{synopsis}
\item[ifft] [ --l $L$ ]  [ --\{ R $|$ I \} ] [ {\em infile} ] 
\end{synopsis}

\begin{qsection}{DESCRIPTION}
The {\em ifft} command reads conplex data from input file,
evaluates its inverse DFT, and sends the results to
the standard output.
The input data format is float, and the real part is allocated
in the first $L$ points and the imaginary part is allocated
next $L$ points for every block.
\begin{center}
 \leavevmode
 \includegraphics{fig/ifft.eps} 
\end{center}
\end{qsection}

\begin{options}
	\argm{l}{L}{FFT size power of 2}{256}
	\argm{R}{}{output only real part}{FALSE}
	\argm{I}{}{output only imaginary part}{FALSE}
\end{options}

\begin{qsection}{EXAMPLE}
In this example, the inverse DFT is evaluated from a data file
{\em data.f} in float format
(real part: 256 points, imaginary part: 256 points),
and the output is written to {\em data.ifft}:
\begin{quote}
  \verb!ifft data.f -l 256 > data.ifft!
\end{quote}
\end{qsection}

\begin{qsection}{SEE ALSO}
 fft, fft2, fftr, fftr2, ifft2
\end{qsection}
