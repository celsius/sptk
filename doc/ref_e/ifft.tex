% ----------------------------------------------------------------
%       Speech Signal Processing Toolkit (SPTK): version 3.0
%                      SPTK Working Group
% 
%                Department of Computer Science
%                Nagoya Institute of Technology
%                             and
%   Interdisciplinary Graduate School of Science and Engineering
%                Tokyo Institute of Technology
%                   Copyright (c) 1984-2000
%                     All Rights Reserved.
% 
% Permission is hereby granted, free of charge, to use and
% distribute this software and its documentation without
% restriction, including without limitation the rights to use,
% copy, modify, merge, publish, distribute, sublicense, and/or
% sell copies of this work, and to permit persons to whom this
% work is furnished to do so, subject to the following conditions:
% 
%   1. The code must retain the above copyright notice, this list
%      of conditions and the following disclaimer.
% 
%   2. Any modifications must be clearly marked as such.
%                                                                        
% NAGOYA INSTITUTE OF TECHNOLOGY, TOKYO INSITITUTE OF TECHNOLOGY,
% SPTK WORKING GROUP, AND THE CONTRIBUTORS TO THIS WORK DISCLAIM
% ALL WARRANTIES WITH REGARD TO THIS SOFTWARE, INCLUDING ALL
% IMPLIED WARRANTIES OF MERCHANTABILITY AND FITNESS, IN NO EVENT
% SHALL NAGOYA INSTITUTE OF TECHNOLOGY, TOKYO INSITITUTE OF
% TECHNOLOGY, SPTK WORKING GROUP, NOR THE CONTRIBUTORS BE LIABLE
% FOR ANY SPECIAL, INDIRECT OR CONSEQUENTIAL DAMAGES OR ANY
% DAMAGES WHATSOEVER RESULTING FROM LOSS OF USE, DATA OR PROFITS,
% WHETHER IN AN ACTION OF CONTRACT, NEGLIGENCE OR OTHER TORTIOUS
% ACTION, ARISING OUT OF OR IN CONNECTION WITH THE USE OR
% PERFORMANCE OF THIS SOFTWARE.
% ----------------------------------------------------------------
%
\hypertarget{ifft}{}
\name{ifft}{inverse FFT for complex sequence}{signal processing}

\begin{synopsis}
\item[ifft] [ --l $L$ ]  [ --\{ R $|$ I \} ] [ {\em infile} ] 
\end{synopsis}

\begin{qsection}{DESCRIPTION}
{\em ifft} calculates the Inverse Discrete Fourier Transform (IDFT) 
of complex-valued data from {\em infile} (or standard input), 
sending the results to standard output.
The input and output data is in float format, arranged as follows.
\begin{center}
 \leavevmode
 \includegraphics{fig/ifft.eps} 
\end{center}
\end{qsection}

\begin{options}
	\argm{l}{L}{FFT size power of 2}{256}
	\argm{R}{}{output only real part}{FALSE}
	\argm{I}{}{output only imaginary part}{FALSE}
\end{options}

\begin{qsection}{EXAMPLE}
In this example, the inverse DFT is evaluated from a data file
{\em data.f} in float format
(real part: 256 points, imaginary part: 256 points),
and the output is written to {\em data.ifft}:
\begin{quote}
  \verb!ifft data.f -l 256 > data.ifft!
\end{quote}
\end{qsection}

\begin{qsection}{SEE ALSO}
\hyperlink{fft}{fft},
\hyperlink{fft2}{fft2},
\hyperlink{fftr}{fftr},
\hyperlink{fftr2}{fftr2},
\hyperlink{ifft2}{ifft2}
\end{qsection}
