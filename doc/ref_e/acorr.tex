\name{acorr}{evaluation short term autocorrelation function}{signal processing}

\begin{synopsis}
 \item[ acorr ] [ --m $M$ ] [ --l $L$ ] [ {\em infile} ]
\end{synopsis}

\begin{qsection}{DESCRIPTION}
 Data is read from the file {\em infile}, the $m$ order autocorrelation
 function sequence is calculated for every frame and sent out to standard
 output. Namely, input data is
\[ x(0),x(1),\cdots,x(L-1) \],
 and evaluating 
\[ r(k)=\sum_{m=0}^{L-1-k}x(m)x(m+k),~~~~~~k=0,1,\cdots,M \],
 we obtain the autocorrelation function sequence.
\[ r(0),r(1),\cdots,r(M) \]
 input and output data format are float.
\end{qsection}

\begin{options}
	\argm{m}{M}{number of coefficient of autocorrelation function}{25}
	\argm{l}{L}{frame length}{256}
\end{options}

\begin{qsection}{EXAMPLE}
In the example below, the input file {\em data.f} is in float format.
the frame length is 256, the frame period 100, and every frame is
passed throught Blackman window, and autocorrelation function sequence
is founds in {\em data.acorr}.
\begin{center}
 \verb!frame -l 256 -p 100 < data.f | window | acorr -m 10 > data.acorr!
\end{center}
\end{qsection}

\begin{qsection}{SEE ALSO}
 c2acr, levdur
\end{qsection}