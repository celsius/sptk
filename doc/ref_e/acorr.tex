% ----------------------------------------------------------------
%       Speech Signal Processing Toolkit (SPTK): version 3.0
%                      SPTK Working Group
% 
%                Department of Computer Science
%                Nagoya Institute of Technology
%                             and
%   Interdisciplinary Graduate School of Science and Engineering
%                Tokyo Institute of Technology
%                   Copyright (c) 1984-2000
%                     All Rights Reserved.
% 
% Permission is hereby granted, free of charge, to use and
% distribute this software and its documentation without
% restriction, including without limitation the rights to use,
% copy, modify, merge, publish, distribute, sublicense, and/or
% sell copies of this work, and to permit persons to whom this
% work is furnished to do so, subject to the following conditions:
% 
%   1. The code must retain the above copyright notice, this list
%      of conditions and the following disclaimer.
% 
%   2. Any modifications must be clearly marked as such.
%                                                                        
% NAGOYA INSTITUTE OF TECHNOLOGY, TOKYO INSITITUTE OF TECHNOLOGY,
% SPTK WORKING GROUP, AND THE CONTRIBUTORS TO THIS WORK DISCLAIM
% ALL WARRANTIES WITH REGARD TO THIS SOFTWARE, INCLUDING ALL
% IMPLIED WARRANTIES OF MERCHANTABILITY AND FITNESS, IN NO EVENT
% SHALL NAGOYA INSTITUTE OF TECHNOLOGY, TOKYO INSITITUTE OF
% TECHNOLOGY, SPTK WORKING GROUP, NOR THE CONTRIBUTORS BE LIABLE
% FOR ANY SPECIAL, INDIRECT OR CONSEQUENTIAL DAMAGES OR ANY
% DAMAGES WHATSOEVER RESULTING FROM LOSS OF USE, DATA OR PROFITS,
% WHETHER IN AN ACTION OF CONTRACT, NEGLIGENCE OR OTHER TORTIOUS
% ACTION, ARISING OUT OF OR IN CONNECTION WITH THE USE OR
% PERFORMANCE OF THIS SOFTWARE.
% ----------------------------------------------------------------
%
\name{acorr}{obtain autocorrelation sequence}{signal processing}

\begin{synopsis}
 \item[ acorr ] [ --m $M$ ] [ --l $L$ ] [ {\em infile} ]
\end{synopsis}

\begin{qsection}{DESCRIPTION}
 Data is read from the file {\em infile}, the $m$ order autocorrelation
 function sequence is calculated for every frame and sent out
 to the standard output. Namely, the input data is given by
\[ x(0),x(1),\cdots,x(L-1), \]
 and the autocorrelation is evaluated as
\[ r(k)=\sum_{m=0}^{L-1-k}x(m)x(m+k),~~~~~~k=0,1,\cdots,M, \]
 and the output is the following autocorrelation function sequence,
\[ r(0),r(1),\cdots,r(M) \]
 The format for input and output is float.
\end{qsection}

\begin{options}
	\argm{m}{M}{order of sequence}{25}
	\argm{l}{L}{frame length}{256}
\end{options}

\begin{qsection}{EXAMPLE}
In the example below, the input file {\em data.f} is in float format.
The frame length is 256, the frame period 100, every frame is
passed through a Blackman window, and the autocorrelation function sequence
is found in {\em data.acorr}.
\begin{center}
 \verb!frame -l 256 -p 100 < data.f | window | acorr -m 10 > data.acorr!
\end{center}
\end{qsection}

\begin{qsection}{SEE ALSO}
 c2acr, levdur
\end{qsection}
