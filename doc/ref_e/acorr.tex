% ----------------------------------------------------------------- %
%             The Speech Signal Processing Toolkit (SPTK)           %
%             developed by SPTK Working Group                       %
%             http://sp-tk.sourceforge.net/                         %
% ----------------------------------------------------------------- %
%                                                                   %
%  Copyright (c) 1984-2007  Tokyo Institute of Technology           %
%                           Interdisciplinary Graduate School of    %
%                           Science and Engineering                 %
%                                                                   %
%                1996-2011  Nagoya Institute of Technology          %
%                           Department of Computer Science          %
%                                                                   %
% All rights reserved.                                              %
%                                                                   %
% Redistribution and use in source and binary forms, with or        %
% without modification, are permitted provided that the following   %
% conditions are met:                                               %
%                                                                   %
% - Redistributions of source code must retain the above copyright  %
%   notice, this list of conditions and the following disclaimer.   %
% - Redistributions in binary form must reproduce the above         %
%   copyright notice, this list of conditions and the following     %
%   disclaimer in the documentation and/or other materials provided %
%   with the distribution.                                          %
% - Neither the name of the SPTK working group nor the names of its %
%   contributors may be used to endorse or promote products derived %
%   from this software without specific prior written permission.   %
%                                                                   %
% THIS SOFTWARE IS PROVIDED BY THE COPYRIGHT HOLDERS AND            %
% CONTRIBUTORS "AS IS" AND ANY EXPRESS OR IMPLIED WARRANTIES,       %
% INCLUDING, BUT NOT LIMITED TO, THE IMPLIED WARRANTIES OF          %
% MERCHANTABILITY AND FITNESS FOR A PARTICULAR PURPOSE ARE          %
% DISCLAIMED. IN NO EVENT SHALL THE COPYRIGHT OWNER OR CONTRIBUTORS %
% BE LIABLE FOR ANY DIRECT, INDIRECT, INCIDENTAL, SPECIAL,          %
% EXEMPLARY, OR CONSEQUENTIAL DAMAGES (INCLUDING, BUT NOT LIMITED   %
% TO, PROCUREMENT OF SUBSTITUTE GOODS OR SERVICES; LOSS OF USE,     %
% DATA, OR PROFITS; OR BUSINESS INTERRUPTION) HOWEVER CAUSED AND ON %
% ANY THEORY OF LIABILITY, WHETHER IN CONTRACT, STRICT LIABILITY,   %
% OR TORT (INCLUDING NEGLIGENCE OR OTHERWISE) ARISING IN ANY WAY    %
% OUT OF THE USE OF THIS SOFTWARE, EVEN IF ADVISED OF THE           %
% POSSIBILITY OF SUCH DAMAGE.                                       %
% ----------------------------------------------------------------- %
\hypertarget{acorr}{}
\name{acorr}{obtain autocorrelation sequence}{signal processing}

\begin{synopsis}
 \item[ acorr ] [ --m $M$ ] [ --l $L$ ] [ {\em infile} ]
\end{synopsis}

\begin{qsection}{DESCRIPTION}
{\em acorr} calculates the $m$-th order autocorrelation function sequence 
for each frame of float data from {\em infile} (or standard input), 
sending the result to standard output.
Namely, the input data is given by
\[ x(0),x(1),\dots,x(L-1), \]
 and the autocorrelation is evaluated as
\[ r(k)=\sum_{m=0}^{L-1-k}x(m)x(m+k), \qquad k=0,1,\dots,M, \]
 and the output is the following autocorrelation function sequence,
\[ r(0),r(1),\dots,r(M) \]
 Both input and output files are in float format.
\end{qsection}

\begin{options}
	\argm{m}{M}{order of sequence}{25}
	\argm{l}{L}{frame length}{256}
\end{options}

\begin{qsection}{EXAMPLE}
In the example below, the input file {\em data.f} is in float format.
Here, the frame length and period are of 256 and 100, respectively.
Also, every frame is passed through a Blackman window and the
autocorrelation function sequence is sent to {\em data.acorr}.
\begin{center}
 \verb!frame -l 256 -p 100 < data.f | window | acorr -m 10 > data.acorr!
\end{center}
\end{qsection}

\begin{qsection}{SEE ALSO}
\hyperlink{c2acr}{c2acr}, 
\hyperlink{levdur}{levdur}
\end{qsection}
