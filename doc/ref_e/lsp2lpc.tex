% ----------------------------------------------------------------
%       Speech Signal Processing Toolkit (SPTK): version 3.0
%                      SPTK Working Group
% 
%                Department of Computer Science
%                Nagoya Institute of Technology
%                             and
%   Interdisciplinary Graduate School of Science and Engineering
%                Tokyo Institute of Technology
%                   Copyright (c) 1984-2000
%                     All Rights Reserved.
% 
% Permission is hereby granted, free of charge, to use and
% distribute this software and its documentation without
% restriction, including without limitation the rights to use,
% copy, modify, merge, publish, distribute, sublicense, and/or
% sell copies of this work, and to permit persons to whom this
% work is furnished to do so, subject to the following conditions:
% 
%   1. The code must retain the above copyright notice, this list
%      of conditions and the following disclaimer.
% 
%   2. Any modifications must be clearly marked as such.
%                                                                        
% NAGOYA INSTITUTE OF TECHNOLOGY, TOKYO INSITITUTE OF TECHNOLOGY,
% SPTK WORKING GROUP, AND THE CONTRIBUTORS TO THIS WORK DISCLAIM
% ALL WARRANTIES WITH REGARD TO THIS SOFTWARE, INCLUDING ALL
% IMPLIED WARRANTIES OF MERCHANTABILITY AND FITNESS, IN NO EVENT
% SHALL NAGOYA INSTITUTE OF TECHNOLOGY, TOKYO INSITITUTE OF
% TECHNOLOGY, SPTK WORKING GROUP, NOR THE CONTRIBUTORS BE LIABLE
% FOR ANY SPECIAL, INDIRECT OR CONSEQUENTIAL DAMAGES OR ANY
% DAMAGES WHATSOEVER RESULTING FROM LOSS OF USE, DATA OR PROFITS,
% WHETHER IN AN ACTION OF CONTRACT, NEGLIGENCE OR OTHER TORTIOUS
% ACTION, ARISING OUT OF OR IN CONNECTION WITH THE USE OR
% PERFORMANCE OF THIS SOFTWARE.
% ----------------------------------------------------------------
%
\name{lsp2lpc}{transform LSP to LPC}{speech parameter transformation}

\begin{synopsis}
\item [lsp2lpc] [ --m $M$ ] [ --s $S$ ] [ --k ] [ --l ] [ --i $I$ ] [ {\em infile} ] 
\end{synopsis}

\begin{qsection}{DESCRIPTION}
{\em lsp2lpc} calculates linear prediction (LPC) coefficients 
from $M$-order line spectral pair (LSP) coefficients 
from {\em infile} (or standard input),
sending the result to standard output.

The LSP input input format is
\begin{displaymath}
   [ K ], l(1), \dots , l(M), 
\end{displaymath}
and the LPC output format is
\begin{displaymath}
   K , a(1), \dots , a(M).
\end{displaymath}

By default, {\em lsp2lpc} assumes that 
the LSP input vectors include the gain $K$, 
and it passes that gain value through to the LPC output vectors.  
However, if the --k option is present, 
{\em lsp2lpc} assumes that $K$ is not present in the LSP input vectors, 
and it sets $K$ to $1.0$ in the LPC output vectors.
\end{qsection}

\begin{options}
	\argm{m}{M}{order of LPC}{25}
	\argm{s}{S}{sampling frequency(kHz)}{10}
	\argm{k}{}{input \& output gain}{TRUE}
	\argm{l}{}{regard input as log gain and output linear gain}{FALSE}
	\argm{i}{I}{input format\\
		\begin{tabular}{ll} \\[-1ex]
			$0$ & normalized frequency $(0 \dots \pi)$ \\
			$1$ & normalized frequency $(0 \dots 0.5)$ \\
			$2$ & frequency (kHz) \\
			$3$ & frequency (Hz)  \\
		\end{tabular}\\\hspace*{\fill}}{0}
\end{options}

\begin{qsection}{EXAMPLE}
In the example below, 10 order LSP coefficients in float format
are read from file {\em data.lsp}, the linear prediction coefficients
are evaluated, and written to {\em data.lpc}:
\begin{quote}
\verb! lsp2lpc -m 10 < data.lsp > data.lpc!
\end{quote}
\end{qsection}

\begin{qsection}{SEE ALSO}
lpc, lpc2lsp
\end{qsection}
