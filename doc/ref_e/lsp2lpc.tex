\name{lsp2lpc}{transform LSP to LPC}{speech parameter transformation}

\begin{synopsis}
\item [lsp2lpc] [ --m $M$ ] [ --s $S$ ] [ --k ] [ --i $I$ ] [ {\em infile} ] 
\end{synopsis}

\begin{qsection}{DESCRIPTION}
The {\em lsp2lpc} command transforms LSP coefficients into
LPC coefficients.
If the gain is not assigned,
then its value is made equal to 1 as follows
\begin{displaymath}
	K = 1,a(1),\ldots,a(M).
 \end{displaymath}
\end{qsection}

\begin{options}
	\argm{m}{M}{order of LPC}{25}
	\argm{s}{S}{sampling frequency(kHz)}{10}
	\argm{k}{}{input \& output gain}{TRUE}
	\argm{i}{I}{input format\\
		\begin{tabular}{ll} \\[-1zh]
			$0$ & normalized frequency $(0 \ldots \pi)$ \\
			$1$ & normalized frequency $(0 \ldots 0.5)$ \\
			$2$ & frequency (kHz) \\
			$3$ & frequency (Hz)  \\
		\end{tabular}\\\hspace*{\fill}}{0}
\end{options}

\begin{qsection}{EXAMPLE}
In the example below, 10 order LSP coefficients in float format
are read from file {\em data.lsp}, the linear prediction coefficients
are evaluated, and written to {\em data.lpc}:
\begin{quote}
\verb! lsp2lpc -m 10 < data.lsp > data.lpc!
\end{quote}
\end{qsection}

\begin{qsection}{SEE ALSO}
lpc, lpc2lsp
\end{qsection}
