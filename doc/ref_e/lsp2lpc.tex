% ----------------------------------------------------------------- %
%             The Speech Signal Processing Toolkit (SPTK)           %
%             developed by SPTK Working Group                       %
%             http://sp-tk.sourceforge.net/                         %
% ----------------------------------------------------------------- %
%                                                                   %
%  Copyright (c) 1984-2007  Tokyo Institute of Technology           %
%                           Interdisciplinary Graduate School of    %
%                           Science and Engineering                 %
%                                                                   %
%                1996-2011  Nagoya Institute of Technology          %
%                           Department of Computer Science          %
%                                                                   %
% All rights reserved.                                              %
%                                                                   %
% Redistribution and use in source and binary forms, with or        %
% without modification, are permitted provided that the following   %
% conditions are met:                                               %
%                                                                   %
% - Redistributions of source code must retain the above copyright  %
%   notice, this list of conditions and the following disclaimer.   %
% - Redistributions in binary form must reproduce the above         %
%   copyright notice, this list of conditions and the following     %
%   disclaimer in the documentation and/or other materials provided %
%   with the distribution.                                          %
% - Neither the name of the SPTK working group nor the names of its %
%   contributors may be used to endorse or promote products derived %
%   from this software without specific prior written permission.   %
%                                                                   %
% THIS SOFTWARE IS PROVIDED BY THE COPYRIGHT HOLDERS AND            %
% CONTRIBUTORS "AS IS" AND ANY EXPRESS OR IMPLIED WARRANTIES,       %
% INCLUDING, BUT NOT LIMITED TO, THE IMPLIED WARRANTIES OF          %
% MERCHANTABILITY AND FITNESS FOR A PARTICULAR PURPOSE ARE          %
% DISCLAIMED. IN NO EVENT SHALL THE COPYRIGHT OWNER OR CONTRIBUTORS %
% BE LIABLE FOR ANY DIRECT, INDIRECT, INCIDENTAL, SPECIAL,          %
% EXEMPLARY, OR CONSEQUENTIAL DAMAGES (INCLUDING, BUT NOT LIMITED   %
% TO, PROCUREMENT OF SUBSTITUTE GOODS OR SERVICES; LOSS OF USE,     %
% DATA, OR PROFITS; OR BUSINESS INTERRUPTION) HOWEVER CAUSED AND ON %
% ANY THEORY OF LIABILITY, WHETHER IN CONTRACT, STRICT LIABILITY,   %
% OR TORT (INCLUDING NEGLIGENCE OR OTHERWISE) ARISING IN ANY WAY    %
% OUT OF THE USE OF THIS SOFTWARE, EVEN IF ADVISED OF THE           %
% POSSIBILITY OF SUCH DAMAGE.                                       %
% ----------------------------------------------------------------- %
\hypertarget{lsp2lpc}{}
\name{lsp2lpc}{transform LSP to LPC}{speech parameter transformation}

\begin{synopsis}
\item [lsp2lpc] [ --m $M$ ] [ --s $S$ ] [ --k ] [ --l ] [ --i $I$ ] [ {\em infile} ] 
\end{synopsis}

\begin{qsection}{DESCRIPTION}
{\em lsp2lpc} calculates linear prediction (LPC) coefficients 
from $M$-th order line spectral pair (LSP) coefficients 
from {\em infile} (or standard input),
sending the result to standard output.

The LSP input input format is
\begin{displaymath}
   [ K ], l(1), \dots , l(M), 
\end{displaymath}
and the LPC output format is
\begin{displaymath}
   K , a(1), \dots , a(M).
\end{displaymath}

By default, {\em lsp2lpc} assumes that 
the LSP input vectors include the gain $K$, 
and it passes that gain value through to the LPC output vectors.  
However, if the --k option is present, 
{\em lsp2lpc} assumes that $K$ is not present in the LSP input vectors, 
and it sets $K$ to $1.0$ in the LPC output vectors.
\end{qsection}

\begin{options}
	\argm{m}{M}{order of LPC}{25}
	\argm{s}{S}{sampling frequency (kHz)}{10.0}
	\argm{k}{}{input \& output gain}{TRUE}
	\argm{l}{}{regard input as log gain and output linear gain}{FALSE}
	\argm{i}{I}{input format\\
		\begin{tabular}{ll} \\[-1ex]
			$0$ & normalized frequency $(0 \dots \pi)$ \\
			$1$ & normalized frequency $(0 \dots 0.5)$ \\
			$2$ & frequency (kHz) \\
			$3$ & frequency (Hz)  \\
		\end{tabular}\\\hspace*{\fill}}{0}
\end{options}

\begin{qsection}{EXAMPLE}
In the example below, 10-th order LSP coefficients in float format
are read from file {\em data.lsp}, the linear prediction coefficients
are evaluated, and written to {\em data.lpc}:
\begin{quote}
\verb! lsp2lpc -m 10 < data.lsp > data.lpc!
\end{quote}
\end{qsection}

\begin{qsection}{SEE ALSO}
\hyperlink{lpc}{lpc},
\hyperlink{lpc2lsp}{lpc2lsp}
\end{qsection}
