% ----------------------------------------------------------------
%       Speech Signal Processing Toolkit (SPTK): version 3.0
%                      SPTK Working Group
% 
%                Department of Computer Science
%                Nagoya Institute of Technology
%                             and
%   Interdisciplinary Graduate School of Science and Engineering
%                Tokyo Institute of Technology
%                   Copyright (c) 1984-2000
%                     All Rights Reserved.
% 
% Permission is hereby granted, free of charge, to use and
% distribute this software and its documentation without
% restriction, including without limitation the rights to use,
% copy, modify, merge, publish, distribute, sublicense, and/or
% sell copies of this work, and to permit persons to whom this
% work is furnished to do so, subject to the following conditions:
% 
%   1. The code must retain the above copyright notice, this list
%      of conditions and the following disclaimer.
% 
%   2. Any modifications must be clearly marked as such.
%                                                                        
% NAGOYA INSTITUTE OF TECHNOLOGY, TOKYO INSITITUTE OF TECHNOLOGY,
% SPTK WORKING GROUP, AND THE CONTRIBUTORS TO THIS WORK DISCLAIM
% ALL WARRANTIES WITH REGARD TO THIS SOFTWARE, INCLUDING ALL
% IMPLIED WARRANTIES OF MERCHANTABILITY AND FITNESS, IN NO EVENT
% SHALL NAGOYA INSTITUTE OF TECHNOLOGY, TOKYO INSITITUTE OF
% TECHNOLOGY, SPTK WORKING GROUP, NOR THE CONTRIBUTORS BE LIABLE
% FOR ANY SPECIAL, INDIRECT OR CONSEQUENTIAL DAMAGES OR ANY
% DAMAGES WHATSOEVER RESULTING FROM LOSS OF USE, DATA OR PROFITS,
% WHETHER IN AN ACTION OF CONTRACT, NEGLIGENCE OR OTHER TORTIOUS
% ACTION, ARISING OUT OF OR IN CONNECTION WITH THE USE OR
% PERFORMANCE OF THIS SOFTWARE.
% ----------------------------------------------------------------
%
\hypertarget{ivq}{}
\name{ivq}{decoder of vector quantization}{vector quantization}

\begin{synopsis}
\item [ivq] [ --l $L$ ] [ --n $N$ ] {\em cbfile}  [ {\em infile} ] 
\end{synopsis}

\begin{qsection}{DESCRIPTION}
{\em ivq} decodes vector-quantized data from a sequence of codebook indexes
from {\em infile} (or standard input), 
using the codebook {\em cbfile}, 
sending the result to standard output. 
The decoded output vector format is
\begin{displaymath}
  c_i(0),c_i(1),\dots,c_i(L-1). 
\end{displaymath}

Input data is in int format, and output data is in float format.
\end{qsection}

\begin{options}
	\argm{l}{L}{length of vector}{26}
	\argm{n}{N}{order of vector}{L-1}
\end{options}

\begin{qsection}{EXAMPLE}
In the following example,
the decoded 25-th order output file {\em data.ivq} is obtained
through the index file {\em data.vq} and codebook {\em cbfile}.
\begin{quote}
\verb! ivq cbfile data.vq > data.ivq !
\end{quote}
\end{qsection}

\begin{qsection}{SEE ALSO}
\hyperlink{vq}{vq},
\hyperlink{imsvq}{imsvq},
\hyperlink{msvq}{msvq}
\end{qsection}
