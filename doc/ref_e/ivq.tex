% ----------------------------------------------------------------- %
%             The Speech Signal Processing Toolkit (SPTK)           %
%             developed by SPTK Working Group                       %
%             http://sp-tk.sourceforge.net/                         %
% ----------------------------------------------------------------- %
%                                                                   %
%  Copyright (c) 1984-2007  Tokyo Institute of Technology           %
%                           Interdisciplinary Graduate School of    %
%                           Science and Engineering                 %
%                                                                   %
%                1996-2010  Nagoya Institute of Technology          %
%                           Department of Computer Science          %
%                                                                   %
% All rights reserved.                                              %
%                                                                   %
% Redistribution and use in source and binary forms, with or        %
% without modification, are permitted provided that the following   %
% conditions are met:                                               %
%                                                                   %
% - Redistributions of source code must retain the above copyright  %
%   notice, this list of conditions and the following disclaimer.   %
% - Redistributions in binary form must reproduce the above         %
%   copyright notice, this list of conditions and the following     %
%   disclaimer in the documentation and/or other materials provided %
%   with the distribution.                                          %
% - Neither the name of the SPTK working group nor the names of its %
%   contributors may be used to endorse or promote products derived %
%   from this software without specific prior written permission.   %
%                                                                   %
% THIS SOFTWARE IS PROVIDED BY THE COPYRIGHT HOLDERS AND            %
% CONTRIBUTORS "AS IS" AND ANY EXPRESS OR IMPLIED WARRANTIES,       %
% INCLUDING, BUT NOT LIMITED TO, THE IMPLIED WARRANTIES OF          %
% MERCHANTABILITY AND FITNESS FOR A PARTICULAR PURPOSE ARE          %
% DISCLAIMED. IN NO EVENT SHALL THE COPYRIGHT OWNER OR CONTRIBUTORS %
% BE LIABLE FOR ANY DIRECT, INDIRECT, INCIDENTAL, SPECIAL,          %
% EXEMPLARY, OR CONSEQUENTIAL DAMAGES (INCLUDING, BUT NOT LIMITED   %
% TO, PROCUREMENT OF SUBSTITUTE GOODS OR SERVICES; LOSS OF USE,     %
% DATA, OR PROFITS; OR BUSINESS INTERRUPTION) HOWEVER CAUSED AND ON %
% ANY THEORY OF LIABILITY, WHETHER IN CONTRACT, STRICT LIABILITY,   %
% OR TORT (INCLUDING NEGLIGENCE OR OTHERWISE) ARISING IN ANY WAY    %
% OUT OF THE USE OF THIS SOFTWARE, EVEN IF ADVISED OF THE           %
% POSSIBILITY OF SUCH DAMAGE.                                       %
% ----------------------------------------------------------------- %
\hypertarget{ivq}{}
\name{ivq}{decoder of vector quantization}{vector quantization}

\begin{synopsis}
\item [ivq] [ --l $L$ ] [ --n $N$ ] {\em cbfile}  [ {\em infile} ] 
\end{synopsis}

\begin{qsection}{DESCRIPTION}
{\em ivq} decodes vector-quantized data from a sequence of codebook indexes
from {\em infile} (or standard input), 
using the codebook {\em cbfile}, 
sending the result to standard output. 
The decoded output vector format is
\begin{displaymath}
  c_i(0),c_i(1),\dots,c_i(L-1). 
\end{displaymath}

Input data is in int format, and output data is in float format.
\end{qsection}

\begin{options}
	\argm{l}{L}{length of vector}{26}
	\argm{n}{N}{order of vector}{L-1}
\end{options}

\begin{qsection}{EXAMPLE}
In the following example,
the decoded 25-th order output file {\em data.ivq} is obtained
through the index file {\em data.vq} and codebook {\em cbfile}.
\begin{quote}
\verb! ivq cbfile data.vq > data.ivq !
\end{quote}
\end{qsection}

\begin{qsection}{SEE ALSO}
\hyperlink{vq}{vq},
\hyperlink{imsvq}{imsvq},
\hyperlink{msvq}{msvq}
\end{qsection}
