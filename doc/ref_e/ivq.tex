\name{ivq}{decoder of vector quantization}{vector quantization}

\begin{synopsis}
\item [ivq] [ --l $L$ ] [ --n $N$ ] {\em cbfile}  [ {\em infile} ] 
\end{synopsis}

\begin{qsection}{DESCRIPTION}
The {\em ivq} command reads the codebook indexes $i$ from 
the assigned {\em infile}, as well as the codebook file {\em cbfile}
and output the decoded vector
\begin{displaymath}
  c_i(0),c_i(1),\ldots,c_i(L-1)
\end{displaymath}
to the starndard output.
\par
Input data is in int format, and output data is in float format.
\end{qsection}

\begin{options}
	\argm{l}{L}{length of vector}{26}
	\argm{n}{N}{order of vector}{L-1}
\end{options}

\begin{qsection}{EXAMPLE}
In the following example,
the decoded 25 order output file {\em data.ivq} is obtained
through the index file {\em data.vq} and codebook {\em cbfile}.
\begin{quote}
\verb! ivq cbfile data.vq > data.ivq !
\end{quote}
\end{qsection}

\begin{qsection}{SEE ALSO}
vq, imsvq, msvq
\end{qsection}
