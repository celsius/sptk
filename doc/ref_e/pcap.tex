% ----------------------------------------------------------------
%       Speech Signal Processing Toolkit (SPTK): version 3.3
%                      SPTK Working Group
% 
%                Department of Computer Science
%                Nagoya Institute of Technology
%                             and
%   Interdisciplinary Graduate School of Science and Engineering
%                Tokyo Institute of Technology
%                   Copyright (c) 1984-2000
%                     All Rights Reserved.
% 
% Permission is hereby granted, free of charge, to use and
% distribute this software and its documentation without
% restriction, including without limitation the rights to use,
% copy, modify, merge, publish, distribute, sublicense, and/or
% sell copies of this work, and to permit persons to whom this
% work is furnished to do so, subject to the following conditions:
% 
%   1. The code must retain the above copyright notice, this list
%      of conditions and the following disclaimer.
% 
%   2. Any modifications must be clearly marked as such.
%                                                                        
% NAGOYA INSTITUTE OF TECHNOLOGY, TOKYO INSITITUTE OF TECHNOLOGY,
% SPTK WORKING GROUP, AND THE CONTRIBUTORS TO THIS WORK DISCLAIM
% ALL WARRANTIES WITH REGARD TO THIS SOFTWARE, INCLUDING ALL
% IMPLIED WARRANTIES OF MERCHANTABILITY AND FITNESS, IN NO EVENT
% SHALL NAGOYA INSTITUTE OF TECHNOLOGY, TOKYO INSITITUTE OF
% TECHNOLOGY, SPTK WORKING GROUP, NOR THE CONTRIBUTORS BE LIABLE
% FOR ANY SPECIAL, INDIRECT OR CONSEQUENTIAL DAMAGES OR ANY
% DAMAGES WHATSOEVER RESULTING FROM LOSS OF USE, DATA OR PROFITS,
% WHETHER IN AN ACTION OF CONTRACT, NEGLIGENCE OR OTHER TORTIOUS
% ACTION, ARISING OUT OF OR IN CONNECTION WITH THE USE OR
% PERFORMANCE OF THIS SOFTWARE.
% ----------------------------------------------------------------
%
\name{pcap}{calculate principal component scores}{data processing}
\def\Vec#1{\mbox{\boldmath $#1$}}

\begin{synopsis}
 \item[pcap] [ --l $L$ ] [ --n $N$] {\em pcafile} [ {\em infile} ] 
\end{synopsis}

\begin{qsection}{DESCRIPTION}
 {\em pcap} calculates principal component scores
 from {\em infile} (or standard input) ,
 sending the result to standard output.

 The input data set must be composed of $L$-dimension,
 mean vector $\Vec{m}$ and eigen vectors $\Vec{e}(i)$:
\[
 \Vec{m}, \Vec{e}(0), \Vec{e}(1), \Vec{e}(2), \cdots
 \]
 \[
 where\;\;\Vec{m} = (m(1), m(2), \cdots, m(L))\;\;and\;\;
 \Vec{e}(i) = (e_{i}(1), e_{i}(2), \cdots, e_{i}(L))
\]

Input and output data are in float format. 
\end{qsection}

\begin{options}
 \argm{l}{L}{dimentionality of vector}{3}
 \argm{n}{N}{number principal components for output}{2}
\end{options}


\begin{qsection}{EXAMPLE}
 In the example below,
 the principal component scores are calculated
 from {\em test.dat} and sent to {\em score.dat},
 using {\em pca.dat}
 in which the mean vectors and the eigen vectors
 are contained.
\begin{quote}
  \verb!pcap pca.dat -l 3 -n 2 < test.dat > score.dat!
\end{quote} 
In the {\em pca.dat}, the mean vector must be in the front of
eigen vectors.
\end{qsection} 
\begin{qsection}{SEE ALSO}
 \hyperlink{pca}{pca}
\end{qsection}
