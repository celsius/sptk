\name{phase}{transform real sequence to phase}{signal processing}

\begin{synopsis}
\item[phase] [ --l $L$ ] [ --p {\em pfile} ] [ --z {\em zfile} ]
             [ --m $M$ ] [ --n $N$ ] [ {\em infile} ]
\end{synopsis}

\begin{qsection}{DESCRIPTION}
This command calculates the phase of the spectrum of a real sequence read
from the standard input.
Assume that the input sequence is
\begin{displaymath}
  x(0), x(1), \ldots, x(L-1)
\end{displaymath}
and the FFT is
\begin{eqnarray*}
  X_k &=& X(e^{j\omega}) \left|
	\begin{array}{c}
	\\
        \omega=\frac{2\pi k}{L}
	\end{array}
    \right. \nonumber \\
         &=& \sum_{m=0}^{L-1}x(m)e^{-j\omega m} \left|
	\begin{array}{c}
	\\
        \omega=\frac{2\pi\, k}{L}
	\end{array}
    \right.,\qquad k=0,1,\ldots,L-1
\end{eqnarray*}
Then the output is given by
\begin{displaymath}
  Y_k=\arg X_k, \qquad k=0,1,\ldots,L/2
\end{displaymath}
In this case the phase is written in continuous form.
The output data angular frequency varies from $0\sim \pi$.
Input and output data are in float format.
\par
If the {\bf --p, --z} options are assigned
then the phase of the corresponding filter related to
the assigned coefficients is calculated
\footnote{
In this case the phase is not evaluated from the filter
impulse response, the phase is evaluated from
the difference between the numerator and denominator phases}.
\end{qsection}

\begin{options}
	\argm{l}{L}{frame length power of 2}{256}
	\argm{p}{pfile}{numerator coefficients file\\
			The {\em pfile} should follow this structure in
                        float format:\\
			\hspace*{2zw}$K, a(1), \ldots, a(M)$\\[-1zh]}{NULL}
	\argm{z}{zfile}{denominator coefficients file\\
			The {\em zfile} should follow this structure in
                        float format:\\
			\hspace*{2zw}$b(0), b(1), \ldots, b(N)$\\
			The contents of {\em pfile} and {\em zfile}
                        should be in a similar form to that used in
                        command {\em dfs}.
			When only the {\bf --p} option is assigned
                        then the denominator is made equal to 1.
                        When only the {\bf --z} option is assigned
                        then the numerator and the gain $K$ are made
                        equal to 1.
                        If neither {\bf --p} nor {\bf --z} are
                        assigned, data is read from the standard input.}{NULL}
	\argm{m}{M}{order of denominator polynomial\\
			In the case where the number of input data
                        values is less then $M+1$, then $M$ is made
                        equal to the number of input data values $-1$.
                        If the input data should not be analyzed in blocks of
                        size $M+1$, then it is not necessary to assign
                        a value to $M$.}{$L-1$}
	\argm{n}{N}{order of numerator polynomial\\
                        Similarly to the --m option,
			in the case where the number of input data
                        values is less then $N+1$, then $N$ is made
                        equal to the number of input data values $-1$.
                        If the input data should not be analyzed in blocks of
                        size $N+1$, then it is not necessary to assign
                        a value to $N$.}{$L-1$}
	\argm{u}{}{unlapping}{TRUE}
\end{options}

\begin{qsection}{EXAMPLE}
In the example below, the phase characteristic of a digital filter
with coefficients assigned by the files {\em data.p, data.z} 
in float format is displayed:
\begin{quote}
  \verb!phase -p data.p -z data.z | fdrw | xgr !
\end{quote}
If the filter defined by {\em data.p}, {\em data.z} is stable
then the following command gives rise to a similar result:
\begin{quote}
  \verb!impulse | dfs -p data.p -z data.z | phase | fdrw | xgr !
\end{quote}
\end{qsection}

\begin{qsection}{SEE ALSO}
  spec, fft, fftr, dfs
\end{qsection}

\begin{qsection}{BUGS}
When the sample interval between FFT points is large
(the value assigned by the --l option is small),
when the phase characteristic includes steep angles
(when zeros and/or poles are close to the unit circle in the $z$
 domain), then sometimes phase is not properly drawn in continuous form.
\end{qsection}
