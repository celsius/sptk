\name[ref:GLSA-IEICEtaikai90s]{iglsadf}{inverse GLSA digital filter}%
{digital filter}

\begin{synopsis}
\item [iglsadf] [ --m $M$ ] [ --g $G$ ] [ --p $P$ ] [ --i $I$ ]
	  	[ --n ] [ --k ] {\em gcfile}  [ {\em infile} ] 
\end{synopsis}

\begin{qsection}{DESCRIPTION}
The {\em iglsadf} command is the inverse GLSA filter.
\par
Input and output data are in float format.

\end{qsection}

\begin{options}
	\argm{m}{M}{order of generalized cepstrum}{25}
	\argm{g}{G}{power parameter $\gamma=-1/G$ of generalized cepstrum}{1}
	\argm{p}{P}{frame period}{100}
	\argm{i}{I}{interpolation period}{1}
	\argm{n}{}{regard input as normalized generalized cepstrum}{FALSE}
	\argm{k}{}{filtering without gain}{FALSE}
\end{options}

\begin{qsection}{EXAMPLE}
In the first line of the example below, 
speech is read in float format from {\em data.f},
the generalized cepstrum coefficients $(M=15$B!$(B\gamma = -1/2)$
are evaluated, and written in {\em data.gc}.
In the second line,
speech data and cepstral coefficients are
inputed to {\em iglsadf} command and the resulting excitation file
is written to {\em data.e}.
\begin{quote}
 \verb!frame < data.f | window | gcep -m 15 -g 2 > data.gc!\\
 \verb!iglsadf -m 15 -g 2 data.gc < data.f > data.e!
\end{quote}
\end{qsection}

\begin{qsection}{BUGS}
Correct response of this command is obtained only when
$\gamma = -1/n$, with $n$ a natural number.
\end{qsection}
\begin{qsection}{SEE ALSO}
glsadf, gcep
\end{qsection}
