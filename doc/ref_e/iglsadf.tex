% ----------------------------------------------------------------
%       Speech Signal Processing Toolkit (SPTK): version 3.0
%                      SPTK Working Group
% 
%                Department of Computer Science
%                Nagoya Institute of Technology
%                             and
%   Interdisciplinary Graduate School of Science and Engineering
%                Tokyo Institute of Technology
%                   Copyright (c) 1984-2000
%                     All Rights Reserved.
% 
% Permission is hereby granted, free of charge, to use and
% distribute this software and its documentation without
% restriction, including without limitation the rights to use,
% copy, modify, merge, publish, distribute, sublicense, and/or
% sell copies of this work, and to permit persons to whom this
% work is furnished to do so, subject to the following conditions:
% 
%   1. The code must retain the above copyright notice, this list
%      of conditions and the following disclaimer.
% 
%   2. Any modifications must be clearly marked as such.
%                                                                        
% NAGOYA INSTITUTE OF TECHNOLOGY, TOKYO INSITITUTE OF TECHNOLOGY,
% SPTK WORKING GROUP, AND THE CONTRIBUTORS TO THIS WORK DISCLAIM
% ALL WARRANTIES WITH REGARD TO THIS SOFTWARE, INCLUDING ALL
% IMPLIED WARRANTIES OF MERCHANTABILITY AND FITNESS, IN NO EVENT
% SHALL NAGOYA INSTITUTE OF TECHNOLOGY, TOKYO INSITITUTE OF
% TECHNOLOGY, SPTK WORKING GROUP, NOR THE CONTRIBUTORS BE LIABLE
% FOR ANY SPECIAL, INDIRECT OR CONSEQUENTIAL DAMAGES OR ANY
% DAMAGES WHATSOEVER RESULTING FROM LOSS OF USE, DATA OR PROFITS,
% WHETHER IN AN ACTION OF CONTRACT, NEGLIGENCE OR OTHER TORTIOUS
% ACTION, ARISING OUT OF OR IN CONNECTION WITH THE USE OR
% PERFORMANCE OF THIS SOFTWARE.
% ----------------------------------------------------------------
%
\name[ref:GLSA-IEICEtaikai90s]{iglsadf}{inverse GLSA digital filter}%
{filters for speech synthesis}

\begin{synopsis}
\item [iglsadf] [ --m $M$ ] [ --g $G$ ] [ --p $P$ ] [ --i $I$ ]
	  	[ --n ] [ --k ] {\em gcfile}  [ {\em infile} ] 
\end{synopsis}

\begin{qsection}{DESCRIPTION}
The {\em iglsadf} command is the inverse GLSA filter.
\par
Input and output data are in float format.

\end{qsection}

\begin{options}
	\argm{m}{M}{order of generalized cepstrum}{25}
	\argm{g}{G}{power parameter $\gamma=-1/G$ of generalized cepstrum}{1}
	\argm{p}{P}{frame period}{100}
	\argm{i}{I}{interpolation period}{1}
	\argm{n}{}{regard input as normalized generalized cepstrum}{FALSE}
	\argm{k}{}{filtering without gain}{FALSE}
\end{options}

\begin{qsection}{EXAMPLE}
In the first line of the example below, 
speech is read in float format from {\em data.f},
the generalized cepstrum coefficients $(M=15,\gamma = -1/2)$
are evaluated, and written in {\em data.gc}.
In the second line,
speech data and cepstral coefficients are
inputed to {\em iglsadf} command and the resulting excitation file
is written to {\em data.e}.
\begin{quote}
 \verb!frame < data.f | window | gcep -m 15 -g 2 > data.gc!\\
 \verb!iglsadf -m 15 -g 2 data.gc < data.f > data.e!
\end{quote}
\end{qsection}

\begin{qsection}{BUGS}
Correct response of this command is obtained only when
$\gamma = -1/n$, with $n$ a natural number.
\end{qsection}
\begin{qsection}{SEE ALSO}
glsadf, gcep
\end{qsection}
