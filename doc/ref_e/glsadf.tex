% ----------------------------------------------------------------- %
%             The Speech Signal Processing Toolkit (SPTK)           %
%             developed by SPTK Working Group                       %
%             http://sp-tk.sourceforge.net/                         %
% ----------------------------------------------------------------- %
%                                                                   %
%  Copyright (c) 1984-2007  Tokyo Institute of Technology           %
%                           Interdisciplinary Graduate School of    %
%                           Science and Engineering                 %
%                                                                   %
%                1996-2011  Nagoya Institute of Technology          %
%                           Department of Computer Science          %
%                                                                   %
% All rights reserved.                                              %
%                                                                   %
% Redistribution and use in source and binary forms, with or        %
% without modification, are permitted provided that the following   %
% conditions are met:                                               %
%                                                                   %
% - Redistributions of source code must retain the above copyright  %
%   notice, this list of conditions and the following disclaimer.   %
% - Redistributions in binary form must reproduce the above         %
%   copyright notice, this list of conditions and the following     %
%   disclaimer in the documentation and/or other materials provided %
%   with the distribution.                                          %
% - Neither the name of the SPTK working group nor the names of its %
%   contributors may be used to endorse or promote products derived %
%   from this software without specific prior written permission.   %
%                                                                   %
% THIS SOFTWARE IS PROVIDED BY THE COPYRIGHT HOLDERS AND            %
% CONTRIBUTORS "AS IS" AND ANY EXPRESS OR IMPLIED WARRANTIES,       %
% INCLUDING, BUT NOT LIMITED TO, THE IMPLIED WARRANTIES OF          %
% MERCHANTABILITY AND FITNESS FOR A PARTICULAR PURPOSE ARE          %
% DISCLAIMED. IN NO EVENT SHALL THE COPYRIGHT OWNER OR CONTRIBUTORS %
% BE LIABLE FOR ANY DIRECT, INDIRECT, INCIDENTAL, SPECIAL,          %
% EXEMPLARY, OR CONSEQUENTIAL DAMAGES (INCLUDING, BUT NOT LIMITED   %
% TO, PROCUREMENT OF SUBSTITUTE GOODS OR SERVICES; LOSS OF USE,     %
% DATA, OR PROFITS; OR BUSINESS INTERRUPTION) HOWEVER CAUSED AND ON %
% ANY THEORY OF LIABILITY, WHETHER IN CONTRACT, STRICT LIABILITY,   %
% OR TORT (INCLUDING NEGLIGENCE OR OTHERWISE) ARISING IN ANY WAY    %
% OUT OF THE USE OF THIS SOFTWARE, EVEN IF ADVISED OF THE           %
% POSSIBILITY OF SUCH DAMAGE.                                       %
% ----------------------------------------------------------------- %
\hypertarget{glsadf}{}
\name[ref:GLSA-IEICEtaikai90s]{glsadf}{GLSA digital filter for speech synthesis}%
{filters for speech synthesis}

\begin{synopsis}
\item [glsadf] [ --m $M$ ] [ --c $C$ ] [ --p $P$ ] [ --i $I$ ] [ --v ] [ --t ] [ --n ] [ --k ] [ --P $Pa$ ]
 {\em gcfile} 
\item [\ ~~~~~~~~~] [ {\em infile} ]
\end{synopsis}

\begin{qsection}{DESCRIPTION}
{\em glsadf} derives a Generalized Log Spectral Approximation digital filter 
from normalized generalized cepstral coefficients in {\em gcfile} 
and uses it to filter an excitation sequence 
from {\em infile} (or standard input) to synthesize speech data, 
sending the result to standard output.
The cepstral coefficients can be be represented as
$K,c_\gamma'(1),\dots,c_\gamma'(M)$. 

Input and output data are in float format.

The transfer function $H(z)$ are synthesis filter based on an $M$ order
normalized generalized cepstral coefficients $c_\gamma'(m)$ is 
\begin{align}
H(z) &= K \cdot D(z) \notag \\
     &= \begin{cases} \;\;\displaystyle
          K \cdot \left( 1+\gamma\sum_{m=1}^{M} c_\gamma'(m) z^{-m}
                \right)^{1/\gamma}, & 0<\gamma\leq -1 \\ 
                \;\;\displaystyle K \cdot \exp \sum_{m=1}^{M} c_\gamma'(m) z^{-m}, & \gamma=0
        \end{cases}\notag
\end{align}
In this case, we are considering only values for the power parameter
$\gamma=-1/C$, where $C$ is a natural number.
The filter $D(z)$ can be realized through a $C$ level cascade as shown
in figure\ref{fig:glsadflt_GLSA}, where
\begin{displaymath}
\frac{1}{C(z)} = \frac{1}
                {\displaystyle 1+\gamma\sum_{m=1}^{M} c_\gamma'(m) z^{-m}}
\end{displaymath}

\setcounter{figure}{0}
\begin{figure}[h]
\setlength{\unitlength}{0.3mm}
\begin{center}
\begin{picture}(300,80)(10,0)
  \thicklines
  \put(40,10){\framebox(50,40){\Large $\frac{1}{C(z)}$}}
  \put(110,10){\framebox(50,40){\Large $\frac{1}{C(z)}$}}
  \put(240,10){\framebox(50,40){\Large $\frac{1}{C(z)}$}}

  \put(10,30){\vector(1,0){30}}
  \put(90,30){\line(1,0){20}}
  \put(160,30){\line(1,0){20}}
  \put(220,30){\line(1,0){20}}
  \put(290,30){\vector(1,0){30}}

  \put(200,30){\makebox(0,0){$\cdot\cdot\cdot$}}
  \put(10,40){\makebox(0,0){Input}}
  \put(320,40){\makebox(0,0){Output}}
  \put(60,65){\makebox(0,0){\bf level 1}}
  \put(140,65){\makebox(0,0){\bf level 2}}
  \put(270,65){\makebox(0,0){\bf level $C$}}

\end{picture}
\caption{Structure of filter $D(z)$}
\label{fig:glsadflt_GLSA}
\end{center}
\end{figure}
\end{qsection}

\newpage
\begin{options}
        \argm{m}{M}{order of generalized cepstrum}{25}
        \argm{c}{C}{power parameter $\gamma=-1/C$ for generalized cepstrum\\
                         if $C==0$ then the LMA filter is used}{1}
        \argm{p}{P}{frame period}{100}
        \argm{i}{I}{interpolation period}{1}
        \argm{n}{}{regard input as normalized generalized cepstrum}{FALSE}
        \argm{v}{}{inverse filter}{FALSE}
        \argm{t}{}{transpose filter}{FALSE}
        \argm{k}{}{filtering without gain}{FALSE}
        \desc[1ex]{The option below only works if $C==0$.}
        \argm{P}{Pa}{order of the Pad\'e approximation\\
                     $Pa$ should be $4$ or $5$}{4}
\end{options}

\begin{qsection}{EXAMPLE}
In this example, excitation is generated through the pitch data
in the file {\em data.pitch} in float format, passed through a
GLSA filter based on generalized cepstral coefficient file
{\em data.gcep}, and the synthesized speech is output to
{\em data.syn}:
\begin{quote}
 \verb!excite < data.pitch | glsadf data.gcep > data.syn!
\end{quote} 
\end{qsection}

\begin{qsection}{SEE ALSO}
\hyperlink{ltcdf}{ltcdf},
\hyperlink{lmadf}{lmadf},
\hyperlink{lspdf}{lspdf},
\hyperlink{mlsadf}{mlsadf},
\hyperlink{mglsadf}{mglsadf}
\end{qsection}
