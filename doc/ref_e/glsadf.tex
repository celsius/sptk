\name[ref:GLSA-IEICEtaikai90s]{glsadf}{GLSA digital filter for speech synthesis}%
{digital filter}

\begin{synopsis}
\item [glsadf] [ --m $M$ ] [ --g $G$ ] [ --p $P$ ] [ --i $I$ ] [ --n ] [ --k ]
	       {\em gcfile}  [ {\em infile} ]
\end{synopsis}

\begin{qsection}{DESCRIPTION}
This command outputs speech data synthesized from excitation information
contained in {\em infile} passed through normalized generalized cepstrum
coefficients $K,c_\gamma'(1),\ldots,c_\gamma'(M)$ in the file {\em gcfile}.
\par
Input and output data are in float format.
\par
The transfer function $H(z)$ are synthesis filter based on an $M$ order
normalized generalized cepstrum coefficients $c_\gamma'(m)$ is 
\begin{eqnarray*}
H(z) &=& K \cdot D(z) \\
     &=& \left\{ 
	\begin{array}{ll} \displaystyle
          K \cdot \left( 1+\gamma\sum_{m=1}^{M} c_\gamma'(m) z^{-m}
		\right)^{1/\gamma}, & 0<\gamma\leq -1 \\ \displaystyle
	  K \cdot \exp \sum_{m=1}^{M} c_\gamma'(m) z^{-m}, & \gamma=0
	\end{array} \right.
\end{eqnarray*}
In this case, we are considering only values for the power parameter
$\gamma=-1/G$, where $G$ is a natural number.
The filter $D(z)$ can be realized through a $G$ level cascade as shown
in figure\ref{fig:glsadflt_GLSA}, where
\begin{displaymath}
\frac{1}{C(z)} = \frac{1}
		{\displaystyle 1+\gamma\sum_{m=1}^{M} c_\gamma'(m) z^{-m}}
\end{displaymath}

\setcounter{figure}{0}
\begin{figure}[h]
\setlength{\unitlength}{0.3mm}
\begin{center}
\begin{picture}(300,80)(10,0)
  \thicklines
  \put(40,10){\framebox(50,40){\Large $\frac{1}{C(z)}$}}
  \put(110,10){\framebox(50,40){\Large $\frac{1}{C(z)}$}}
  \put(240,10){\framebox(50,40){\Large $\frac{1}{C(z)}$}}

  \put(10,30){\vector(1,0){30}}
  \put(90,30){\line(1,0){20}}
  \put(160,30){\line(1,0){20}}
  \put(220,30){\line(1,0){20}}
  \put(290,30){\vector(1,0){30}}

  \put(200,30){\makebox(0,0){$B!&!&!&(B}}
  \put(10,40){\makebox(0,0){Input}}
  \put(320,40){\makebox(0,0){Output}}
  \put(60,65){\makebox(0,0){\bf level 1}}
  \put(140,65){\makebox(0,0){\bf level 2}}
  \put(270,65){\makebox(0,0){\bf level $G$}}

\end{picture}
\caption{Structure of filter $D(z)$}
\label{fig:glsadflt_GLSA}
\end{center}
\end{figure}
\end{qsection}

\newpage
\begin{options}
	\argm{m}{M}{order of generalized cepstrum}{25}
	\argm{g}{G}{power parameter $\gamma=-1/G$ for generalized cepstrum}{1}
	\argm{p}{P}{frame period}{100}
	\argm{i}{I}{interpolation period}{1}
	\argm{n}{}{regard input as normalized generalized cepstrum}{FALSE}
	\argm{k}{}{filtering without gain}{FALSE}
\end{options}

\begin{qsection}{EXAMPLE}
In this example, excitation is generated through the pitch data
in the file {\em data.pitch} in float format, passed through a
GLSA filter based on generalized cepstrum coefficient file
{\em data.gcep}, and the synthesized speech is outputed to
{\em data.syn}:
\begin{quote}
 \verb!excite < data.pitch | glsadf data.gcep > data.syn!
\end{quote} 
\end{qsection}

\begin{qsection}{BUGS}
Correct response of this command is obtained only when
$\gamma = -1/n$, with $n$ a natural number.
\end{qsection}

\begin{qsection}{SEE ALSO}
 ltcdf, lmadf, lspdf, mlsadf, mglsadf
\end{qsection}
