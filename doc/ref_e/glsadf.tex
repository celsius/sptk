% ----------------------------------------------------------------
%       Speech Signal Processing Toolkit (SPTK): version 3.0
%                      SPTK Working Group
% 
%                Department of Computer Science
%                Nagoya Institute of Technology
%                             and
%   Interdisciplinary Graduate School of Science and Engineering
%                Tokyo Institute of Technology
%                   Copyright (c) 1984-2000
%                     All Rights Reserved.
% 
% Permission is hereby granted, free of charge, to use and
% distribute this software and its documentation without
% restriction, including without limitation the rights to use,
% copy, modify, merge, publish, distribute, sublicense, and/or
% sell copies of this work, and to permit persons to whom this
% work is furnished to do so, subject to the following conditions:
% 
%   1. The code must retain the above copyright notice, this list
%      of conditions and the following disclaimer.
% 
%   2. Any modifications must be clearly marked as such.
%                                                                        
% NAGOYA INSTITUTE OF TECHNOLOGY, TOKYO INSITITUTE OF TECHNOLOGY,
% SPTK WORKING GROUP, AND THE CONTRIBUTORS TO THIS WORK DISCLAIM
% ALL WARRANTIES WITH REGARD TO THIS SOFTWARE, INCLUDING ALL
% IMPLIED WARRANTIES OF MERCHANTABILITY AND FITNESS, IN NO EVENT
% SHALL NAGOYA INSTITUTE OF TECHNOLOGY, TOKYO INSITITUTE OF
% TECHNOLOGY, SPTK WORKING GROUP, NOR THE CONTRIBUTORS BE LIABLE
% FOR ANY SPECIAL, INDIRECT OR CONSEQUENTIAL DAMAGES OR ANY
% DAMAGES WHATSOEVER RESULTING FROM LOSS OF USE, DATA OR PROFITS,
% WHETHER IN AN ACTION OF CONTRACT, NEGLIGENCE OR OTHER TORTIOUS
% ACTION, ARISING OUT OF OR IN CONNECTION WITH THE USE OR
% PERFORMANCE OF THIS SOFTWARE.
% ----------------------------------------------------------------
%
\hypertarget{glsadf}{}
\name[ref:GLSA-IEICEtaikai90s]{glsadf}{GLSA digital filter for speech synthesis}%
{filters for speech synthesis}

\begin{synopsis}
\item [glsadf] [ --m $M$ ] [ --g $G$ ] [ --p $P$ ] [ --i $I$ ] [ --n ] [ --k ] [ --P $Pa$ ]
 {\em gcfile}  [ {\em infile} ]
\end{synopsis}

\begin{qsection}{DESCRIPTION}
{\em glsadf} derives a Generalized Log Spectral Approximation digital filter 
from normalized generalized cepstral coefficients in {\em gcfile} 
and uses it to filter an excitation sequence 
from {\em infile} (or standard input) to synthesize speech data, 
sending the result to standard output.
The cepstral coefficients can be be represented as
$K,c_\gamma'(1),\dots,c_\gamma'(M)$. 

Input and output data are in float format.

The transfer function $H(z)$ are synthesis filter based on an $M$ order
normalized generalized cepstral coefficients $c_\gamma'(m)$ is 
\begin{align}
H(z) &= K \cdot D(z) \notag \\
     &= \begin{cases} \;\;\displaystyle
          K \cdot \left( 1+\gamma\sum_{m=1}^{M} c_\gamma'(m) z^{-m}
		\right)^{1/\gamma}, & 0<\gamma\leq -1 \\ 
		\;\;\displaystyle K \cdot \exp \sum_{m=1}^{M} c_\gamma'(m) z^{-m}, & \gamma=0
	\end{cases}\notag
\end{align}
In this case, we are considering only values for the power parameter
$\gamma=-1/G$, where $G$ is a natural number.
The filter $D(z)$ can be realized through a $G$ level cascade as shown
in figure\ref{fig:glsadflt_GLSA}, where
\begin{displaymath}
\frac{1}{C(z)} = \frac{1}
		{\displaystyle 1+\gamma\sum_{m=1}^{M} c_\gamma'(m) z^{-m}}
\end{displaymath}

\setcounter{figure}{0}
\begin{figure}[h]
\setlength{\unitlength}{0.3mm}
\begin{center}
\begin{picture}(300,80)(10,0)
  \thicklines
  \put(40,10){\framebox(50,40){\Large $\frac{1}{C(z)}$}}
  \put(110,10){\framebox(50,40){\Large $\frac{1}{C(z)}$}}
  \put(240,10){\framebox(50,40){\Large $\frac{1}{C(z)}$}}

  \put(10,30){\vector(1,0){30}}
  \put(90,30){\line(1,0){20}}
  \put(160,30){\line(1,0){20}}
  \put(220,30){\line(1,0){20}}
  \put(290,30){\vector(1,0){30}}

  \put(200,30){\makebox(0,0){$\cdot\cdot\cdot$}}
  \put(10,40){\makebox(0,0){Input}}
  \put(320,40){\makebox(0,0){Output}}
  \put(60,65){\makebox(0,0){\bf level 1}}
  \put(140,65){\makebox(0,0){\bf level 2}}
  \put(270,65){\makebox(0,0){\bf level $G$}}

\end{picture}
\caption{Structure of filter $D(z)$}
\label{fig:glsadflt_GLSA}
\end{center}
\end{figure}
\end{qsection}

\newpage
\begin{options}
	\argm{m}{M}{order of generalized cepstrum}{25}
	\argm{g}{G}{power parameter $\gamma=-1/G$ for generalized cepstrum\\
			 if $G==0$ then the LMA filter is used}{1}
	\argm{p}{P}{frame period}{100}
	\argm{i}{I}{interpolation period}{1}
	\argm{n}{}{regard input as normalized generalized cepstrum}{FALSE}
	\argm{k}{}{filtering without gain}{FALSE}
	\desc[1ex]{The option below only works if $G==0$.}
	\argm{P}{Pa}{order of the Pad\'e approximation\\
                     $Pa$ should be $4$ or $5$}{4}
\end{options}

\begin{qsection}{EXAMPLE}
In this example, excitation is generated through the pitch data
in the file {\em data.pitch} in float format, passed through a
GLSA filter based on generalized cepstral coefficient file
{\em data.gcep}, and the synthesized speech is output to
{\em data.syn}:
\begin{quote}
 \verb!excite < data.pitch | glsadf data.gcep > data.syn!
\end{quote} 
\end{qsection}

\begin{qsection}{SEE ALSO}
\hyperlink{ltcdf}{ltcdf},
\hyperlink{lmadf}{lmadf},
\hyperlink{lspdf}{lspdf},
\hyperlink{mlsadf}{mlsadf},
\hyperlink{mglsadf}{mglsadf}
\end{qsection}
