\name{mc2b}{transform mel-cepstrum to MLSA digital filter coefficients}%
{speech parameter transformation}

\begin{synopsis}
 \item [mc2b] [ --a $A$ ] [ --m $M$ ] [ {\em infile} ]
\end{synopsis}

\begin{qsection}{DESCRIPTION}
This command transforms mel-cepstrum coefficients $c_\alpha(m)$ 
into MLSA filter coefficients $b(m)$ and sends the results
to the standard output.
\par
Input and output data are in float format.
\par
The equations are used for this transformation follows.
\begin{displaymath}
b(m) = \left\{
	\begin{array}{ll}
	  c_\alpha(M), & m=M \\
	  c_\alpha(m) - \alpha b(m+1), & 0 \leq m < M \\
	\end{array} \right.
\end{displaymath}
These coefficients $b(m)$ can be directory used in the
implementation of a MLSA filter.
This transformation is the inverse transformation undertaken
by the command ``b2mc''.
\end{qsection}

\begin{options}
	\argm{a}{A}{all-pass constant $\alpha$��}{0.35}
	\argm{m}{M}{order of mel-cepstrum}{25}
\end{options}

\begin{qsection}{EXAMPLE}
Speech data is read in float format from {\em data.f},
a 12 order mel-cepstrum analysis is undertaken,
these mel-cepstrum coefficients are transformed into
MLSA filter coefficients, and these coefficients $b(m)$
are written to {\em data.b}:
\begin{quote}
 \verb!frame < data.f | window | mcep -m 12 | mc2b -m 12 > data.b!
\end{quote} 
\end{qsection}

\begin{qsection}{SEE ALSO}
 mlsadf, mglsadf, b2mc, mcep, mgcep, amcep
\end{qsection}
