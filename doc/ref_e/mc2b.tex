% ----------------------------------------------------------------
%       Speech Signal Processing Toolkit (SPTK): version 3.0
%                      SPTK Working Group
% 
%                Department of Computer Science
%                Nagoya Institute of Technology
%                             and
%   Interdisciplinary Graduate School of Science and Engineering
%                Tokyo Institute of Technology
%                   Copyright (c) 1984-2000
%                     All Rights Reserved.
% 
% Permission is hereby granted, free of charge, to use and
% distribute this software and its documentation without
% restriction, including without limitation the rights to use,
% copy, modify, merge, publish, distribute, sublicense, and/or
% sell copies of this work, and to permit persons to whom this
% work is furnished to do so, subject to the following conditions:
% 
%   1. The code must retain the above copyright notice, this list
%      of conditions and the following disclaimer.
% 
%   2. Any modifications must be clearly marked as such.
%                                                                        
% NAGOYA INSTITUTE OF TECHNOLOGY, TOKYO INSITITUTE OF TECHNOLOGY,
% SPTK WORKING GROUP, AND THE CONTRIBUTORS TO THIS WORK DISCLAIM
% ALL WARRANTIES WITH REGARD TO THIS SOFTWARE, INCLUDING ALL
% IMPLIED WARRANTIES OF MERCHANTABILITY AND FITNESS, IN NO EVENT
% SHALL NAGOYA INSTITUTE OF TECHNOLOGY, TOKYO INSITITUTE OF
% TECHNOLOGY, SPTK WORKING GROUP, NOR THE CONTRIBUTORS BE LIABLE
% FOR ANY SPECIAL, INDIRECT OR CONSEQUENTIAL DAMAGES OR ANY
% DAMAGES WHATSOEVER RESULTING FROM LOSS OF USE, DATA OR PROFITS,
% WHETHER IN AN ACTION OF CONTRACT, NEGLIGENCE OR OTHER TORTIOUS
% ACTION, ARISING OUT OF OR IN CONNECTION WITH THE USE OR
% PERFORMANCE OF THIS SOFTWARE.
% ----------------------------------------------------------------
%
\hypertarget{mc2b}{}
\name{mc2b}{transform mel-cepstrum to MLSA digital filter coefficients}%
{speech parameter transformation}

\begin{synopsis}
 \item [mc2b] [ --a $A$ ] [ --m $M$ ] [ {\em infile} ]
\end{synopsis}

\begin{qsection}{DESCRIPTION}
{\em mc2b} calculates MLSA filter coefficients $b(m)$ 
from mel-cepstral coefficients $c_\alpha(m)$ 
from {\em infile} (or standard input), 
sending the result to standard output.

Input and output data are in float format.

The equations are used for this transformation follows.
\begin{displaymath}
b(m) = \begin{cases}
	  \;\; c_\alpha(M), & m=M \\
	  \;\; c_\alpha(m) - \alpha b(m+1), & 0 \leq m < M \\
	\end{cases}
\end{displaymath}
These coefficients $b(m)$ can be directory used in the
implementation of a MLSA filter.
This transformation is the inverse transformation undertaken
by the command \hyperlink{b2mc}{b2mc}.
\end{qsection}

\begin{options}
	\argm{a}{A}{all-pass constant $\alpha$}{0.35}
	\argm{m}{M}{order of mel-cepstrum}{25}
\end{options}

\begin{qsection}{EXAMPLE}
Speech data is read in float format from {\em data.f},
a 12-th order mel-cepstral analysis is undertaken,
these mel-cepstral coefficients are transformed into
MLSA filter coefficients, and these coefficients $b(m)$
are written to {\em data.b}:
\begin{quote}
 \verb!frame < data.f | window | mcep -m 12 | mc2b -m 12 > data.b!
\end{quote} 
\end{qsection}

\begin{qsection}{SEE ALSO}
\hyperlink{mlsadf}{mlsadf},
\hyperlink{mglsadf}{mglsadf},
\hyperlink{b2mc}{b2mc},
\hyperlink{mcep}{mcep},
\hyperlink{mgcep}{mgcep},
\hyperlink{amcep}{amcep}
\end{qsection}
