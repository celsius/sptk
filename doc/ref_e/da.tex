% ----------------------------------------------------------------- %
%             The Speech Signal Processing Toolkit (SPTK)           %
%             developed by SPTK Working Group                       %
%             http://sp-tk.sourceforge.net/                         %
% ----------------------------------------------------------------- %
%                                                                   %
%  Copyright (c) 1984-2007  Tokyo Institute of Technology           %
%                           Interdisciplinary Graduate School of    %
%                           Science and Engineering                 %
%                                                                   %
%                1996-2009  Nagoya Institute of Technology          %
%                           Department of Computer Science          %
%                                                                   %
% All rights reserved.                                              %
%                                                                   %
% Redistribution and use in source and binary forms, with or        %
% without modification, are permitted provided that the following   %
% conditions are met:                                               %
%                                                                   %
% - Redistributions of source code must retain the above copyright  %
%   notice, this list of conditions and the following disclaimer.   %
% - Redistributions in binary form must reproduce the above         %
%   copyright notice, this list of conditions and the following     %
%   disclaimer in the documentation and/or other materials provided %
%   with the distribution.                                          %
% - Neither the name of the SPTK working group nor the names of its %
%   contributors may be used to endorse or promote products derived %
%   from this software without specific prior written permission.   %
%                                                                   %
% THIS SOFTWARE IS PROVIDED BY THE COPYRIGHT HOLDERS AND            %
% CONTRIBUTORS "AS IS" AND ANY EXPRESS OR IMPLIED WARRANTIES,       %
% INCLUDING, BUT NOT LIMITED TO, THE IMPLIED WARRANTIES OF          %
% MERCHANTABILITY AND FITNESS FOR A PARTICULAR PURPOSE ARE          %
% DISCLAIMED. IN NO EVENT SHALL THE COPYRIGHT OWNER OR CONTRIBUTORS %
% BE LIABLE FOR ANY DIRECT, INDIRECT, INCIDENTAL, SPECIAL,          %
% EXEMPLARY, OR CONSEQUENTIAL DAMAGES (INCLUDING, BUT NOT LIMITED   %
% TO, PROCUREMENT OF SUBSTITUTE GOODS OR SERVICES; LOSS OF USE,     %
% DATA, OR PROFITS; OR BUSINESS INTERRUPTION) HOWEVER CAUSED AND ON %
% ANY THEORY OF LIABILITY, WHETHER IN CONTRACT, STRICT LIABILITY,   %
% OR TORT (INCLUDING NEGLIGENCE OR OTHERWISE) ARISING IN ANY WAY    %
% OUT OF THE USE OF THIS SOFTWARE, EVEN IF ADVISED OF THE           %
% POSSIBILITY OF SUCH DAMAGE.                                       %
% ----------------------------------------------------------------- %
\hypertarget{da}{}
\name{da}{play 16-bit linear PCM data}{DA transformation}
\begin{synopsis}
\item [da] [ --s $S$ ] [ --c $C$ ] [ --g $G$ ] [ --a $A$ ] [ --o $O$ ] 
	   [ --w ] [ --H $H$ ]
\item [\ ~~~] [ --v ] [ +$type$ ] [ {\em infile1} ] [ {\em infile2} ] ...
\end{synopsis}

\begin{qsection}{DESCRIPTION}
{\em da} plays a series of input files (or standard input) 
on a system-dependent audio output device.
If the system does not support the specified sampling frequency, 
{\em da} up-samples the data to a supported frequency.
This command can be used under
Linux (i386), FreeBSD (i386 newpcm driver), SunOS 4.1.x, SunOS 5.x (SPARC).
 
It is possible to change environment setting through following options

\begin{tabular}{ll}
%% DA\_SAMPFREQ & sampling frequency\\
DA\_GAIN & gain\\
DA\_AMPGAIN & amplitude gain\\
DA\_PORT & output port\\
DA\_HDRSIZE & header size\\
DA\_FLOAT & set the input data to float\\
\end{tabular}

\end{qsection}

\begin{options}
	\argm{s}{S}{sampling frequency, it can be used the following
 sampling frequencies 8, 10, 11.025, 12, 16, 20, 22.05, 32, 44.1, 48 (kHz).}{10}
	\argm{g}{G}{gain}{0}
	\argm{a}{A}{amplitude gain(0..100)}{N/A}
	\argm{o}{O}{output port(s : speaker, h : headphone)}{s}
	\argm{w}{}{execute byte swap}{FALSE}
	\argm{H}{H}{header size in byte}{0}
	\argm{v}{}{display filename}{FALSE}
	\argp{type}{data format (s: short, f: float)}{s}
\end{options}

\begin{qsection}{EXAMPLE}
In the following example, the speech data file {\em data.s}
is played on the headphone.
The sampling frequency is 8 kHz, and data is in short format.
\begin{quote}
\verb! da +s -s 8 -o h data.s!
\end{quote}
\end{qsection}

\begin{qsection}{BUGS}
In the Linux operating system, the output port can not be assigned.
\end{qsection}

