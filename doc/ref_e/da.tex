\name{da}{play 16-bit linear PCM data}{DA transformation}
\begin{synopsis}
\item [da] [ --s $S$ ] [ --c $C$ ] [ --g $G$ ] [ --a $A$ ] [ --o $O$ ] 
	   [ --w ] [ --H $H$ ]
\item [\ ~~~] [ --v ] [ +$type$ ] [ {\em infile1} ] [ {\em infile2} ] ...
\end{synopsis}

\begin{qsection}{DESCRIPTION}
 This command plays the input file.
 The output can be sent to the speaker or the headphone.
 If the input file is not assigned, data is read from the starndard
 input.
 If the sampling frequency of the audio device is not supported,
 then up sampling to an appropriate sampling frequency is done, and
 the sound is played.

 This command can be used under
 Linux (i386), SunOS 4.1.x��SunOS 5.x (SPARC).
 
 It is possible to change environment setting through following options

\begin{tabular}{ll}
%% DA\_SAMPFREQ & sampling frequency\\
DA\_GAIN & gain\\
DA\_AMPGAIN & amplitude gain\\
DA\_PORT & output port\\
DA\_HDRSIZE & header size\\
DA\_FLOAT & set the input data to float\\
\end{tabular}

\end{qsection}

\begin{options}
	\argm{s}{S}{sampling frequency, we can be used the following
 sampling frequency 8,10,11.025,12,16,22.05,32,44.1,48(kHz).}{16}
	\argm{g}{G}{gain}{0}
	\argm{a}{A}{amplitude gain(0..100)}{N/A}
	\argm{o}{O}{output port(s : speaker, h : head phone)}{s}
	\argm{w}{}{execute byte swap}{FALSE}
	\argm{H}{H}{header size in byte}{0}
	\argm{v}{}{display filename}{FALSE}
	\argp{type}{data format(s : short, f : float)}{s}
\end{options}

\begin{qsection}{EXAMPLE}
In the following example, the speech data file {\em data.s}
is played on the head phone.
The sampling frequency is 8 kHz, and data is short format.
\begin{quote}
\verb! da -s 8 -o h data.s!
\end{quote}
\end{qsection}

\begin{qsection}{BUGS}
In the Linux operating system, the output port can not be assigned.
\end{qsection}

\begin{qsection}{SEE ALSO}
\end{qsection}
