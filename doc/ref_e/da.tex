% ----------------------------------------------------------------
%       Speech Signal Processing Toolkit (SPTK): version 3.0
%                      SPTK Working Group
% 
%                Department of Computer Science
%                Nagoya Institute of Technology
%                             and
%   Interdisciplinary Graduate School of Science and Engineering
%                Tokyo Institute of Technology
%                   Copyright (c) 1984-2000
%                     All Rights Reserved.
% 
% Permission is hereby granted, free of charge, to use and
% distribute this software and its documentation without
% restriction, including without limitation the rights to use,
% copy, modify, merge, publish, distribute, sublicense, and/or
% sell copies of this work, and to permit persons to whom this
% work is furnished to do so, subject to the following conditions:
% 
%   1. The code must retain the above copyright notice, this list
%      of conditions and the following disclaimer.
% 
%   2. Any modifications must be clearly marked as such.
%                                                                        
% NAGOYA INSTITUTE OF TECHNOLOGY, TOKYO INSITITUTE OF TECHNOLOGY,
% SPTK WORKING GROUP, AND THE CONTRIBUTORS TO THIS WORK DISCLAIM
% ALL WARRANTIES WITH REGARD TO THIS SOFTWARE, INCLUDING ALL
% IMPLIED WARRANTIES OF MERCHANTABILITY AND FITNESS, IN NO EVENT
% SHALL NAGOYA INSTITUTE OF TECHNOLOGY, TOKYO INSITITUTE OF
% TECHNOLOGY, SPTK WORKING GROUP, NOR THE CONTRIBUTORS BE LIABLE
% FOR ANY SPECIAL, INDIRECT OR CONSEQUENTIAL DAMAGES OR ANY
% DAMAGES WHATSOEVER RESULTING FROM LOSS OF USE, DATA OR PROFITS,
% WHETHER IN AN ACTION OF CONTRACT, NEGLIGENCE OR OTHER TORTIOUS
% ACTION, ARISING OUT OF OR IN CONNECTION WITH THE USE OR
% PERFORMANCE OF THIS SOFTWARE.
% ----------------------------------------------------------------
%
\name{da}{play 16-bit linear PCM data}{DA transformation}
\begin{synopsis}
\item [da] [ --s $S$ ] [ --c $C$ ] [ --g $G$ ] [ --a $A$ ] [ --o $O$ ] 
	   [ --w ] [ --H $H$ ]
\item [\ ~~~] [ --v ] [ +$type$ ] [ {\em infile1} ] [ {\em infile2} ] ...
\end{synopsis}

\begin{qsection}{DESCRIPTION}
{\em da} plays a series of input files (or standard input) 
on a system-dependent audio output device.
If the system does not support the specified sampling frequency, 
{\em da} upsamples the data to a supported frequency.
This command can be used under
Linux (i386), FreeBSD (i386 newpcm driver), SunOS 4.1.x, SunOS 5.x (SPARC).
 
It is possible to change environment setting through following options

\begin{tabular}{ll}
%% DA\_SAMPFREQ & sampling frequency\\
DA\_GAIN & gain\\
DA\_AMPGAIN & amplitude gain\\
DA\_PORT & output port\\
DA\_HDRSIZE & header size\\
DA\_FLOAT & set the input data to float\\
\end{tabular}

\end{qsection}

\begin{options}
	\argm{s}{S}{sampling frequency, it can be used the following
 sampling frequencies 8,10,11.025,12,16,22.05,32,44.1,48(kHz).}{16}
	\argm{g}{G}{gain}{0}
	\argm{a}{A}{amplitude gain(0..100)}{N/A}
	\argm{o}{O}{output port(s : speaker, h : headphone)}{s}
	\argm{w}{}{execute byte swap}{FALSE}
	\argm{H}{H}{header size in byte}{0}
	\argm{v}{}{display filename}{FALSE}
	\argp{type}{data format(s : short, f : float)}{s}
\end{options}

\begin{qsection}{EXAMPLE}
In the following example, the speech data file {\em data.s}
is played on the headphone.
The sampling frequency is 8 kHz, and data is in short format.
\begin{quote}
\verb! da -s 8 -o h data.s!
\end{quote}
\end{qsection}

\begin{qsection}{BUGS}
In the Linux operating system, the output port can not be assigned.
\end{qsection}

\begin{qsection}{SEE ALSO}
\end{qsection}
