\name{impulse}{generate impulse sequence}{signal processing}

\begin{synopsis}
\item[impulse] [ --l $L$ ] [ --n $N$ ]
\end{synopsis}

\begin{qsection}{DESCRIPTION}
This command generates the unit impulse sequence
of length $L$ and sends the output to the standard output.
That is, the following output data in float format is outputed
\begin{displaymath}
\underbrace{1, 0, 0, \ldots, 0}_{L}
\end{displaymath}
\par
If the options --l and --n are used at the same time,
the value of the pulse length is taken from the last option.
\end{qsection}

\begin{options}
	\argm{l}{L}{length of unit impulse\\
		    if $L < 0$ then endless sequence is generated.}{256}
	\argm{n}{N}{order of unit impulse}{255}
\end{options}

\begin{qsection}{EXAMPLE}
In the example below, an unit impulse sequence is passed through 
a digital filter and the results is presented on the screen.
\begin{quote}
 \verb!impulse | dfs -a 1 0.9 -b 1 2 1 | dmp!
\end{quote}
\end{qsection}

\begin{qsection}{SEE ALSO}
  step, train, ramp, sin, nrand
\end{qsection}
