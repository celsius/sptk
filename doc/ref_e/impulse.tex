% ----------------------------------------------------------------- %
%             The Speech Signal Processing Toolkit (SPTK)           %
%             developed by SPTK Working Group                       %
%             http://sp-tk.sourceforge.net/                         %
% ----------------------------------------------------------------- %
%                                                                   %
%  Copyright (c) 1984-2007  Tokyo Institute of Technology           %
%                           Interdisciplinary Graduate School of    %
%                           Science and Engineering                 %
%                                                                   %
%                1996-2011  Nagoya Institute of Technology          %
%                           Department of Computer Science          %
%                                                                   %
% All rights reserved.                                              %
%                                                                   %
% Redistribution and use in source and binary forms, with or        %
% without modification, are permitted provided that the following   %
% conditions are met:                                               %
%                                                                   %
% - Redistributions of source code must retain the above copyright  %
%   notice, this list of conditions and the following disclaimer.   %
% - Redistributions in binary form must reproduce the above         %
%   copyright notice, this list of conditions and the following     %
%   disclaimer in the documentation and/or other materials provided %
%   with the distribution.                                          %
% - Neither the name of the SPTK working group nor the names of its %
%   contributors may be used to endorse or promote products derived %
%   from this software without specific prior written permission.   %
%                                                                   %
% THIS SOFTWARE IS PROVIDED BY THE COPYRIGHT HOLDERS AND            %
% CONTRIBUTORS "AS IS" AND ANY EXPRESS OR IMPLIED WARRANTIES,       %
% INCLUDING, BUT NOT LIMITED TO, THE IMPLIED WARRANTIES OF          %
% MERCHANTABILITY AND FITNESS FOR A PARTICULAR PURPOSE ARE          %
% DISCLAIMED. IN NO EVENT SHALL THE COPYRIGHT OWNER OR CONTRIBUTORS %
% BE LIABLE FOR ANY DIRECT, INDIRECT, INCIDENTAL, SPECIAL,          %
% EXEMPLARY, OR CONSEQUENTIAL DAMAGES (INCLUDING, BUT NOT LIMITED   %
% TO, PROCUREMENT OF SUBSTITUTE GOODS OR SERVICES; LOSS OF USE,     %
% DATA, OR PROFITS; OR BUSINESS INTERRUPTION) HOWEVER CAUSED AND ON %
% ANY THEORY OF LIABILITY, WHETHER IN CONTRACT, STRICT LIABILITY,   %
% OR TORT (INCLUDING NEGLIGENCE OR OTHERWISE) ARISING IN ANY WAY    %
% OUT OF THE USE OF THIS SOFTWARE, EVEN IF ADVISED OF THE           %
% POSSIBILITY OF SUCH DAMAGE.                                       %
% ----------------------------------------------------------------- %
\hypertarget{impulse}{}
\name{impulse}{generate impulse sequence}{signal processing}

\begin{synopsis}
\item[impulse] [ --l $L$ ] [ --n $N$ ]
\end{synopsis}

\begin{qsection}{DESCRIPTION}
{\em impulse} generates the unit impulse sequence of length $L$, 
sending the output to standard output. 
The output is in float format as follows.
\begin{displaymath}
\underbrace{1, 0, 0, \dots, 0}_{L}
\end{displaymath}

If both --l and --n options are given, the last one is used.
\end{qsection}

\begin{options}
	\argm{l}{L}{length of unit impulse\\
		    if $L < 0$ then endless sequence is generated.}{256}
	\argm{n}{N}{order of unit impulse}{255}
\end{options}

\begin{qsection}{EXAMPLE}
In the example below, an unit impulse sequence is passed through 
a digital filter and the results is presented on the screen.
\begin{quote}
 \verb!impulse | dfs -a 1 0.9 -b 1 2 1 | dmp +f!
\end{quote}
\end{qsection}

\begin{qsection}{SEE ALSO}
\hyperlink{step}{step},
\hyperlink{train}{train},
\hyperlink{ramp}{ramp},
\hyperlink{sin}{sin},
\hyperlink{nrand}{nrand}
\end{qsection}
