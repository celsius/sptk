% ----------------------------------------------------------------
%       Speech Signal Processing Toolkit (SPTK): version 3.0
%                      SPTK Working Group
% 
%                Department of Computer Science
%                Nagoya Institute of Technology
%                             and
%   Interdisciplinary Graduate School of Science and Engineering
%                Tokyo Institute of Technology
%                   Copyright (c) 1984-2000
%                     All Rights Reserved.
% 
% Permission is hereby granted, free of charge, to use and
% distribute this software and its documentation without
% restriction, including without limitation the rights to use,
% copy, modify, merge, publish, distribute, sublicense, and/or
% sell copies of this work, and to permit persons to whom this
% work is furnished to do so, subject to the following conditions:
% 
%   1. The code must retain the above copyright notice, this list
%      of conditions and the following disclaimer.
% 
%   2. Any modifications must be clearly marked as such.
%                                                                        
% NAGOYA INSTITUTE OF TECHNOLOGY, TOKYO INSITITUTE OF TECHNOLOGY,
% SPTK WORKING GROUP, AND THE CONTRIBUTORS TO THIS WORK DISCLAIM
% ALL WARRANTIES WITH REGARD TO THIS SOFTWARE, INCLUDING ALL
% IMPLIED WARRANTIES OF MERCHANTABILITY AND FITNESS, IN NO EVENT
% SHALL NAGOYA INSTITUTE OF TECHNOLOGY, TOKYO INSITITUTE OF
% TECHNOLOGY, SPTK WORKING GROUP, NOR THE CONTRIBUTORS BE LIABLE
% FOR ANY SPECIAL, INDIRECT OR CONSEQUENTIAL DAMAGES OR ANY
% DAMAGES WHATSOEVER RESULTING FROM LOSS OF USE, DATA OR PROFITS,
% WHETHER IN AN ACTION OF CONTRACT, NEGLIGENCE OR OTHER TORTIOUS
% ACTION, ARISING OUT OF OR IN CONNECTION WITH THE USE OR
% PERFORMANCE OF THIS SOFTWARE.
% ----------------------------------------------------------------
%
\name{window}{data windowing}{signal processing,speech analysis and synthesis}

\begin{synopsis}
\item[window] [ --l $L_1$ ] [ --L $L_2$] [ --n $N$ ] [ --w $W$ ] [ {\em infile} ]
\end{synopsis}

\begin{qsection}{DESCRIPTION}
{\em window} multiplies, 
on an element-by-element basis, 
length $L$ input vectors from {\em infile} (or standard input) 
by a specified windowing function, 
sending the result to standard output.

For the input data
\begin{displaymath}
  x(0), x(1), \ldots, x(L_1-1)
\end{displaymath}
and the windowing function
\begin{displaymath}
  w(0), w(1), \ldots, w(L_1-1), 
\end{displaymath}
the output is calculated as follows:
\begin{displaymath}
  x(0)\cdot w(0),\,x(1)\cdot w(1),\,\ldots,\,x(L_1-1)\cdot w(L_1-1). 
\end{displaymath}
If $L_2$ is greater then $L_1$, then 0s are added to the output as follows.
\begin{displaymath}
  \underbrace{x(0)\cdot w(0),\,x(1)\cdot w(1),\,\ldots,\,x(L_1-1)\cdot w(L_1-1),0,\ldots,0}_{L_2}
\end{displaymath}

Input and output data are in float format.
\end{qsection}

\begin{options}
	\argm{l}{L_1}{frame length of input $(L\leq 2048)$}{256}
	\argm{L}{L_2}{frame length of output}{$L_1$}
	\argm{n}{N}{type of normalization\\
			\begin{tabular}{ll}\\ [-1ex]
			 0 & no normalization\\
			 1 & normalization such as
                             $\displaystyle \sum_{n=0}^{L-1} w^2(n) = 1$\\
			 2 & normalization such as
                             $\displaystyle \sum_{n=0}^{L-1} w(n) = 1$\\
			 \end{tabular}\\\hspace*{\fill}}{1}
	\argm{w}{W}{type of window\\
			\begin{tabular}{ll}\\ [-1ex]
			 0 & Blackman \\
			 1 & Hamming \\
			 2 & Hanning \\
			 3 & Bartlett \\
			 4 & trapezoid \\
			 5 & rectangular \\
			\end{tabular}\\\hspace*{\fill}}{0}
\end{options}

\begin{qsection}{EXAMPLE}
This example prints in the screen a sin wave function
with period 20 after windowing it with a Blackman window:
\begin{quote}
  \verb!sin -p 20 | window | fdrw | xgr !
\end{quote}
\par
This example passes the excitation generated through a train pulse
by a digital filter, applies to it a Blackman windowing function,
evaluates the log magnitude spectrum through 512 points FFT,
and plots the results in the screen:
\begin{quote}
\verb!train -p 50 | dfs -a 1 0.9 | window -l 50 -L 512 |\! \\
\verb!spec -l 512 | fdrw | xgr!
\end{quote}
\end{qsection}

\begin{qsection}{SEE ALSO}
  fftr, spec
\end{qsection}
