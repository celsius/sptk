% ----------------------------------------------------------------- %
%             The Speech Signal Processing Toolkit (SPTK)           %
%             developed by SPTK Working Group                       %
%             http://sp-tk.sourceforge.net/                         %
% ----------------------------------------------------------------- %
%                                                                   %
%  Copyright (c) 1984-2007  Tokyo Institute of Technology           %
%                           Interdisciplinary Graduate School of    %
%                           Science and Engineering                 %
%                                                                   %
%                1996-2017  Nagoya Institute of Technology          %
%                           Department of Computer Science          %
%                                                                   %
% All rights reserved.                                              %
%                                                                   %
% Redistribution and use in source and binary forms, with or        %
% without modification, are permitted provided that the following   %
% conditions are met:                                               %
%                                                                   %
% - Redistributions of source code must retain the above copyright  %
%   notice, this list of conditions and the following disclaimer.   %
% - Redistributions in binary form must reproduce the above         %
%   copyright notice, this list of conditions and the following     %
%   disclaimer in the documentation and/or other materials provided %
%   with the distribution.                                          %
% - Neither the name of the SPTK working group nor the names of its %
%   contributors may be used to endorse or promote products derived %
%   from this software without specific prior written permission.   %
%                                                                   %
% THIS SOFTWARE IS PROVIDED BY THE COPYRIGHT HOLDERS AND            %
% CONTRIBUTORS "AS IS" AND ANY EXPRESS OR IMPLIED WARRANTIES,       %
% INCLUDING, BUT NOT LIMITED TO, THE IMPLIED WARRANTIES OF          %
% MERCHANTABILITY AND FITNESS FOR A PARTICULAR PURPOSE ARE          %
% DISCLAIMED. IN NO EVENT SHALL THE COPYRIGHT OWNER OR CONTRIBUTORS %
% BE LIABLE FOR ANY DIRECT, INDIRECT, INCIDENTAL, SPECIAL,          %
% EXEMPLARY, OR CONSEQUENTIAL DAMAGES (INCLUDING, BUT NOT LIMITED   %
% TO, PROCUREMENT OF SUBSTITUTE GOODS OR SERVICES; LOSS OF USE,     %
% DATA, OR PROFITS; OR BUSINESS INTERRUPTION) HOWEVER CAUSED AND ON %
% ANY THEORY OF LIABILITY, WHETHER IN CONTRACT, STRICT LIABILITY,   %
% OR TORT (INCLUDING NEGLIGENCE OR OTHERWISE) ARISING IN ANY WAY    %
% OUT OF THE USE OF THIS SOFTWARE, EVEN IF ADVISED OF THE           %
% POSSIBILITY OF SUCH DAMAGE.                                       %
% ----------------------------------------------------------------- %
\hypertarget{mgc2mgclsp}{}
\name{mgc2mgclsp}{transform MGC to MGC-LSP}{speech parameter transformation}

\begin{synopsis}
\item [mgc2mgclsp] [ --m $M$ ] [ --a $A$] [ --g $G$ ] [ --c $C$ ] [ --o $O$ ]
 [ --s $S$ ] [ --k ] [ --L ] [ {\em infile} ]
\end{synopsis}

\begin{qsection}{DESCRIPTION}
{\em mgc2mgclsp} transforms mel-generalized cepstral coefficients
$c_{\alpha,\gamma}(0), \dots, c_{\alpha,\gamma}(M)$
from {\em infile} (or standard input)
into line spectral pair coefficients (MGC-LSPs) $K, l(1), \dots, l(M)$
sending the result to standard output.

$\alpha$ characterizes the frequency-warping transform,
while $\gamma$ characterizes the generalized log magnitude transform
and $K$ is the gain.

{\em mgc2mgclsp} does not check for stability of the MGC-LSPs.
One should use the command {\em lspcheck} to check the stability of the
MGC-LSPs.

\end{qsection}

\begin{options}
	\argm{m}{M}{order of mel-generalized cepstrum}{25}
	\argm{a}{A}{alpha of mel-generalized cepstrum}{0.35}
        \argm{g}{G_1}{gamma of mel-generalized cepstrum \\
                        $\gamma = G$}{-1}
        \argm{c}{C_1}{gamma of mel-generalized cepstrum (input)\\
                        $\gamma =-1 / $(int)$ C$\\
                        $C$ must be $C \geq 1$}{}
        \argm{o}{O}{output format \\
                \begin{tabular}{ll} \\[-1ex]
                        $0$ & normalized frequency $(0 \dots \pi)$ \\
                        $1$ & normalized frequency $(0 \dots 0.5)$ \\
                        $2$ & frequency (kHz) \\
                        $3$ & frequency (Hz)  \\
                \end{tabular}\\\hspace*{\fill}}{0}
	\argm{s}{S}{sampling frequency (kHz)}{10}
        \argm{k}{}{output gain}{TRUE}
        \argm{L}{}{output log gain instead of linear gain}{FALSE}
  	\desc[0.6ex]{Usually, the options below do not need to be assigned.}
	\argm{n}{N}{split number of unit circle}{128}
	\argm{p}{P}{maximum number of interpolation}{4}
	\argm{d}{D}{end condition of interpolation}{1e-06}
\end{options}

\begin{qsection}{EXAMPLE}
In the following example, speech data is read in float format from
{\em data.f}, analyzed with $\alpha = 0.35, \gamma = -1$
and the MGC-LSP coefficients are evaluated and written to {\em data.mgclsp}:
\begin{quote}
\verb!frame < data.f | window | mgcep -a 0.35 -g -1 |\!\\
\verb!mgc2mgclsp -a 0.35 -g -1 > data.mgclsp!
\end{quote}
Also, the stability of the MGC-LSPs can be checked by using the following:
\begin{quote}
\verb!frame < data.f | window | mgcep -a 0.35 -g -1 |\!\\
\verb!mgc2mgclsp -a 0.35 -g -1 | lspcheck -r 0.01 > data.mgclsp !
\end{quote}
\end{qsection}

\begin{qsection}{SEE ALSO}
\hyperlink{lpc}{lpc},
\hyperlink{lsp2lpc}{lsp2lpc},
\hyperlink{lspcheck}{lspcheck},
\hyperlink{mgc2mgc}{mgc2mgc},
\hyperlink{mgcep}{mgcep}
\end{qsection}
