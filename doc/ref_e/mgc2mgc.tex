% ----------------------------------------------------------------- %
%             The Speech Signal Processing Toolkit (SPTK)           %
%             developed by SPTK Working Group                       %
%             http://sp-tk.sourceforge.net/                         %
% ----------------------------------------------------------------- %
%                                                                   %
%  Copyright (c) 1984-2007  Tokyo Institute of Technology           %
%                           Interdisciplinary Graduate School of    %
%                           Science and Engineering                 %
%                                                                   %
%                1996-2017  Nagoya Institute of Technology          %
%                           Department of Computer Science          %
%                                                                   %
% All rights reserved.                                              %
%                                                                   %
% Redistribution and use in source and binary forms, with or        %
% without modification, are permitted provided that the following   %
% conditions are met:                                               %
%                                                                   %
% - Redistributions of source code must retain the above copyright  %
%   notice, this list of conditions and the following disclaimer.   %
% - Redistributions in binary form must reproduce the above         %
%   copyright notice, this list of conditions and the following     %
%   disclaimer in the documentation and/or other materials provided %
%   with the distribution.                                          %
% - Neither the name of the SPTK working group nor the names of its %
%   contributors may be used to endorse or promote products derived %
%   from this software without specific prior written permission.   %
%                                                                   %
% THIS SOFTWARE IS PROVIDED BY THE COPYRIGHT HOLDERS AND            %
% CONTRIBUTORS "AS IS" AND ANY EXPRESS OR IMPLIED WARRANTIES,       %
% INCLUDING, BUT NOT LIMITED TO, THE IMPLIED WARRANTIES OF          %
% MERCHANTABILITY AND FITNESS FOR A PARTICULAR PURPOSE ARE          %
% DISCLAIMED. IN NO EVENT SHALL THE COPYRIGHT OWNER OR CONTRIBUTORS %
% BE LIABLE FOR ANY DIRECT, INDIRECT, INCIDENTAL, SPECIAL,          %
% EXEMPLARY, OR CONSEQUENTIAL DAMAGES (INCLUDING, BUT NOT LIMITED   %
% TO, PROCUREMENT OF SUBSTITUTE GOODS OR SERVICES; LOSS OF USE,     %
% DATA, OR PROFITS; OR BUSINESS INTERRUPTION) HOWEVER CAUSED AND ON %
% ANY THEORY OF LIABILITY, WHETHER IN CONTRACT, STRICT LIABILITY,   %
% OR TORT (INCLUDING NEGLIGENCE OR OTHERWISE) ARISING IN ANY WAY    %
% OUT OF THE USE OF THIS SOFTWARE, EVEN IF ADVISED OF THE           %
% POSSIBILITY OF SUCH DAMAGE.                                       %
% ----------------------------------------------------------------- %
\hypertarget{mgc2mgc}{}
\name{mgc2mgc}{frequency and generalized cepstral transformation}%
{speech parameter transformation}

\begin{synopsis}
 \item [mgc2mgc] [ --m $M_1$ ] [ --a $A_1$ ] [ --g $G_1$ ] [ --c
   $C_1$ ] [ --n ] [ --u ]
 \item [\ ~~~~~~~~~~~] [ --M $M_2$ ] [ --A $A_2$ ] [ --G $G_2$ ]
   [ --C $C_2$ ] [ --N ] [ --U ] [ {\em infile} ] 
\end{synopsis}

\begin{qsection}{DESCRIPTION}
{\em mgc2mgc} transforms mel-generalized cepstral coefficients
$c_{\alpha_1,\gamma_1}(0), \dots, c_{\alpha_1,\gamma_1}(M_1)$
from {\em infile} (or standard input) 
into a different set of mel-generalized cepstral coefficients
$c_{\alpha_2,\gamma_2}(0), \dots, c_{\alpha_2,\gamma_2}(M_2)$
sending the result to standard output.

$\alpha$ characterizes the frequency-warping transform,
while $\gamma$ characterizes the generalized log magnitude transform.

Input and output data are in float format.

First, a frequency transformation ($\alpha_1 \rightarrow \alpha_2$)
is undertaken in the input mel-generalized cepstral
coefficients $c_{\alpha_1,\gamma_1}(m)$,
and $c_{\alpha_2,\gamma_1}(m)$ is calculated as follows.
\begin{align} 
\alpha &= (\alpha_2-\alpha_1)/(1-\alpha_1\alpha_2) \notag \\
c_{\alpha_2,\gamma_1}^{(i)}(m) &= \begin{cases}
          \;\; c_{\alpha_1,\gamma_1}(-i)
            +\alpha\,c_{\alpha_2,\gamma_1}^{(i-1)}(0), &  m=0 \\
          \;\; (1-\alpha^2)\,c_{\alpha_2,\gamma_1}^{(i-1)}(0)
            +\alpha\,c_{\alpha_2,\gamma_1}^{(i-1)}(1), &  m=1 \\
          \;\; c_{\alpha_2,\gamma_1}^{(i-1)}(m-1) 
            +\alpha\, \left(c_{\alpha_2,\gamma_1}^{(i-1)}(m)
            -c_{\alpha_2,\gamma_1}^{(i)}(m-1)\right), &   m=2,\dots,M_2
         \end{cases} \notag \\
&\hspace{70mm} i = -M_1,\dots,-1,0 \notag
\end{align}

Then the gain is normalized and $c_{\alpha_2,\gamma_1}'(m)$ 
is evaluated.
\begin{align}
K_{\alpha_2} &= 
        s_{\gamma_1}^{-1}\left(c_{\alpha_2,\gamma_1}^{(0)}(0)\right), \notag \\ 
c_{\alpha_2,\gamma_1}'(m) &=
          c_{\alpha_2,\gamma_1}^{(0)}(m)/\left(1+\gamma_1\,
          c_{\alpha_2,\gamma_1}^{(0)}(0)\right), \qquad m = 1,2,\dots, M_2 \notag
\end{align}

Afterwards, $c_{\alpha_2,\gamma_1}'(m)$ is transformed into 
$c_{\alpha_2,\gamma_2}'(m)$ through a generalized log transformation
( $\gamma_1 \rightarrow \gamma_2$ ).
\begin{align}
c_{\alpha_2,\gamma_2}'(m) &=
        c_{\alpha_2,\gamma_1}'(m)+\sum_{k=1}^{m-1} \frac{k}{m}
          \left\{ \gamma_2\,c_{\alpha_2,\gamma_1}(k)\,
          c_{\alpha_2,\gamma_2}'(m-k) 
 -\gamma_1\,c_{\alpha_2,\gamma_2}(k)\,
          c_{\alpha_2,\gamma_1}'(m-k) \right\},  \notag \\
          &\hspace{70mm} m = 1, 2, \dots, M_2 \notag
\end{align}

Finally, the gain is inversely normalized and $c_{\alpha_2,\gamma_2}(m)$
is calculated.
\begin{align}
c_{\alpha_2,\gamma_2}(0) &= 
        s_{\gamma_2}\left(K_{\alpha_2}\right), \notag \\
c_{\alpha_2,\gamma_2}(m) &= 
          c_{\alpha_2,\gamma_2}'(m)\,\left(1+\gamma_2\, 
          c_{\alpha_2,\gamma_2}(0)\right), 
          \qquad m = 1,2,\dots, M_2 \notag
\end{align}

In case we represent input and output with $\gamma$,
if the coefficients $c_{\alpha,\gamma}(m)$ are not normalized, then
the following representation is assumed
\begin{displaymath}
1+\gamma c_{\alpha,\gamma}(0), \gamma c_{\alpha,\gamma}(1), \dots, \gamma c_{\alpha,\gamma}(M),
\end{displaymath}
if they are normalized, then
the following representation is assumed
\begin{displaymath}
K_\alpha,\gamma c_{\alpha,\gamma}'(1),\dots, \gamma c_{\alpha,\gamma}'(M).
\end{displaymath}

\end{qsection}

\begin{options}
        \argm{m}{M_1}{order of mel-generalized cepstrum (input)}{25}
        \argm{a}{A_1}{alpha of mel-generalized cepstrum (input)}{0}
        \argm{g}{G_1}{gamma of mel-generalized cepstrum (input)\\
                        $\gamma_1 = G_1$}{0}
        \argm{c}{C_1}{gamma of mel-generalized cepstrum (input)\\
                        $\gamma_1 =-1 / $(int)$ C_1$\\
                        $C_1$ must be $C_1 \geq 1$}{}
        \argm{n}{}{regard input as normalized mel-generalized cepstrum}{FALSE}
        \argm{u}{}{regard input as multiplied by gamma}{FALSE}
        \argm{M}{M_2}{order of mel-generalized cepstrum (output)}{25}
        \argm{A}{A_2}{alpha of mel-generalized cepstrum (output)}{0}
        \argm{G}{G_2}{gamma of mel-generalized cepstrum (output)\\
                        $\gamma_2 = G_2$}{1}
        \argm{C}{C_2}{gamma of mel-generalized cepstrum (output)\\
                        $\gamma_2 =-1 / $(int)$ G_2$\\
                        $C_2$ must be $C_2 \geq 1$}{}
        \argm{N}{}{regard output as normalized mel-generalized cepstrum}{FALSE}
        \argm{U}{}{regard input as multiplied by gamma}{FALSE}
\end{options}

\begin{qsection}{EXAMPLE}
In the example below, 12-th order LPC coefficients are read in
float format from {\em data.lpc}, and 30-th order mel-cepstral
coefficients are calculated and written to {\em data.mcep}:
\begin{quote}
 \verb!mgc2mgc -m 12 -a 0 -g -1 -M 30 -A 0.31 -G 0!\\
 \verb!                     < data.lpc > data.mcep!
\end{quote} 
\end{qsection}

\begin{qsection}{NOTICE}
Value of $C_1$ and $C_2$ must be $C_1 \geq 1$, $C_2 \geq 1$.
\end{qsection}

\begin{qsection}{SEE ALSO}
\hyperlink{uels}{uels},
\hyperlink{gcep}{gcep},
\hyperlink{mcep}{mcep},
\hyperlink{mgcep}{mgcep},
\hyperlink{gc2gc}{gc2gc},
\hyperlink{freqt}{freqt},
\hyperlink{lpc2c}{lpc2c}
\end{qsection}
