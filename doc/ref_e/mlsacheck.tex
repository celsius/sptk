% ----------------------------------------------------------------- %
%             The Speech Signal Processing Toolkit (SPTK)           %
%             developed by SPTK Working Group                       %
%             http://sp-tk.sourceforge.net/                         %
% ----------------------------------------------------------------- %
%                                                                   %
%  Copyright (c) 1984-2007  Tokyo Institute of Technology           %
%                           Interdisciplinary Graduate School of    %
%                           Science and Engineering                 %
%                                                                   %
%                1996-2013  Nagoya Institute of Technology          %
%                           Department of Computer Science          %
%                                                                   %
% All rights reserved.                                              %
%                                                                   %
% Redistribution and use in source and binary forms, with or        %
% without modification, are permitted provided that the following   %
% conditions are met:                                               %
%                                                                   %
% - Redistributions of source code must retain the above copyright  %
%   notice, this list of conditions and the following disclaimer.   %
% - Redistributions in binary form must reproduce the above         %
%   copyright notice, this list of conditions and the following     %
%   disclaimer in the documentation and/or other materials provided %
%   with the distribution.                                          %
% - Neither the name of the SPTK working group nor the names of its %
%   contributors may be used to endorse or promote products derived %
%   from this software without specific prior written permission.   %
%                                                                   %
% THIS SOFTWARE IS PROVIDED BY THE COPYRIGHT HOLDERS AND            %
% CONTRIBUTORS "AS IS" AND ANY EXPRESS OR IMPLIED WARRANTIES,       %
% INCLUDING, BUT NOT LIMITED TO, THE IMPLIED WARRANTIES OF          %
% MERCHANTABILITY AND FITNESS FOR A PARTICULAR PURPOSE ARE          %
% DISCLAIMED. IN NO EVENT SHALL THE COPYRIGHT OWNER OR CONTRIBUTORS %
% BE LIABLE FOR ANY DIRECT, INDIRECT, INCIDENTAL, SPECIAL,          %
% EXEMPLARY, OR CONSEQUENTIAL DAMAGES (INCLUDING, BUT NOT LIMITED   %
% TO, PROCUREMENT OF SUBSTITUTE GOODS OR SERVICES; LOSS OF USE,     %
% DATA, OR PROFITS; OR BUSINESS INTERRUPTION) HOWEVER CAUSED AND ON %
% ANY THEORY OF LIABILITY, WHETHER IN CONTRACT, STRICT LIABILITY,   %
% OR TORT (INCLUDING NEGLIGENCE OR OTHERWISE) ARISING IN ANY WAY    %
% OUT OF THE USE OF THIS SOFTWARE, EVEN IF ADVISED OF THE           %
% POSSIBILITY OF SUCH DAMAGE.                                       %
% ----------------------------------------------------------------- %
\hypertarget{mlsacheck}{}
\name{mlsacheck}%
{check stability of MLSA filter}{speech parameter transformation}

\begin{synopsis}
\item [mlsacheck] [ --m $M$ ] [ --a $A$ ] [ --c $C$ ] [ --r ] [ --l $L$]
[ --R ] [--P $Pa$ ] [ {\em infile} ]
\end{synopsis}

\begin{qsection}{DESCRIPTION}
 {\em mlsacheck} tests the stability of
 the Mel Log Spectral Approximation (MLSA) digital filter
 of the mel-cepstrum coefficients in {\em infile} (or standard input).
 The result sends to standard output.

 Both input and output are in float format.

 As described in \hyperlink{mlsadf}{mlsadf},
 the transfer function $H(z)$ is expressed as
\begin{align}
H(z) &= \exp \sum_{m=0}^M b(m) {\it\Phi}_m(z) \notag \\
     &= K \cdot D(z)\notag
\end{align}
where
\begin{displaymath}
{\it\Phi}_m(z) = \begin{cases}
	  \;\;1, & m=0 \\ \;\;\displaystyle
	  \frac{\displaystyle (1-\alpha^2)z^{-1}}
	    {\displaystyle 1-\alpha z^{-1}}
	    \tilde{z}^{-(m-1)},& m\geq 1
	\end{cases}
\end{displaymath}
and
\begin{align}
\tilde{z}^{-1} &= \frac{z^{-1}-\alpha}{1-\alpha z^{-1}}, \notag \\
K    &= \exp b(0) \notag, \\
D(z) &= \exp \sum_{m=1}^M b(m) {\it\Phi}_m(z).  \notag
\end{align}
To construct the exponential transfer function $H(z)$,
 Pad\'e approximation is used to approximate complex exponential function
 $\exp\:w$
 by a following rational function:
  \begin{displaymath}
   \exp\:w \simeq  R_L(w) = \frac{1+\sum_{l=1}^{L}A_{L,l}w^{l}}{1+\sum_{l=1}^{L}A_{L,l}(-w)^{l}}
  \end{displaymath}
 Then $D(z)$ is approximated by
  \begin{displaymath}
   D(z) = \exp(F(z)) \simeq R_{L}(F(z))
  \end{displaymath}
 where
  \begin{displaymath}
   F(z) = \sum_{m=0}^{M}b(m)\tilde{z}^{-m}.
  \end{displaymath}

 The stability of the MLSA synthesis filter
 is related to the accuracy of the approximation.
 When \begin{math} |F(e^{j\omega})| < r = 4.5 \end{math} and
 \begin{math} L = 4 \end{math}
 for \begin{math} R_{L}(w)\end{math},
 the log approximation error does not exceed 0.24 dB.
 The corresponding synthesis filter
 \begin{math} R_{L}(F(z))\simeq \exp(F(z))=D(z)\end{math}
 is stable when \begin{math} |F(e^{j\omega})| < r_{max} = 6.2 \end{math}.
 Also, the log approximation error does not exceed 0.2735 dB
 when \begin{math} r = 6.0 \end{math} and
 \begin{math}L = 5 \end{math}.
 The corresponding synthesis filter is stable
 when \begin{math}r_{max} = 7.65 \end{math}.

 In spite of whether specifying --c option or not,
 {\em mlsacheck} tests the stability and sends
 an ASCII report of the number of unstable frame to standard error.
 When specifying --c option,
 {\em mlsacheck} modifies the filter coefficients
 if unstable frame is found.
 When specifying --r option,
 the stable condition can be selected as follows:
 When '--r 0', {\em mlsacheck} keeps the log approximation
 not exceeding 0.24 dB ($Pa=4$) or 0.2735 dB ($Pa=5$),
 where $Pa$ is the order of Pad\'e approximation.
 When '--r 1', {\em mlsacheck} keeps the MLSA filter stable
 although the accuracy of log approximation is lost.

 The ways of check and modification is specified by the -c option.
 When the -c option is 1 or 4, MLSA filter coefficients are checked and modified without FFT.
 This method evaluates stability by summation MLSA filter coefficients.
\end{qsection}

\begin{options}
 \argm{m}{M}{order of mel-cepstrum}{25}
 \argm{a}{A}{all-pass constant $\alpha$}{0.35}
 \argm{l}{L}{FFT length}{256}
 \argm{c}{C}{stability check and modification of MLSA filter coefficients \\
            \begin{tabular}{ll} \\[-1ex]
	     $0$ & only check \\
	     $1$ & only check (fast mode) \\
	     $2$ & check and modification by clipping \\
	     $3$ & check and modification by scaling \\
	     $4$ & check and modification (fast mode) \\
	    \end{tabular}\\
            \hspace*{\fill}}{0}
 \argm{r}{R}{stable condition for MLSA filter\\
            \begin{tabular}{ll} \\[-1ex]
             $0$ & keep log approximation error not exceeding\\
             & 0.24 dB ($Pa = 4$) or 0.2735 dB ($Pa = 5$) \\
             $1$ & keep MLSA filter stable\\
            \end{tabular}\\
            \hspace*{\fill}}{0}
 \argm{P}{Pa}{order of the Pad\'e approximation\\
                     $Pa$ should be $4$ or $5$}{4}
 \argm{R}{R}{threshold value for modification \\
             If this option wasn't specified, R is set as follows : \\
            \begin{tabular}{ll} \\[-1ex]
             when $r=0$ and $P=4$ & ,$R=4.5$ \\
	     when $r=1$ and $P=4$ & ,$R=6.2$ \\
	     when $r=0$ and $P=5$ & ,$R=6.0$ \\
	     when $r=1$ and $P=5$ & ,$R=7.65$ \\
            \end{tabular}\\
            \hspace*{\fill}}{N/A}

\end{options}

\begin{qsection}{EXAMPLE}
In the following example,
39-th order mel-cepstrum coefficients are
read from {\em data.mcep} in float format,
then the stability of MLSA filter is checked,
 and the results are written to {\em data.mlsachk}.
\begin{quote}
 \verb! mlsacheck -a 0.48 -m 39 -c 0 data.mcep > data.mlsachk !
\end{quote}
 Also, in the following example, the stability of MLSA filter of 49-th order
 mel-cepstrum coefficients read from {\em data.mcep} is checked
 under the condition that
 frequency warping is 0.55, Pad\'e order is 5, and FFT length is 4096.
 In this example, the coefficients are modified in unstable frames by -r option.
\begin{quote}
 \verb! mlsacheck -m 49 -a 0.55 -P 5 -l 4096 -c 2 \ !\\
 \verb! -r 1 data.mcep > data.mlsachk!
\end{quote}
\end{qsection}

\begin{qsection}{SEE ALSO}
\hyperlink{mcep}{mcep},
\hyperlink{amcep}{amcep},
\hyperlink{poledf}{poledf},
\hyperlink{zerodf}{zerodf},
\hyperlink{ltcdf}{ltcdf},
\hyperlink{lmadf}{lmadf},
\hyperlink{glsadf}{glsadf},
\hyperlink{mglsadf}{mglsadf}
\end{qsection}
