% ----------------------------------------------------------------- %
%             The Speech Signal Processing Toolkit (SPTK)           %
%             developed by SPTK Working Group                       %
%             http://sp-tk.sourceforge.net/                         %
% ----------------------------------------------------------------- %
%                                                                   %
%  Copyright (c) 1984-2007  Tokyo Institute of Technology           %
%                           Interdisciplinary Graduate School of    %
%                           Science and Engineering                 %
%                                                                   %
%                1996-2015  Nagoya Institute of Technology          %
%                           Department of Computer Science          %
%                                                                   %
% All rights reserved.                                              %
%                                                                   %
% Redistribution and use in source and binary forms, with or        %
% without modification, are permitted provided that the following   %
% conditions are met:                                               %
%                                                                   %
% - Redistributions of source code must retain the above copyright  %
%   notice, this list of conditions and the following disclaimer.   %
% - Redistributions in binary form must reproduce the above         %
%   copyright notice, this list of conditions and the following     %
%   disclaimer in the documentation and/or other materials provided %
%   with the distribution.                                          %
% - Neither the name of the SPTK working group nor the names of its %
%   contributors may be used to endorse or promote products derived %
%   from this software without specific prior written permission.   %
%                                                                   %
% THIS SOFTWARE IS PROVIDED BY THE COPYRIGHT HOLDERS AND            %
% CONTRIBUTORS "AS IS" AND ANY EXPRESS OR IMPLIED WARRANTIES,       %
% INCLUDING, BUT NOT LIMITED TO, THE IMPLIED WARRANTIES OF          %
% MERCHANTABILITY AND FITNESS FOR A PARTICULAR PURPOSE ARE          %
% DISCLAIMED. IN NO EVENT SHALL THE COPYRIGHT OWNER OR CONTRIBUTORS %
% BE LIABLE FOR ANY DIRECT, INDIRECT, INCIDENTAL, SPECIAL,          %
% EXEMPLARY, OR CONSEQUENTIAL DAMAGES (INCLUDING, BUT NOT LIMITED   %
% TO, PROCUREMENT OF SUBSTITUTE GOODS OR SERVICES; LOSS OF USE,     %
% DATA, OR PROFITS; OR BUSINESS INTERRUPTION) HOWEVER CAUSED AND ON %
% ANY THEORY OF LIABILITY, WHETHER IN CONTRACT, STRICT LIABILITY,   %
% OR TORT (INCLUDING NEGLIGENCE OR OTHERWISE) ARISING IN ANY WAY    %
% OUT OF THE USE OF THIS SOFTWARE, EVEN IF ADVISED OF THE           %
% POSSIBILITY OF SUCH DAMAGE.                                       %
% ----------------------------------------------------------------- %
\hypertarget{wavsplit}{}
\name{wavsplit}{split a stereo WAV file}{data operation}

\begin{synopsis}
\item[wavsplit] [ --i $I$ ][ --o $O$ ] 
\end{synopsis}

\begin{qsection}{DESCRIPTION}
{\em wavsplit} splits a stereo WAV file into two monaural WAV files.

\end{qsection}

\begin{options}
	\argm{i}{I}{Input WAV file or directory}{}
	\argm{o}{O}{Output WAV files or directories}{}
\end{options}

\begin{qsection}{EXAMPLE}
 In the following command, the stereo wav file {\em file.wav} is split into
 two monaural WAV files {\em file\_channel0.wav} and {\em file\_channel1.wav}.
\begin{quote}
 \verb!wavsplit -i file.wav -o file_channel0.wav file_channel1.wav!
\end{quote}
 If an input directory is specified, {\em wavsplit} splits all the WAV files in the directory.
 When the two output directories are given as follows, {\em wavsplit} outputs the monaural wav files separately for each channel.
 The output file names are the same as the input one.
 \begin{quote}
  \verb!wavsplit -i input_directory -o output_directory0 output_directory1!
 \end{quote}
 If an output directory is specified, {\em wavsplit} suffixes a channel number to the output file name.
 For example, {\em file.wav} in {\em input\_directory} is split into two WAV files {\em file\_0.wav} and {\em file\_1.wav} in {\em output\_directory}
\begin{quote}
  \verb!wavsplit -i input_directory -o output_directory!
\end{quote}
\end{qsection}

\begin{qsection}{NOTICE}
 {\em wavsplit} does not distinguish between small and capital letters of the file extension.
 The first output WAV file or directory is related to channel 0, and the other is related to channel 1.
\end{qsection}

\begin{qsection}{SEE ALSO}
\hyperlink{raw2wav}{raw2wav},
\hyperlink{wavjoin}{wavjoin}
\end{qsection}
