%  ---------------------------------------------------------------  %
%            Speech Signal Processing Toolkit (SPTK)                %
%                      SPTK Working Group                           %
%                                                                   %
%                  Department of Computer Science                   %
%                  Nagoya Institute of Technology                   %
%                               and                                 %
%   Interdisciplinary Graduate School of Science and Engineering    %
%                  Tokyo Institute of Technology                    %
%                                                                   %
%                     Copyright (c) 1984-2007                       %
%                       All Rights Reserved.                        %
%                                                                   %
%  Permission is hereby granted, free of charge, to use and         %
%  distribute this software and its documentation without           %
%  restriction, including without limitation the rights to use,     %
%  copy, modify, merge, publish, distribute, sublicense, and/or     %
%  sell copies of this work, and to permit persons to whom this     %
%  work is furnished to do so, subject to the following conditions: %
%                                                                   %
%    1. The source code must retain the above copyright notice,     %
%       this list of conditions and the following disclaimer.       %
%                                                                   %
%    2. Any modifications to the source code must be clearly        %
%       marked as such.                                             %
%                                                                   %
%    3. Redistributions in binary form must reproduce the above     %
%       copyright notice, this list of conditions and the           %
%       following disclaimer in the documentation and/or other      %
%       materials provided with the distribution.  Otherwise, one   %
%       must contact the SPTK working group.                        %
%                                                                   %
%  NAGOYA INSTITUTE OF TECHNOLOGY, TOKYO INSTITUTE OF TECHNOLOGY,   %
%  SPTK WORKING GROUP, AND THE CONTRIBUTORS TO THIS WORK DISCLAIM   %
%  ALL WARRANTIES WITH REGARD TO THIS SOFTWARE, INCLUDING ALL       %
%  IMPLIED WARRANTIES OF MERCHANTABILITY AND FITNESS, IN NO EVENT   %
%  SHALL NAGOYA INSTITUTE OF TECHNOLOGY, TOKYO INSTITUTE OF         %
%  TECHNOLOGY, SPTK WORKING GROUP, NOR THE CONTRIBUTORS BE LIABLE   %
%  FOR ANY SPECIAL, INDIRECT OR CONSEQUENTIAL DAMAGES OR ANY        %
%  DAMAGES WHATSOEVER RESULTING FROM LOSS OF USE, DATA OR PROFITS,  %
%  WHETHER IN AN ACTION OF CONTRACT, NEGLIGENCE OR OTHER TORTUOUS   %
%  ACTION, ARISING OUT OF OR IN CONNECTION WITH THE USE OR          %
%  PERFORMANCE OF THIS SOFTWARE.                                    %
%                                                                   %
%  ---------------------------------------------------------------  %
%
\hypertarget{dawrite}{}
\name{dawrite}{output waveform to audio device}{DA transformation}
\begin{synopsis}
\item [dawrite] [ --s $S$ ] [ --c $C$ ] [ --g $G$ ] [ --a $A$ ] [ --o $O$ ]
           [ --w ] [ --H $H$ ] [ --v ] [ +$type$ ]
\item [\ ~~~] [ {\em infile1} ] [ {\em infile2} ] ...
\end{synopsis}

\begin{qsection}{DESCRIPTION}
{\em dawrite} plays a series of input files (or standard input) 
on a system-dependent audio output device.

If $G$ is gain, this program multiplies input data by $2^G$
and outputs it.
Amplitude gain can be set within 0--100.
You can set sampling frequency to those suported in your audio device.
The sampling frequencies, 11.025, 22.05, and 44.1 kHz can be abbreviated to 
11, 22, and 44, respectively.
If the system does not support the specified sampling frequency, 
{\em da} upsamples the data to a supported frequency.
This command can be used under
Linux (i386), FreeBSD (i386 newpcm driver), SunOS 4.1.x, SunOS 5.x (SPARC).
 
It is possible to change environment setting through following options

\begin{tabular}{ll}
DA\_SAMPFREQ & sampling frequency\\
DA\_GAIN & gain\\
DA\_AMPGAIN & amplitude gain\\
DA\_PORT & output port\\
DA\_HDRSIZE & header size\\
DA\_FLOAT & set the input data to float\\
\end{tabular}

\end{qsection}

\begin{options}
	\argm{s}{S}{sampling frequency, it can be used the following
 sampling frequencies 8, 10, 11.025, 12, 16, 20, 22.05, 32, 44.1, 48 (kHz).}{10}
	\argm{g}{G}{gain}{0}
	\argm{a}{A}{amplitude gain(0..100)}{N/A}
	\argm{o}{O}{output port(s : speaker, h : headphone)}{s}
	\argm{w}{}{execute byte swap}{FALSE}
	\argm{H}{H}{header size in byte}{0}
	\argm{v}{}{display filename}{FALSE}
	\argp{type}{data format (s: short, f: float, d: double)}{s}
\end{options}

\begin{qsection}{EXAMPLE}
In the following example, the speech data file {\em data.s}
is played on the headphone.
The sampling frequency is 8 kHz, and data is in short format.
\begin{quote}
\verb! dawrite +s -s 8 -o h data.s!
\end{quote}
\end{qsection}

\begin{qsection}{BUGS}
In the Linux operating system, the output port can not be assigned.
\end{qsection}

\begin{qsection}{SEE ALSO}
\hyperlink{da}{da},
\hyperlink{us}{us}
\end{qsection}
