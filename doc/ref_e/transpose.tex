% ----------------------------------------------------------------- %
%             The Speech Signal Processing Toolkit (SPTK)           %
%             developed by SPTK Working Group                       %
%             http://sp-tk.sourceforge.net/                         %
% ----------------------------------------------------------------- %
%                                                                   %
%  Copyright (c) 1984-2007  Tokyo Institute of Technology           %
%                           Interdisciplinary Graduate School of    %
%                           Science and Engineering                 %
%                                                                   %
%                1996-2013  Nagoya Institute of Technology          %
%                           Department of Computer Science          %
%                                                                   %
% All rights reserved.                                              %
%                                                                   %
% Redistribution and use in source and binary forms, with or        %
% without modification, are permitted provided that the following   %
% conditions are met:                                               %
%                                                                   %
% - Redistributions of source code must retain the above copyright  %
%   notice, this list of conditions and the following disclaimer.   %
% - Redistributions in binary form must reproduce the above         %
%   copyright notice, this list of conditions and the following     %
%   disclaimer in the documentation and/or other materials provided %
%   with the distribution.                                          %
% - Neither the name of the SPTK working group nor the names of its %
%   contributors may be used to endorse or promote products derived %
%   from this software without specific prior written permission.   %
%                                                                   %
% THIS SOFTWARE IS PROVIDED BY THE COPYRIGHT HOLDERS AND            %
% CONTRIBUTORS "AS IS" AND ANY EXPRESS OR IMPLIED WARRANTIES,       %
% INCLUDING, BUT NOT LIMITED TO, THE IMPLIED WARRANTIES OF          %
% MERCHANTABILITY AND FITNESS FOR A PARTICULAR PURPOSE ARE          %
% DISCLAIMED. IN NO EVENT SHALL THE COPYRIGHT OWNER OR CONTRIBUTORS %
% BE LIABLE FOR ANY DIRECT, INDIRECT, INCIDENTAL, SPECIAL,          %
% EXEMPLARY, OR CONSEQUENTIAL DAMAGES (INCLUDING, BUT NOT LIMITED   %
% TO, PROCUREMENT OF SUBSTITUTE GOODS OR SERVICES; LOSS OF USE,     %
% DATA, OR PROFITS; OR BUSINESS INTERRUPTION) HOWEVER CAUSED AND ON %
% ANY THEORY OF LIABILITY, WHETHER IN CONTRACT, STRICT LIABILITY,   %
% OR TORT (INCLUDING NEGLIGENCE OR OTHERWISE) ARISING IN ANY WAY    %
% OUT OF THE USE OF THIS SOFTWARE, EVEN IF ADVISED OF THE           %
% POSSIBILITY OF SUCH DAMAGE.                                       %
% ----------------------------------------------------------------- %
\hypertarget{transpose}{}
\name{transpose}{transpose a matrix}{data operation}

\begin{synopsis}
\item[transpose] [ --m $m$ ] [ --n $n$ ] [ {\em infile} ]
\end{synopsis}

\begin{qsection}{DESCRIPTION}
{\em transpose} assumes the input data from {\em infile} (or standard
input) as $m \times n$ matrix and transposes the matrix to
$n \times m$ matrix.
Then, sends the result to standard output. 
You have to define the number of rows and columns and
if the file length is not a multiple of $m \times n$, 
leftover values are discarded as shown in the example below.

\mbox{Input sequence}
\begin{center}
\begin{tabular}{cccccccc}
$x(0,0)$&$,$&$x(0,1)$&$,$&$\ldots$&$,$&$x(0,n-1)$&,\\
$x(1,0)$&$,$&$x(1,1)$&$,$&$\ldots$&$,$&$x(1,n-1)$&,\\
$\vdots$&   &$\vdots$&   &        &   &$\vdots$& \\
$x(m-1,0)$&$,$&$x(m-1,1)$&$,$&$\ldots$&$,$&$x(m-1,n-1)$&
\end{tabular}
\end{center}

\mbox{Output sequence}
\begin{center}
\begin{tabular}{cccccccc}
$x(0,0)$&$,$&$x(1,0)$&$,$&$\ldots$&$,$&$x(m-1,0)$&,\\
$x(0,1)$&$,$&$x(1,1)$&$,$&$\ldots$&$,$&$x(m-1,1)$&,\\
$\vdots$&   &$\vdots$&   &        &   &$\vdots$& \\
$x(0,n-1)$&$,$&$x(1,n-1)$&$,$&$\ldots$&$,$&$x(m-1,n-1)$&
\end{tabular}
\end{center}

\end{qsection}

\begin{options}
	\argm{m}{m}{number of rows}{N/A}
	\argm{n}{n}{number of columns}{N/A}
\end{options}

\begin{qsection}{EXAMPLE}
Let's assume that the following data
is read from {\em data.in} file in float format.
\begin{displaymath}
 \underbrace{0.0, ~1.0, ~2.0}, ~
 \underbrace{3.0, ~4.0, ~5.0}, ~6.0
\end{displaymath}
The command
\begin{quote}
\verb!transpose -m 2 -n 3 data.in > data.out!
\end{quote}
will write the following output to {\em data.out}.
\begin{displaymath}
 \underbrace{0.0, ~3.0}, ~
 \underbrace{1.0, ~4.0}, ~
 \underbrace{2.0, ~5.0}
\end{displaymath}
\end{qsection}
