% ----------------------------------------------------------------
%       Speech Signal Processing Toolkit (SPTK): version 3.0
%                      SPTK Working Group
% 
%                Department of Computer Science
%                Nagoya Institute of Technology
%                             and
%   Interdisciplinary Graduate School of Science and Engineering
%                Tokyo Institute of Technology
%                   Copyright (c) 1984-2000
%                     All Rights Reserved.
% 
% Permission is hereby granted, free of charge, to use and
% distribute this software and its documentation without
% restriction, including without limitation the rights to use,
% copy, modify, merge, publish, distribute, sublicense, and/or
% sell copies of this work, and to permit persons to whom this
% work is furnished to do so, subject to the following conditions:
% 
%   1. The code must retain the above copyright notice, this list
%      of conditions and the following disclaimer.
% 
%   2. Any modifications must be clearly marked as such.
%                                                                        
% NAGOYA INSTITUTE OF TECHNOLOGY, TOKYO INSITITUTE OF TECHNOLOGY,
% SPTK WORKING GROUP, AND THE CONTRIBUTORS TO THIS WORK DISCLAIM
% ALL WARRANTIES WITH REGARD TO THIS SOFTWARE, INCLUDING ALL
% IMPLIED WARRANTIES OF MERCHANTABILITY AND FITNESS, IN NO EVENT
% SHALL NAGOYA INSTITUTE OF TECHNOLOGY, TOKYO INSITITUTE OF
% TECHNOLOGY, SPTK WORKING GROUP, NOR THE CONTRIBUTORS BE LIABLE
% FOR ANY SPECIAL, INDIRECT OR CONSEQUENTIAL DAMAGES OR ANY
% DAMAGES WHATSOEVER RESULTING FROM LOSS OF USE, DATA OR PROFITS,
% WHETHER IN AN ACTION OF CONTRACT, NEGLIGENCE OR OTHER TORTIOUS
% ACTION, ARISING OUT OF OR IN CONNECTION WITH THE USE OR
% PERFORMANCE OF THIS SOFTWARE.
% ----------------------------------------------------------------
%
\hypertarget{vsum}{}
\name{vsum}{summation of vector}{data processing}

\begin{synopsis}
\item[vsum] [ --l $L$ ] [ --n $N$ ] [ {\em infile} ]
\end{synopsis}

\begin{qsection}{DESCRIPTION}
{\em vsum} calculates the vector sum of groups of $N$ input vectors 
of length $L$ from {\em infile} (or standard input), 
sending the result to standard output.
That is, if the input data is given by
\begin{displaymath}
\overbrace{
  \overbrace{a_1(1),\dots,a_1(L)}^{L},
  \overbrace{a_2(1),\dots,a_2(L)}^{L},\dots,
  \overbrace{a_N(1),\dots,a_N(L)}^{L}
}^{N \cdot L},\dots
\end{displaymath}
then the output is 
\begin{displaymath}
  \overbrace{s(1),\dots,s(L)}^{L},\dots
\end{displaymath}
where $s(n)$ is
\begin{displaymath}
  s(n)=\sum_{k=1}^{N} a_k(n)
\end{displaymath}

Input and output data are in float format.
\end{qsection}

\begin{options}
	\argm{l}{L}{order of vector}{1}
	\argm{n}{N}{number of vector}{EOD}
\end{options}

\begin{qsection}{EXAMPLE}
The output file {\em data.sum} contains the summation for
the whole data in file {\em data.f} read in float format:
\begin{quote}
  \verb!vsum data.f > data.sum!
\end{quote}
\par
In this example, the norm of 10-th order vectors are
evaluated and written to {\em data.n}:
\begin{quote}
  \verb!sopr data.f -P | vsum -n 10 | sopr -R > data.n!
\end{quote}
\par
In the next example, 15-th order coefficients vectors are read
from {\em data.f}, the average for every 3 frame is evaluated,
and output to {\em data.av}:
\begin{quote}
  \verb!vsum -l 15 -n 3 data.f | sopr -d 3 > data.av!
\end{quote}
\end{qsection}

\begin{qsection}{SEE ALSO}
\hyperlink{sopr}{sopr}
\end{qsection}
