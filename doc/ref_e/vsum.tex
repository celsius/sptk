% ----------------------------------------------------------------- %
%             The Speech Signal Processing Toolkit (SPTK)           %
%             developed by SPTK Working Group                       %
%             http://sp-tk.sourceforge.net/                         %
% ----------------------------------------------------------------- %
%                                                                   %
%  Copyright (c) 1984-2007  Tokyo Institute of Technology           %
%                           Interdisciplinary Graduate School of    %
%                           Science and Engineering                 %
%                                                                   %
%                1996-2012  Nagoya Institute of Technology          %
%                           Department of Computer Science          %
%                                                                   %
% All rights reserved.                                              %
%                                                                   %
% Redistribution and use in source and binary forms, with or        %
% without modification, are permitted provided that the following   %
% conditions are met:                                               %
%                                                                   %
% - Redistributions of source code must retain the above copyright  %
%   notice, this list of conditions and the following disclaimer.   %
% - Redistributions in binary form must reproduce the above         %
%   copyright notice, this list of conditions and the following     %
%   disclaimer in the documentation and/or other materials provided %
%   with the distribution.                                          %
% - Neither the name of the SPTK working group nor the names of its %
%   contributors may be used to endorse or promote products derived %
%   from this software without specific prior written permission.   %
%                                                                   %
% THIS SOFTWARE IS PROVIDED BY THE COPYRIGHT HOLDERS AND            %
% CONTRIBUTORS "AS IS" AND ANY EXPRESS OR IMPLIED WARRANTIES,       %
% INCLUDING, BUT NOT LIMITED TO, THE IMPLIED WARRANTIES OF          %
% MERCHANTABILITY AND FITNESS FOR A PARTICULAR PURPOSE ARE          %
% DISCLAIMED. IN NO EVENT SHALL THE COPYRIGHT OWNER OR CONTRIBUTORS %
% BE LIABLE FOR ANY DIRECT, INDIRECT, INCIDENTAL, SPECIAL,          %
% EXEMPLARY, OR CONSEQUENTIAL DAMAGES (INCLUDING, BUT NOT LIMITED   %
% TO, PROCUREMENT OF SUBSTITUTE GOODS OR SERVICES; LOSS OF USE,     %
% DATA, OR PROFITS; OR BUSINESS INTERRUPTION) HOWEVER CAUSED AND ON %
% ANY THEORY OF LIABILITY, WHETHER IN CONTRACT, STRICT LIABILITY,   %
% OR TORT (INCLUDING NEGLIGENCE OR OTHERWISE) ARISING IN ANY WAY    %
% OUT OF THE USE OF THIS SOFTWARE, EVEN IF ADVISED OF THE           %
% POSSIBILITY OF SUCH DAMAGE.                                       %
% ----------------------------------------------------------------- %
\hypertarget{vsum}{}
\name{vsum}{summation of vector}{data processing}

\begin{synopsis}
\item[vsum] [ --l $L$ ] [ --n $N$ ] [ {\em infile} ]
\end{synopsis}

\begin{qsection}{DESCRIPTION}
{\em vsum} calculates the vector sum of groups of $N$ input vectors 
of length $L$ from {\em infile} (or standard input), 
sending the result to standard output.
That is, if the input data is given by
\begin{displaymath}
\overbrace{
  \overbrace{a_1(1),\dots,a_1(L)}^{L},
  \overbrace{a_2(1),\dots,a_2(L)}^{L},\dots,
  \overbrace{a_N(1),\dots,a_N(L)}^{L}
}^{N \cdot L},\dots
\end{displaymath}
then the output is 
\begin{displaymath}
  \overbrace{s(1),\dots,s(L)}^{L},\dots
\end{displaymath}
,where $s(n)$ can be written as
\begin{displaymath}
  s(n)=\sum_{k=1}^{N} a_k(n)
\end{displaymath}

Input and output data are in float format.
\end{qsection}

\begin{options}
	\argm{l}{L}{order of vector}{1}
	\argm{n}{N}{number of vector}{EOD}
\end{options}

\begin{qsection}{EXAMPLE}
The output file {\em data.sum} contains the summation of
the whole data in file {\em data.f} read in float format:
\begin{quote}
  \verb!vsum data.f > data.sum!
\end{quote}
\par
In this example, the norm of 10-th order vectors are
evaluated and written to {\em data.n}:
\begin{quote}
  \verb!sopr data.f -P | vsum -n 10 | sopr -R > data.n!
\end{quote}
\par
In the next example, 15-th order coefficients vectors are read
from {\em data.f}, the average for every 3 frames is evaluated,
and output to {\em data.av}:
\begin{quote}
  \verb!vsum -l 15 -n 3 data.f | sopr -d 3 > data.av!
\end{quote}
\end{qsection}

\begin{qsection}{SEE ALSO}
\hyperlink{sopr}{sopr}
\end{qsection}
