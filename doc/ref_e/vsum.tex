\name{vsum}{summation of vector}{data processing}

\begin{synopsis}
\item[vsum] [ --l $L$ ] [ --n $N$ ] [ {\em infile} ]
\end{synopsis}

\begin{qsection}{DESCRIPTION}
This command reads $L$ dimension vectors from the assigned file,
evaluates their summations for every $N$ vectors and sends the
results to the standard output.
That is, if the input data is given by
\begin{displaymath}
\overbrace{
  \overbrace{a_1(1),\ldots,a_1(L)}^{L},
  \overbrace{a_2(1),\ldots,a_2(L)}^{L},\ldots,
  \overbrace{a_N(1),\ldots,a_N(L)}^{L}
}^{N \cdot L},\ldots
\end{displaymath}
then the output is 
\begin{displaymath}
  \overbrace{s(1),\ldots,s(L)}^{L},\ldots
\end{displaymath}
where $s(n)$ is
\begin{displaymath}
  s(n)=\sum_{k=1}^{N} a_k(n)
\end{displaymath}
\par
If the input file is omitted, then data is read from the standard input.
\par
Input and output data are in float format.
\end{qsection}

\begin{options}
	\argm{l}{L}{order of vector}{1}
	\argm{n}{N}{number of vector}{EOD}
\end{options}

\begin{qsection}{EXAMPLE}
The output file {\em data.sum} contains the summation for
the whole data in file {\em data.f} read in float format:
\begin{quote}
  \verb!vsum data.f > data.sum!
\end{quote}
\par
In this example, the norm of 10 order vectors are
evaluated and written to {\em data.n}:
\begin{quote}
  \verb!sopr data.f -P | vsum -n 10 | sopr -R > data.n!
\end{quote}
\par
In the next example, 15 order coefficients vectors are read
from {\em data.f}, the average for every 3 frame is evaluated,
and outputed to {\em data.av}:
\begin{quote}
  \verb!vsum -l 15 -n 3 data.f | sopr -d 3 > data.av!
\end{quote}
\end{qsection}

\begin{qsection}{SEE ALSO}
  sopr
\end{qsection}
