%  ---------------------------------------------------------------  %
%            Speech Signal Processing Toolkit (SPTK)                %
%                      SPTK Working Group                           %
%                                                                   %
%                  Department of Computer Science                   %
%                  Nagoya Institute of Technology                   %
%                               and                                 %
%   Interdisciplinary Graduate School of Science and Engineering    %
%                  Tokyo Institute of Technology                    %
%                                                                   %
%                     Copyright (c) 1984-2007                       %
%                       All Rights Reserved.                        %
%                                                                   %
%  Permission is hereby granted, free of charge, to use and         %
%  distribute this software and its documentation without           %
%  restriction, including without limitation the rights to use,     %
%  copy, modify, merge, publish, distribute, sublicense, and/or     %
%  sell copies of this work, and to permit persons to whom this     %
%  work is furnished to do so, subject to the following conditions: %
%                                                                   %
%    1. The source code must retain the above copyright notice,     %
%       this list of conditions and the following disclaimer.       %
%                                                                   %
%    2. Any modifications to the source code must be clearly        %
%       marked as such.                                             %
%                                                                   %
%    3. Redistributions in binary form must reproduce the above     %
%       copyright notice, this list of conditions and the           %
%       following disclaimer in the documentation and/or other      %
%       materials provided with the distribution.  Otherwise, one   %
%       must contact the SPTK working group.                        %
%                                                                   %
%  NAGOYA INSTITUTE OF TECHNOLOGY, TOKYO INSTITUTE OF TECHNOLOGY,   %
%  SPTK WORKING GROUP, AND THE CONTRIBUTORS TO THIS WORK DISCLAIM   %
%  ALL WARRANTIES WITH REGARD TO THIS SOFTWARE, INCLUDING ALL       %
%  IMPLIED WARRANTIES OF MERCHANTABILITY AND FITNESS, IN NO EVENT   %
%  SHALL NAGOYA INSTITUTE OF TECHNOLOGY, TOKYO INSTITUTE OF         %
%  TECHNOLOGY, SPTK WORKING GROUP, NOR THE CONTRIBUTORS BE LIABLE   %
%  FOR ANY SPECIAL, INDIRECT OR CONSEQUENTIAL DAMAGES OR ANY        %
%  DAMAGES WHATSOEVER RESULTING FROM LOSS OF USE, DATA OR PROFITS,  %
%  WHETHER IN AN ACTION OF CONTRACT, NEGLIGENCE OR OTHER TORTUOUS   %
%  ACTION, ARISING OUT OF OR IN CONNECTION WITH THE USE OR          %
%  PERFORMANCE OF THIS SOFTWARE.                                    %
%                                                                   %
%  ---------------------------------------------------------------  %
%
\hypertarget{linear_intpl}{}
\name{linear\_intpl}{linear interpolation of data}{data processing}

\begin{synopsis}
\item[linear\_intpl] [ --l $L$ ] [ --m $M$ ] [ --x $x_{min} \; x_{max}$ ] 
[ --i $x_{min}$ ] [ --j $x_{max}$ ] [ {\em infile} ]
\end{synopsis}

\begin{qsection}{DESCRIPTION}
{\em linear\_intpl} reads a 2-dimensional input data sequence
from {\em infile} (or standard input) and outputs $y$-axis values
when $x$-axis are linearly interpolated by equally-spaced $L-1$ points.

If the input data is
\begin{displaymath}
   \begin{matrix}
	x_0, y_0 \\
	x_1, y_1 \\
	\vdots   \\
	x_K, y_K \\
	\end{matrix}
\end{displaymath}
then the output data is
\begin{displaymath}
y_0, y_1, \dots, y_{L-1}
\end{displaymath}

\par
Input and output data are in float format.
\par
This command can interpolate data sequence whose $x$-axis is not equally-spaced,
such as digital filter characteristics.

\end{qsection}

\begin{options}
        \argm{l}{L}{output length}{256}
        \argm{m}{M}{number of interpolation points}{L-1}
        \argm{x}{x_{min} \; x_{max}}{minimum and maximum values of $x$-axis in 
        input data}{$0.0 \, 0.5$}
        \argm{i}{x_{min}}{minimum values of $x$-axis in input data}{$0.0$}
        \argm{j}{x_{max}}{maximum values of $x$-axis in input data}{$0.5$}
\end{options}

\begin{qsection}{EXAMPLE}
This example decimates input data from {\em data.f} file with interval 2,
interpolates 0 with interval 2, and then outputs it to {\em
data.di} file:

When input data {\em data.f} contains the following data,
\begin{eqnarray*}
&& 0, 2 \nonumber \\
&& 2, 2 \nonumber \\
&& 3, 0   \\
&& 5, 1 \nonumber \\
\end{eqnarray*}
this example linearly interpolates input data and outputs it to {\em data.intpl}
\begin{quote}
 \verb!linear_intpl -m 10 -x 0 5 < data.f > data.intpl!
\end{quote}
The result becomes
\begin{displaymath}
2, 2, 2, 2, 2, 1, 0, 0.25, 0.5, 0.75, 1
\end{displaymath}
\end{qsection}
% 
% \begin{qsection}{SEE ALSO}
%   decimate, interpolate
% \end{qsection}
