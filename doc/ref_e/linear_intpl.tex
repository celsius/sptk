% ----------------------------------------------------------------- %
%             The Speech Signal Processing Toolkit (SPTK)           %
%             developed by SPTK Working Group                       %
%             http://sp-tk.sourceforge.net/                         %
% ----------------------------------------------------------------- %
%                                                                   %
%  Copyright (c) 1984-2007  Tokyo Institute of Technology           %
%                           Interdisciplinary Graduate School of    %
%                           Science and Engineering                 %
%                                                                   %
%                1996-2009  Nagoya Institute of Technology          %
%                           Department of Computer Science          %
%                                                                   %
% All rights reserved.                                              %
%                                                                   %
% Redistribution and use in source and binary forms, with or        %
% without modification, are permitted provided that the following   %
% conditions are met:                                               %
%                                                                   %
% - Redistributions of source code must retain the above copyright  %
%   notice, this list of conditions and the following disclaimer.   %
% - Redistributions in binary form must reproduce the above         %
%   copyright notice, this list of conditions and the following     %
%   disclaimer in the documentation and/or other materials provided %
%   with the distribution.                                          %
% - Neither the name of the SPTK working group nor the names of its %
%   contributors may be used to endorse or promote products derived %
%   from this software without specific prior written permission.   %
%                                                                   %
% THIS SOFTWARE IS PROVIDED BY THE COPYRIGHT HOLDERS AND            %
% CONTRIBUTORS "AS IS" AND ANY EXPRESS OR IMPLIED WARRANTIES,       %
% INCLUDING, BUT NOT LIMITED TO, THE IMPLIED WARRANTIES OF          %
% MERCHANTABILITY AND FITNESS FOR A PARTICULAR PURPOSE ARE          %
% DISCLAIMED. IN NO EVENT SHALL THE COPYRIGHT OWNER OR CONTRIBUTORS %
% BE LIABLE FOR ANY DIRECT, INDIRECT, INCIDENTAL, SPECIAL,          %
% EXEMPLARY, OR CONSEQUENTIAL DAMAGES (INCLUDING, BUT NOT LIMITED   %
% TO, PROCUREMENT OF SUBSTITUTE GOODS OR SERVICES; LOSS OF USE,     %
% DATA, OR PROFITS; OR BUSINESS INTERRUPTION) HOWEVER CAUSED AND ON %
% ANY THEORY OF LIABILITY, WHETHER IN CONTRACT, STRICT LIABILITY,   %
% OR TORT (INCLUDING NEGLIGENCE OR OTHERWISE) ARISING IN ANY WAY    %
% OUT OF THE USE OF THIS SOFTWARE, EVEN IF ADVISED OF THE           %
% POSSIBILITY OF SUCH DAMAGE.                                       %
% ----------------------------------------------------------------- %
\hypertarget{linear_intpl}{}
\name{linear\_intpl}{linear interpolation of data}{data processing}

\begin{synopsis}
\item[linear\_intpl] [ --l $L$ ] [ --m $M$ ] [ --x $x_{min} \; x_{max}$ ] 
[ --i $x_{min}$ ] [ --j $x_{max}$ ] [ {\em infile} ]
\end{synopsis}

\begin{qsection}{DESCRIPTION}
{\em linear\_intpl} reads a 2-dimensional input data sequence
from {\em infile} (or standard input) and outputs $y$-axis values
when $x$-axis are linearly interpolated by equally-spaced $L-1$ points.

If the input data is
\begin{displaymath}
   \begin{matrix}
	x_0, y_0 \\
	x_1, y_1 \\
	\vdots   \\
	x_K, y_K \\
	\end{matrix}
\end{displaymath}
then the output data is
\begin{displaymath}
y_0, y_1, \dots, y_{L-1}
\end{displaymath}

\par
Input and output data are in float format.
\par
This command can interpolate data sequence whose $x$-axis is not equally-spaced,
such as digital filter characteristics.

\end{qsection}

\begin{options}
        \argm{l}{L}{output length}{256}
        \argm{m}{M}{number of interpolation points}{L-1}
        \argm{x}{x_{min} \; x_{max}}{minimum and maximum values of $x$-axis in 
        input data}{$0.0 \, 0.5$}
        \argm{i}{x_{min}}{minimum values of $x$-axis in input data}{$0.0$}
        \argm{j}{x_{max}}{maximum values of $x$-axis in input data}{$0.5$}
\end{options}

\begin{qsection}{EXAMPLE}
This example decimates input data from {\em data.f} file with interval 2,
interpolates 0 with interval 2, and then outputs it to {\em
data.di} file:

When input data {\em data.f} contains the following data,
\begin{eqnarray*}
&& 0, 2 \nonumber \\
&& 2, 2 \nonumber \\
&& 3, 0   \\
&& 5, 1 \nonumber \\
\end{eqnarray*}
this example linearly interpolates input data and outputs it to {\em data.intpl}
\begin{quote}
 \verb!linear_intpl -m 10 -x 0 5 < data.f > data.intpl!
\end{quote}
The result becomes
\begin{displaymath}
2, 2, 2, 2, 2, 1, 0, 0.25, 0.5, 0.75, 1
\end{displaymath}
\end{qsection}
% 
% \begin{qsection}{SEE ALSO}
%   decimate, interpolate
% \end{qsection}
