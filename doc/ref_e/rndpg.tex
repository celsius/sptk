\name{rndpg}%
{obtain randomized parameter sequence from PDF sequence}{parameter generation}

\def\bmath#1{\mbox{\boldmath{$#1$}}}
\def\bc{\bmath{c}}
\def\bo{\bmath{o}}
\def\bC{\bmath{C}}
\def\bO{\bmath{O}}
\def\bU{\bmath{U}}
\def\bmu{\bmath{\mu}}

\begin{synopsis}
	\item [rndpg] [ --l $L$ ] [ --m $M$ ] 
		[--d ($fn$ $|$ $d_0$ [$d_1$ $\ldots$]) ]
		[--r $N_R$ $W_1$ [$W_2$] ]
	\item [\ ~~~~] [ --i $I$ ] [ --s $S$ ] [--R] [ {\em infile} ] 
\end{synopsis}

\begin{qsection}{DESCRIPTION}
        The {\em rndpg} command reads the means and diagonal covariances
        of Gaussian distributions from the assigned file {\em infile}
        such as
 \begin{eqnarray}
	&& \ldots, \mu_t(0), \ldots, \mu_t(M),
		\mu^{(1)}_t(0), \ldots, \mu^{(1)}_t(M),
		\ldots, \mu^{(N)}_t(M),
		\nonumber\\
	&& ~~~~~~\sigma^2_t(0), \ldots, \sigma^2_t(M),
		{\sigma^{(1)}}^2_t(0), \ldots, {\sigma^{(1)}}^2_t(M),
		\ldots, {\sigma^{(N)}}^2_t(M),
		\ldots \nonumber
 \end{eqnarray}
        and evaluates the randomized parameters according to the 
	Gaussian distributions 
        sending them to the standard output.

	Input and output data are in float format.

%        The speech parameter vector $\bo_t$ for
%        every frame $t$ is composed of the static feature
%        vector $\bc_t$, where 
% \begin{displaymath}
%	\bc_t = [c_t(0), c_t(1), \ldots, c_t(M)]'
% \end{displaymath}
%	and the dynamic feature vector 
%	\ $\Delta^{(1)}\bc_t, \ldots, \Delta^{(N)}\bc_t$,
%	that is,
% \begin{displaymath}
%	\bo_t = [\bc_t', \Delta^{(1)}\bc_t', \ldots, \Delta^{(N)}\bc_t']'.
% \end{displaymath}
%        The dynamic feature vector $\Delta^{(n)}\bc_t$ is
%        obtained from the static feature vector as follows.
% \begin{displaymath}
%	\Delta^{(n)}\bc_t 
%	= \sum_{\tau=-L^{(n)}}^{L^{(n)}} w^{(n)}(\tau)\bc_{t+\tau}
% \end{displaymath}
%        where $n$ is the order of dynamic feature vector, for
%        example, when we evaluate $\Delta^2$ parameter, n=2.
%        The mlpg command reads probability density functions
%	sequence
%	$\left((\bmu_1, \bU_1), (\bmu_2, \bU_2), \ldots, (\bmu_T, \bU_T)
%	\right)$, where
% \begin{eqnarray}
%	\bmu_t 
%	& = & [\bmu^{\prime(0)}_t, \bmu^{\prime(1)}_t, 
%		\ldots, \bmu^{\prime(N)}_t]'
%		\nonumber\\
%	\bU_t 
%	& = & \mbox{diag}[\bU^{(0)}_t, \bU^{(1)}_t, \ldots, \bU^{(1)}_t]
%		\nonumber
% \end{eqnarray}
%	and evaluates the maximum likelihood parameter sequence
%	\ $(\bo_1, \bo_2, \ldots, \bo_T)$,
%	\ and sends the static feature vector sequence $\bc_t$,
%        that is
%	\ $(\bc_1, \bc_2, \ldots, \bc_T)$,
%	to the output.
%	In the example above,
%	$\bmu^{(0)}, \bU^{(0)}$ represent the static feature vector
%        mean and covariance matrix, respectively,
%        and $\bmu^{(n)}, \bU^{(n)}$ represent the $n$ order dynamic
%        feature vector mean and covariance matrix, respectively.
\end{qsection}

\begin{options}
	\argm{l}{L}{order of vector }{26}
	\argm{m}{M}{length of vector}{$L-1$}
	\argm{d}{(fn~|~d_0~[d_1~\ldots])}{$fn$ is 
                the file name of the parameters $w^{(n)}(\tau)$
               	used when evaluating the dynamic feature vector.
                It is assume that the number of coefficients
                to the left and to the right have the same length,
                if this is not true than zeros are added to the
                short side.
                For example, if the coefficients are
	 \begin{displaymath}
		w(-1), w(0), w(1), w(2), w(3)
	 \end{displaymath}
		then zeros are added to the left as follows.
	 \begin{displaymath}
		0, 0, w(-1), w(0), w(1), w(2), w(3)
	 \end{displaymath}
		Instead of entering the filename $fn$,
                the coefficients(which compose the file $fn$)
                can be directly inputed in the command line.
                When the order of the dynamic feature vector is higher
                then one, then the sets of coefficients can be inputed
                one after the other as shown on the last example below.
		this option can not be used with the --r option.}{N/A}
	\argm{r}{N_R~W_1~[W_2]}{
		This option is used when $N_R$ order dynamic parameters
                are used and the weighting coefficients $w^{(n)}(\tau)$
                are evaluated by regression.
                $N_R$ can be made equal to 1 or 2.
		The variables $W_1$ and $W_2$ represent the
                widths of the first and second order regression
                coefficients, respectively.
		The first order regression coefficients for
                $\Delta\bc_t$ at frame $t$ are evaluated as follows.
	 \begin{displaymath}
		\Delta\bc_t
		= \frac{\sum_{\tau=-W_1}^{W_1}\tau \bc_{t+\tau}}%
			{\sum_{\tau=-W_1}^{W_1}\tau^2}
	 \end{displaymath}
		For the second order regression coefficients,
		$a_2 = \sum_{\tau=-W_2}^{W_2} \tau^4$,
		$a_1 = \sum_{\tau=-W_2}^{W_2} \tau^2$,
		$a_0 = \sum_{\tau=-W_2}^{W_2} 1$
                and 
	 \begin{displaymath}
		\Delta^2\bc_t
		= \frac{\sum_{\tau=-W_2}^{W_2}
				(a_0\tau^2 - a_1) \bc_{t+\tau}}
			{2(a_2a_0-a_1^2)}
	 \end{displaymath}
		this option can not be used with the --d option.}{N/A}
	\argm{i}{I}{type of input PDFs\\
	  \begin{tabular}{l@{\hspace{2\tabcolsep}$($}l@{ }l@{$)$}}
		$I = 0$ & $\bmu,$         & $\bU$ \\
		$I = 1$ & $\bmu,$         & $\bU^{-1}$ \\
		$I = 2$ & $\bmu\bU^{-1},$ & $\bU^{-1}$ \\[1zh]
	  \end{tabular}
		}{0}
	\argm{s}{S}{range of influenced frames}{30}
	\argm{R}{}{output randomized parameters}{TRUE}
\end{options}

\begin{qsection}{EXAMPLE}
	In the example below,
        the number of parameters is 15, the width of
        the window for 1st or 2nd order dynamic feature
        evaluation is 1, and the parameter sequence is
        evaluated from the probability density function.
 \begin{quote}
	\verb!rndpg -m 15 -r 2 1 1 data.pdf > data.par!
 \end{quote}
	or
 \begin{quote}
	\verb!echo "-0.5 0 0.5" | x2x +af > delta! \\
	\verb!echo "0.25 -0.5 0.25" | x2x +af > accel! \\
	\verb!rndpg -m 15 -d delta -d accel data.pdf > data.par!
 \end{quote}
\end{qsection}

%\begin{qsection}{SEE ALSO}
%\end{qsection}
