% ----------------------------------------------------------------- %
%             The Speech Signal Processing Toolkit (SPTK)           %
%             developed by SPTK Working Group                       %
%             http://sp-tk.sourceforge.net/                         %
% ----------------------------------------------------------------- %
%                                                                   %
%  Copyright (c) 1984-2007  Tokyo Institute of Technology           %
%                           Interdisciplinary Graduate School of    %
%                           Science and Engineering                 %
%                                                                   %
%                1996-2008  Nagoya Institute of Technology          %
%                           Department of Computer Science          %
%                                                                   %
% All rights reserved.                                              %
%                                                                   %
% Redistribution and use in source and binary forms, with or        %
% without modification, are permitted provided that the following   %
% conditions are met:                                               %
%                                                                   %
% - Redistributions of source code must retain the above copyright  %
%   notice, this list of conditions and the following disclaimer.   %
% - Redistributions in binary form must reproduce the above         %
%   copyright notice, this list of conditions and the following     %
%   disclaimer in the documentation and/or other materials provided %
%   with the distribution.                                          %
% - Neither the name of the SPTK working group nor the names of its %
%   contributors may be used to endorse or promote products derived %
%   from this software without specific prior written permission.   %
%                                                                   %
% THIS SOFTWARE IS PROVIDED BY THE COPYRIGHT HOLDERS AND            %
% CONTRIBUTORS "AS IS" AND ANY EXPRESS OR IMPLIED WARRANTIES,       %
% INCLUDING, BUT NOT LIMITED TO, THE IMPLIED WARRANTIES OF          %
% MERCHANTABILITY AND FITNESS FOR A PARTICULAR PURPOSE ARE          %
% DISCLAIMED. IN NO EVENT SHALL THE COPYRIGHT OWNER OR CONTRIBUTORS %
% BE LIABLE FOR ANY DIRECT, INDIRECT, INCIDENTAL, SPECIAL,          %
% EXEMPLARY, OR CONSEQUENTIAL DAMAGES (INCLUDING, BUT NOT LIMITED   %
% TO, PROCUREMENT OF SUBSTITUTE GOODS OR SERVICES; LOSS OF USE,     %
% DATA, OR PROFITS; OR BUSINESS INTERRUPTION) HOWEVER CAUSED AND ON %
% ANY THEORY OF LIABILITY, WHETHER IN CONTRACT, STRICT LIABILITY,   %
% OR TORT (INCLUDING NEGLIGENCE OR OTHERWISE) ARISING IN ANY WAY    %
% OUT OF THE USE OF THIS SOFTWARE, EVEN IF ADVISED OF THE           %
% POSSIBILITY OF SUCH DAMAGE.                                       %
% ----------------------------------------------------------------- %
\hypertarget{rmse}{}
\name{rmse}{calculation of root mean squared error}{data processing}

\begin{synopsis}
\item [rmse] [ --l $L$ ] {\em file1} [ {\em infile} ]
\end{synopsis}

\begin{qsection}{DESCRIPTION}
{\em rmse} calculates RMSE (Root Mean Square Error) of input data sequences 
from {\em infile} (or standard input) and {\em file1}, 
sending the results to standard output.

From given two files, $L$-length time series 
\begin{displaymath}
  \underbrace{x_1(0),x_1(1),\dots,x_1(L-1)},\underbrace{x_2(0),x_2(1),\dots}
\end{displaymath}
and
\begin{displaymath}
  \underbrace{y_1(0),y_1(1),\dots,y_1(L-1)},\underbrace{y_2(0),y_2(1),\dots}
\end{displaymath}
are read,
and then RMSE of these two series are calculated and output
\begin{displaymath}
\mathrm{RMSE}_j = \sqrt{\sum_{m=0}^{L-1} (x_j(m)-y_j(m))^2/L}
\end{displaymath}

Input and output data are in float format.
\end{qsection}

\begin{options}
        \argm{l}{L}{data length to calculate RMSE.\\
                    If $L=0$, RMSE of whole input data is output.}{0}
\end{options}

\begin{qsection}{EXAMPLE}
This example calculates the RMSE of input data files {\em data.f1} and {\em
data.f2}, and output its maximum and minimum values:
\begin{quote}
 \verb!rmse -l 26 data.f1 data.f2 | minmax | dmp +f!
\end{quote}
\end{qsection}

\begin{qsection}{SEE ALSO}
\hyperlink{histogram}{histogram},
\hyperlink{minmax}{minmax}
\end{qsection}
