%  ---------------------------------------------------------------  %
%            Speech Signal Processing Toolkit (SPTK)                %
%                      SPTK Working Group                           %
%                                                                   %
%                  Department of Computer Science                   %
%                  Nagoya Institute of Technology                   %
%                               and                                 %
%   Interdisciplinary Graduate School of Science and Engineering    %
%                  Tokyo Institute of Technology                    %
%                                                                   %
%                     Copyright (c) 1984-2007                       %
%                       All Rights Reserved.                        %
%                                                                   %
%  Permission is hereby granted, free of charge, to use and         %
%  distribute this software and its documentation without           %
%  restriction, including without limitation the rights to use,     %
%  copy, modify, merge, publish, distribute, sublicense, and/or     %
%  sell copies of this work, and to permit persons to whom this     %
%  work is furnished to do so, subject to the following conditions: %
%                                                                   %
%    1. The source code must retain the above copyright notice,     %
%       this list of conditions and the following disclaimer.       %
%                                                                   %
%    2. Any modifications to the source code must be clearly        %
%       marked as such.                                             %
%                                                                   %
%    3. Redistributions in binary form must reproduce the above     %
%       copyright notice, this list of conditions and the           %
%       following disclaimer in the documentation and/or other      %
%       materials provided with the distribution.  Otherwise, one   %
%       must contact the SPTK working group.                        %
%                                                                   %
%  NAGOYA INSTITUTE OF TECHNOLOGY, TOKYO INSTITUTE OF TECHNOLOGY,   %
%  SPTK WORKING GROUP, AND THE CONTRIBUTORS TO THIS WORK DISCLAIM   %
%  ALL WARRANTIES WITH REGARD TO THIS SOFTWARE, INCLUDING ALL       %
%  IMPLIED WARRANTIES OF MERCHANTABILITY AND FITNESS, IN NO EVENT   %
%  SHALL NAGOYA INSTITUTE OF TECHNOLOGY, TOKYO INSTITUTE OF         %
%  TECHNOLOGY, SPTK WORKING GROUP, NOR THE CONTRIBUTORS BE LIABLE   %
%  FOR ANY SPECIAL, INDIRECT OR CONSEQUENTIAL DAMAGES OR ANY        %
%  DAMAGES WHATSOEVER RESULTING FROM LOSS OF USE, DATA OR PROFITS,  %
%  WHETHER IN AN ACTION OF CONTRACT, NEGLIGENCE OR OTHER TORTUOUS   %
%  ACTION, ARISING OUT OF OR IN CONNECTION WITH THE USE OR          %
%  PERFORMANCE OF THIS SOFTWARE.                                    %
%                                                                   %
%  ---------------------------------------------------------------  %
%
\hypertarget{rmse}{}
\name{rmse}{calculate RMSE}{data processing}

\begin{synopsis}
\item [rmse] [ --l $L$ ] {\em file1} [ {\em infile} ]
\end{synopsis}

\begin{qsection}{DESCRIPTION}
{\em rmse} calculates RMSE (Root Mean Square Error) of input data sequences 
from {\em infile} (or standard input) and {\em file1}, 
sending the results to standard output.

From given two files, $L$-length time series 
\begin{displaymath}
  \underbrace{x_1(0),x_1(1),\dots,x_1(L-1)},\underbrace{x_2(0),x_2(1),\dots}
\end{displaymath}
and
\begin{displaymath}
  \underbrace{y_1(0),y_1(1),\dots,y_1(L-1)},\underbrace{y_2(0),y_2(1),\dots}
\end{displaymath}
are read,
and then RMSE of these two series are calculated and output
\begin{displaymath}
\mathrm{RMSE}_j = \sqrt{\sum_{m=0}^{L-1} (x_j(m)-y_j(m))^2/L}
\end{displaymath}

Input and output data are in float format.
\end{qsection}

\begin{options}
        \argm{l}{L}{data length to calculate RMSE.\\
                    If $L=0$, RMSE of whole input data is output.}{0}
\end{options}

\begin{qsection}{EXAMPLE}
This example calculates the RMSE of input data files {\em data.f1} and {\em
data.f2}, and output its maximum and minimum values:
\begin{quote}
 \verb!rmse -l 26 data.f1 data.f2 | minmax | dmp !
\end{quote}
\end{qsection}

\begin{qsection}{SEE ALSO}
\hyperlink{histogram}{histogram},
\hyperlink{minmax}{minmax}
\end{qsection}
