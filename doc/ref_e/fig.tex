\name{fig}{plot a graph}{graph}

\begin{synopsis}
 \item[fig] [ --F $F$ ] [ --R $R$ ] [ --W $W$ ] [ --H $H$] [ --o $xo$ $yo$ ]
	    [ --g $G$ ]  [ --p $P$ ] 
 \item[\ ~~~] [ --s $S$ ] [ --f $file$ ] [ --t ] [ {\em infile} ]
\end{synopsis}

\begin{qsection}{DESCRIPTION}
This command draws a graph using information from the input file
{\em infile}. 
If {\em infile} is not assigned, data is read from the starndard input.
This command is similar to the unix command ``graph'' but includes
some labeling functions.
The output file comes with a sequence of commands for direct printing.
For this reason, in case the output is printed on the screen,
one should use the ``xgr'' command.
In case the output is printed in postscript, one should use the
``psgr'' command.
\end{qsection}

\begin{options}
	\argm{F}{F}{factor}{1}
	\argm{R}{R}{rotation angle}{0}
	\argm{W}{W}{width of figure($\times 100$mm)}{1}
	\argm{H}{H}{height of figure($\times 100$mm)}{1}
	\argm{o}{xo \; yo}{origin in mm}{20 20}
	\argm{g}{G}{draw grid($0 \sim 2$)\\
			\begin{tabular}{cccc}
			&\epsfxsize=2cm \epsffile{fig/g0.eps}
			&\epsfxsize=2cm \epsffile{fig/g1.eps}
			&\epsfxsize=2cm \epsffile{fig/g2.eps}\\
			$G$&0&1&2	 
			\end{tabular}\\\hspace*{\fill}}{2}
	\argm{p}{P}{pen number($1 \sim 10$)}{1}
	\argm{s}{S}{font size($1 \sim 4$)}{1}
	\argm{f}{file}{The file assigned after this option is read
                       before {\em infile}, that is, this option gives
                       preference.}{NULL}
	\argm{t}{}{transpose $x$ and $y$ axes}{FALSE}
\end{options}

\begin{qsection}{EXAMPLE}
Data in {\em data.fig} file is plotted in an X terminal in the
example below:
\vspace{-3mm}
\begin{quote}
 \verb!fig data.fig |xgr!
\end{quote}
\vspace{-3mm}
In this example, data in {\em data.fig} file is written in postscript,
and seen with ghostview:
\vspace{-3mm}
\begin{quote}
 \verb!fig data.fig | psgr | ghostview -!
\end{quote}
\vspace{-3mm}
\end{qsection}

\vspace{-1cm}
\begin{qsection}{USAGE}
$B!!(B\vspace{-1cm}
\end{qsection}

\begin{qsection}{\ ~~~COMMAND}
The input data file can contain commands and data.
Commands can be used for labeling, scaling, etc.
Data is written in the ($x ~y$) coordinate pair form.
Command values can be overwritten by entering new command values.
\end{qsection}

\vspace{-1cm}
\begin{qsection}{\ ~~~COMMAND PART}
\begin{minipage}[t]{5.5cm}
x [mel $\alpha$]~ $xmin$ ~$ymax$ [$xa$]\\
y [mel $\alpha$]~ $xmin$ ~$ymax$ [$ya$]\\
\end{minipage}
\begin{minipage}[t]{9cm}
Assigns $x$ and $y$ scalings.\\
Marks can be assigned in $x$ and $y$ axes through $xa$ and $ya$.
If no assigned of $xa$ and $ya$ is done,
then $xa = xmin$ and $ya = ymin$.
If the optional ``mel $\alpha$'', where $\alpha$ must be
a number (for example, mel 0.35), is used,
then labeling is undertaken as frequancy transformation of
minimum phase first order all-pass filter.
\end{minipage} \\

\begin{minipage}[t]{5.5cm}
xscale ~$x_1$ $x_2$ $x_3$ $\cdots$\\
yscale ~$y_1$ $y_2$ $y_3$ $\cdots$
\end{minipage}
\begin{minipage}[t]{9cm}
Assigns the points $x_1, x_2,$\\$x_3,\cdots$
and $y_1,y_2,y_3,\cdots$ in $x$ and $y$ axes.
These points can be assigned with numbers or marks,
Also, when a mixture of not-a-number $+$ number (for example, '2,*3.14)
is needed, following function should be used:

\begin{tabular}{cl}
s & draws marks with half size.\\
$\backslash$& only writes number.\\
@ & does not write anything but assigns positions of marks.\\
none of the above & only marks are written.
\end{tabular}\\

Whenever character is inside quotes,
it appear in the position assigned
by the string that precedes it.
Please refer to the commands ``x/yname'' for information on
special characters.\\
(Example)\\
x ~0 ~5\\
xscale 0~1.0 ~s1.5 ~'2 ~$\backslash$2.5 ~'3.14 ~''$\backslash$pi'' ~@4 ~''x'' ~5\\

\epsfxsize=9cm
\epsffile{fig/scale.eps}
\end{minipage}\\

\begin{minipage}[t]{5.5cm}
xname ~''$text$''\\
yname ~''$text$''\\
\end{minipage}
\begin{minipage}[t]{9cm}
Labels $x$ and $y$ axes.
$text$ should be inside quote.
Inside $text$, \TeX commands can be used.
Also, characters such as those that can be obtained
with \TeX can be written with this command.
\end{minipage}\\

\begin{minipage}[t]{5.5cm}
 print ~x ~y ~''$text$'' [$th$]\\
 printc ~x ~y ~''$text$'' [$th$]
\end{minipage}
\begin{minipage}[t]{9cm}
This command writes $text$ in the assigned position (x ~y).
The option $th$ assigned the rotation degree.

\begin{tabular}{cc}
\epsffile{fig/fig-print1.eps}&  
\epsffile{fig/fig-print2.eps}\\
print&printc
\end{tabular}p
\end{minipage}\\

\begin{minipage}[t]{5.5cm}
title ~x ~y ~''$text$'' [$th$]\\
titlec ~x ~y ~''$text$'' [$th$]
\end{minipage}
\begin{minipage}[t]{9cm}
This command is same as print(c).
However, the basic unit is expressed in absolute value mm.
The reference point is on the botton-left.
\end{minipage}\\

\begin{minipage}[t]{5.5cm}
csize ~h [w]
\end{minipage}
\begin{minipage}[t]{9cm}
This command assigns in mm the character width and height,
to be used in the following commands:\\
x/yscale$B!$(Bx/yname$B!$(Bprint/c$B!$(Btitle/c\\
When the value of $w$ is omitted, $w$ is made equal to $h$.
The default values for the option --{\bf s} follows:
\begin{tabular}{ccc}

--{\bf s} &w &h  \\ \hline
1 &2.5 &2.2\\
2&5&2.6\\
3&2.5&4.4\\
4&5&4.4
\end{tabular}\\

\end{minipage}\\

\begin{minipage}[t]{5.5cm}
 pen ~$penno$
\end{minipage}
\begin{minipage}[t]{9cm}
This command chooses the variable $penno$.
1$B!e(B$penno$$B!e(B10.   %????????????????????????????????????????????
Please refer to appendix.
\end{minipage}\\

\begin{minipage}[t]{5.5cm}
line ~$ltype$ [$lpt$]
\end{minipage}
\begin{minipage}[t]{9cm}
This command assigns the type $ltype$ of the line which will connect
data as well as the $lpt$ pace. $lpt$ is on mm unit.
When $ltype$=0: no line is used to connect coordinate points.
1:~solid~~2:~dotted~~3:~dot and dash~~4:~broken~~5:~dash
Please refer to the appendix.\\
	
\end{minipage}\\

\begin{minipage}[t]{5.5cm}
xgrid ~$x_1$ ~$x_2$ ~$\cdots$\\
ygrid ~$y_1$ ~$y_2$ ~$\cdots$
\end{minipage}
\begin{minipage}[t]{9cm}
$x_1$ $x_2$ $\cdots$$B!$(B $y_1$ $y_2$ $\cdots$ $B$N0LCV$K!$$=$l$>$l=D$H2#$N%0%j%C%I(B
$B$r0z$-$^$9!%(B\\
($BNc(B)\\
\begin{minipage}[t]{4.3cm}
 \epsfxsize=4cm
 \epsffile{fig/grid.eps}
\end{minipage}
\begin{minipage}[b]{4.5cm}
\baselineskip 5pt
x 0 5\\
y 0 6\\
xscale 0 1 2 3 4 5\\
yscale 0 2 4 6

\vspace{3mm}
xgrid 1 2 3 4\\
ygrid 2 4
\vspace*{1cm}
\end{minipage}
\end{minipage}\\


\begin{minipage}[t]{5.5cm}
mark ~$label$ [$th$]
\end{minipage}
\begin{minipage}[t]{9cm}
$B%G!<%?9T$GM?$($i$l$k:BI8$KJ8;zNs(B $label$ $B$r=q$-$^$9!%(B
$th$ $B$OJ8;zNs$r=q$/3QEY(B(deg)$B$G$9!%(B
$label$ $B$K(B $\backslash 0$ $B$r;XDj$9$k$H2r=|$7$^$9!%(B
$B%^!<%/!$FC<lJ8;z$N=q$-J}$O%G!<%?9T$N(B $label$ $B9`$r;2>H$7$F2<$5$$!%(B
\end{minipage}\\

\begin{minipage}[t]{5.5cm}
hight ~$h$ [$w$]\\
italic ~$th$
\end{minipage}
\begin{minipage}[t]{9cm}
$B%G!<%?9T$G(B$label$ $B$r;XDj$7$?>l9g$K=q$/J8;z$NBg$-$5(B $h$(mm)$B!$4V3V(B $w$(mm)
$B$H%$%?%j%C%/;XDj(B $th$(deg)$B$r$7$^$9!%(B 
\end{minipage}\\

\begin{minipage}[t]{5.5cm}
circle ~x ~y ~$r_1$ ~$r_2$ ~$\cdots$\\
xcircle ~x ~y ~$r_1$ ~$r_2$ ~$\cdots$\\
ycircle ~x ~y ~$r_1$ ~$r_2$ ~$\cdots$

\end{minipage}
\begin{minipage}[t]{9cm}
(x~ y)$B$rCf?4$H$7H>7B$,(B$r_1$ ~$r_2$ ~$\cdots$ $B$N1_$rIA$-$^$9!%(B
$BC"$7!$(B$r_x$ $B$NC10L$O(Bcircle$B$O(Bmm$B!$(Bxcircle$B$O%0%i%U$N(B $x$ $B%9%1!<%k!$(Bycircle$B$O%0%i%U(B
$B$N(B $y$ $B%9%1!<%k$G$9!%(B\\

($BNc(B)\\
\begin{minipage}[t]{4.3cm}
 \epsfxsize=4cm
 \epsffile{fig/circle.eps}
\end{minipage}
\begin{minipage}[b]{4.5cm}
\baselineskip 5pt
x 0 5\\
y 0 20\\
xscale 0 5\\
yscale 0 20

\vspace*{3mm}
xcircle 3 10 1 2\\
ycircle 1 3 1 2\\
circle  1.5 15 13\\
\vspace*{7mm}
\end{minipage}
\end{minipage}\\

\begin{minipage}[t]{5.5cm}
box ~$x_0$ $y_0$ $x_1$ ~$y_1$ [~$x_2$ $y_2$ $\cdots$ ]\\
paint ~$type$
\end{minipage}
\begin{minipage}[t]{9cm}
 ($x_0$ $y_0$)$B!$(B($x_1$ $y_1$)$B$r7k$VD>@~$rBP3Q@~$H$9$kD9J}7A$r(B paint$B$G;XDj(B
$B$5$l$?(B $type$$B$GIA$-$^$9!%(B
$BC"$7!$(B~$x_2$ $y_2$ $\cdots$ $B$,;XDj$5$l$?>l9g!$(B ($x_0$ $y_0$),($x_1$ $y_1$),
($x_2$ $y_2$),$\cdots$ $B$r7k$VB?3Q7A$rIA$-$^$9(B(paint$B$O(B $type$ 1 $B0J30;XDj$7$J$$(B
$B$G2<$5$$(B)$B!%(B
paint $B$N%G%U%)%k%H$O(B1$B$G$9!%(B\\

($BNc(B)\\
\begin{minipage}[t]{4.3cm}
 \epsfxsize=4cm
 \epsffile{fig/box.eps}
\end{minipage}
\begin{minipage}[b]{4.5cm}
\baselineskip 5pt
x 0 10\\
y 0 10\\
xscale 0 10\\
yscale 0 10

\vspace*{3mm}
paint 18\\
box 2.5 0 3.5 6\\
paint -18 \\
box 4 0 5 8\\
paint 1\\
box  2 2 8 8 8 2 4 7
\end{minipage}
\end{minipage}\\

\begin{minipage}[t]{5.5cm}
clip ~$x_0$ $y_0$ $x_1$ ~$y_1$ 
\end{minipage}
\begin{minipage}[t]{9cm}
  ($x_0$ $y_0$)$B!$(B($x_1$ $y_1$)$B$r7k$VD>@~$rBP3Q@~$H$9$kD9J}7A$NHO0O$NCf$@$1(B
$BIA$-$^$9!%(B
  ($x_0$ $y_0$)$B!$(B($x_1$ $y_1$)$B$,>JN,$5$l$?;~!$(Bclip$B$r2r=|$7$^$9!%(B\\

($BNc(B)\\
\begin{minipage}[t]{4.3cm}
 \epsfxsize=4cm
 \epsffile{fig/clip.eps}
\end{minipage}
\begin{minipage}[b]{4.5cm}
\baselineskip 5pt
x 0 10\\
y 0 10\\
xscale 0 10\\
yscale 0 10

\vspace*{3mm}
clip 2 3 9 7\\
paint 18\\
box 2.5 0 3.5 6\\
paint -18 \\
box 4 0 5 8\\
paint 1\\
box  2 2 8 8 8 2 4 7
\end{minipage}
\end{minipage}\\

\begin{minipage}[t]{5.5cm}
\# any commet
\end{minipage}
\begin{minipage}[t]{9cm}
 $B%3%a%s%H9T$G$9!%F0:n$K1F6A$r5Z$\$7$^$;$s!%(B
\end{minipage}\\
\end{qsection}

\begin{qsection}{\ ~~~$B%G!<%?9T(B}
\begin{minipage}[t]{5.5cm}
 x ~y [$label$~ [$th$]]
\end{minipage}
\begin{minipage}[t]{9cm}
(x ~y)$B:BI8$O!$%3%^%s%I9T(B x, y $B$G;XDj$5$l$??tCM$G%9%1!<%j%s%0$5$l$^$9!%(B
$label$ $B$N0LCV$KJ8;zNs$r=q$/$H!$$=$NJ8;zNs$,(B(x ~y)$B$N0LCV$K=q$+$l$^$9!%(B
$label$ $B$O6uGrJ8;z$r4^$s$G$O$$$1$^$;$s!%(B
$B%3%^%s%I9T(B mark $B$G(B$label$$B$,;XDj$5$l$F$$$k>l9g$O!$$3$N:BI8$K8B$j;XDj$N(B$label$
$B$KCV$-49$o$j$^$9!%(B$th$$B$O3QEY$rM?$($^$9!%(B\\
$label$ $BJ8;zNs$K(B ~$\backslash n$ ~~$0$B!e(Bn$B!e(B15$ $B$r;XDj$9$k$H!$BP1~$9$kHV9f$N(B
$B%^!<%/$rIA$-$^$9(B($B%^!<%/$N<oN`$OIUO?$r;2>H$7$F2<$5$$(B)$B!%(B
$B%^!<%/HV9f$K(B-- ($B%^%$%J%9(B)$B$rIU$1$k$H!$%^!<%/$NCf?4$r@~$G7k$S$^$9!%(B
$BDL>o$O!$%^!<%/$H@~$,=E$J$i$J$$$h$&$KIA$-$^$9!%(B\\
$n=16(\backslash 16)$ $B$G(B $B%3%^%s%I9T(B hight $B$G;XDj$5$l$?(B h $B$rD>7B$H$9$k>.1_$r(B
$BIA$-$^$9!%(B
$B$^$?!$(B $n>32$ $B$G;XDj%3!<%IHV9f$N(BASCII$BJ8;z$^$?$OFC<lJ8;z$rIA$-$^$9!%(B
\end{minipage}\\

\begin{minipage}[t]{5.5cm}
 eod\\
EOD
\end{minipage}
\begin{minipage}[t]{9cm}
 $B%G!<%?$N6h@Z$j$G$9!%$3$NA08e$N:BI84V$K$O@~$r0z$-$^$;$s!%(B
\end{minipage}
\end{qsection}
\newpage
\begin{qsection}{$BIUO?(B}
{\large \hspace{-1.5ex}$\bullet$ mark $BKt$O%G!<%?9T$G(B $label$ $B$,;XDj$5$l$?>l9g(B
$B$K=PNO$5$l$k%^!<%/(B:}

\begin{center}
\leavevmode
\epsfxsize=12cm
\epsffile{fig/mark.eps} \\
\end{center}


{\large \hspace{-1.5ex}$\bullet$ pen $B$H(B line $B$N%?%$%W(B:}\\
\hspace{3mm}[psgr$B$G=PNO$7$?>l9g(B]

\leavevmode
\epsffile{fig/pen-line.eps} \\

($BCm(B)~~ pen$B$N%?%$%W$O=PNO$9$k%W%j%s%?$K0MB8$7$^$9!%(B($B$3$N%Z!<%8$r=PNO$7$F$_$F(B
$B2<$5$$(B)\\
\newpage
[xgr $B$G=PNO$7$?>l9g(B]\\
$B2<I=$K<($9?'$GI=<($5$l$^$9!%(B\\

\begin{center}
\begin{tabular}{|c|c|c|c|c|c|c|c|c|c|c|}
 \hline
 pen$B%?%$%W(B& 1& 2& 3& 4& 5& 6& 7& 8& 9&10  \\ \hline
 $B?'(B& $B9u(B& $B@D(B& $B@V(B& $BNP(B& $B%T%s%/(B& $B%*%l%s%8(B& $B%(%a%i%k%I(B& $B3%(B &$BCc(B & $B:0(B 
 \\ \hline
\end{tabular}\\
\end{center}

\vspace{5mm}
{\large \hspace{-1.5ex}$\bullet$ paint$B$N%?%$%W(B:}\\
\epsffile{fig/paint.eps}\\
($BCm(B)~~~$1 \sim 3$ $B$OOH$N$_!$(B$-9 , -19$ $B$OOHL5$7$NGr$G$9!%(B

\end{qsection}
