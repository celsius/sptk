% ----------------------------------------------------------------- %
%             The Speech Signal Processing Toolkit (SPTK)           %
%             developed by SPTK Working Group                       %
%             http://sp-tk.sourceforge.net/                         %
% ----------------------------------------------------------------- %
%                                                                   %
%  Copyright (c) 1984-2007  Tokyo Institute of Technology           %
%                           Interdisciplinary Graduate School of    %
%                           Science and Engineering                 %
%                                                                   %
%                1996-2011  Nagoya Institute of Technology          %
%                           Department of Computer Science          %
%                                                                   %
% All rights reserved.                                              %
%                                                                   %
% Redistribution and use in source and binary forms, with or        %
% without modification, are permitted provided that the following   %
% conditions are met:                                               %
%                                                                   %
% - Redistributions of source code must retain the above copyright  %
%   notice, this list of conditions and the following disclaimer.   %
% - Redistributions in binary form must reproduce the above         %
%   copyright notice, this list of conditions and the following     %
%   disclaimer in the documentation and/or other materials provided %
%   with the distribution.                                          %
% - Neither the name of the SPTK working group nor the names of its %
%   contributors may be used to endorse or promote products derived %
%   from this software without specific prior written permission.   %
%                                                                   %
% THIS SOFTWARE IS PROVIDED BY THE COPYRIGHT HOLDERS AND            %
% CONTRIBUTORS "AS IS" AND ANY EXPRESS OR IMPLIED WARRANTIES,       %
% INCLUDING, BUT NOT LIMITED TO, THE IMPLIED WARRANTIES OF          %
% MERCHANTABILITY AND FITNESS FOR A PARTICULAR PURPOSE ARE          %
% DISCLAIMED. IN NO EVENT SHALL THE COPYRIGHT OWNER OR CONTRIBUTORS %
% BE LIABLE FOR ANY DIRECT, INDIRECT, INCIDENTAL, SPECIAL,          %
% EXEMPLARY, OR CONSEQUENTIAL DAMAGES (INCLUDING, BUT NOT LIMITED   %
% TO, PROCUREMENT OF SUBSTITUTE GOODS OR SERVICES; LOSS OF USE,     %
% DATA, OR PROFITS; OR BUSINESS INTERRUPTION) HOWEVER CAUSED AND ON %
% ANY THEORY OF LIABILITY, WHETHER IN CONTRACT, STRICT LIABILITY,   %
% OR TORT (INCLUDING NEGLIGENCE OR OTHERWISE) ARISING IN ANY WAY    %
% OUT OF THE USE OF THIS SOFTWARE, EVEN IF ADVISED OF THE           %
% POSSIBILITY OF SUCH DAMAGE.                                       %
% ----------------------------------------------------------------- %
\hypertarget{mcep}{}
\name[ref:mcep-IEICE,ref:amcep-ICASSP92]{mcep}{mel cepstral analysis}%
{speech analysis}

\begin{synopsis}
\item [mcep] [ --a $A$ ] [ --m $M$ ] [ --l $L$ ] [ --q $Q$ ] [ --i $I$ ] [ --j $J$ ] 
	     [ --d $D$ ] [ --e $E$ ] [ --f $F$ ]
\item [\ ~~~~~~~~] [ {\em infile} ]
\end{synopsis}

\begin{qsection}{DESCRIPTION}
{\em mcep} uses mel-cepstral analysis 
to calculate mel-cepstral coefficients $c_{\alpha}(m)$ 
from $L$-length framed windowed data from {\em infile} (or standard input), 
sending the result to standard output.

Input and output data are in float format.

In the mel-cepstral analysis, the spectrum of the speech signal
is modeled by $M$-th order mel-cepstral coefficients $c_{\alpha}(m)$
as follows.
\begin{displaymath}
H(z) = \exp \sum_{m=0}^M c_{\alpha}(m) \tilde{z}^{-m} 
\end{displaymath}
For this command ``mcep'', it is applied a cost function
based on the unbiased estimation log spectrum method.
The variable $\tilde{z}^{-1}$ can be expressed as the following
first order all-pass function
\begin{displaymath}
\tilde{z}^{-1} = \frac{z^{-1}-\alpha}{1-\alpha z^{-1}}.
\end{displaymath}
The phase characteristic is given by the variable $\alpha$.
For a sampling rate 16 kHz, $\alpha$ is made equal to $0.42$.
For a sampling rate 10 kHz, $\alpha$ is made equal to $0.35$.
For a sampling rate 8 kHz, $\alpha$ is made equal to $0.31$.
By making these choices for $\alpha$,
the mel-scale becomes the good approximation to human
sensitivity to the loudness speech sound.

The Newton-Raphson method is used to minimize the cost function
when evaluating mel-cepstral coefficients.
\end{qsection}

\begin{options}
	\argm{a}{A}{all-pass constant $\alpha$}{0.35}
	\argm{m}{M}{order of mel cepstrum}{25}
	\argm{l}{L}{frame length}{256}
    \argm{q}{Q}{input data style\\
        \begin{tabular}{ll} \\[-1ex]
            $Q=0$ & windowed data sequence \\
            $Q=1$ & $20 \times \log |f(w)|$ \\
            $Q=2$ & $\ln |f(w)|$ \\
            $Q=3$ & $|f(w)|$ \\
            $Q=4$ & $|f(w)|^2$ \\[1ex]
        \end{tabular}\\\hspace*{\fill}}{0}
	\desc[1ex]{Usually, the options below do not need to be assigned.}
	\argm{i}{I}{minimum iteration of Newton-Raphson method}{2}
	\argm{j}{J}{maximum iteration of Newton-Raphson method}{30}
	\argm{d}{D}{end condition of Newton-Raphson}{0.001}
	\argm{e}{E}{small value added to periodgram}{0.0}
	\argm{f}{F}{minimum value of the determinant of the normal matrix}{0.000001}
\end{options}

\begin{qsection}{EXAMPLE}
Speech data is read in float format from {\em data.f} and 
analyzed, and mel-cepstral coefficients are written to {\em data.mcep}:
\begin{quote}
 \verb!frame +f < data.f | window | mcep > data.mcep !
\end{quote}
\begin{quote}
 \verb!frame +f < data.f | window | fftr -A -H | mcep -q 3 > data.mcep !
\end{quote}
\end{qsection}

\begin{qsection}{SEE ALSO}
\hyperlink{uels}{uels},
\hyperlink{gcep}{gcep},
\hyperlink{mgcep}{mgcep},
\hyperlink{mlsadf}{mlsadf}
\end{qsection}
