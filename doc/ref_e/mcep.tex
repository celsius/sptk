% ----------------------------------------------------------------
%       Speech Signal Processing Toolkit (SPTK): version 3.0
%                      SPTK Working Group
% 
%                Department of Computer Science
%                Nagoya Institute of Technology
%                             and
%   Interdisciplinary Graduate School of Science and Engineering
%                Tokyo Institute of Technology
%                   Copyright (c) 1984-2000
%                     All Rights Reserved.
% 
% Permission is hereby granted, free of charge, to use and
% distribute this software and its documentation without
% restriction, including without limitation the rights to use,
% copy, modify, merge, publish, distribute, sublicense, and/or
% sell copies of this work, and to permit persons to whom this
% work is furnished to do so, subject to the following conditions:
% 
%   1. The code must retain the above copyright notice, this list
%      of conditions and the following disclaimer.
% 
%   2. Any modifications must be clearly marked as such.
%                                                                        
% NAGOYA INSTITUTE OF TECHNOLOGY, TOKYO INSITITUTE OF TECHNOLOGY,
% SPTK WORKING GROUP, AND THE CONTRIBUTORS TO THIS WORK DISCLAIM
% ALL WARRANTIES WITH REGARD TO THIS SOFTWARE, INCLUDING ALL
% IMPLIED WARRANTIES OF MERCHANTABILITY AND FITNESS, IN NO EVENT
% SHALL NAGOYA INSTITUTE OF TECHNOLOGY, TOKYO INSITITUTE OF
% TECHNOLOGY, SPTK WORKING GROUP, NOR THE CONTRIBUTORS BE LIABLE
% FOR ANY SPECIAL, INDIRECT OR CONSEQUENTIAL DAMAGES OR ANY
% DAMAGES WHATSOEVER RESULTING FROM LOSS OF USE, DATA OR PROFITS,
% WHETHER IN AN ACTION OF CONTRACT, NEGLIGENCE OR OTHER TORTIOUS
% ACTION, ARISING OUT OF OR IN CONNECTION WITH THE USE OR
% PERFORMANCE OF THIS SOFTWARE.
% ----------------------------------------------------------------
%
\hypertarget{mcep}{}
\name[ref:mcep-IEICE,ref:amcep-ICASSP92]{mcep}{mel cepstral analysis}%
{speech analysis}

\begin{synopsis}
\item [mcep] [ --a $A$ ] [ --m $M$ ] [ --l $L$ ] [ --i $I$ ] [ --j $J$ ] 
	     [ --d $D$ ] [ --e $E$ ] [ {\em infile} ]
\end{synopsis}

\begin{qsection}{DESCRIPTION}
{\em mcep} uses mel-cepstral analysis 
to calculate mel-cepstral coefficients $c_{\alpha}(m)$ 
from $L$-length framed windowed data from {\em infile} (or standard input), 
sending the result to standard output.

Input and output data are in float format.

In the mel-cepstral analysis, the spectrum of the speech signal
is modeled by $M$-th order mel-cepstral coefficients $c_{\alpha}(m)$
as follows.
\begin{displaymath}
H(z) = \exp \sum_{m=0}^M c_{\alpha}(m) \tilde{z}^{-m} 
\end{displaymath}
For this command ``mcep'', it is applied a cost function
based on the unbiased estimation log spectrum method.
The variable $\tilde{z}^{-1}$ can be expressed as the following
first order all-pass function
\begin{displaymath}
\tilde{z}^{-1} = \frac{z^{-1}-\alpha}{1-\alpha z^{-1}}.
\end{displaymath}
The phase characteristic is given by the variable $\alpha$.
For a sampling rate 16 kHz, $\alpha$ is made equal to $0.42$.
For a sampling rate 10 kHz, $\alpha$ is made equal to $0.35$.
For a sampling rate 8 kHz, $\alpha$ is made equal to $0.31$.
By making these choices for $\alpha$,
the mel-scale becomes the good approximation to human
sensitivity to the loudness speech sound.

The Newton-Raphson method is used to minimize the cost function
when evaluating mel-cepstral coefficients.
\end{qsection}

\begin{options}
	\argm{a}{A}{all-pass constant $\alpha$}{0.35}
	\argm{m}{M}{order of mel cepstrum}{25}
	\argm{l}{L}{frame length}{256}
	\desc[1ex]{Usually, the options below do not need to be assigned.}
	\argm{i}{I}{minimum iteration of Newton-Raphson method}{2}
	\argm{j}{J}{maximum iteration of Newton-Raphson method}{30}
	\argm{d}{D}{end condition of Newton-Raphson}{0.001}
	\argm{e}{E}{small value added to periodgram}{0.0}
\end{options}

\begin{qsection}{EXAMPLE}
Speech data is read in float format from {\em data.f} and 
analyzed, and mel-cepstral coefficients are written to {\em data.mcep}:
\begin{quote}
 \verb!frame < data.f | window | mcep > data.mcep !
\end{quote}
\end{qsection}

\begin{qsection}{SEE ALSO}
\hyperlink{uels}{uels},
\hyperlink{gcep}{gcep},
\hyperlink{mgcep}{mgcep},
\hyperlink{mlsadf}{mlsadf}
\end{qsection}
