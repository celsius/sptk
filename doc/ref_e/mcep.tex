\name[ref:mcep-IEICE,ref:amcep-ICASSP92]{mcep}{mel cepstral analysis}%
{speech analysis}

\begin{synopsis}
\item [mcep] [ --a $A$ ] [ --m $M$ ] [ --l $L$ ] [ --i $I$ ] [ --j $J$ ] 
	     [ --d $D$ ] [ --e $E$ ] [ {\em infile} ]
\end{synopsis}

\begin{qsection}{DESCRIPTION}
This command undertakes the mel-cepstrum analysis,
and sends mel-cepstrum coefficients $c_{\alpha}(m)$ 
to the standard output.
When input signal has length $L$,
then the time sequence is given by
\begin{displaymath}
  x(0),x(1),\ldots,x(L-1).
\end{displaymath}
\par
Input and output data are in float format.
\par
In the mel-cepstrum analysis, the spectrum of the speech signal
is modeled by $M$ order mel-cepstrum coefficients $c_{\alpha}(m)$
as follows.
\begin{displaymath}
H(z) = \exp \sum_{m=0}^M c_{\alpha}(m) \tilde{z}^{-m} 
\end{displaymath}
For this command ``mcep'', it is applied a cost function
based on the unbiased estimation log spectrum method.
The variable $\tilde{z}^{-1}$ can be expressed as the following
first order all-pass function
\begin{displaymath}
\tilde{z}^{-1} = \frac{z^{-1}-\alpha}{1-\alpha z^{-1}}.
\end{displaymath}
The phase characteristic is given by the variable $\alpha$.
For a sampling rate 10kHz, $\alpha$ is made equal to $0.35$.
For a sampling rate 8kHz, $\alpha$ is made equal to $0.31$.
By making these choices for $\alpha$,
the mel-scale becomes the good approximation to human
sensitivity to the loudness speech sound.
\par
The Newton-Raphson method is used to minimize the cost function
when evaluating mel-cepstrum coefficients.
\end{qsection}

\begin{options}
	\argm{a}{A}{all-pass constant $\alpha$}{0.35}
	\argm{m}{M}{order of mel cepstrum}{25}
	\argm{l}{L}{frame length}{256}
	\desc[1zh]{Usually, the options below do not need to be assigned.}
	\argm{i}{I}{minimum iteration of Newton-Raphson method}{2}
	\argm{j}{J}{maximum iteration of Newton-Raphson method}{30}
	\argm{d}{D}{end condition of Newton-Raphson}{0.001}
	\argm{e}{E}{small value added to periodgram}{0.0}
\end{options}

\begin{qsection}{EXAMPLE}
Speech data is read in float format from {\em data.f} and 
analyzed, and mel-cepstrum coefficients are written to {\em data.mcep}:
\begin{quote}
 \verb!frame < data.f | window | mcep > data.mcep !
\end{quote}
\end{qsection}

\begin{qsection}{SEE ALSO}
 uels, gcep, mgcep, mlsadf
\end{qsection}
