% ----------------------------------------------------------------
%       Speech Signal Processing Toolkit (SPTK): version 3.0
%                      SPTK Working Group
% 
%                Department of Computer Science
%                Nagoya Institute of Technology
%                             and
%   Interdisciplinary Graduate School of Science and Engineering
%                Tokyo Institute of Technology
%                   Copyright (c) 1984-2000
%                     All Rights Reserved.
% 
% Permission is hereby granted, free of charge, to use and
% distribute this software and its documentation without
% restriction, including without limitation the rights to use,
% copy, modify, merge, publish, distribute, sublicense, and/or
% sell copies of this work, and to permit persons to whom this
% work is furnished to do so, subject to the following conditions:
% 
%   1. The code must retain the above copyright notice, this list
%      of conditions and the following disclaimer.
% 
%   2. Any modifications must be clearly marked as such.
%                                                                        
% NAGOYA INSTITUTE OF TECHNOLOGY, TOKYO INSITITUTE OF TECHNOLOGY,
% SPTK WORKING GROUP, AND THE CONTRIBUTORS TO THIS WORK DISCLAIM
% ALL WARRANTIES WITH REGARD TO THIS SOFTWARE, INCLUDING ALL
% IMPLIED WARRANTIES OF MERCHANTABILITY AND FITNESS, IN NO EVENT
% SHALL NAGOYA INSTITUTE OF TECHNOLOGY, TOKYO INSITITUTE OF
% TECHNOLOGY, SPTK WORKING GROUP, NOR THE CONTRIBUTORS BE LIABLE
% FOR ANY SPECIAL, INDIRECT OR CONSEQUENTIAL DAMAGES OR ANY
% DAMAGES WHATSOEVER RESULTING FROM LOSS OF USE, DATA OR PROFITS,
% WHETHER IN AN ACTION OF CONTRACT, NEGLIGENCE OR OTHER TORTIOUS
% ACTION, ARISING OUT OF OR IN CONNECTION WITH THE USE OR
% PERFORMANCE OF THIS SOFTWARE.
% ----------------------------------------------------------------
%
\hypertarget{average}{}
\name{average}{calculate mean for each block}{data processing}

\begin{synopsis}
\item [average] [ --l $L$ ] [ --n $N$ ] [ {\em infile} ] 
\end{synopsis}

\begin{qsection}{DESCRIPTION}
{\em average} calculates the mean value for every $L$-length block 
from {\em infile} (or standard input),
sending the result to standard output.

For the input data
\begin{displaymath}
  x(0),x(1),\ldots,x(L-1)
\end{displaymath}
the output is calculated as follows:
\begin{displaymath}
\frac{x(0)+x(1)+\ldots+x(L-1)}{L}
\end{displaymath}
If $L=0$ then average of whole input data is calculated.
\par
Input and output data are in float format.
\end{qsection}

\begin{options}
	\argm{l}{L}{number of items contained 1 block}{0}
	\argm{n}{N}{order of items contained 1 block}{L-1}
\end{options}

\begin{qsection}{EXAMPLE}
The output file {\em data.av} contains the mean taken from the whole data in
{\em data.f} read in float format.
\begin{quote}
 \verb!average < data.f > data.av!
\end{quote} 
\end{qsection}

\begin{qsection}{SEE ALSO}
\hyperlink{histogram}{histogram},
\hyperlink{vsum}{vsum},
\hyperlink{vstat}{vstat}
\end{qsection}
