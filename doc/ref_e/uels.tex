% ----------------------------------------------------------------- %
%             The Speech Signal Processing Toolkit (SPTK)           %
%             developed by SPTK Working Group                       %
%             http://sp-tk.sourceforge.net/                         %
% ----------------------------------------------------------------- %
%                                                                   %
%  Copyright (c) 1984-2007  Tokyo Institute of Technology           %
%                           Interdisciplinary Graduate School of    %
%                           Science and Engineering                 %
%                                                                   %
%                1996-2008  Nagoya Institute of Technology          %
%                           Department of Computer Science          %
%                                                                   %
% All rights reserved.                                              %
%                                                                   %
% Redistribution and use in source and binary forms, with or        %
% without modification, are permitted provided that the following   %
% conditions are met:                                               %
%                                                                   %
% - Redistributions of source code must retain the above copyright  %
%   notice, this list of conditions and the following disclaimer.   %
% - Redistributions in binary form must reproduce the above         %
%   copyright notice, this list of conditions and the following     %
%   disclaimer in the documentation and/or other materials provided %
%   with the distribution.                                          %
% - Neither the name of the SPTK working group nor the names of its %
%   contributors may be used to endorse or promote products derived %
%   from this software without specific prior written permission.   %
%                                                                   %
% THIS SOFTWARE IS PROVIDED BY THE COPYRIGHT HOLDERS AND            %
% CONTRIBUTORS "AS IS" AND ANY EXPRESS OR IMPLIED WARRANTIES,       %
% INCLUDING, BUT NOT LIMITED TO, THE IMPLIED WARRANTIES OF          %
% MERCHANTABILITY AND FITNESS FOR A PARTICULAR PURPOSE ARE          %
% DISCLAIMED. IN NO EVENT SHALL THE COPYRIGHT OWNER OR CONTRIBUTORS %
% BE LIABLE FOR ANY DIRECT, INDIRECT, INCIDENTAL, SPECIAL,          %
% EXEMPLARY, OR CONSEQUENTIAL DAMAGES (INCLUDING, BUT NOT LIMITED   %
% TO, PROCUREMENT OF SUBSTITUTE GOODS OR SERVICES; LOSS OF USE,     %
% DATA, OR PROFITS; OR BUSINESS INTERRUPTION) HOWEVER CAUSED AND ON %
% ANY THEORY OF LIABILITY, WHETHER IN CONTRACT, STRICT LIABILITY,   %
% OR TORT (INCLUDING NEGLIGENCE OR OTHERWISE) ARISING IN ANY WAY    %
% OUT OF THE USE OF THIS SOFTWARE, EVEN IF ADVISED OF THE           %
% POSSIBILITY OF SUCH DAMAGE.                                       %
% ----------------------------------------------------------------- %
\hypertarget{uels}{}
\name[ref:UELS-IEICE,ref:UELS-SignalProcessingIV]{uels}%
{unbiased estimation of log spectrum}{speech analysis}

\begin{synopsis}
\item [uels] [ --m $M$ ] [ --l $L$ ] [ --q $Q$ ] [ --i $I$ ] 
	     [ --j $J$ ] [ --d $D$ ] [ --e $E$ ] [ {\em infile} ]
\end{synopsis}

\begin{qsection}{DESCRIPTION}
{\em uels} uses the unbiased estimation of log spectrum method 
to calculate cepstral coefficients $c(m)$ 
from $L$-length framed windowed input data
from {\rm infile} (or standard input), 
sending the result to standard output.

Input and output data are in float format.

Until the proposition of the unbiased estimation of
log spectrum method, the conventional methods had
two main problems.
Firstly the importance of smoothing of the log spectrum
was not clear.
Secondly it could not be guaranteed that
the bias of the estimated value would be sufficiently small.

The evaluation procedure to obtain the unbiased estimation
log spectrum values is similar to other improved methods to
calculate cepstral coefficients.
The main difference is that in UELS method a non-linear smoothing
is used to guaranty that the estimation will be unbiased.
\end{qsection}

\begin{options}
	\argm{m}{M}{order of cepstrum}{25}
	\argm{l}{L}{frame length}{256}
    \argm{q}{Q}{input data style\\
        \begin{tabular}{ll} \\[-1ex]
            $Q=0$ & windowed data sequence \\
            $Q=1$ & $20 \times \log |f(w)|$ \\
            $Q=2$ & $\ln |f(w)|$ \\
            $Q=3$ & $|f(w)|$ \\
            $Q=4$ & $|f(w)|^2$ \\[1ex]
        \end{tabular}\\\hspace*{\fill}}{0}
	\desc[1ex]{Usually, the options below do not need to be assigned.}
	\argm{i}{I}{minimum iteration}{2}
	\argm{j}{J}{maximum iteration}{30}
	\argm{d}{D}{end condition}{0.001}
	\argm{e}{E}{small value added to periodgram}{0.0}
\end{options}

\begin{qsection}{EXAMPLE}
The example below reads data in float format,
evaluates 15-th order log spectrum through UELS method,
and sends spectrum coefficients to {\em data.cep}:
\begin{quote}
 \verb!frame +f < data.f | window | uels -m 15 > data.cep!
\end{quote} 
\end{qsection}

\begin{qsection}{SEE ALSO}
\hyperlink{gcep}{gcep},
\hyperlink{mcep}{mcep},
\hyperlink{mgcep}{mgcep},
\hyperlink{lmadf}{lmadf}
\end{qsection}
