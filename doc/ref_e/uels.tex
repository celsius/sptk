% ----------------------------------------------------------------
%       Speech Signal Processing Toolkit (SPTK): version 3.0
%                      SPTK Working Group
% 
%                Department of Computer Science
%                Nagoya Institute of Technology
%                             and
%   Interdisciplinary Graduate School of Science and Engineering
%                Tokyo Institute of Technology
%                   Copyright (c) 1984-2000
%                     All Rights Reserved.
% 
% Permission is hereby granted, free of charge, to use and
% distribute this software and its documentation without
% restriction, including without limitation the rights to use,
% copy, modify, merge, publish, distribute, sublicense, and/or
% sell copies of this work, and to permit persons to whom this
% work is furnished to do so, subject to the following conditions:
% 
%   1. The code must retain the above copyright notice, this list
%      of conditions and the following disclaimer.
% 
%   2. Any modifications must be clearly marked as such.
%                                                                        
% NAGOYA INSTITUTE OF TECHNOLOGY, TOKYO INSITITUTE OF TECHNOLOGY,
% SPTK WORKING GROUP, AND THE CONTRIBUTORS TO THIS WORK DISCLAIM
% ALL WARRANTIES WITH REGARD TO THIS SOFTWARE, INCLUDING ALL
% IMPLIED WARRANTIES OF MERCHANTABILITY AND FITNESS, IN NO EVENT
% SHALL NAGOYA INSTITUTE OF TECHNOLOGY, TOKYO INSITITUTE OF
% TECHNOLOGY, SPTK WORKING GROUP, NOR THE CONTRIBUTORS BE LIABLE
% FOR ANY SPECIAL, INDIRECT OR CONSEQUENTIAL DAMAGES OR ANY
% DAMAGES WHATSOEVER RESULTING FROM LOSS OF USE, DATA OR PROFITS,
% WHETHER IN AN ACTION OF CONTRACT, NEGLIGENCE OR OTHER TORTIOUS
% ACTION, ARISING OUT OF OR IN CONNECTION WITH THE USE OR
% PERFORMANCE OF THIS SOFTWARE.
% ----------------------------------------------------------------
%
\name[ref:UELS-IEICE,ref:UELS-SignalProcessingIV]{uels}%
{unbiased estimation of log spectrum}{speech analysis}

\begin{synopsis}
\item [uels] [ --m $M$ ] [ --l $L$ ] [ --i $I$ ] 
	     [ --j $J$ ] [ --d $D$ ] [ --e $E$ ] [ {\em infile} ]
\end{synopsis}

\begin{qsection}{DESCRIPTION}
{\em uels} uses the unbiased estimation of log spectrum method 
to calculate cepstrum coefficients $c(m)$ 
from $L$-length framed windowed input data
from {\rm infile} (or standard input), 
sending the result to standard output.

Input and output data are in float format.

Until the proposition of the unbiased estimation of
log spectrum method, the conventional methods had
two main problems.
Firstly the importance of smoothing of the log spectrum
was not clear.
Secondly it could not be guaranteed that
the bias of the estimated value would be sufficiently small.

The evaluation procedure to obtain the unbiased estimation
log spectrum values is similar to other improved methods to
calculate cepstrum coefficients.
The main difference is that in UELS method a non-linear smoothing
is used to guaranty that the estimation will be unbiased.
\end{qsection}

\begin{options}
	\argm{m}{M}{order of cepstrum}{25}
	\argm{l}{L}{frame length}{256}
	\desc[1ex]{Usually, the options below do not need to be assigned.}
	\argm{i}{I}{minimum iteration}{2}
	\argm{j}{J}{maximum iteration}{30}
	\argm{d}{D}{end condition}{0.001}
	\argm{e}{E}{small value added to periodgram}{0.0}
\end{options}

\begin{qsection}{EXAMPLE}
The example below reads data in float format,
evaluates 15-order log spectrum through UELS method,
and sends spectrum coefficients to {\em data.cep}:
\begin{quote}
 \verb!frame < data.f | window | uels -m 15 > data.cep!
\end{quote} 
\end{qsection}

\begin{qsection}{SEE ALSO}
 icep, gcep, mcep, mgcep, lmadf
\end{qsection}
