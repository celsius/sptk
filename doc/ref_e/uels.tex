\name[ref:UELS-IEICE,ref:UELS-SignalProcessingIV]{uels}%
{unbiased estimation of log spectrum}{speech analysis}

\begin{synopsis}
\item [uels] [ --m $M$ ] [ --l $L$ ] [ --i $I$ ] 
	     [ --j $J$ ] [ --d $D$ ] [ --e $E$ ] [ {\em infile} ]
\end{synopsis}

\begin{qsection}{DESCRIPTION}
This command uses the unbiased estimation of log spectrum method
to calculate cepstrum coefficients $c(m)$ and
sends them to the standard output.
Assume that the input sequence is a windowed sequence of length $L$
as follows.
\begin{displaymath}
  x(0),x(1),\ldots,x(L-1)
\end{displaymath}
\par
Input and output data are in float format.
\par
Until the proposition of the unbiased estimation of
log spectrum method, the conventional methods had
two main problems.
Firstly the importance of smoothing of the log spectrum
was not clear.
Secondly it could not be guaranteed that
the bias of the estimated value would be sufficiently small.
\par
The evaluation procedure to obtain the unbiased estimation
log spectrum values is similar to other improved methods to
calculate cepstrum coefficients.
The main difference is that in UELS method a non-linear smoothing
is used to guaranty that the estimation will be unbiased.
\end{qsection}

\begin{options}
	\argm{m}{M}{order of cepstrum}{25}
	\argm{l}{L}{frame length}{256}
	\desc[1zh]{Usually, the options below do not need to be assigned.}
	\argm{i}{I}{minimum iteration}{2}
	\argm{j}{J}{maximum iteration}{30}
	\argm{d}{D}{end condition}{0.001}
	\argm{e}{E}{small value added to periodgram}{0.0}
\end{options}

\begin{qsection}{EXAMPLE}
The example below reads data in float format,
evaluates 15-order log spectrum through UELS method,
and sends spectrum coefficients to {\em data.cep}:
\begin{quote}
 \verb!frame < data.f | window | uels -m 15 > data.cep!
\end{quote} 
\end{qsection}

\begin{qsection}{SEE ALSO}
 icep, gcep, mcep, mgcep, lmadf
\end{qsection}
