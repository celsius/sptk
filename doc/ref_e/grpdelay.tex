\name{grpdelay}{group delay of digital filter}{signal processing}
\begin{synopsis}
 \item[grpdelay] [ --l $L$ ] [ --m $M$ ] [ --a ] [ {\em infile} ] 
\end{synopsis}

\begin{qsection}{DESCRIPTION}
This command reads the filter coefficients from {\em infile},
or from the standard input if {\em infile} is not specified.
Input and output data are in float format.
\par
When the {\bf --m} option is omitted
and the input data sequence length is less than FFT size,
then the input file is padded with 0 and the FFT is evaluated
as exemplified below.
When the {\bf --a} option is assigned,
the gain is obtained from zero order input.???????????????????????
\par
\[
\begin{array}{lll}
\mbox{Input sequence} & 
\overbrace{\framebox[4.5cm]{$x_0, x_1, \ldots, x_{M}, 0,
					\ldots,0$}}^{L}  & \mbox{filter coefficients}\\
		& \makebox[4.5cm]{0\hfill $L-1$} &
\end{array}
\]
\[
\begin{array}{lll}
\mbox{Output sequence} & \overbrace{\framebox[4.5cm]{$\tau(\omega)$}}^{L/2+1} &
	   \mbox{group delay}\\
		& \makebox[4.5cm]{0\hfill $L-1$} &

\end{array}
\]
\end{qsection}

\begin{options}
	\argm{l}{L}{FFT size power of 2}{256}
	\argm{m}{M}{order of filter}{L-1}
	\argm{a}{}{ARMA filter}{FALSE}
\end{options}


\begin{qsection}{EXAMPLE}
This example plots in the screen the group delay of impulse response
of the filter with the following transfer function.
\begin{displaymath}
  H(z)=\frac{1}{1+0.9z^{-1}}
\end{displaymath}
\begin{quote}
\verb! impluse | dfs -a 1 0.9 | grpdelay | fdrw | xgr !
\end{quote}  
\end{qsection}

\begin{qsection}{SEE ALSO}
delay, phase
\end{qsection}
