% ----------------------------------------------------------------- %
%             The Speech Signal Processing Toolkit (SPTK)           %
%             developed by SPTK Working Group                       %
%             http://sp-tk.sourceforge.net/                         %
% ----------------------------------------------------------------- %
%                                                                   %
%  Copyright (c) 1984-2007  Tokyo Institute of Technology           %
%                           Interdisciplinary Graduate School of    %
%                           Science and Engineering                 %
%                                                                   %
%                1996-2016  Nagoya Institute of Technology          %
%                           Department of Computer Science          %
%                                                                   %
% All rights reserved.                                              %
%                                                                   %
% Redistribution and use in source and binary forms, with or        %
% without modification, are permitted provided that the following   %
% conditions are met:                                               %
%                                                                   %
% - Redistributions of source code must retain the above copyright  %
%   notice, this list of conditions and the following disclaimer.   %
% - Redistributions in binary form must reproduce the above         %
%   copyright notice, this list of conditions and the following     %
%   disclaimer in the documentation and/or other materials provided %
%   with the distribution.                                          %
% - Neither the name of the SPTK working group nor the names of its %
%   contributors may be used to endorse or promote products derived %
%   from this software without specific prior written permission.   %
%                                                                   %
% THIS SOFTWARE IS PROVIDED BY THE COPYRIGHT HOLDERS AND            %
% CONTRIBUTORS "AS IS" AND ANY EXPRESS OR IMPLIED WARRANTIES,       %
% INCLUDING, BUT NOT LIMITED TO, THE IMPLIED WARRANTIES OF          %
% MERCHANTABILITY AND FITNESS FOR A PARTICULAR PURPOSE ARE          %
% DISCLAIMED. IN NO EVENT SHALL THE COPYRIGHT OWNER OR CONTRIBUTORS %
% BE LIABLE FOR ANY DIRECT, INDIRECT, INCIDENTAL, SPECIAL,          %
% EXEMPLARY, OR CONSEQUENTIAL DAMAGES (INCLUDING, BUT NOT LIMITED   %
% TO, PROCUREMENT OF SUBSTITUTE GOODS OR SERVICES; LOSS OF USE,     %
% DATA, OR PROFITS; OR BUSINESS INTERRUPTION) HOWEVER CAUSED AND ON %
% ANY THEORY OF LIABILITY, WHETHER IN CONTRACT, STRICT LIABILITY,   %
% OR TORT (INCLUDING NEGLIGENCE OR OTHERWISE) ARISING IN ANY WAY    %
% OUT OF THE USE OF THIS SOFTWARE, EVEN IF ADVISED OF THE           %
% POSSIBILITY OF SUCH DAMAGE.                                       %
% ----------------------------------------------------------------- %
\hypertarget{csm2lpc}{}
\name{csm2lpc}{transform CSM to LPC}{speech parameter transformation}

\begin{synopsis}
\item [csm2lpc] [ --m $M$ ] [ --f $F$ ] [ {\em infile} ] 
\end{synopsis}

\begin{qsection}{DESCRIPTION}
{\em csm2lpc} calculates linear prediction coefficients (LPC) 
from $M$-th order composite sinusoidal modeling (CSM) parameters 
from {\em infile} (or standard input),
sending the result to standard output.

The CSM input format is
\begin{displaymath}
   \omega(1), \dots , \omega\biggl(\frac{M+1}{2}\biggr), m(1), \dots , m\biggl(\frac{M+1}{2}\biggr).
\end{displaymath}
The autocorrelation function $v(l)$ is obtained from the CSM parameters:
\begin{displaymath}
   v(l)=\sum_{i=1}^{\frac{M+1}{2}}m(i)\cos(l\cdot\omega(i)), \qquad l=0,1,\dots,M,
\end{displaymath}
By solving the autocorrelation matrix consists of $v(l)$ using the
Levinson-Durbin algorithm (see \hyperlink{levdur}{levdur}), the corresponding LPC are obtained as
\begin{displaymath}
   K , a(1), \dots , a(M).
\end{displaymath}

\end{qsection}

\begin{options}
	\argm{m}{M}{order of LPC}{25}
	\argm{f}{F}{minimum value of the determinant of the normal matrix}{0}
\end{options}

\begin{qsection}{EXAMPLE}
In the example below, 15-th order CSM coefficients in float format
are read from {\em data.csm}, and then the LPC are written to {\em data.lpc}:
\begin{quote}
\verb! csm2lpc -m 15 data.csm > data.lpc!
\end{quote}
\end{qsection}

\begin{qsection}{SEE ALSO}
\hyperlink{lpc}{lpc},
\hyperlink{levdur}{levdur}
\end{qsection}
