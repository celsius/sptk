% ----------------------------------------------------------------
%       Speech Signal Processing Toolkit (SPTK): version 3.0
%                      SPTK Working Group
% 
%                Department of Computer Science
%                Nagoya Institute of Technology
%                             and
%   Interdisciplinary Graduate School of Science and Engineering
%                Tokyo Institute of Technology
%                   Copyright (c) 1984-2000
%                     All Rights Reserved.
% 
% Permission is hereby granted, free of charge, to use and
% distribute this software and its documentation without
% restriction, including without limitation the rights to use,
% copy, modify, merge, publish, distribute, sublicense, and/or
% sell copies of this work, and to permit persons to whom this
% work is furnished to do so, subject to the following conditions:
% 
%   1. The code must retain the above copyright notice, this list
%      of conditions and the following disclaimer.
% 
%   2. Any modifications must be clearly marked as such.
%                                                                        
% NAGOYA INSTITUTE OF TECHNOLOGY, TOKYO INSITITUTE OF TECHNOLOGY,
% SPTK WORKING GROUP, AND THE CONTRIBUTORS TO THIS WORK DISCLAIM
% ALL WARRANTIES WITH REGARD TO THIS SOFTWARE, INCLUDING ALL
% IMPLIED WARRANTIES OF MERCHANTABILITY AND FITNESS, IN NO EVENT
% SHALL NAGOYA INSTITUTE OF TECHNOLOGY, TOKYO INSITITUTE OF
% TECHNOLOGY, SPTK WORKING GROUP, NOR THE CONTRIBUTORS BE LIABLE
% FOR ANY SPECIAL, INDIRECT OR CONSEQUENTIAL DAMAGES OR ANY
% DAMAGES WHATSOEVER RESULTING FROM LOSS OF USE, DATA OR PROFITS,
% WHETHER IN AN ACTION OF CONTRACT, NEGLIGENCE OR OTHER TORTIOUS
% ACTION, ARISING OUT OF OR IN CONNECTION WITH THE USE OR
% PERFORMANCE OF THIS SOFTWARE.
% ----------------------------------------------------------------
%
\hypertarget{sp2mgc}{}
\name[ref:mgcep-IEICE,ref:mgcep-ICSLP94]{sp2mgc}%
{extract mel-generalized cepstrum from periodgram}{speech analysis}

\begin{synopsis}
\item[sp2mgc]   [ --a $A$ ] [ --g $G$ ] [ --m $M$ ] [ --l $L$ ] 
	       [ --i $I$ ] [ --o $O$ ]
\item[\ ~~~~~~~] [ --j $J$ ] [ --k $K$ ] [ --d $D$ ] [ --p $P$ ] [ -- e $E$ ] 
		 [ {\em infile} ]
\end{synopsis}

\begin{qsection}{DESCRIPTION}
{\em sp2mgc} uses mel-generalized cepstral analysis 
to calculate mel-generalized cepstral coefficients 
from $L$-length periodgram input data 
from {\em infile} (or standard input), 
sending the result to standard output. 
There are several different input and output formats,
controlled by the --i and --o options, respectively.
\end{qsection}

\begin{options}
	\argm{a}{A}{alpha $\alpha$}{0.35}
	\argm{g}{G}{power parameter of generalized cepstrum $\gamma$\\
			 if $G>1.0$ then $\gamma=-1/G$.}{0}
	\argm{m}{M}{order of mel-generalized cepstrum}{25}
	\argm{l}{L}{FFT length power of 2}{256}
	\argm{i}{I}{input data style\\
			\begin{tabular}{ll} \\[-1ex]
				$I=0$ & $20 \times \log |H(z)|$ \\
				$I=1$ & $\ln |H(z)|$ \\
				$I=2$ & $|H(z)|$ \\
				$I=3$ & $|H(z)|^2$ \\[1ex]
        	        \end{tabular} \\
			}{0}
	\argm{o}{O}{output data style\\
			\begin{tabular}{ll} \\[-1ex]
				$O=0$ & $c_{\alpha, \gamma}(0), c_{\alpha, \gamma}(1), \dots, c_{\alpha, \gamma}(M)$ \\
				$O=1$ & $b_\gamma(0), b_\gamma(1), \dots, b_\gamma(M)$ \\
				$O=2$ & $K_\alpha, c_{\alpha, \gamma}'(1), \dots, c_{\alpha, \gamma}'(M)$ \\
				$O=3$ & $K, b_\gamma'(1), \dots, b_\gamma'(M)$ \\
				$O=4$ & $K_\alpha, \gamma\,c_{\alpha, \gamma}'(1), \dots, \gamma\,c_{\alpha, \gamma}'(M)$ \\
				$O=5$ & $K, \gamma\,b_\gamma'(1), \dots, \gamma\,b_\gamma'(M)$ \\[1ex]
			\end{tabular} \\
			}{0}
	\desc[1ex]{Usually, the options below do not need to be assigned.}
	\argm{j}{J}{minimum iteration of Newton-Raphson method}{2}
	\argm{k}{K}{maximum iteration of Newton-Raphson method}{30}
	\argm{d}{D}{end condition of Newton-Raphson method}{0.001}
	\argm{p}{P}{order of recursions}{$L-1$}
	\argm{e}{E}{small value added to periodgram}{0}	
\end{options}

\begin{qsection}{EXAMPLE}
In the following speech data in float format is read
from {\em data.f}, and analyzed with $\gamma=0$, $\alpha=0$
(which correspond to UELS method for log spectrum estimation)
and the resulting cepstral coefficients are written {\em data.cep}:
\begin{quote}
  \verb!frame < data.f | window | fftr -A -H | sp2mgc -i 2 > data.cep !
\end{quote}
\begin{quote}
  \verb!frame < data.f | window | fftr -P -H | sp2mgc -i 3 > data.cep !
\end{quote}
\end{qsection}

\begin{qsection}{SEE ALSO}
\hyperlink{mgcep}{mgcep},
\hyperlink{uels}{uels},
\hyperlink{gcep}{gcep},
\hyperlink{mcep}{mcep},
\hyperlink{freqt}{freqt},
\hyperlink{gc2gc}{gc2gc},
\hyperlink{mgc2mgc}{mgc2mgc},
\hyperlink{gnorm}{gnorm},
\hyperlink{mglsadf}{mglsadf}
\end{qsection}
