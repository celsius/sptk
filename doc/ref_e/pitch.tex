\name{pitch}{pitch extraction}{�������,����ʬ�Ϲ�����Ϣ}

\begin{synopsis}
\item[pitch] [ --s $S$ ] [ --l $L$ ] [ --t $T$ ]
 [ --L $Lo$ ] [ --H $Hi$ ] [ --e $E$ ]
\item[\ ~~~~~] [ --i $I$ ] [ --j $J$ ] [ --d $D$ ] [ {\em infile} ] 
\end{synopsis}

\begin{qsection}{DESCRIPTION}
This command extracts the pitch $p(t)$ using
cepstrum method and sends it to the standard output.
Assume that the windowed input sequance of length $L$ is
\begin{displaymath}
  x(0),x(1),\ldots,x(L -1)
\end{displaymath}
\par
Input and output data are in float format.
\par
To discriminate between voiced and unvoiced sounds,
the unbiased estimation of log spectrum method is applied
to evaluate $(S/10 \times 25)$ order cepstrum.
Then from these coefficients, the magnitude of log spectrum
$\hat{g}_i(\Omega_k)$ is evaluated.
Finally the mean value $v_i$ for every band is calculated.%?????????????????
\begin{displaymath}
v_i = \frac{1}{14 n}\sum_{k = 4 n}^{17 n}\hat{g}_i(\Omega_k),~~~~~ (\Omega_k = \frac{2 \pi k}{N},n = N /256)
\end{displaymath}

Here the FFT size $N$ is square number greater then $L$.

If the speech sound is voiced $(v_i > T)$,
then the FFT cepstrum coefficients $c(m)$ are transformed
into $c(m) \times m$,
and the peak frequency between $Lo$ (Hz) and $Hi$ (Hz)
is the pitch.
If the speech sound is unvoiced $(v_i < T)$
then $0$ is outputed.

\end{qsection}

\begin{options}
	\argm{s}{S}{sampling frequency (kHz)}{10}
	\argm{l}{L}{frame length}{400}
	\argm{t}{T}{voiced/unvoiced threshold}{6.0}
	\argm{L}{Lo}{minimum fundamental
                     frequency to serach for (Hz)}{60}
	\argm{H}{Hi}{minimum fundamental
                     frequency to serach for (Hz)}{240}
	\argm{e}{E}{small value for calculate
                    log-spectral envelope}{0.0}
        \desc[1zh]{Usually, the options below do not need to be assigned.}
	\argm{i}{I}{minimum number of iteration}{2}
	\argm{j}{J}{maximum number of iteration}{30}
	\argm{d}{D}{end condition}{0.1}
\end{options}

\begin{qsection}{EXAMPLE}
Speech data with sampling rate 10kHz is read in float format
from {\em data.f}, the pitch is evaluated, and
the output is written to {\em data.pitch}:
\begin{quote}
  \verb!frame -l 400 < data.f | window -l 400 | pitch -l 400 > data.pitch !
\end{quote}
\end{qsection}

\begin{qsection}{SEE ALSO}
  excite
\end{qsection}
