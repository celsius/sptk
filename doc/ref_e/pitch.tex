% ----------------------------------------------------------------
%       Speech Signal Processing Toolkit (SPTK): version 3.0
%                      SPTK Working Group
% 
%                Department of Computer Science
%                Nagoya Institute of Technology
%                             and
%   Interdisciplinary Graduate School of Science and Engineering
%                Tokyo Institute of Technology
%                   Copyright (c) 1984-2000
%                     All Rights Reserved.
% 
% Permission is hereby granted, free of charge, to use and
% distribute this software and its documentation without
% restriction, including without limitation the rights to use,
% copy, modify, merge, publish, distribute, sublicense, and/or
% sell copies of this work, and to permit persons to whom this
% work is furnished to do so, subject to the following conditions:
% 
%   1. The code must retain the above copyright notice, this list
%      of conditions and the following disclaimer.
% 
%   2. Any modifications must be clearly marked as such.
%                                                                        
% NAGOYA INSTITUTE OF TECHNOLOGY, TOKYO INSITITUTE OF TECHNOLOGY,
% SPTK WORKING GROUP, AND THE CONTRIBUTORS TO THIS WORK DISCLAIM
% ALL WARRANTIES WITH REGARD TO THIS SOFTWARE, INCLUDING ALL
% IMPLIED WARRANTIES OF MERCHANTABILITY AND FITNESS, IN NO EVENT
% SHALL NAGOYA INSTITUTE OF TECHNOLOGY, TOKYO INSITITUTE OF
% TECHNOLOGY, SPTK WORKING GROUP, NOR THE CONTRIBUTORS BE LIABLE
% FOR ANY SPECIAL, INDIRECT OR CONSEQUENTIAL DAMAGES OR ANY
% DAMAGES WHATSOEVER RESULTING FROM LOSS OF USE, DATA OR PROFITS,
% WHETHER IN AN ACTION OF CONTRACT, NEGLIGENCE OR OTHER TORTIOUS
% ACTION, ARISING OUT OF OR IN CONNECTION WITH THE USE OR
% PERFORMANCE OF THIS SOFTWARE.
% ----------------------------------------------------------------
%
\hypertarget{pitch}{}
\name{pitch}{pitch extraction}{signal processing,speech analysis and synthesis}

\begin{synopsis}
\item[pitch] [ --s $S$ ] [ --l $L$ ] [ --t $T$ ]
 [ --L $Lo$ ] [ --H $Hi$ ] [ --e $E$ ]
\item[\ ~~~~~] [ --i $I$ ] [ --j $J$ ] [ --d $D$ ] [ {\em infile} ] 
\end{synopsis}

\begin{qsection}{DESCRIPTION}
{\em pitch} uses the cepstrum method to calculate the pitch period values
corresponding to frames of input data of length $L$ 
from {\em infile} (or standard input), 
sending the result to standard output. 
For unvoiced frames, the output value is 0.0. 
For voiced frames, the output value is proportional to the pitch period.

Input and output data are in float format.

To discriminate between voiced and unvoiced sounds,
the unbiased estimation of log spectrum method is applied
to evaluate $(S/10 \times 25)$ order cepstrum.
Then from these coefficients, the magnitude of log spectrum
$\hat{g}_i(\Omega_k)$ is evaluated.
Finally the mean value $v_i$ for every band is calculated.
\begin{displaymath}
v_i = \frac{1}{14 n}\sum_{k = 4 n}^{17 n}\hat{g}_i(\Omega_k),\qquad (\Omega_k = \frac{2 \pi k}{N},n = N /256)
\end{displaymath}

Here the FFT size $N$ is square number greater then $L$.

If the speech sound is voiced $(v_i > T)$,
then the FFT cepstrum coefficients $c(m)$ are transformed
into $c(m) \times m$,
and the peak frequency between $Lo$ (Hz) and $Hi$ (Hz)
is the pitch.
If the speech sound is unvoiced $(v_i < T)$
then $0$ is outputed.

\end{qsection}

\begin{options}
	\argm{s}{S}{sampling frequency (kHz)}{10}
	\argm{l}{L}{frame length}{400}
	\argm{t}{T}{voiced/unvoiced threshold}{6.0}
	\argm{L}{Lo}{minimum fundamental
                     frequency to search for (Hz)}{60}
	\argm{H}{Hi}{minimum fundamental
                     frequency to search for (Hz)}{240}
	\argm{e}{E}{small value for calculate
                    log-spectral envelope}{0.0}
        \desc[1ex]{Usually, the options below do not need to be assigned.}
	\argm{i}{I}{minimum number of iteration}{2}
	\argm{j}{J}{maximum number of iteration}{30}
	\argm{d}{D}{end condition}{0.1}
\end{options}

\begin{qsection}{EXAMPLE}
Speech data with sampling rate 10kHz is read in float format
from {\em data.f}, the pitch is evaluated, and
the output is written to {\em data.pitch}:
\begin{quote}
  \verb!frame -l 400 < data.f | window -l 400 | pitch -l 400 > data.pitch !
\end{quote}
\end{qsection}

\begin{qsection}{SEE ALSO}
\hyperlink{excite}{excite}
\end{qsection}
