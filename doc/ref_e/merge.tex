\name{merge}{data merge}{operation}

\begin{synopsis}
 \item[merge] [ --s $S$ ] [ --l $L_1$ ] [ --n $N_1$ ] [ --L $L_2$ ]
 [ --N $N_2$ ]
 \item[\ ~~~]  [ --o ] [ +{\em type} ] {\em file1} [ {\em infile} ] 
\end{synopsis}

\begin{qsection}{DESCRIPTION}
This command reads data from the standard input {\em infile}
and from the assigned input {\em file1} for every frame of the
file {\em infile}, a frame of {\em file1} is inserted or
overwritten as shown in the figure below.

\hspace{1cm}\epsffile{fig/merge.eps}
\end{qsection}

\begin{options}
      -s s  : insert point                [0]
       -l l  : frame length of input data  [25]
       -n n  : l - 1                       [24]
       -L L  : frame length of insert data [10]
       -N N  : L - 1                       [9]
       -o    : over write mode             [FALSE]
       +type : data type                   [f]
                c (char)      s (short)
                i (int)       l (long)
                f (float)     d (double)
       -h    : print this message
	\argm{s}{S}{insert point}{0}
	\argm{l}{L_1}{frame length of input data}{25}
	\argm{n}{N_1}{order of input data}{$L_1-1$}
	\argm{L}{L_2}{frame length of insert data}{10}
	\argm{N}{N_2}{order of insert data}{$L_2-1$}
	\argm{o}{}{overwrite mode}{FALSE}
	\argp{t}{input data format\\ 
		\begin{tabular}{llcll} \\[-1zh]
			c & char (1byte) & \quad &
			s & short (2bytes) \\
			i & int (4bytes) & \quad &
			l & long (4bytes) \\
			f & float (4bytes) & \quad &
			d & double (8bytes) \\
		\end{tabular}\\\hspace*{\fill}}{f}
\end{options}


\begin{qsection}{EXAMPLE}
The following example inserts blocks of 2 samples from {\em data.f2}
in short format into {\em data.f1}, also in short format,
the frame length of the file {\em data.f1} is 3, and the blocks
from {\em data.f2} will be inserted from the 3rd sample of
every frame.
The result is {\em data.merge}.
\begin{quote}
 \verb!merge -s 2 -l 3 -L 2 +s data.f2 < data.f1 > data.merge!
\end{quote}
For example, if the {\em data.f1} file is 
\[ 1,1,1,2,2,2,\cdots \]��
and the {\em data.f2} file is 
\[ 2,3,5,6,\cdots \]
then the output {\em data.merge} will be 
\[ 1,1,2,3,1,~ 2,2,5,6,2,\cdots \] 

The next example overwrites blocks of 2 samples from {\em data.f2}
in long format into {\em data.f1}, also in long format,
the frame length of the file {\em data.f1} is 4, and the blocks
from {\em data.f2} will be inserted from the 2nd sample of
every frame.
The result is {\em data.merge}.
\begin{quote}
 \verb!merge -s 2 -l 4 -L 2 +l -o data.f2 < data.f1 > data.merge!
\end{quote}
For example, if the {\em data.f1} file is 
\[ 1,1,1,1,2,2,2,2,\cdots \]��
and the {\em data.f2} file is 
\[ 3,4,5,6,\cdots \]
then the output {\em data.merge} will be 
\[  1,3,4,1,~ 2,5,6,2,\cdots \] 

\end{qsection}
