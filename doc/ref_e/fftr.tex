\name{fftr}{FFT for real sequence}{signal processing}

\begin{synopsis}
 \item[fftr] [ --l $L$ ] [ --m $M$] [ --\{ A $|$ R $|$ I $|$ P \} ] [ --H ]
	     [ {\em infile} ] 
\end{synopsis}

\begin{qsection}{DESCRIPTION}
This command reads a sequence of real numbers from {\em infile}
and undertakes its DFT through the FFT algorithm.
When the input file is not assigned, data is read from the standard
input.
Input and output data are in float format.
\par
When the {\bf --m} option is omitted
and the input data sequence length is less than the FFT size,
then the input file is padded with 0 and the FFT is evaluated
as exemplified below.

\begin{displaymath}
\begin{array}{lll}
\mbox{Input sequence} & 
\overbrace{\framebox[4.5cm]{$x_0, x_1, \ldots, x_{M}, 0,
					\ldots,0$}}^{L}  & \\
		& \makebox[4.5cm]{0\hfill $L-1$} &
\end{array}
\end{displaymath}
\begin{displaymath}
\begin{array}{lll}
\mbox{Output sequence} & \overbrace{\framebox[4.5cm]{real part}}^{L} &
	   \overbrace{\framebox[4.5cm]{imaginary part}}^{L} \\
		& \makebox[4.5cm]{0\hfill $L-1$} &
		\makebox[4.5cm]{0\hfill $L-1$}
\end{array}
\end{displaymath}
\end{qsection}

\begin{options}
	\argm{l}{L}{FFT size power of 2}{256}
	\argm{m}{M}{order of sequence}{L-1}
	\argm{A}{}{output magnitude}{FALSE}
	\argm{R}{}{output real part}{FALSE}
	\argm{I}{}{output imaginary part}{FALSE}
	\argm{P}{}{output power spectrum}{FALSE}
	\argm{H}{}{output half size}{FALSE}
\end{options}

\begin{qsection}{EXAMPLE}
In the example below, a sine wave is passed through a Blackman window,
its DFT is evaluated and the magnitude is plotted:
\begin{quote}
  \verb!sin -t 30 | window | fftr -A | fdrw | xgr!
\end{quote}

\end{qsection}

\begin{qsection}{SEE ALSO}
  fft, spec, phase
\end{qsection}
