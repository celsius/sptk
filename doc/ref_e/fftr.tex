% ----------------------------------------------------------------
%       Speech Signal Processing Toolkit (SPTK): version 3.0
%                      SPTK Working Group
% 
%                Department of Computer Science
%                Nagoya Institute of Technology
%                             and
%   Interdisciplinary Graduate School of Science and Engineering
%                Tokyo Institute of Technology
%                   Copyright (c) 1984-2000
%                     All Rights Reserved.
% 
% Permission is hereby granted, free of charge, to use and
% distribute this software and its documentation without
% restriction, including without limitation the rights to use,
% copy, modify, merge, publish, distribute, sublicense, and/or
% sell copies of this work, and to permit persons to whom this
% work is furnished to do so, subject to the following conditions:
% 
%   1. The code must retain the above copyright notice, this list
%      of conditions and the following disclaimer.
% 
%   2. Any modifications must be clearly marked as such.
%                                                                        
% NAGOYA INSTITUTE OF TECHNOLOGY, TOKYO INSITITUTE OF TECHNOLOGY,
% SPTK WORKING GROUP, AND THE CONTRIBUTORS TO THIS WORK DISCLAIM
% ALL WARRANTIES WITH REGARD TO THIS SOFTWARE, INCLUDING ALL
% IMPLIED WARRANTIES OF MERCHANTABILITY AND FITNESS, IN NO EVENT
% SHALL NAGOYA INSTITUTE OF TECHNOLOGY, TOKYO INSITITUTE OF
% TECHNOLOGY, SPTK WORKING GROUP, NOR THE CONTRIBUTORS BE LIABLE
% FOR ANY SPECIAL, INDIRECT OR CONSEQUENTIAL DAMAGES OR ANY
% DAMAGES WHATSOEVER RESULTING FROM LOSS OF USE, DATA OR PROFITS,
% WHETHER IN AN ACTION OF CONTRACT, NEGLIGENCE OR OTHER TORTIOUS
% ACTION, ARISING OUT OF OR IN CONNECTION WITH THE USE OR
% PERFORMANCE OF THIS SOFTWARE.
% ----------------------------------------------------------------
%
\name{fftr}{FFT for real sequence}{signal processing}

\begin{synopsis}
 \item[fftr] [ --l $L$ ] [ --m $M$] [ --\{ A $|$ R $|$ I $|$ P \} ] [ --H ]
	     [ {\em infile} ] 
\end{synopsis}

\begin{qsection}{DESCRIPTION}
{\em fftr} uses the Fast Fourier Transform (FFT) algorithm 
to calculate the Discrete Fourier Transform (DFT) 
of real-valued input data from {\em infile} (or standard input), 
sending the result to standard output. 
When the --m option is omitted 
and the input data sequence length is less than the FFT size, 
the input data is padded with zeros.
The input and output data is in float format, 
arranged as shown below.
\begin{displaymath}
\begin{array}{lll}
\mbox{Input sequence} & 
\overbrace{\framebox[4.5cm]{$x_0, x_1, \ldots, x_{M}, 0,
					\ldots,0$}}^{L}  & \\
		& \makebox[4.5cm]{0\hfill $L-1$} &
\end{array}
\end{displaymath}
\begin{displaymath}
\begin{array}{lll}
\mbox{Output sequence} & \overbrace{\framebox[4.5cm]{real part}}^{L} &
	   \overbrace{\framebox[4.5cm]{imaginary part}}^{L} \\
		& \makebox[4.5cm]{0\hfill $L-1$} &
		\makebox[4.5cm]{0\hfill $L-1$}
\end{array}
\end{displaymath}
\end{qsection}

\begin{options}
	\argm{l}{L}{FFT size power of 2}{256}
	\argm{m}{M}{order of sequence}{L-1}
	\argm{A}{}{output magnitude}{FALSE}
	\argm{R}{}{output real part}{FALSE}
	\argm{I}{}{output imaginary part}{FALSE}
	\argm{P}{}{output power spectrum}{FALSE}
	\argm{H}{}{output half size}{FALSE}
\end{options}

\begin{qsection}{EXAMPLE}
In the example below, a sine wave is passed through a Blackman window,
its DFT is evaluated and the magnitude is plotted:
\begin{quote}
  \verb!sin -t 30 | window | fftr -A | fdrw | xgr!
\end{quote}

\end{qsection}

\begin{qsection}{SEE ALSO}
  fft, spec, phase
\end{qsection}
