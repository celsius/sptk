% ----------------------------------------------------------------
%       Speech Signal Processing Toolkit (SPTK): version 3.0
%                      SPTK Working Group
% 
%                Department of Computer Science
%                Nagoya Institute of Technology
%                             and
%   Interdisciplinary Graduate School of Science and Engineering
%                Tokyo Institute of Technology
%                   Copyright (c) 1984-2000
%                     All Rights Reserved.
% 
% Permission is hereby granted, free of charge, to use and
% distribute this software and its documentation without
% restriction, including without limitation the rights to use,
% copy, modify, merge, publish, distribute, sublicense, and/or
% sell copies of this work, and to permit persons to whom this
% work is furnished to do so, subject to the following conditions:
% 
%   1. The code must retain the above copyright notice, this list
%      of conditions and the following disclaimer.
% 
%   2. Any modifications must be clearly marked as such.
%                                                                        
% NAGOYA INSTITUTE OF TECHNOLOGY, TOKYO INSITITUTE OF TECHNOLOGY,
% SPTK WORKING GROUP, AND THE CONTRIBUTORS TO THIS WORK DISCLAIM
% ALL WARRANTIES WITH REGARD TO THIS SOFTWARE, INCLUDING ALL
% IMPLIED WARRANTIES OF MERCHANTABILITY AND FITNESS, IN NO EVENT
% SHALL NAGOYA INSTITUTE OF TECHNOLOGY, TOKYO INSITITUTE OF
% TECHNOLOGY, SPTK WORKING GROUP, NOR THE CONTRIBUTORS BE LIABLE
% FOR ANY SPECIAL, INDIRECT OR CONSEQUENTIAL DAMAGES OR ANY
% DAMAGES WHATSOEVER RESULTING FROM LOSS OF USE, DATA OR PROFITS,
% WHETHER IN AN ACTION OF CONTRACT, NEGLIGENCE OR OTHER TORTIOUS
% ACTION, ARISING OUT OF OR IN CONNECTION WITH THE USE OR
% PERFORMANCE OF THIS SOFTWARE.
% ----------------------------------------------------------------
%
\name{norm0}{normalize coefficients}{signal processing}

\begin{synopsis}
\item[norm0] [ --m $M$ ] [ {\em infile} ]
\end{synopsis}

\begin{qsection}{DESCRIPTION}
Assume that the input data sequance read from the standard input
is
\begin{displaymath}
x(0), x(1), \ldots, x(M).
\end{displaymath}
The normalization taken from 0 order coefficients is evaluated as
\begin{displaymath}
1/x(0), x(1)/x(0), \ldots, x(M)/x(0)
\end{displaymath}
and sent to the output.
\par
Input and output data are in float format.
\end{qsection}

\begin{options}
        \argm{m}{M}{order of input data}{25}
\end{options}

\begin{qsection}{EXAMPLE}
Speech data is read from {\em data.f} in float format,
the 15-order autocorrelation coefficients are evaluated
and normalized, and the results is written to {\em data.nacorr}:
\begin{quote}
  \verb!frame < data.f | window | acorr -m 15 |\ !\\
  \verb!norm0 -m 15 > data.nacorr!
\end{quote}
\end{qsection}

\begin{qsection}{SEE ALSO}
 linear\_intpl
\end{qsection}
