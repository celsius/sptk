% ----------------------------------------------------------------- %
%             The Speech Signal Processing Toolkit (SPTK)           %
%             developed by SPTK Working Group                       %
%             http://sp-tk.sourceforge.net/                         %
% ----------------------------------------------------------------- %
%                                                                   %
%  Copyright (c) 1984-2007  Tokyo Institute of Technology           %
%                           Interdisciplinary Graduate School of    %
%                           Science and Engineering                 %
%                                                                   %
%                1996-2011  Nagoya Institute of Technology          %
%                           Department of Computer Science          %
%                                                                   %
% All rights reserved.                                              %
%                                                                   %
% Redistribution and use in source and binary forms, with or        %
% without modification, are permitted provided that the following   %
% conditions are met:                                               %
%                                                                   %
% - Redistributions of source code must retain the above copyright  %
%   notice, this list of conditions and the following disclaimer.   %
% - Redistributions in binary form must reproduce the above         %
%   copyright notice, this list of conditions and the following     %
%   disclaimer in the documentation and/or other materials provided %
%   with the distribution.                                          %
% - Neither the name of the SPTK working group nor the names of its %
%   contributors may be used to endorse or promote products derived %
%   from this software without specific prior written permission.   %
%                                                                   %
% THIS SOFTWARE IS PROVIDED BY THE COPYRIGHT HOLDERS AND            %
% CONTRIBUTORS "AS IS" AND ANY EXPRESS OR IMPLIED WARRANTIES,       %
% INCLUDING, BUT NOT LIMITED TO, THE IMPLIED WARRANTIES OF          %
% MERCHANTABILITY AND FITNESS FOR A PARTICULAR PURPOSE ARE          %
% DISCLAIMED. IN NO EVENT SHALL THE COPYRIGHT OWNER OR CONTRIBUTORS %
% BE LIABLE FOR ANY DIRECT, INDIRECT, INCIDENTAL, SPECIAL,          %
% EXEMPLARY, OR CONSEQUENTIAL DAMAGES (INCLUDING, BUT NOT LIMITED   %
% TO, PROCUREMENT OF SUBSTITUTE GOODS OR SERVICES; LOSS OF USE,     %
% DATA, OR PROFITS; OR BUSINESS INTERRUPTION) HOWEVER CAUSED AND ON %
% ANY THEORY OF LIABILITY, WHETHER IN CONTRACT, STRICT LIABILITY,   %
% OR TORT (INCLUDING NEGLIGENCE OR OTHERWISE) ARISING IN ANY WAY    %
% OUT OF THE USE OF THIS SOFTWARE, EVEN IF ADVISED OF THE           %
% POSSIBILITY OF SUCH DAMAGE.                                       %
% ----------------------------------------------------------------- %
\hypertarget{wav2raw}{}
\name{wav2raw}{wav (RIFF) to raw}{data operation}

\begin{synopsis}
\item[wav2raw] [ --swab ] [ --d $D$ ] [ --n ] [ --N ] [ {\em +type} ] [ {\em infile} ]
\end{synopsis}

\begin{qsection}{DESCRIPTION}
{\em wav2raw} converts file format from wav to raw.

\end{qsection}

\begin{options}
	\argm{swab}{}{change ``endiannes''}{FALSE}
	\argm{d}{D}{destination directory}{N/A}
	\argm{n}{}{normalization with the maximum value\\
                   according to bit/sample of the wav file \\
                   if max $>$= 255 (8bit), 32767 (16bit), \\
                   8388067 (24bit) or 2147483647 (32bit)}{FALSE}
	\argm{N}{}{normalization with the maximum value}{FALSE}
	\argp{type}{output data type\\ 
		\begin{tabular}{llcll} \\[-1ex]
         c & char (1 byte) & \quad &
         C & unsigned char (1 byte) \\
         s & short (2 bytes) & \quad &
                     S & unsigned short (2 bytes) \\
         i3 & int (3 bytes) & \quad &
                     I3 & unsigned int (3 bytes) \\
         i & int (4 bytes) & \quad &
                     I & unsigned int (4 bytes) \\
         l & long (4 bytes) & \quad &
                     L & unsigned long (4 bytes) \\
         f & float (4 bytes) & \quad &
                     d & double (8 bytes) \\
			a & ascii & \quad & & \\
		 \end{tabular}}{f}
\end{options}

\begin{qsection}{EXAMPLE}
 In the following example, the file {\em data.wav} is converted
 to {\em data.raw}
 and normalized with the maximum value.
 The output will be saved in the same directory as {\em data.wav}
 unless the {\em -d} option is given:
\begin{quote}
  \verb!wav2raw -N data.wav!
\end{quote}
\end{qsection}

\begin{qsection}{SEE ALSO}
\hyperlink{raw2wav}{raw2wav},
\hyperlink{swab}{swab}
\end{qsection}
