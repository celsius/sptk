% ----------------------------------------------------------------
%       Speech Signal Processing Toolkit (SPTK): version 3.0
%                      SPTK Working Group
% 
%                Department of Computer Science
%                Nagoya Institute of Technology
%                             and
%   Interdisciplinary Graduate School of Science and Engineering
%                Tokyo Institute of Technology
%                   Copyright (c) 1984-2000
%                     All Rights Reserved.
% 
% Permission is hereby granted, free of charge, to use and
% distribute this software and its documentation without
% restriction, including without limitation the rights to use,
% copy, modify, merge, publish, distribute, sublicense, and/or
% sell copies of this work, and to permit persons to whom this
% work is furnished to do so, subject to the following conditions:
% 
%   1. The code must retain the above copyright notice, this list
%      of conditions and the following disclaimer.
% 
%   2. Any modifications must be clearly marked as such.
%                                                                        
% NAGOYA INSTITUTE OF TECHNOLOGY, TOKYO INSITITUTE OF TECHNOLOGY,
% SPTK WORKING GROUP, AND THE CONTRIBUTORS TO THIS WORK DISCLAIM
% ALL WARRANTIES WITH REGARD TO THIS SOFTWARE, INCLUDING ALL
% IMPLIED WARRANTIES OF MERCHANTABILITY AND FITNESS, IN NO EVENT
% SHALL NAGOYA INSTITUTE OF TECHNOLOGY, TOKYO INSITITUTE OF
% TECHNOLOGY, SPTK WORKING GROUP, NOR THE CONTRIBUTORS BE LIABLE
% FOR ANY SPECIAL, INDIRECT OR CONSEQUENTIAL DAMAGES OR ANY
% DAMAGES WHATSOEVER RESULTING FROM LOSS OF USE, DATA OR PROFITS,
% WHETHER IN AN ACTION OF CONTRACT, NEGLIGENCE OR OTHER TORTIOUS
% ACTION, ARISING OUT OF OR IN CONNECTION WITH THE USE OR
% PERFORMANCE OF THIS SOFTWARE.
% ----------------------------------------------------------------
%
\name{fdrw}{draw a graph}{plotting graphs}

\begin{synopsis}
\item[fdrw] [ --F $F$ ] [ --R $R$ ] [ --W $W$ ] [ --H $H$ ] [ --o $xo \; yo$ ] 
            [ --g $G$ ] [ --m $M$ ]   
\item[\ ~~~~~] [ --l $L$ ] [ --p $P$ ] [ --n $N$ ] [ --t $T$ ] 
	       [ --y $ymin \; ymax$ ] [ --z $Z$ ] [ --b ]  
\item[\ ~~~~~] [ {\em infile} ]
\end{synopsis}

\begin{qsection}{DESCRIPTION}
This command connects through a straight line the input data in float
format.
\par
The output includes a sequence of commands so that it can be plotted
(FP5301 protocol).
\end{qsection}

\begin{options}
	\argm{F}{F}{factor}{1}
	\argm{R}{R}{rotation angle}{0}
	\argm{W}{W}{width of figure($\times 100$mm)}{1}
	\argm{H}{H}{height of figure($\times 100$mm)}{1}
	\argm{o}{xo \; yo}{origin in mm}{20 25}
	\argm{g}{G}{draw grid($0 \sim 2$)
                    Please refer to ``fig'' command.}{1}
	\argm{m}{M}{line type($1 \sim 5$)\\
	\hspace*{2mm}1:~solid~~2:~dotted~~3:~dot and dash~~4:~broken~~5:~dash}{0}
	\argm{l}{L}{line pitch}{0}
	\argm{p}{P}{pen number($1 \sim 10$)}{1}
	\argm{n}{N}{number of sample}{0}
	\argm{t}{T}{rotation of coordinate axis. When $T=-1$, the
                    reference point is on the top-left. When $T=1$
                    the reference point is on the bottom-right.}{0}
	\argm{y}{ymin \; ymax}{scaling factor for $y$ axis}{-1 1}
	\argm{z}{Z}{This option is used when data is written
                    recursively in the $y$ axis. The distance between
                    two graphs in the $y$ axis is given by $Z$.}{0}
	\argm{b}{}{bar graph mode}{FALSE}
	\desc[1zh]{The $x$ axis scaling is automatically done so that
                every point in the input file is plotted in equal interval
                for the assigned width.
                When {\bf --n} option is omitted and the number of
                input samples is below 5000, then the block size is made
                equal to the number of samples.
                When the number of samples is above 5000,
                then the block size is made equal to 5000.}
	\desc{When the {\bf --y} option is omitted,
		the input data minimum value is made equal to $ymin$
                and the maximum value is made equal to $ymax$.}
\end{options}

\begin{qsection}{EXAMPLE}
In the example below, the impulse response of a digital filter is
drawed on the X window environment:
\begin{quote}
  \verb!impulse | dfs -a 1 0.8 0.5 | fdrw -H 0.3 | xgr!
\end{quote}
The graph width is 10cm and its height is 3cm.
\par
The next example draws on the X window environment the magnitude of
the frequency response of a digital filter:
\begin{quote}
  \verb!impulse | dfs -a 1 0.8 0.5 | spec | fdrw -y -60 40 | xgr!
\end{quote}
The $y$ axis goes from $-60$dB to $40$dB.
\par
The running spectrum can be draw on the X window environment by:
\begin{quote}
 \verb!fig -g 0 -w 0.4 << EOF ! \\
 \verb!����x 0 5 !\\
 \verb!����xscale 0 1 2 3 4 5 !\\
 \verb!����xname "FREQUENCY (kHz)"!\\
 \verb!EOF!\\
 \verb!spec < data |\ !\\
 \verb!fdrw -w 0.4 -h 0.2 -g 0 -n 129 -y -30 30 -z 3 |\ !\\
 \verb!xgr !
\end{quote}
The command {\em psgr} prints the output in a laser printer in the
same way that it is printed on the screen.
Since the {\em fdrw} command includes a sequence of commands
for a plotter machine(FP5301 protocol) in the output file,
its output can be directly sent to a printer.
\end{qsection}

\begin{qsection}{SEE ALSO}
 fig, xgr, psgr
\end{qsection}
