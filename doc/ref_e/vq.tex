% ----------------------------------------------------------------
%       Speech Signal Processing Toolkit (SPTK): version 3.0
%                      SPTK Working Group
% 
%                Department of Computer Science
%                Nagoya Institute of Technology
%                             and
%   Interdisciplinary Graduate School of Science and Engineering
%                Tokyo Institute of Technology
%                   Copyright (c) 1984-2000
%                     All Rights Reserved.
% 
% Permission is hereby granted, free of charge, to use and
% distribute this software and its documentation without
% restriction, including without limitation the rights to use,
% copy, modify, merge, publish, distribute, sublicense, and/or
% sell copies of this work, and to permit persons to whom this
% work is furnished to do so, subject to the following conditions:
% 
%   1. The code must retain the above copyright notice, this list
%      of conditions and the following disclaimer.
% 
%   2. Any modifications must be clearly marked as such.
%                                                                        
% NAGOYA INSTITUTE OF TECHNOLOGY, TOKYO INSITITUTE OF TECHNOLOGY,
% SPTK WORKING GROUP, AND THE CONTRIBUTORS TO THIS WORK DISCLAIM
% ALL WARRANTIES WITH REGARD TO THIS SOFTWARE, INCLUDING ALL
% IMPLIED WARRANTIES OF MERCHANTABILITY AND FITNESS, IN NO EVENT
% SHALL NAGOYA INSTITUTE OF TECHNOLOGY, TOKYO INSITITUTE OF
% TECHNOLOGY, SPTK WORKING GROUP, NOR THE CONTRIBUTORS BE LIABLE
% FOR ANY SPECIAL, INDIRECT OR CONSEQUENTIAL DAMAGES OR ANY
% DAMAGES WHATSOEVER RESULTING FROM LOSS OF USE, DATA OR PROFITS,
% WHETHER IN AN ACTION OF CONTRACT, NEGLIGENCE OR OTHER TORTIOUS
% ACTION, ARISING OUT OF OR IN CONNECTION WITH THE USE OR
% PERFORMANCE OF THIS SOFTWARE.
% ----------------------------------------------------------------
%
\name{vq}{vector quantization}{vector quantization}

\begin{synopsis}
\item [vq] [ --l $L$ ] [ --n $N$ ] [ --q ] {\em cbfile} [{\em infile}]
\end{synopsis}

\begin{qsection}{DESCRIPTION}
{\em vq} uses vector quantization to compress vectors 
from {\em infile} (or standard input)
according to the codebook {\em cbfile}, 
sending either codebook indices or quantized vectors to standard output.

For each length $L$ input vector
\begin{displaymath}
  x(0),x(1),\ldots,x(L-1)
\end{displaymath}
{\em vq} finds the codebook vector $\mbox{\boldmath $c$}_i$ 
that minimizes the Euclidian distance
\begin{displaymath}
d_i = \frac{1}{L}\sum_{m=0}^{L-1} (x(m)-c_i(m))^2. 
\end{displaymath}

Input data is in float format.
If the --q option is given, 
the output is the code vector $[c_i(0), c_i(1), \cdots, c_i(L-1)]$ 
in float format.
If the --q option is not given, 
the output is codebook index $i$ in int format.
\end{qsection}

\begin{options}
	\argm{l}{L}{length of vector}{26}
	\argm{n}{N}{order of vector}{25}
	\argm{q}{}{output quantized vector}{FALSE}
\end{options}

\begin{qsection}{EXAMPLE}
In this example, a sequence of length 25 is read from {\em data.f}
in float format.
it is quantized using codebook {\em cbfile},
and the results are written to {\em data.vq}:
\begin{quote}
 \verb!vq -q cbfile < data.f > data.vq!
\end{quote} 
\end{qsection}

\begin{qsection}{SEE ALSO}
ivq, msvq, imsvq, lbg
\end{qsection}
