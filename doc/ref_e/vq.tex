% ----------------------------------------------------------------- %
%             The Speech Signal Processing Toolkit (SPTK)           %
%             developed by SPTK Working Group                       %
%             http://sp-tk.sourceforge.net/                         %
% ----------------------------------------------------------------- %
%                                                                   %
%  Copyright (c) 1984-2007  Tokyo Institute of Technology           %
%                           Interdisciplinary Graduate School of    %
%                           Science and Engineering                 %
%                                                                   %
%                1996-2008  Nagoya Institute of Technology          %
%                           Department of Computer Science          %
%                                                                   %
% All rights reserved.                                              %
%                                                                   %
% Redistribution and use in source and binary forms, with or        %
% without modification, are permitted provided that the following   %
% conditions are met:                                               %
%                                                                   %
% - Redistributions of source code must retain the above copyright  %
%   notice, this list of conditions and the following disclaimer.   %
% - Redistributions in binary form must reproduce the above         %
%   copyright notice, this list of conditions and the following     %
%   disclaimer in the documentation and/or other materials provided %
%   with the distribution.                                          %
% - Neither the name of the SPTK working group nor the names of its %
%   contributors may be used to endorse or promote products derived %
%   from this software without specific prior written permission.   %
%                                                                   %
% THIS SOFTWARE IS PROVIDED BY THE COPYRIGHT HOLDERS AND            %
% CONTRIBUTORS "AS IS" AND ANY EXPRESS OR IMPLIED WARRANTIES,       %
% INCLUDING, BUT NOT LIMITED TO, THE IMPLIED WARRANTIES OF          %
% MERCHANTABILITY AND FITNESS FOR A PARTICULAR PURPOSE ARE          %
% DISCLAIMED. IN NO EVENT SHALL THE COPYRIGHT OWNER OR CONTRIBUTORS %
% BE LIABLE FOR ANY DIRECT, INDIRECT, INCIDENTAL, SPECIAL,          %
% EXEMPLARY, OR CONSEQUENTIAL DAMAGES (INCLUDING, BUT NOT LIMITED   %
% TO, PROCUREMENT OF SUBSTITUTE GOODS OR SERVICES; LOSS OF USE,     %
% DATA, OR PROFITS; OR BUSINESS INTERRUPTION) HOWEVER CAUSED AND ON %
% ANY THEORY OF LIABILITY, WHETHER IN CONTRACT, STRICT LIABILITY,   %
% OR TORT (INCLUDING NEGLIGENCE OR OTHERWISE) ARISING IN ANY WAY    %
% OUT OF THE USE OF THIS SOFTWARE, EVEN IF ADVISED OF THE           %
% POSSIBILITY OF SUCH DAMAGE.                                       %
% ----------------------------------------------------------------- %
\hypertarget{vq}{}
\name{vq}{vector quantization}{vector quantization}

\begin{synopsis}
\item [vq] [ --l $L$ ] [ --n $N$ ] [ --q ] {\em cbfile} [{\em infile}]
\end{synopsis}

\begin{qsection}{DESCRIPTION}
{\em vq} uses vector quantization to compress vectors 
from {\em infile} (or standard input)
according to the codebook {\em cbfile}, 
sending either codebook indexes or quantized vectors to standard output.

For each length $L$ input vector
\begin{displaymath}
  x(0),x(1),\dots,x(L-1)
\end{displaymath}
{\em vq} finds the codebook vector $\bc_i$ 
that minimizes the Euclidean distance
\begin{displaymath}
d_i = \frac{1}{L}\sum_{m=0}^{L-1} (x(m)-c_i(m))^2. 
\end{displaymath}

Input data is in float format.
If the --q option is given, 
the output is the code vector $[c_i(0), c_i(1), \cdots, c_i(L-1)]$ 
in float format.
If the --q option is not given, 
the output is codebook index $i$ in int format.
\end{qsection}

\begin{options}
	\argm{l}{L}{length of vector}{26}
	\argm{n}{N}{order of vector}{25}
	\argm{q}{}{output quantized vector}{FALSE}
\end{options}

\begin{qsection}{EXAMPLE}
In this example, a sequence of length 25 is read from {\em data.f}
in float format.
it is quantized using codebook {\em cbfile},
and the results are written to {\em data.vq}:
\begin{quote}
 \verb!vq -q cbfile < data.f > data.vq!
\end{quote} 
\end{qsection}

\begin{qsection}{SEE ALSO}
\hyperlink{ivq}{ivq},
\hyperlink{msvq}{msvq},
\hyperlink{imsvq}{imsvq},
\hyperlink{lbg}{lbg}
\end{qsection}
