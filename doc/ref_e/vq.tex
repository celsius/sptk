\name{vq}{vector quantization}{vector quantization}

\begin{synopsis}
\item [vq] [ --l $L$ ] [ --n $N$ ] [ --q ] {\em cbfile} [{\em infile}]
\end{synopsis}

\begin{qsection}{DESCRIPTION}
This command reads data from the assigned file sequence of length $L$
\begin{displaymath}
  x(0),x(1),\ldots,x(L-1)
\end{displaymath}
compares each value with every code vector read from
codebook file {\em cbfile} through the Euclidean distance $d_i$,
\begin{displaymath}
d_i = \frac{1}{L}\sum_{m=0}^{L-1} (x(m)-c_i(m))^2
\end{displaymath}
and sends to the output the index which minimizes this distance.
If -q options is assigned, then the code vector
$[c_i(0), c_i(1), \cdots, c_i(L-1)]$ is outputed.
\par
Input data is in float format and output data is in int format.
If the -q options is assigned, then output data is in float format.
\end{qsection}

\begin{options}
	\argm{l}{L}{length of vector}{26}
	\argm{n}{N}{order of vector}{25}
	\argm{q}{}{output quantized vector}{FALSE}
\end{options}

\begin{qsection}{EXAMPLE}
In this example, a sequence of length 25 is read from {\em data.f}
in float format.
it is quantized using codebook {\em cbfile},
and the results are written to {\em data.vq}:
\begin{quote}
 \verb!vq -q cbfile < data.f > data.vq!
\end{quote} 
\end{qsection}

\begin{qsection}{SEE ALSO}
ivq, msvq, imsvq, lbg
\end{qsection}
