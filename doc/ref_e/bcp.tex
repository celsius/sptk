% ----------------------------------------------------------------
%       Speech Signal Processing Toolkit (SPTK): version 3.0
%                      SPTK Working Group
% 
%                Department of Computer Science
%                Nagoya Institute of Technology
%                             and
%   Interdisciplinary Graduate School of Science and Engineering
%                Tokyo Institute of Technology
%                   Copyright (c) 1984-2000
%                     All Rights Reserved.
% 
% Permission is hereby granted, free of charge, to use and
% distribute this software and its documentation without
% restriction, including without limitation the rights to use,
% copy, modify, merge, publish, distribute, sublicense, and/or
% sell copies of this work, and to permit persons to whom this
% work is furnished to do so, subject to the following conditions:
% 
%   1. The code must retain the above copyright notice, this list
%      of conditions and the following disclaimer.
% 
%   2. Any modifications must be clearly marked as such.
%                                                                        
% NAGOYA INSTITUTE OF TECHNOLOGY, TOKYO INSITITUTE OF TECHNOLOGY,
% SPTK WORKING GROUP, AND THE CONTRIBUTORS TO THIS WORK DISCLAIM
% ALL WARRANTIES WITH REGARD TO THIS SOFTWARE, INCLUDING ALL
% IMPLIED WARRANTIES OF MERCHANTABILITY AND FITNESS, IN NO EVENT
% SHALL NAGOYA INSTITUTE OF TECHNOLOGY, TOKYO INSITITUTE OF
% TECHNOLOGY, SPTK WORKING GROUP, NOR THE CONTRIBUTORS BE LIABLE
% FOR ANY SPECIAL, INDIRECT OR CONSEQUENTIAL DAMAGES OR ANY
% DAMAGES WHATSOEVER RESULTING FROM LOSS OF USE, DATA OR PROFITS,
% WHETHER IN AN ACTION OF CONTRACT, NEGLIGENCE OR OTHER TORTIOUS
% ACTION, ARISING OUT OF OR IN CONNECTION WITH THE USE OR
% PERFORMANCE OF THIS SOFTWARE.
% ----------------------------------------------------------------
%
\hypertarget{bcp}{}
\name{bcp}{block copy}{data operation}

\begin{synopsis}
\item[bcp] [ --l $l$ ]  [ --L $L$ ]  [ --n $n$ ]  [ --N $N$ ]
           [ --s $s$ ]  [ --S $S$ ]  [ --e $e$ ]  [ --f $f$ ]
\item[\ ~~~~~~~] [ +{\em type} ] [ {\em infile} ] 
\end{synopsis}

\begin{qsection}{DESCRIPTION}
	{\em bcp} copies data blocks from {\em infile} (or standard input) 
	to standard output, 
	reformatting them according to command line options as shown below.

	If the input format is ASCII, 
	the basic input unit is a sequence of letters
	and the output block is partitioned with carriage returns.
\end{qsection}

\begin{center}
\leavevmode
\begin{figure}[h]
\includegraphics{fig/bcp.eps}
\caption{Example of the bcp command}
\end{figure}
\end{center}

\begin{options}
	\argm{l}{l}{number of items contained 1 block}{512}
	\argm{L}{L}{number of destination block size}{N/A}
	\argm{n}{n}{order of items contained 1 block}{l-1}
	\argm{N}{N}{order of destination block size}{N/A}
	\argm{s}{s}{start number}{0}
	\argm{S}{S}{start number in destination block}{0}
	\argm{e}{e}{end number}{EOF}
	\argm{f}{f}{fill into empty block}{0}
	\argp{t}{data type\\ 
		\begin{tabular}{llcll} \\[-1ex]
			c & char (1byte) & \quad &
			s & short (2bytes) \\
			i & int (4bytes) & \quad &
			l & long (4bytes) \\
			f & float (4bytes) & \quad &
			d & double (8bytes) \\
			a & ASCII letter sequence\\
		\end{tabular}}{f}
\end{options}

\begin{qsection}{EXAMPLE}
Assume that data of the input file {\em data.f}
{a(0), a(1), a(2), ... , a(20)} is recursively written in float format.\\
If it is desired to copy the array {a(1), a(2), ... , a(10)}, 
the following command can be used.
\begin{quote}
\verb!bcp data.f +f -l 21 -s 1 -e 10 > data.bcp!
\end{quote}

\par
A different example with respect to the same input file {\em data.f}
follows

\begin{quote}
\verb!bcp data.f +f -l 21 -s 3 -e 5 -S 6 -L 10 > data.bcp!
\end{quote}

In this example, the output block is
\begin{quote}
\verb!0, 0, 0, 0, 0, 0, a(3), a(4), a(5), 0!
\end{quote}
\end{qsection}

\begin{qsection}{SEE ALSO}
\hyperlink{bcut}{bcut},
\hyperlink{merge}{merge},
\hyperlink{reverse}{reverse}
\end{qsection}
