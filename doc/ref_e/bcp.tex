\name{bcp}{block copy}{data operation}

\begin{synopsis}
\item[bcp] [ --l $l$ ]  [ --L $L$ ]  [ --n $n$ ]  [ --N $N$ ]
           [ --s $s$ ]  [ --S $S$ ]  [ --e $e$ ]  [ --f $f$ ]
\item[\ ~~~~~~~] [ +{\em type} ] [ {\em infile} ] 
\end{synopsis}

\begin{qsection}{DESCRIPTION}
The {\em bcp} command sends to the standard output
the whole or part of an input file.
The input file as well as the block length are determined by
the option in the command line.
\par
In case data is in ASCII format, the unit is
a sequence of letters of the input file.
The output block is partitioned by the carriage return.
\par
In case the input file is not assigned,
data is read from the standard input.
\end{qsection}

\begin{center}
\leavevmode
\begin{figure}[h]
\includegraphics{fig/bcp.eps}
\caption{Example of the bcp command}
\end{figure}
\end{center}

\begin{options}
	\argm{l}{l}{number of items contained 1 block}{512}
	\argm{L}{L}{number of destination block size}{N/A}
	\argm{n}{n}{order of items contained 1 block}{l-1}
	\argm{N}{N}{order of destination block size}{N/A}
	\argm{s}{s}{start number}{0}
	\argm{S}{S}{start number in destination block}{0}
	\argm{e}{e}{end number}{EOF}
	\argm{f}{f}{fill into empty block}{0}
	\argp{t}{data type\\ 
		\begin{tabular}{llcll} \\[-1zh]
			c & char (1byte) & \quad &
			s & short (2bytes) \\
			i & int (4bytes) & \quad &
			l & long (4bytes) \\
			f & float (4bytes) & \quad &
			d & double (8bytes) \\
			a & ASCII letter sequence\\
		\end{tabular}}{f}
\end{options}

\begin{qsection}{EXAMPLE}
Assume that data of the input file {\em data.f}
{a(0), a(1), a(2), ... , a(20)} is recursively written in float format.\\
If it is desired to copy the array {a(1), a(2), ... , a(10)}, 
the following command can be used.
\begin{quote}
\verb!bcp data.f +f -l 21 -s 1 -e 10 > data.bcp!
\end{quote}

\par
A different example with respect to the same input file {\em data.f}
follows

\begin{quote}
\verb!bcp data.f +f -l 21 -s 3 -e 5 -S 6 -L 10 > data.bcp!
\end{quote}

In this example, the output block is
\begin{quote}
\verb!0, 0, 0, 0, 0, 0, a(3), a(4), a(5), 0!
\end{quote}
\end{qsection}

\begin{qsection}{SEE ALSO}
bcut, merge, reverse
\end{qsection}
