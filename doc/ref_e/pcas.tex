% ----------------------------------------------------------------- %
%             The Speech Signal Processing Toolkit (SPTK)           %
%             developed by SPTK Working Group                       %
%             http://sp-tk.sourceforge.net/                         %
% ----------------------------------------------------------------- %
%                                                                   %
%  Copyright (c) 1984-2007  Tokyo Institute of Technology           %
%                           Interdisciplinary Graduate School of    %
%                           Science and Engineering                 %
%                                                                   %
%                1996-2010  Nagoya Institute of Technology          %
%                           Department of Computer Science          %
%                                                                   %
% All rights reserved.                                              %
%                                                                   %
% Redistribution and use in source and binary forms, with or        %
% without modification, are permitted provided that the following   %
% conditions are met:                                               %
%                                                                   %
% - Redistributions of source code must retain the above copyright  %
%   notice, this list of conditions and the following disclaimer.   %
% - Redistributions in binary form must reproduce the above         %
%   copyright notice, this list of conditions and the following     %
%   disclaimer in the documentation and/or other materials provided %
%   with the distribution.                                          %
% - Neither the name of the SPTK working group nor the names of its %
%   contributors may be used to endorse or promote products derived %
%   from this software without specific prior written permission.   %
%                                                                   %
% THIS SOFTWARE IS PROVIDED BY THE COPYRIGHT HOLDERS AND            %
% CONTRIBUTORS "AS IS" AND ANY EXPRESS OR IMPLIED WARRANTIES,       %
% INCLUDING, BUT NOT LIMITED TO, THE IMPLIED WARRANTIES OF          %
% MERCHANTABILITY AND FITNESS FOR A PARTICULAR PURPOSE ARE          %
% DISCLAIMED. IN NO EVENT SHALL THE COPYRIGHT OWNER OR CONTRIBUTORS %
% BE LIABLE FOR ANY DIRECT, INDIRECT, INCIDENTAL, SPECIAL,          %
% EXEMPLARY, OR CONSEQUENTIAL DAMAGES (INCLUDING, BUT NOT LIMITED   %
% TO, PROCUREMENT OF SUBSTITUTE GOODS OR SERVICES; LOSS OF USE,     %
% DATA, OR PROFITS; OR BUSINESS INTERRUPTION) HOWEVER CAUSED AND ON %
% ANY THEORY OF LIABILITY, WHETHER IN CONTRACT, STRICT LIABILITY,   %
% OR TORT (INCLUDING NEGLIGENCE OR OTHERWISE) ARISING IN ANY WAY    %
% OUT OF THE USE OF THIS SOFTWARE, EVEN IF ADVISED OF THE           %
% POSSIBILITY OF SUCH DAMAGE.                                       %
% ----------------------------------------------------------------- %
\name{pcas}{calculate principal component scores}{data processing}
\def\Vec#1{\mbox{\boldmath $#1$}}

\begin{synopsis}
 \item[pcas] [ --l $L$ ] [ --n $N$] {\em pcafile} [ {\em infile} ] 
\end{synopsis}

\begin{qsection}{DESCRIPTION}
 {\em pcas} calculates principal component scores
 from {\em infile} (or standard input) ,
 sending the result to standard output.

 The input data set must be composed of $L$-dimension,
 mean vector $\Vec{m}$ and eigen vectors $\Vec{e}(i)$:
\[
 \Vec{m}, \Vec{e}(0), \Vec{e}(1), \Vec{e}(2), \cdots
 \]
 \[
 where\;\;\Vec{m} = (m(1), m(2), \cdots, m(L))\;\;and\;\;
 \Vec{e}(i) = (e_{i}(1), e_{i}(2), \cdots, e_{i}(L))
\]

Input and output data are in float format. 
\end{qsection}

\begin{options}
 \argm{l}{L}{dimentionality of vector}{3}
 \argm{n}{N}{number principal components for output}{2}
\end{options}


\begin{qsection}{EXAMPLE}
 In the example below,
 the principal component scores are calculated
 from {\em test.dat} and sent to {\em score.dat},
 using {\em pca.dat}
 in which the mean vectors and the eigen vectors
 are contained.
\begin{quote}
  \verb!pcas pca.dat -l 3 -n 2 < test.dat > score.dat!
\end{quote} 
In the {\em pca.dat}, the mean vector must be in the front of
eigen vectors.
\end{qsection} 
\begin{qsection}{SEE ALSO}
 \hyperlink{pca}{pca}
\end{qsection}
