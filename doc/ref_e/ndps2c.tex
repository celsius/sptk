% ----------------------------------------------------------------- %
%             The Speech Signal Processing Toolkit (SPTK)           %
%             developed by SPTK Working Group                       %
%             http://sp-tk.sourceforge.net/                         %
% ----------------------------------------------------------------- %
%                                                                   %
%  Copyright (c) 1984-2007  Tokyo Institute of Technology           %
%                           Interdisciplinary Graduate School of    %
%                           Science and Engineering                 %
%                                                                   %
%                1996-2016  Nagoya Institute of Technology          %
%                           Department of Computer Science          %
%                                                                   %
% All rights reserved.                                              %
%                                                                   %
% Redistribution and use in source and binary forms, with or        %
% without modification, are permitted provided that the following   %
% conditions are met:                                               %
%                                                                   %
% - Redistributions of source code must retain the above copyright  %
%   notice, this list of conditions and the following disclaimer.   %
% - Redistributions in binary form must reproduce the above         %
%   copyright notice, this list of conditions and the following     %
%   disclaimer in the documentation and/or other materials provided %
%   with the distribution.                                          %
% - Neither the name of the SPTK working group nor the names of its %
%   contributors may be used to endorse or promote products derived %
%   from this software without specific prior written permission.   %
%                                                                   %
% THIS SOFTWARE IS PROVIDED BY THE COPYRIGHT HOLDERS AND            %
% CONTRIBUTORS "AS IS" AND ANY EXPRESS OR IMPLIED WARRANTIES,       %
% INCLUDING, BUT NOT LIMITED TO, THE IMPLIED WARRANTIES OF          %
% MERCHANTABILITY AND FITNESS FOR A PARTICULAR PURPOSE ARE          %
% DISCLAIMED. IN NO EVENT SHALL THE COPYRIGHT OWNER OR CONTRIBUTORS %
% BE LIABLE FOR ANY DIRECT, INDIRECT, INCIDENTAL, SPECIAL,          %
% EXEMPLARY, OR CONSEQUENTIAL DAMAGES (INCLUDING, BUT NOT LIMITED   %
% TO, PROCUREMENT OF SUBSTITUTE GOODS OR SERVICES; LOSS OF USE,     %
% DATA, OR PROFITS; OR BUSINESS INTERRUPTION) HOWEVER CAUSED AND ON %
% ANY THEORY OF LIABILITY, WHETHER IN CONTRACT, STRICT LIABILITY,   %
% OR TORT (INCLUDING NEGLIGENCE OR OTHERWISE) ARISING IN ANY WAY    %
% OUT OF THE USE OF THIS SOFTWARE, EVEN IF ADVISED OF THE           %
% POSSIBILITY OF SUCH DAMAGE.                                       %
% ----------------------------------------------------------------- %
\hypertarget{ndps2c}{} 
\name[ref:NDPS-SignalProcessing]
{ndps2c}{Negative Derivative of Phase Spectrum (NDPS) to cepstrum}
{speech parameter transformation}

\begin{synopsis}
 \item[ndps2c] [ --m $M$ ] [ --l $L$ ] [ {\em infile} ]
\end{synopsis}

\begin{qsection}{DESCRIPTION}
{\em ndps2c} calculates the minimum phase cepstrum 
from the Negative Derivative of Phase Spectrum (NDPS)
in the {\em infile} (or standard input), 
sending the result to standard output.
For example, if the input sequence is
\begin{displaymath}
   n(0),n(1),n(2),\dots,n(L/2)
\end{displaymath}
then the cepstrum $c(m)$is calculated from
\begin{displaymath}
 n(k) = Re \left[ \sum^{M}_{m=0}mc(m)\mathrm{e}^{-j\frac{2\pi km}{N}} \right]\hspace{10mm} (k=0,\cdots,N-1).
\end{displaymath}

Both input and output files are is float format.

\end{qsection}

\begin{options}
	\argm{m}{M}{order of cepstrum}{25}
	\argm{l}{L}{FFT length}{256}
\end{options}

\begin{qsection}{EXAMPLE}
The output file {\em data.c} contains the cepstrum
in the range $n = 0 \sim 30$ obtained from the NDPS
 file {\em data.ndps}, in float format:
 \begin{quote}
  \verb!ndps2c -l 2048 -m 30 data.ndps > data.cep!
 \end{quote}
\end{qsection}

\begin{qsection}{SEE ALSO}
\hyperlink{mgc2sp}{mgc2sp},
\hyperlink{c2ndps}{c2ndps}
\end{qsection}
