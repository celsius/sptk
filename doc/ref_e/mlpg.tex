% ----------------------------------------------------------------- %
%             The Speech Signal Processing Toolkit (SPTK)           %
%             developed by SPTK Working Group                       %
%             http://sp-tk.sourceforge.net/                         %
% ----------------------------------------------------------------- %
%                                                                   %
%  Copyright (c) 1984-2007  Tokyo Institute of Technology           %
%                           Interdisciplinary Graduate School of    %
%                           Science and Engineering                 %
%                                                                   %
%                1996-2010  Nagoya Institute of Technology          %
%                           Department of Computer Science          %
%                                                                   %
% All rights reserved.                                              %
%                                                                   %
% Redistribution and use in source and binary forms, with or        %
% without modification, are permitted provided that the following   %
% conditions are met:                                               %
%                                                                   %
% - Redistributions of source code must retain the above copyright  %
%   notice, this list of conditions and the following disclaimer.   %
% - Redistributions in binary form must reproduce the above         %
%   copyright notice, this list of conditions and the following     %
%   disclaimer in the documentation and/or other materials provided %
%   with the distribution.                                          %
% - Neither the name of the SPTK working group nor the names of its %
%   contributors may be used to endorse or promote products derived %
%   from this software without specific prior written permission.   %
%                                                                   %
% THIS SOFTWARE IS PROVIDED BY THE COPYRIGHT HOLDERS AND            %
% CONTRIBUTORS "AS IS" AND ANY EXPRESS OR IMPLIED WARRANTIES,       %
% INCLUDING, BUT NOT LIMITED TO, THE IMPLIED WARRANTIES OF          %
% MERCHANTABILITY AND FITNESS FOR A PARTICULAR PURPOSE ARE          %
% DISCLAIMED. IN NO EVENT SHALL THE COPYRIGHT OWNER OR CONTRIBUTORS %
% BE LIABLE FOR ANY DIRECT, INDIRECT, INCIDENTAL, SPECIAL,          %
% EXEMPLARY, OR CONSEQUENTIAL DAMAGES (INCLUDING, BUT NOT LIMITED   %
% TO, PROCUREMENT OF SUBSTITUTE GOODS OR SERVICES; LOSS OF USE,     %
% DATA, OR PROFITS; OR BUSINESS INTERRUPTION) HOWEVER CAUSED AND ON %
% ANY THEORY OF LIABILITY, WHETHER IN CONTRACT, STRICT LIABILITY,   %
% OR TORT (INCLUDING NEGLIGENCE OR OTHERWISE) ARISING IN ANY WAY    %
% OUT OF THE USE OF THIS SOFTWARE, EVEN IF ADVISED OF THE           %
% POSSIBILITY OF SUCH DAMAGE.                                       %
% ----------------------------------------------------------------- %
\hypertarget{mlpg}{}
\name[ref:synHMM-EUROSPEECH95]{mlpg}%
{obtain parameter sequence from PDF sequence}{parameter generation}

\begin{synopsis}
	\item [mlpg] [ --l $L$ ] [ --m $M$ ] 
		[--d ($fn$ $|$ $d_0$ [$d_1$ $\dots$]) ]
		[--r $N_R$ $W_1$ [$W_2$] ]
	\item [\ ~~~~] [ --i $I$ ] [ --s $S$ ] [ {\em infile} ] 
\end{synopsis}

\begin{qsection}{DESCRIPTION}
	{\em mlpg} calculates the maximum likelihood parameters 
	from the means and diagonal covariances of Gaussian distributions 
	from {\em infile} (or standard input),
	sending the result to standard output.
	The input format is
 \begin{align}
	& \dots, \mu_t(0), \dots, \mu_t(M),
		\mu^{(1)}_t(0), \dots, \mu^{(1)}_t(M),
		\dots, \mu^{(N)}_t(M),
		\notag\\
	& \qquad \sigma^2_t(0), \dots, \sigma^2_t(M),
		{\sigma^{(1)}}^2_t(0), \dots, {\sigma^{(1)}}^2_t(M),
		\dots, {\sigma^{(N)}}^2_t(M),
		\dots \notag
 \end{align}

	Input and output data are in float format.

        The speech parameter vector $\bo_t$ for
        every frame $t$ is composed of the static feature
        vector $\bc_t$, where 
 \begin{displaymath}
	\bc_t = [c_t(0), c_t(1), \dots, c_t(M)]^\top
 \end{displaymath}
	and the dynamic feature vector 
	\ $\Delta^{(1)}\bc_t, \dots, \Delta^{(N)}\bc_t$,
	that is,
 \begin{displaymath}
	\bo_t = [\bc_t', \Delta^{(1)}\bc_t', \dots, \Delta^{(N)}\bc_t']^\top.
 \end{displaymath}
        The dynamic feature vector $\Delta^{(n)}\bc_t$ is
        obtained from the static feature vector as follows.
 \begin{displaymath}
	\Delta^{(n)}\bc_t 
	= \sum_{\tau=-L^{(n)}}^{L^{(n)}} w^{(n)}(\tau)\bc_{t+\tau}
 \end{displaymath}
        where $n$ is the order of dynamic feature vector, for
        example, when we evaluate $\Delta^2$ parameter, n=2.
        The mlpg command reads probability density functions
	sequence
\begin{displaymath}
	\left((\bmu_1, \bSigma_1), (\bmu_2, \bSigma_2), \dots, (\bmu_T, \bSigma_T)
	\right),
\end{displaymath}
where
 \begin{align}
	\bmu_t 
	& =  \left[\bmu^{\prime(0)}_t, \bmu^{\prime(1)}_t, 
		\dots, \bmu^{\prime(N)}_t\right]^\top
		\notag \\
	\bSigma_t 
	& = \diag \left[ \bSigma^{(0)}_t, \bSigma^{(1)}_t, \dots, \bSigma^{(1)}_t \right]
		\notag
 \end{align}
	and evaluates the maximum likelihood parameter sequence
	\ $(\bo_1, \bo_2, \dots, \bo_T)$,
	\ and sends the static feature vector sequence $\bc_t$,
        that is
	\ $(\bc_1, \bc_2, \dots, \bc_T)$,
	to the output.
	In the example above,
	$\bmu^{(0)}, \bSigma^{(0)}$ represent the static feature vector
        mean and covariance matrix, respectively,
        and $\bmu^{(n)}, \bSigma^{(n)}$ represent the $n$-th order dynamic
        feature vector mean and covariance matrix, respectively.
\end{qsection}

\begin{options}
	\argm{m}{M}{order of vector}{25}
	\argm{l}{L}{length of vector }{$M+1$}
	\argm{d}{(fn~|~d_0~[d_1~\dots])}{$fn$ is 
                the file name of the parameters $w^{(n)}(\tau)$
               	used when evaluating the dynamic feature vector.
                It is assume that the number of coefficients
                to the left and to the right have the same length,
                if this is not true than zeros are added to the
                short side.
                For example, if the coefficients are
	 \begin{displaymath}
		w(-1), w(0), w(1), w(2), w(3)
	 \end{displaymath}
		then zeros are added to the left as follows.
	 \begin{displaymath}
		0, 0, w(-1), w(0), w(1), w(2), w(3)
	 \end{displaymath}
		Instead of entering the filename $fn$,
                the coefficients(which compose the file $fn$)
                can be directly inputed in the command line.
                When the order of the dynamic feature vector is higher
                then one, then the sets of coefficients can be inputed
                one after the other as shown on the last example below.
		this option can not be used with the --r option.}{N/A}
	\argm{r}{N_R~W_1~[W_2]}{
		This option is used when $N_R$-th order dynamic parameters
                are used and the weighting coefficients $w^{(n)}(\tau)$
                are evaluated by regression.
                $N_R$ can be made equal to 1 or 2.
		The variables $W_1$ and $W_2$ represent the
                widths of the first and second order regression
                coefficients, respectively.
		The first order regression coefficients for
                $\Delta\bc_t$ at frame $t$ are evaluated as follows.
	 \begin{displaymath}
		\Delta\bc_t
		= \frac{\sum_{\tau=-W_1}^{W_1}\tau \bc_{t+\tau}}%
			{\sum_{\tau=-W_1}^{W_1}\tau^2}
	 \end{displaymath}
		For the second order regression coefficients,
		$a_2 = \sum_{\tau=-W_2}^{W_2} \tau^4$,
		$a_1 = \sum_{\tau=-W_2}^{W_2} \tau^2$,
		$a_0 = \sum_{\tau=-W_2}^{W_2} 1$
                and 
	 \begin{displaymath}
		\Delta^2\bc_t
		= \frac{\sum_{\tau=-W_2}^{W_2}
				(a_0\tau^2 - a_1) \bc_{t+\tau}}
			{2(a_2a_0-a_1^2)}
	 \end{displaymath}
		this option can not be used with the --d option.}{N/A}
	\argm{i}{I}{type of input PDFs\\
	 \begin{displaymath}
             \begin{matrix}
             I = 0 & \bmu,        & \bSigma \\
             I = 1 & \bmu,        & \bSigma^{-1} \\
             I = 2 & \bmu\bSigma^{-1}, & \bSigma^{-1} \\
             \end{matrix}
     \end{displaymath}
		}{0}
	\argm{s}{S}{range of influenced frames}{30}
\end{options}

\begin{qsection}{EXAMPLE}
	In the example below,
        the number of parameters is 15, the width of
        the window for first or second order dynamic feature
        evaluation is 1, and the parameter sequence is
        evaluated from the probability density function.
 \begin{quote}
	\verb!mlpg -m 15 -r 2 1 1 data.pdf > data.par!
 \end{quote}
	or
 \begin{quote}
	\verb!echo "-0.5 0 0.5" | x2x +af > delta! \\
	\verb!echo "0.25 -0.5 0.25" | x2x +af > accel! \\
	\verb!mlpg -m 15 -d delta -d accel data.pdf > data.par!
 \end{quote}
\end{qsection}

%\begin{qsection}{SEE ALSO}
%\end{qsection}
