% ----------------------------------------------------------------- %
%             The Speech Signal Processing Toolkit (SPTK)           %
%             developed by SPTK Working Group                       %
%             http://sp-tk.sourceforge.net/                         %
% ----------------------------------------------------------------- %
%                                                                   %
%  Copyright (c) 1984-2007  Tokyo Institute of Technology           %
%                           Interdisciplinary Graduate School of    %
%                           Science and Engineering                 %
%                                                                   %
%                1996-2010  Nagoya Institute of Technology          %
%                           Department of Computer Science          %
%                                                                   %
% All rights reserved.                                              %
%                                                                   %
% Redistribution and use in source and binary forms, with or        %
% without modification, are permitted provided that the following   %
% conditions are met:                                               %
%                                                                   %
% - Redistributions of source code must retain the above copyright  %
%   notice, this list of conditions and the following disclaimer.   %
% - Redistributions in binary form must reproduce the above         %
%   copyright notice, this list of conditions and the following     %
%   disclaimer in the documentation and/or other materials provided %
%   with the distribution.                                          %
% - Neither the name of the SPTK working group nor the names of its %
%   contributors may be used to endorse or promote products derived %
%   from this software without specific prior written permission.   %
%                                                                   %
% THIS SOFTWARE IS PROVIDED BY THE COPYRIGHT HOLDERS AND            %
% CONTRIBUTORS "AS IS" AND ANY EXPRESS OR IMPLIED WARRANTIES,       %
% INCLUDING, BUT NOT LIMITED TO, THE IMPLIED WARRANTIES OF          %
% MERCHANTABILITY AND FITNESS FOR A PARTICULAR PURPOSE ARE          %
% DISCLAIMED. IN NO EVENT SHALL THE COPYRIGHT OWNER OR CONTRIBUTORS %
% BE LIABLE FOR ANY DIRECT, INDIRECT, INCIDENTAL, SPECIAL,          %
% EXEMPLARY, OR CONSEQUENTIAL DAMAGES (INCLUDING, BUT NOT LIMITED   %
% TO, PROCUREMENT OF SUBSTITUTE GOODS OR SERVICES; LOSS OF USE,     %
% DATA, OR PROFITS; OR BUSINESS INTERRUPTION) HOWEVER CAUSED AND ON %
% ANY THEORY OF LIABILITY, WHETHER IN CONTRACT, STRICT LIABILITY,   %
% OR TORT (INCLUDING NEGLIGENCE OR OTHERWISE) ARISING IN ANY WAY    %
% OUT OF THE USE OF THIS SOFTWARE, EVEN IF ADVISED OF THE           %
% POSSIBILITY OF SUCH DAMAGE.                                       %
% ----------------------------------------------------------------- %
\hypertarget{swab}{}
\name{swab}{swap bytes}{data operation}

\begin{synopsis}
\item [swab] [ --S $S_1$ ] [ --s $S_2$ ] [ --E $E_1$ ] [ --e $E_2$ ] 
 	    [ +$type$ ] [ {\em infile} ] 
\end{synopsis}

\begin{qsection}{DESCRIPTION}
{\em swab} changes the byte order 
(from big-endian to little-endian or vice versa) 
of the input data from {\em infile} (or standard input), 
sending the result to standard output.

The range of input data that is changed can be restricted 
with the --S, --E or --s, --e options.

The +$type$ option specifies the input and output data formats.
\end{qsection}

\begin{options}
	\argm{S}{S_1}{start address}{0}
	\argm{s}{S_2}{start offset number}{0}
	\argm{E}{E_1}{end address}{EOF}
	\argm{e}{E_2}{end offset number}{0}
	\argp{type}{Input and output data format\\
		\begin{tabular}{llcll} \\[-1ex]
         s & short (2 bytes) & \quad &
                     S & unsigned short (2 bytes) \\
         i3 & int (3 bytes) & \quad &
                     I3 & unsigned int (3 bytes) \\
         i & int (4 bytes) & \quad &
                     I & unsigned int (4 bytes) \\
         l & long (4 bytes) & \quad &
                     L & unsigned long (4 bytes) \\
         le & long long (8 bytes) & \quad &
                     LE & unsigned long long (8 bytes) \\
         f & float (4 bytes) & \quad &
                     d & double (8 bytes) \\
		\end{tabular}\\\hspace*{\fill}}{s}
\end{options}

\begin{qsection}{EXAMPLE}
In the example below the byte order of the file {\em data.f} in
float format is changed and written to {\em data.swab}:
\begin{quote}
 \verb!swab +f data.f > data.swab!
\end{quote} 
\end{qsection}

%\begin{qsection}{SEE ALSO}
% 
%\end{qsection}
