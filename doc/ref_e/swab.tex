% ----------------------------------------------------------------
%       Speech Signal Processing Toolkit (SPTK): version 3.0
%                      SPTK Working Group
% 
%                Department of Computer Science
%                Nagoya Institute of Technology
%                             and
%   Interdisciplinary Graduate School of Science and Engineering
%                Tokyo Institute of Technology
%                   Copyright (c) 1984-2000
%                     All Rights Reserved.
% 
% Permission is hereby granted, free of charge, to use and
% distribute this software and its documentation without
% restriction, including without limitation the rights to use,
% copy, modify, merge, publish, distribute, sublicense, and/or
% sell copies of this work, and to permit persons to whom this
% work is furnished to do so, subject to the following conditions:
% 
%   1. The code must retain the above copyright notice, this list
%      of conditions and the following disclaimer.
% 
%   2. Any modifications must be clearly marked as such.
%                                                                        
% NAGOYA INSTITUTE OF TECHNOLOGY, TOKYO INSITITUTE OF TECHNOLOGY,
% SPTK WORKING GROUP, AND THE CONTRIBUTORS TO THIS WORK DISCLAIM
% ALL WARRANTIES WITH REGARD TO THIS SOFTWARE, INCLUDING ALL
% IMPLIED WARRANTIES OF MERCHANTABILITY AND FITNESS, IN NO EVENT
% SHALL NAGOYA INSTITUTE OF TECHNOLOGY, TOKYO INSITITUTE OF
% TECHNOLOGY, SPTK WORKING GROUP, NOR THE CONTRIBUTORS BE LIABLE
% FOR ANY SPECIAL, INDIRECT OR CONSEQUENTIAL DAMAGES OR ANY
% DAMAGES WHATSOEVER RESULTING FROM LOSS OF USE, DATA OR PROFITS,
% WHETHER IN AN ACTION OF CONTRACT, NEGLIGENCE OR OTHER TORTIOUS
% ACTION, ARISING OUT OF OR IN CONNECTION WITH THE USE OR
% PERFORMANCE OF THIS SOFTWARE.
% ----------------------------------------------------------------
%
\name{swab}{swap bytes}{data operation}

\begin{synopsis}
\item [swab] [ --S $S_1$ ] [ --s $S_2$ ] [ --E $E_1$ ] [ --e $E_2$ ] 
 	    [ +$type$ ] [ {\em infile} ] 
\end{synopsis}

\begin{qsection}{DESCRIPTION}
The command {\em swab} transforms the byte order from
little endian to big endian, and vice versa(byte swap).
If input file is not assigned, then data is read from
the standard input.
\par
The range where the swap is undertaken can be assigned by
the option {\bf --S, --E} or {\bf --s, --e}.
\par
Input and output data formats are assigned by +$type$.
\end{qsection}

\begin{options}
	\argm{S}{S_1}{start address}{0}
	\argm{s}{S_2}{start offset number}{0}
	\argm{E}{E_1}{end address}{EOF}
	\argm{e}{E_2}{end offset number}{0}
	\argp{type}{Input and output data format\\
		\begin{tabular}{llcll} \\[-1zh]
			s & short (2bytes) & \quad &
			l & long (4bytes) \\
			f & float (4bytes) & \quad &
			d & double (8bytes) \\
		\end{tabular}\\\hspace*{\fill}}{s}
\end{options}

\begin{qsection}{EXAMPLE}
In the example below the byte order of the file {\em data.f} in
float format is changed and written to {\em data.swab}:
\begin{quote}
 \verb!swab +f data.f > data.swab!
\end{quote} 
\end{qsection}

%\begin{qsection}{SEE ALSO}
% 
%\end{qsection}
