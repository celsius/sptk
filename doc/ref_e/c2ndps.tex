% ----------------------------------------------------------------- %
%             The Speech Signal Processing Toolkit (SPTK)           %
%             developed by SPTK Working Group                       %
%             http://sp-tk.sourceforge.net/                         %
% ----------------------------------------------------------------- %
%                                                                   %
%  Copyright (c) 1984-2007  Tokyo Institute of Technology           %
%                           Interdisciplinary Graduate School of    %
%                           Science and Engineering                 %
%                                                                   %
%                1996-2013  Nagoya Institute of Technology          %
%                           Department of Computer Science          %
%                                                                   %
% All rights reserved.                                              %
%                                                                   %
% Redistribution and use in source and binary forms, with or        %
% without modification, are permitted provided that the following   %
% conditions are met:                                               %
%                                                                   %
% - Redistributions of source code must retain the above copyright  %
%   notice, this list of conditions and the following disclaimer.   %
% - Redistributions in binary form must reproduce the above         %
%   copyright notice, this list of conditions and the following     %
%   disclaimer in the documentation and/or other materials provided %
%   with the distribution.                                          %
% - Neither the name of the SPTK working group nor the names of its %
%   contributors may be used to endorse or promote products derived %
%   from this software without specific prior written permission.   %
%                                                                   %
% THIS SOFTWARE IS PROVIDED BY THE COPYRIGHT HOLDERS AND            %
% CONTRIBUTORS "AS IS" AND ANY EXPRESS OR IMPLIED WARRANTIES,       %
% INCLUDING, BUT NOT LIMITED TO, THE IMPLIED WARRANTIES OF          %
% MERCHANTABILITY AND FITNESS FOR A PARTICULAR PURPOSE ARE          %
% DISCLAIMED. IN NO EVENT SHALL THE COPYRIGHT OWNER OR CONTRIBUTORS %
% BE LIABLE FOR ANY DIRECT, INDIRECT, INCIDENTAL, SPECIAL,          %
% EXEMPLARY, OR CONSEQUENTIAL DAMAGES (INCLUDING, BUT NOT LIMITED   %
% TO, PROCUREMENT OF SUBSTITUTE GOODS OR SERVICES; LOSS OF USE,     %
% DATA, OR PROFITS; OR BUSINESS INTERRUPTION) HOWEVER CAUSED AND ON %
% ANY THEORY OF LIABILITY, WHETHER IN CONTRACT, STRICT LIABILITY,   %
% OR TORT (INCLUDING NEGLIGENCE OR OTHERWISE) ARISING IN ANY WAY    %
% OUT OF THE USE OF THIS SOFTWARE, EVEN IF ADVISED OF THE           %
% POSSIBILITY OF SUCH DAMAGE.                                       %
% ----------------------------------------------------------------- %
\hypertarget{c2ndps}{}
\name[ref:NDPS-SignalProcessing]
{c2ndps}{cepstrum to Negative Derivative of Phase Spectrum (NDPS)}%
{speech parameter transformation}

\begin{synopsis}
 \item[c2ndps] [ --l $L$ ] [ --m $M$ ] [ --p ] [ --z ] [ {\em infile} ]
\end{synopsis}

\begin{qsection}{DESCRIPTION}
{\em c2ndps} calculates the Negative Derivative of Phase Spectrum (NDPS)
from the real mixed phase cepstrum coefficients in the {\em infile} (or standard input),
sending the result to standard output.
For example, if the input sequence is
\begin{displaymath}
   c(0),c(1),c(2),\dots,c(M)
\end{displaymath}
then the log spectrum is calculated as
\begin{displaymath}
\ln S(\omega) = \sum^{M}_{m=0}c(m)\mathrm{e}^{-j\omega m}.
\end{displaymath}
$\ln S(\omega)$ can be decomposed into the real part and imaginary part, that is, the magnitude and phase spectrum as
\begin{displaymath}
 \ln |S(\omega)| + j\arg S(\omega) = \sum^{M}_{m=0}c(m)\mathrm{e}^{-j\omega m}.
\end{displaymath}
 Then, partially differentiate the both sides of the above equation by $\omega$, one can obtain
\begin{displaymath}
 \frac{\partial}{\partial \omega}\ln |S(\omega)| + j\frac{\partial}{\partial \omega}\arg S(\omega) = -j \sum^{M}_{m=0}mc(m)\mathrm{e}^{-j\omega m}.
\end{displaymath}
 Finally, from the imaginary part of the above equation, Negative Derivative of Phase Spectrum (NDPS) can be obtained as
\begin{displaymath}
 -\frac{\partial}{\partial \omega}\arg S(\omega) = \sum^{M}_{m=0}mc(m)\cos \omega m.
\end{displaymath}
 From the above derivation, NDPS is also equivalent to the real part of DFT of $mc(m)$:
\begin{displaymath}
 n(k) = Re\left[ \sum^{M}_{m=0}mc(m)\mathrm{e}^{-j\frac{2\pi km}{N}} \right]\hspace{10mm} (k=0,\cdots,N-1).
\end{displaymath}
Both input and output files are in float format. The output file contains the $n(k)$ in the range $k=0,\cdots,N/2$.

Additionally, the -p or -z option can be used to output NDPS as follows.
If the -p option is specified,
\begin{align}
n(k) = \begin{cases} \;\; \displaystyle
	n(k), & n(k) > 0 \\
	\;\;\displaystyle 0 & n(k) < 0
       \end{cases}. \notag
\end{align}
If the -z option is specified,
\begin{align}
n(k) = \begin{cases} \;\; \displaystyle
	0, & n(k) > 0 \\
	\;\;\displaystyle n(k) & n(k) < 0
       \end{cases}. \notag
\end{align}
$n(k)$ doesn't comprehend the c(0).
\end{qsection}

\begin{options}
	\argm{m}{M}{order of cepstrum}{25}
	\argm{l}{L}{FFT length}{256}
        \desc[1ex]{(level 2)}
        \argm{p}{}{extract only pole part}{FALSE}
        \argm{z}{}{extract only zero part}{FALSE}
\end{options}

\begin{qsection}{EXAMPLE}
The output file {\em data.ir} contains the $n(k)$
in the range $k = 0,\cdots,1024$ obtained from the 30-th order cepstral
coefficients file {\em data.cep}, in float format:
 \begin{quote}
  \verb!c2ndps -l 2048 -m 30 data.cep > data.ndps!
 \end{quote}
\end{qsection}

\begin{qsection}{SEE ALSO}
\hyperlink{mgcep}{mgcep},
\hyperlink{ndps2c}{ndps2c}
\end{qsection}
