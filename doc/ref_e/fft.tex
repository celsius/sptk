\name{fft}{FFT for complex sequence}{signal processing}

\begin{synopsis}
\item[fft] [ --l $L$ ] [ --m $M$] [ --\{ A $|$ R $|$ I $|$ P \} ] 
	   [ {\em infile} ] 
\end{synopsis}

\begin{qsection}{DESCRIPTION}
This command reads a sequence of complex numbers from {\em infile}
and undertakes its DFT through the FFT algorithm.
When the inputfile is not assigned, data is read from the starndard
input.
Input and output data are in float format, and the order of input
 and output data are as follows.
\[
 \begin{array}{lll}
\mbox{Input sequence} & \overbrace{\framebox[4.5cm]{real part}}^{M+1} &
	   \overbrace{\framebox[4.5cm]{imaginary part}}^{M+1} \\
		& \makebox[4.5cm]{0\hfill $M$} &
		\makebox[4.5cm]{0\hfill $M$}
\end{array}
\]
\[
\begin{array}{lll}
\mbox{Output sequence} & \overbrace{\framebox[4.5cm]{real part}}^{L} &
	   \overbrace{\framebox[4.5cm]{imaginary part}}^{L} \\
		& \makebox[4.5cm]{0\hfill $L-1$} &
		\makebox[4.5cm]{0\hfill $L-1$}
\end{array}
\]
\end{qsection}

\begin{options}
	\argm{l}{L}{FFT size  power of 2}{256}
	\argm{m}{M}{order of sequence}{L-1}
	\argm{A}{}{amplitude}{FALSE}
	\argm{R}{}{real part}{FALSE}
	\argm{I}{}{imaginary part}{FALSE}
	\argm{P}{}{output power spectrum}{FALSE}
\end{options}

\begin{qsection}{EXAMPLE}
This example reads a sequence of complex number in float format from
{\em data.f} file (real part with 256 points and imaginary part with
256 points), evaluates its DFT and outputs it to {\em data.dft} file:
\begin{quote}
  \verb!fft data.f -l 256 -A > data.dft!
\end{quote}
\end{qsection}

\begin{qsection}{SEE ALSO}
  fftr, spec, phase
\end{qsection}
