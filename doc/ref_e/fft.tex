%  ---------------------------------------------------------------  %
%            Speech Signal Processing Toolkit (SPTK)                %
%                      SPTK Working Group                           %
%                                                                   %
%                  Department of Computer Science                   %
%                  Nagoya Institute of Technology                   %
%                               and                                 %
%   Interdisciplinary Graduate School of Science and Engineering    %
%                  Tokyo Institute of Technology                    %
%                                                                   %
%                     Copyright (c) 1984-2007                       %
%                       All Rights Reserved.                        %
%                                                                   %
%  Permission is hereby granted, free of charge, to use and         %
%  distribute this software and its documentation without           %
%  restriction, including without limitation the rights to use,     %
%  copy, modify, merge, publish, distribute, sublicense, and/or     %
%  sell copies of this work, and to permit persons to whom this     %
%  work is furnished to do so, subject to the following conditions: %
%                                                                   %
%    1. The source code must retain the above copyright notice,     %
%       this list of conditions and the following disclaimer.       %
%                                                                   %
%    2. Any modifications to the source code must be clearly        %
%       marked as such.                                             %
%                                                                   %
%    3. Redistributions in binary form must reproduce the above     %
%       copyright notice, this list of conditions and the           %
%       following disclaimer in the documentation and/or other      %
%       materials provided with the distribution.  Otherwise, one   %
%       must contact the SPTK working group.                        %
%                                                                   %
%  NAGOYA INSTITUTE OF TECHNOLOGY, TOKYO INSTITUTE OF TECHNOLOGY,   %
%  SPTK WORKING GROUP, AND THE CONTRIBUTORS TO THIS WORK DISCLAIM   %
%  ALL WARRANTIES WITH REGARD TO THIS SOFTWARE, INCLUDING ALL       %
%  IMPLIED WARRANTIES OF MERCHANTABILITY AND FITNESS, IN NO EVENT   %
%  SHALL NAGOYA INSTITUTE OF TECHNOLOGY, TOKYO INSTITUTE OF         %
%  TECHNOLOGY, SPTK WORKING GROUP, NOR THE CONTRIBUTORS BE LIABLE   %
%  FOR ANY SPECIAL, INDIRECT OR CONSEQUENTIAL DAMAGES OR ANY        %
%  DAMAGES WHATSOEVER RESULTING FROM LOSS OF USE, DATA OR PROFITS,  %
%  WHETHER IN AN ACTION OF CONTRACT, NEGLIGENCE OR OTHER TORTUOUS   %
%  ACTION, ARISING OUT OF OR IN CONNECTION WITH THE USE OR          %
%  PERFORMANCE OF THIS SOFTWARE.                                    %
%                                                                   %
%  ---------------------------------------------------------------  %
%
\hypertarget{fft}{}
\name{fft}{FFT for complex sequence}{signal processing}

\begin{synopsis}
\item[fft] [ --l $L$ ] [ --m $M$] [ --\{ A $|$ R $|$ I $|$ P \} ] 
	   [ {\em infile} ] 
\end{synopsis}

\begin{qsection}{DESCRIPTION}
{\em fft} uses the Fast Fourier Transform (FFT) algorithm 
to calculate the Discrete Fourier Transform (DFT) 
of complex-valued input data from {\em infile} (or standard input), 
sending the result to standard output. 
The input and output data is in float format, arranged as follows.
\[
 \begin{array}{lll}
\mbox{Input sequence} & \overbrace{\framebox[4.5cm]{real part}}^{M+1} &
	   \overbrace{\framebox[4.5cm]{imaginary part}}^{M+1} \\
		& \makebox[4.5cm]{0\hfill $M$} &
		\makebox[4.5cm]{0\hfill $M$}
\end{array}
\]
\[
\begin{array}{lll}
\mbox{Output sequence} & \overbrace{\framebox[4.5cm]{real part}}^{L} &
	   \overbrace{\framebox[4.5cm]{imaginary part}}^{L} \\
		& \makebox[4.5cm]{0\hfill $L-1$} &
		\makebox[4.5cm]{0\hfill $L-1$}
\end{array}
\]
\end{qsection}

\begin{options}
	\argm{l}{L}{FFT size  power of 2}{256}
	\argm{m}{M}{order of sequence}{L-1}
	\argm{A}{}{amplitude}{FALSE}
	\argm{R}{}{real part}{FALSE}
	\argm{I}{}{imaginary part}{FALSE}
	\argm{P}{}{output power spectrum}{FALSE}
\end{options}

\begin{qsection}{EXAMPLE}
This example reads a sequence of complex numbers in float format from
{\em data.f} file (real part with 256 points and imaginary part with
256 points), evaluates its DFT and outputs it to {\em data.dft} file:
\begin{quote}
  \verb!fft data.f -l 256 -A > data.dft!
\end{quote}
\end{qsection}

\begin{qsection}{SEE ALSO}
\hyperlink{fftr}{fftr},
\hyperlink{spec}{spec},
\hyperlink{phase}{phase}
\end{qsection}
