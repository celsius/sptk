% ----------------------------------------------------------------- %
%             The Speech Signal Processing Toolkit (SPTK)           %
%             developed by SPTK Working Group                       %
%             http://sp-tk.sourceforge.net/                         %
% ----------------------------------------------------------------- %
%                                                                   %
%  Copyright (c) 1984-2007  Tokyo Institute of Technology           %
%                           Interdisciplinary Graduate School of    %
%                           Science and Engineering                 %
%                                                                   %
%                1996-2014  Nagoya Institute of Technology          %
%                           Department of Computer Science          %
%                                                                   %
% All rights reserved.                                              %
%                                                                   %
% Redistribution and use in source and binary forms, with or        %
% without modification, are permitted provided that the following   %
% conditions are met:                                               %
%                                                                   %
% - Redistributions of source code must retain the above copyright  %
%   notice, this list of conditions and the following disclaimer.   %
% - Redistributions in binary form must reproduce the above         %
%   copyright notice, this list of conditions and the following     %
%   disclaimer in the documentation and/or other materials provided %
%   with the distribution.                                          %
% - Neither the name of the SPTK working group nor the names of its %
%   contributors may be used to endorse or promote products derived %
%   from this software without specific prior written permission.   %
%                                                                   %
% THIS SOFTWARE IS PROVIDED BY THE COPYRIGHT HOLDERS AND            %
% CONTRIBUTORS "AS IS" AND ANY EXPRESS OR IMPLIED WARRANTIES,       %
% INCLUDING, BUT NOT LIMITED TO, THE IMPLIED WARRANTIES OF          %
% MERCHANTABILITY AND FITNESS FOR A PARTICULAR PURPOSE ARE          %
% DISCLAIMED. IN NO EVENT SHALL THE COPYRIGHT OWNER OR CONTRIBUTORS %
% BE LIABLE FOR ANY DIRECT, INDIRECT, INCIDENTAL, SPECIAL,          %
% EXEMPLARY, OR CONSEQUENTIAL DAMAGES (INCLUDING, BUT NOT LIMITED   %
% TO, PROCUREMENT OF SUBSTITUTE GOODS OR SERVICES; LOSS OF USE,     %
% DATA, OR PROFITS; OR BUSINESS INTERRUPTION) HOWEVER CAUSED AND ON %
% ANY THEORY OF LIABILITY, WHETHER IN CONTRACT, STRICT LIABILITY,   %
% OR TORT (INCLUDING NEGLIGENCE OR OTHERWISE) ARISING IN ANY WAY    %
% OUT OF THE USE OF THIS SOFTWARE, EVEN IF ADVISED OF THE           %
% POSSIBILITY OF SUCH DAMAGE.                                       %
% ----------------------------------------------------------------- %
\hypertarget{gnorm}{}
\name{gnorm}{gain normalization}{speech parameter transformation}

\begin{synopsis}
\item [gnorm] [ --m $M$ ] [ --g $G$ ] [ --c $C$ ] [ {\em infile} ]
\end{synopsis}

\begin{qsection}{DESCRIPTION}
{\em gnorm} normalizes generalized cepstral coefficients $c_\gamma(m)$ 
from {\em infile} (or standard input), 
sending the normalized generalized cepstral coefficients to standard output.

Both input and output files are in float format.

The normalized generalized cepstral coefficients $c_\gamma'(m)$
can be written as
\begin{displaymath}
c_\gamma'(m) = \frac{c_\gamma(m)}{1+\gamma c_\gamma(0)}, \qquad m>0
\end{displaymath}
Also, the gain $K = c_\gamma'(0)$ is given by:
\begin{displaymath}
K = \begin{cases} \;\;\displaystyle
          \left(\frac{1}{1+\gamma c_\gamma(0)}\right)^{1/\gamma},
                & 0<|\gamma|\leq 1 \\ \;\;\displaystyle
          \exp c_\gamma(0),  & \gamma=0
        \end{cases}
\end{displaymath}
\end{qsection}

\begin{options}
        \argm{m}{M}{order of generalized cepstrum}{25}
        \argm{g}{G}{power parameter $\gamma$ of generalized cepstrum,\\
                         $\gamma=G$}{0}
        \argm{c}{C}{power parameter $\gamma$ of generalized cepstrum,\\
                        $\gamma =-1 / $(int)$ C$\\
                        $C$ must be $C \geq 1$}{}
\end{options}

\begin{qsection}{EXAMPLE}
In this example, generalized cepstral coefficients in float format
are read from file {\em data.gcep} $(M=15, \gamma=-0.5)$,
normalized and output to {\em data.ngcep}:
\begin{quote}
 \verb!gnorm -m 15 -c 2 < data.gcep > data.ngcep!
\end{quote} 
\end{qsection}

\begin{qsection}{NOTICE}
Value of $C$ must be $C \geq 1$
\end{qsection}

\begin{qsection}{SEE ALSO}
\hyperlink{ignorm}{ignorm},
\hyperlink{gcep}{gcep},
\hyperlink{mgcep}{mgcep},
\hyperlink{gc2gc}{gc2gc},
\hyperlink{mgc2mgc}{mgc2mgc},
\hyperlink{freqt}{freqt}
\end{qsection}
