\name{gnorm}{gain normalization}{speech parameter transformation}

\begin{synopsis}
\item [gnorm] [ --m $M$ ] [ --g $G$ ] [ {\em infile} ]
\end{synopsis}

\begin{qsection}{DESCRIPTION}
This command reads generalized cepstrum coefficient $c_\gamma(m)$,
normalizes them, and sends the normalized generalized
cepstrum coefficients to the standard output.
\par
Input and output data are in float format.
\par
The normalized generalized cepstrum coefficients $c_\gamma'(m)$
can be written as
\begin{displaymath}
c_\gamma'(m) = \frac{c_\gamma(m)}{1+\gamma c_\gamma(0)}, ~~~m>0
\end{displaymath}
Also, the gain $K = c_\gamma'(0)$ is
\begin{displaymath}
K = \left\{
	\begin{array}{ll} \displaystyle
	  \left(\frac{1}{1+\gamma c_\gamma(0)}\right)^{1/\gamma},
		& 0<|\gamma|\leq 1 \\ \displaystyle
	  \exp c_\gamma(0),  & \gamma=0
	\end{array} \right.
\end{displaymath}
\end{qsection}

\begin{options}
	\argm{m}{M}{order of generalized cepstrum}{25}
	\argm{g}{G}{power parameter $\gamma$ of generalized cepstrum,\\
		    if $G>1.0$ then $\gamma=-1/G$}{0}
\end{options}

\begin{qsection}{EXAMPLE}
In this example, generalized cepstrum coefficients in float format
are read from file {\em data.gcep} )$(M=15, \gamma=-0.5)$,
normalized and outputed to {\em data.ngcep}:
\begin{quote}
 \verb!gnorm -m 15 -g 2 < data.gcep > data.ngcep!
\end{quote} 
\end{qsection}

\begin{qsection}{SEE ALSO}
 ignorm, gcep, mgcep, gc2gc, mgc2mgc, freqt
\end{qsection}
