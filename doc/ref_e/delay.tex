%  ---------------------------------------------------------------  %
%            Speech Signal Processing Toolkit (SPTK)                %
%                      SPTK Working Group                           %
%                                                                   %
%                  Department of Computer Science                   %
%                  Nagoya Institute of Technology                   %
%                               and                                 %
%   Interdisciplinary Graduate School of Science and Engineering    %
%                  Tokyo Institute of Technology                    %
%                                                                   %
%                     Copyright (c) 1984-2007                       %
%                       All Rights Reserved.                        %
%                                                                   %
%  Permission is hereby granted, free of charge, to use and         %
%  distribute this software and its documentation without           %
%  restriction, including without limitation the rights to use,     %
%  copy, modify, merge, publish, distribute, sublicense, and/or     %
%  sell copies of this work, and to permit persons to whom this     %
%  work is furnished to do so, subject to the following conditions: %
%                                                                   %
%    1. The source code must retain the above copyright notice,     %
%       this list of conditions and the following disclaimer.       %
%                                                                   %
%    2. Any modifications to the source code must be clearly        %
%       marked as such.                                             %
%                                                                   %
%    3. Redistributions in binary form must reproduce the above     %
%       copyright notice, this list of conditions and the           %
%       following disclaimer in the documentation and/or other      %
%       materials provided with the distribution.  Otherwise, one   %
%       must contact the SPTK working group.                        %
%                                                                   %
%  NAGOYA INSTITUTE OF TECHNOLOGY, TOKYO INSTITUTE OF TECHNOLOGY,   %
%  SPTK WORKING GROUP, AND THE CONTRIBUTORS TO THIS WORK DISCLAIM   %
%  ALL WARRANTIES WITH REGARD TO THIS SOFTWARE, INCLUDING ALL       %
%  IMPLIED WARRANTIES OF MERCHANTABILITY AND FITNESS, IN NO EVENT   %
%  SHALL NAGOYA INSTITUTE OF TECHNOLOGY, TOKYO INSTITUTE OF         %
%  TECHNOLOGY, SPTK WORKING GROUP, NOR THE CONTRIBUTORS BE LIABLE   %
%  FOR ANY SPECIAL, INDIRECT OR CONSEQUENTIAL DAMAGES OR ANY        %
%  DAMAGES WHATSOEVER RESULTING FROM LOSS OF USE, DATA OR PROFITS,  %
%  WHETHER IN AN ACTION OF CONTRACT, NEGLIGENCE OR OTHER TORTUOUS   %
%  ACTION, ARISING OUT OF OR IN CONNECTION WITH THE USE OR          %
%  PERFORMANCE OF THIS SOFTWARE.                                    %
%                                                                   %
%  ---------------------------------------------------------------  %
%
\hypertarget{delay}{}
\name{delay}{delay sequence}{signal processing}

\begin{synopsis}
\item [delay] [ --s $S$ ] [ --f ] [ {\em infile} ] 
\end{synopsis}

\begin{qsection}{DESCRIPTION}
{\em delay} delays the data from {\em infile} (or standard input) 
by inserting a specified number of zero samples at the beginning, 
sending the result to standard output.
 For example, if we want to delay the following data
\begin{displaymath}
   x(0), x(1), \ldots , x(T)
\end{displaymath}
as
\begin{displaymath}
   \underbrace{0, \dots , 0}_{S}, x(0), x(1), \dots , x(T).
\end{displaymath}
We only need to set the ``--s'' option to $S$
\begin{displaymath}
   \underbrace{0, \dots , 0}_{S}, x(0), x(1), \dots , x(T-S).
\end{displaymath}
\par
The format of input and output is float.
\end{qsection}

\begin{options}
	\argm{s}{S}{start sample}{0}
	\argm{f}{}{keep file length}{False}
\end{options}

\begin{qsection}{EXAMPLE}
If we have the following data in the input {\em data.f} file
\begin{displaymath}
 1.0, 2.0, 3.0, 4.0, 5.0, 6.0
\end{displaymath}
and we use the command below
\begin{quote}
 \verb!delay -s 3 < data.f > data.delay!
\end{quote}
then the output file {\em data.delay} is 
\begin{displaymath}
 0.0, 0.0, 0.0, 1.0, 2.0, 3.0, 4.0, 5.0, 6.0
\end{displaymath}
As another example, if we want to keep the same size of the input file,
we can use the following command,
\begin{quote}
\verb!delay -s 3 -f < data.f > data.delay!
\end{quote}
and the output {\em data.delay} is
\begin{displaymath}
 0.0, 0.0, 0.0, 1.0, 2.0, 3.0
\end{displaymath}
\end{qsection}

%\begin{qsection}{SEE ALSO}
%none
%\end{qsection}
