% ----------------------------------------------------------------
%       Speech Signal Processing Toolkit (SPTK): version 3.0
%                      SPTK Working Group
% 
%                Department of Computer Science
%                Nagoya Institute of Technology
%                             and
%   Interdisciplinary Graduate School of Science and Engineering
%                Tokyo Institute of Technology
%                   Copyright (c) 1984-2000
%                     All Rights Reserved.
% 
% Permission is hereby granted, free of charge, to use and
% distribute this software and its documentation without
% restriction, including without limitation the rights to use,
% copy, modify, merge, publish, distribute, sublicense, and/or
% sell copies of this work, and to permit persons to whom this
% work is furnished to do so, subject to the following conditions:
% 
%   1. The code must retain the above copyright notice, this list
%      of conditions and the following disclaimer.
% 
%   2. Any modifications must be clearly marked as such.
%                                                                        
% NAGOYA INSTITUTE OF TECHNOLOGY, TOKYO INSITITUTE OF TECHNOLOGY,
% SPTK WORKING GROUP, AND THE CONTRIBUTORS TO THIS WORK DISCLAIM
% ALL WARRANTIES WITH REGARD TO THIS SOFTWARE, INCLUDING ALL
% IMPLIED WARRANTIES OF MERCHANTABILITY AND FITNESS, IN NO EVENT
% SHALL NAGOYA INSTITUTE OF TECHNOLOGY, TOKYO INSITITUTE OF
% TECHNOLOGY, SPTK WORKING GROUP, NOR THE CONTRIBUTORS BE LIABLE
% FOR ANY SPECIAL, INDIRECT OR CONSEQUENTIAL DAMAGES OR ANY
% DAMAGES WHATSOEVER RESULTING FROM LOSS OF USE, DATA OR PROFITS,
% WHETHER IN AN ACTION OF CONTRACT, NEGLIGENCE OR OTHER TORTIOUS
% ACTION, ARISING OUT OF OR IN CONNECTION WITH THE USE OR
% PERFORMANCE OF THIS SOFTWARE.
% ----------------------------------------------------------------
%
\name{delay}{delay sequence}{signal processing}

\begin{synopsis}
\item [delay] [ --s $S$ ] [ --f ] [ {\em infile} ] 
\end{synopsis}

\begin{qsection}{DESCRIPTION}
 This command includes a delay on the signal in the input file.
 For example, if we want to delay the following data
\[ x(0), x(1), \ldots , x(T) \]
as
\[ \underbrace{0, \ldots , 0}_{S}, x(0), x(1), \ldots , x(T). \]
we only need to set the ``--s'' option to $S$
\[ \underbrace{0, \ldots , 0}_{S}, x(0), x(1), \ldots , x(T-S) \]
\par
The format of input and output is float.
\end{qsection}

\begin{options}
	\argm{s}{S}{start sample}{0}
	\argm{f}{}{keep file length}{False}
\end{options}

\begin{qsection}{EXAMPLE}
If we have the following data in the input {\em data.f} file
\begin{displaymath}
 1.0, 2.0, 3.0, 4.0, 5.0, 6.0
\end{displaymath}
and we use the command below
\begin{quote}
 \verb!delay -s 3 < data.f > data.delay!
\end{quote}
then the output file {\em data.delay} is 
\begin{displaymath}
 0.0, 0.0, 0.0, 1.0, 2.0, 3.0, 4.0, 5.0, 6.0
\end{displaymath}
As another example, if we want to keep the same size of the input file,
we can use the following command,
\begin{quote}
\verb!delay -s 3 -f < data.f > data.delay!
\end{quote}
and the output {\em data.delay} is
\begin{displaymath}
 0.0, 0.0, 0.0, 1.0, 2.0, 3.0
\end{displaymath}
\end{qsection}

%\begin{qsection}{SEE ALSO}
%none
%\end{qsection}
