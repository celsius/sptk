% ----------------------------------------------------------------- %
%             The Speech Signal Processing Toolkit (SPTK)           %
%             developed by SPTK Working Group                       %
%             http://sp-tk.sourceforge.net/                         %
% ----------------------------------------------------------------- %
%                                                                   %
%  Copyright (c) 1984-2007  Tokyo Institute of Technology           %
%                           Interdisciplinary Graduate School of    %
%                           Science and Engineering                 %
%                                                                   %
%                1996-2013  Nagoya Institute of Technology          %
%                           Department of Computer Science          %
%                                                                   %
% All rights reserved.                                              %
%                                                                   %
% Redistribution and use in source and binary forms, with or        %
% without modification, are permitted provided that the following   %
% conditions are met:                                               %
%                                                                   %
% - Redistributions of source code must retain the above copyright  %
%   notice, this list of conditions and the following disclaimer.   %
% - Redistributions in binary form must reproduce the above         %
%   copyright notice, this list of conditions and the following     %
%   disclaimer in the documentation and/or other materials provided %
%   with the distribution.                                          %
% - Neither the name of the SPTK working group nor the names of its %
%   contributors may be used to endorse or promote products derived %
%   from this software without specific prior written permission.   %
%                                                                   %
% THIS SOFTWARE IS PROVIDED BY THE COPYRIGHT HOLDERS AND            %
% CONTRIBUTORS "AS IS" AND ANY EXPRESS OR IMPLIED WARRANTIES,       %
% INCLUDING, BUT NOT LIMITED TO, THE IMPLIED WARRANTIES OF          %
% MERCHANTABILITY AND FITNESS FOR A PARTICULAR PURPOSE ARE          %
% DISCLAIMED. IN NO EVENT SHALL THE COPYRIGHT OWNER OR CONTRIBUTORS %
% BE LIABLE FOR ANY DIRECT, INDIRECT, INCIDENTAL, SPECIAL,          %
% EXEMPLARY, OR CONSEQUENTIAL DAMAGES (INCLUDING, BUT NOT LIMITED   %
% TO, PROCUREMENT OF SUBSTITUTE GOODS OR SERVICES; LOSS OF USE,     %
% DATA, OR PROFITS; OR BUSINESS INTERRUPTION) HOWEVER CAUSED AND ON %
% ANY THEORY OF LIABILITY, WHETHER IN CONTRACT, STRICT LIABILITY,   %
% OR TORT (INCLUDING NEGLIGENCE OR OTHERWISE) ARISING IN ANY WAY    %
% OUT OF THE USE OF THIS SOFTWARE, EVEN IF ADVISED OF THE           %
% POSSIBILITY OF SUCH DAMAGE.                                       %
% ----------------------------------------------------------------- %
\hypertarget{delay}{}
\name{delay}{delay sequence}{signal processing}

\begin{synopsis}
\item [delay] [ --s $S$ ] [ --f ] [ {\em infile} ] 
\end{synopsis}

\begin{qsection}{DESCRIPTION}
{\em delay} delays the data in {\em infile} (or standard input) 
by inserting a specified number of zero samples at the beginning, 
and sends the result to standard output.
 For example, if we want to delay the following data
\begin{displaymath}
   x(0), x(1), \ldots , x(T)
\end{displaymath}
as in
\begin{displaymath}
   \underbrace{0, \dots , 0}_{S}, x(0), x(1), \dots , x(T).
\end{displaymath}
We only need to set the ``--s'' option to $S$
\begin{displaymath}
   \underbrace{0, \dots , 0}_{S}, x(0), x(1), \dots , x(T-S).
\end{displaymath}
\par
Both input and output files are in float format.
\end{qsection}

\begin{options}
	\argm{s}{S}{start sample}{0}
	\argm{f}{}{keep file length}{FALSE}
\end{options}

\begin{qsection}{EXAMPLE}
If we have the following data in the input {\em data.f} file
\begin{displaymath}
 1.0, 2.0, 3.0, 4.0, 5.0, 6.0
\end{displaymath}
and we use the command below
\begin{quote}
 \verb!delay -s 3 < data.f > data.delay!
\end{quote}
then the output file {\em data.delay} will be 
\begin{displaymath}
 0.0, 0.0, 0.0, 1.0, 2.0, 3.0, 4.0, 5.0, 6.0
\end{displaymath}
As another example, if we want to keep the same size of the input file,
we can use the following command,
\begin{quote}
\verb!delay -s 3 -f < data.f > data.delay!
\end{quote}
and the output {\em data.delay} will be
\begin{displaymath}
 0.0, 0.0, 0.0, 1.0, 2.0, 3.0
\end{displaymath}
\end{qsection}

%\begin{qsection}{SEE ALSO}
%none
%\end{qsection}
