\name{delay}{delay sequence}{signal processing}

\begin{synopsis}
\item [delay] [ --s $S$ ] [ --f ] [ {\em infile} ] 
\end{synopsis}

\begin{qsection}{DESCRIPTION}
 This command delaies the signal in the input file.
 For example, if we want to delay the following data
\[ x(0), x(1), \ldots , x(T) \]
as
\[ \underbrace{0, \ldots , 0}_{S}, x(0), x(1), \ldots , x(T) \].
we only need to set the ``-f'' option to $S$
\[ \underbrace{0, \ldots , 0}_{S}, x(0), x(1), \ldots , x(T-S) \]
\par
Input and output data is in float format.
\end{qsection}

\begin{options}
	\argm{s}{S}{start sample}{0}
	\argm{f}{}{keep file length}{False}
\end{options}

\begin{qsection}{EXAMPLE}
If we have the following data in the input {\em data.f} file
\begin{displaymath}
 1.0, 2.0, 3.0, 4.0, 5.0, 6.0
\end{displaymath}
and we use the command below
\begin{quote}
 \verb!delay -s 3 < data.f > data.delay!
\end{quote}
then the output file {\em data.delay} is 
\begin{displaymath}
 0.0, 0.0, 0.0, 1.0, 2.0, 3.0, 4.0, 5.0, 6.0
\end{displaymath}
As another example, if we want to keep the same size of the input file,
we can use the following command,
\begin{quote}
\verb!delay -s 3 -f < data.f > data.delay!
\end{quote}
and the output {\em data.delay} is
\begin{displaymath}
 0.0, 0.0, 0.0, 1.0, 2.0, 3.0
\end{displaymath}
\end{qsection}

%\begin{qsection}{SEE ALSO}
%none
%\end{qsection}
