% ----------------------------------------------------------------
%       Speech Signal Processing Toolkit (SPTK): version 3.3
%                      SPTK Working Group
% 
%                Department of Computer Science
%                Nagoya Institute of Technology
%                             and
%   Interdisciplinary Graduate School of Science and Engineering
%                Tokyo Institute of Technology
%                   Copyright (c) 1984-2000
%                     All Rights Reserved.
% 
% Permission is hereby granted, free of charge, to use and
% distribute this software and its documentation without
% restriction, including without limitation the rights to use,
% copy, modify, merge, publish, distribute, sublicense, and/or
% sell copies of this work, and to permit persons to whom this
% work is furnished to do so, subject to the following conditions:
% 
%   1. The code must retain the above copyright notice, this list
%      of conditions and the following disclaimer.
% 
%   2. Any modifications must be clearly marked as such.
%                                                                        
% NAGOYA INSTITUTE OF TECHNOLOGY, TOKYO INSITITUTE OF TECHNOLOGY,
% SPTK WORKING GROUP, AND THE CONTRIBUTORS TO THIS WORK DISCLAIM
% ALL WARRANTIES WITH REGARD TO THIS SOFTWARE, INCLUDING ALL
% IMPLIED WARRANTIES OF MERCHANTABILITY AND FITNESS, IN NO EVENT
% SHALL NAGOYA INSTITUTE OF TECHNOLOGY, TOKYO INSITITUTE OF
% TECHNOLOGY, SPTK WORKING GROUP, NOR THE CONTRIBUTORS BE LIABLE
% FOR ANY SPECIAL, INDIRECT OR CONSEQUENTIAL DAMAGES OR ANY
% DAMAGES WHATSOEVER RESULTING FROM LOSS OF USE, DATA OR PROFITS,
% WHETHER IN AN ACTION OF CONTRACT, NEGLIGENCE OR OTHER TORTIOUS
% ACTION, ARISING OUT OF OR IN CONNECTION WITH THE USE OR
% PERFORMANCE OF THIS SOFTWARE.
% ----------------------------------------------------------------
%
\name{pca}{principal component analysis}{data processing}
\def\Vec#1{\mbox{\boldmath $#1$}}

\begin{synopsis}
 \item[pca] [ --l $L$ ] [ --n $N$] [ --i $I$] [ --e $e$]
 [ --v ]  [ --V $fn$ ] [ {\em infile} ] 
\end{synopsis}

\begin{qsection}{DESCRIPTION}
 {\em pca} carries out principal component analysis
 from {\em infile} (or standard input) with jacobi method,
 sending the result to standard output.
 {\em pca} can also calculate contribution ratio with the eigen values.

 In {\em infile},
 the input training data set consists of $L$-dimension vectors:
 \[
 \Vec{x}(0), \Vec{x}(1), \Vec{x}(2), \Vec{x}(3), \cdots \;\;\;
 where\;\;\Vec{x}(i) = (x_{i}(1), x_{i}(2), \cdots, x_{i}(L))
 \]

Input and output data are in float format. 
\end{qsection}

\begin{options}
 \argm{l}{L}{dimentionality of vector}{3}
 \argm{n}{N}{number of output principal components}{2}
 \argm{i}{I}{limit of iteration on jacobi method}{10000}
 \argm{e}{e}{threshold of convergence on jacobi method}{0.000001}
 \argm{v}{}{output eigen vectors and mean vector of the training data}{FALSE}
 \argm{V}{fn}{output eigen values and contribution rate (output filename = fn)}{FALSE}
\end{options}

\begin{qsection}{EXAMPLE}
 In the example below,
 the eigen vectors and the eigen values are
 calculated  from {\em data.f}
 which contains three-dimentional training vectors.
 The mean vectors and eigen vectors are sent to
 {\em pca.dat}, and the eigen values are sent to {\em eigen.dat}.
\begin{quote}
  \verb!pca data.f -n 2 -l 3 -v -V eigen.dat > pca.dat!
\end{quote} 
In the {\em pca.dat}, the mean vector is in the front of
eigen vectors.
In the {\em eigen.dat}, 
the eigen values and their contribution ratio are bound per
the same principal component and ordered according to the
magnitude of the eigen values.
\end{qsection} 
\begin{qsection}{SEE ALSO}
 \hyperlink{pcap}{pcap}
\end{qsection}
