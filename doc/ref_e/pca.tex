% ----------------------------------------------------------------- %
%             The Speech Signal Processing Toolkit (SPTK)           %
%             developed by SPTK Working Group                       %
%             http://sp-tk.sourceforge.net/                         %
% ----------------------------------------------------------------- %
%                                                                   %
%  Copyright (c) 1984-2007  Tokyo Institute of Technology           %
%                           Interdisciplinary Graduate School of    %
%                           Science and Engineering                 %
%                                                                   %
%                1996-2015  Nagoya Institute of Technology          %
%                           Department of Computer Science          %
%                                                                   %
% All rights reserved.                                              %
%                                                                   %
% Redistribution and use in source and binary forms, with or        %
% without modification, are permitted provided that the following   %
% conditions are met:                                               %
%                                                                   %
% - Redistributions of source code must retain the above copyright  %
%   notice, this list of conditions and the following disclaimer.   %
% - Redistributions in binary form must reproduce the above         %
%   copyright notice, this list of conditions and the following     %
%   disclaimer in the documentation and/or other materials provided %
%   with the distribution.                                          %
% - Neither the name of the SPTK working group nor the names of its %
%   contributors may be used to endorse or promote products derived %
%   from this software without specific prior written permission.   %
%                                                                   %
% THIS SOFTWARE IS PROVIDED BY THE COPYRIGHT HOLDERS AND            %
% CONTRIBUTORS "AS IS" AND ANY EXPRESS OR IMPLIED WARRANTIES,       %
% INCLUDING, BUT NOT LIMITED TO, THE IMPLIED WARRANTIES OF          %
% MERCHANTABILITY AND FITNESS FOR A PARTICULAR PURPOSE ARE          %
% DISCLAIMED. IN NO EVENT SHALL THE COPYRIGHT OWNER OR CONTRIBUTORS %
% BE LIABLE FOR ANY DIRECT, INDIRECT, INCIDENTAL, SPECIAL,          %
% EXEMPLARY, OR CONSEQUENTIAL DAMAGES (INCLUDING, BUT NOT LIMITED   %
% TO, PROCUREMENT OF SUBSTITUTE GOODS OR SERVICES; LOSS OF USE,     %
% DATA, OR PROFITS; OR BUSINESS INTERRUPTION) HOWEVER CAUSED AND ON %
% ANY THEORY OF LIABILITY, WHETHER IN CONTRACT, STRICT LIABILITY,   %
% OR TORT (INCLUDING NEGLIGENCE OR OTHERWISE) ARISING IN ANY WAY    %
% OUT OF THE USE OF THIS SOFTWARE, EVEN IF ADVISED OF THE           %
% POSSIBILITY OF SUCH DAMAGE.                                       %
% ----------------------------------------------------------------- %
\hypertarget{pca}{}
\name{pca}{principal component analysis}{data processing}
\def\Vec#1{\mbox{\boldmath $#1$}}

\begin{synopsis}
 \item[pca] [ --l $L$ ] [ --n $N$] [ --i $I$] [ --e $e$]
 [ --v ]  [ --V $fn$ ] [ {\em infile} ] 
\end{synopsis}

\begin{qsection}{DESCRIPTION}
 {\em pca} applies principal component analysis
 in the data from {\em infile} (or standard input) using the Jacobi method,
 and sends the result to standard output.
 {\em pca} can also calculate contribution ratio with the eigen values.

 In {\em infile},
 the input training data set consists of $L$-dimension vectors of the form:
 \[
 \Vec{x}(0), \Vec{x}(1), \Vec{x}(2), \Vec{x}(3), \cdots \;\;\;
 \mathrm{where}\;\;\Vec{x}(i) = (x_{i}(1), x_{i}(2), \cdots, x_{i}(L))
 \]

Input and output data are in float format. 
\end{qsection}

\begin{options}
 \argm{l}{L}{dimension of vector}{3}
 \argm{n}{N}{number of output principal components}{2}
 \argm{i}{I}{limit of iterations of the Jacobi method}{10000}
 \argm{e}{e}{threshold of convergence of the Jacobi method}{0.000001}
 \argm{v}{}{output eigenvectors and mean vector of the training data}{FALSE}
 \argm{V}{fn}{output eigenvalues and contribution rate (output filename = fn)}{FALSE}
\end{options}

\begin{qsection}{EXAMPLE}
 In the example below,
 the eigenvectors and the eigenvalues are
 calculated  from {\em data.f}
 which contains three-dimensional training vectors.
 The mean vectors and eigenvectors are sent to
 {\em pca.dat}, and the eigenvalues are sent to {\em eigen.dat}.
\begin{quote}
  \verb!pca data.f -n 2 -l 3 -v -V eigen.dat > pca.dat!
\end{quote} 
Note that in the {\em pca.dat}, the mean vector is written in front of
the eigenvectors.
In the {\em eigen.dat}, 
the eigenvalues and their contribution ratio are bound by
the same principal component and ordered according to the
magnitude of the eigen values.
\end{qsection} 
\begin{qsection}{SEE ALSO}
 \hyperlink{pcas}{pcas}
\end{qsection}
