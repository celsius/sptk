\name{c2ir}{cepstrum to minimum phase impulse response}%
{speech parameter transformation}

\begin{synopsis}
 \item[c2ir] [ --l $L$ ] [ --m $M_1$ ] [ --M $M_2$ ] [ --i ] [ {\em infile} ]
\end{synopsis}

\begin{qsection}{DESCRIPTION}
The minimum phase impulse response is evaluated from the minimum phase
cepstrum contained in the input file {\em infile}(if input file name
is omitted, the starndard input is used).
For example, if the input sequence is
\[c(0),c(1),c(2),\cdots,c(M_1)\]
then the impulse response is caluclated as
\[ h(n)=\left\{
\begin{array}{lc}
 h(0)=\exp(c(0))&\\
 h(n)=\displaystyle \sum_{k=1}^{M_1} \frac{k}{n}c(k)h(n-k)& n \geq 1
\end{array}
\right. \]
and the output will contain
\[ h(0),h(1),h(2),\cdots,h(L -1) \].
Input and output formats are float.
\end{qsection}

\begin{options}
	\argm{l}{L}{order of impulse response}{256}
	\argm{m}{M_1}{order of cepstrum}{25}
	\argm{M}{M_2}{length of impulse response}{L-1}
	\argm{i}{}{input minimum phase sequence}{FALSE}
	\desc{If the number of cepstrum coefficients $M$ is not
             assigend and the number of each cepstrum coefficients
             inputs is less then $L$, then the number of coefficients
             read is made equal to $M$.}
\end{options}

\begin{qsection}{EXAMPLE}
The output file {\em data.ir} contains the impulse response
in the range $n = 0 \sim 99$ obtained from the 30th order cepstrum
coefficients file {\em data.cep}, in float format:
 \begin{quote}
  \verb!c2ir -l 100 -m 30 data.cep > data.ir!
 \end{quote}
\end{qsection}

\begin{qsection}{SEE ALSO}
 c2sp, c2acr
\end{qsection}
