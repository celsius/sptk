% ----------------------------------------------------------------- %
%             The Speech Signal Processing Toolkit (SPTK)           %
%             developed by SPTK Working Group                       %
%             http://sp-tk.sourceforge.net/                         %
% ----------------------------------------------------------------- %
%                                                                   %
%  Copyright (c) 1984-2007  Tokyo Institute of Technology           %
%                           Interdisciplinary Graduate School of    %
%                           Science and Engineering                 %
%                                                                   %
%                1996-2011  Nagoya Institute of Technology          %
%                           Department of Computer Science          %
%                                                                   %
% All rights reserved.                                              %
%                                                                   %
% Redistribution and use in source and binary forms, with or        %
% without modification, are permitted provided that the following   %
% conditions are met:                                               %
%                                                                   %
% - Redistributions of source code must retain the above copyright  %
%   notice, this list of conditions and the following disclaimer.   %
% - Redistributions in binary form must reproduce the above         %
%   copyright notice, this list of conditions and the following     %
%   disclaimer in the documentation and/or other materials provided %
%   with the distribution.                                          %
% - Neither the name of the SPTK working group nor the names of its %
%   contributors may be used to endorse or promote products derived %
%   from this software without specific prior written permission.   %
%                                                                   %
% THIS SOFTWARE IS PROVIDED BY THE COPYRIGHT HOLDERS AND            %
% CONTRIBUTORS "AS IS" AND ANY EXPRESS OR IMPLIED WARRANTIES,       %
% INCLUDING, BUT NOT LIMITED TO, THE IMPLIED WARRANTIES OF          %
% MERCHANTABILITY AND FITNESS FOR A PARTICULAR PURPOSE ARE          %
% DISCLAIMED. IN NO EVENT SHALL THE COPYRIGHT OWNER OR CONTRIBUTORS %
% BE LIABLE FOR ANY DIRECT, INDIRECT, INCIDENTAL, SPECIAL,          %
% EXEMPLARY, OR CONSEQUENTIAL DAMAGES (INCLUDING, BUT NOT LIMITED   %
% TO, PROCUREMENT OF SUBSTITUTE GOODS OR SERVICES; LOSS OF USE,     %
% DATA, OR PROFITS; OR BUSINESS INTERRUPTION) HOWEVER CAUSED AND ON %
% ANY THEORY OF LIABILITY, WHETHER IN CONTRACT, STRICT LIABILITY,   %
% OR TORT (INCLUDING NEGLIGENCE OR OTHERWISE) ARISING IN ANY WAY    %
% OUT OF THE USE OF THIS SOFTWARE, EVEN IF ADVISED OF THE           %
% POSSIBILITY OF SUCH DAMAGE.                                       %
% ----------------------------------------------------------------- %
\hypertarget{c2ir}{}
\name{c2ir}{cepstrum to minimum phase impulse response}%
{speech parameter transformation}

\begin{synopsis}
 \item[c2ir] [ --l $L$ ] [ --m $M_1$ ] [ --M $M_2$ ] [ --i ] [ {\em infile} ]
\end{synopsis}

\begin{qsection}{DESCRIPTION}
{\em c2ir} calculates the minimum phase impulse response 
from minimum phase cepstral coefficients 
from {\em infile} (or standard input), 
sending the result to standard output.
For example, if the input sequence is
\begin{displaymath}
   c(0),c(1),c(2),\dots,c(M_1)
\end{displaymath}
then the impulse response is calculated as
\begin{displaymath}
 h(n)= \begin{cases}
 \;\; h(0)=\exp(c(0)) & \\
 \;\; h(n)=\displaystyle \sum_{k=1}^{M_1} \frac{k}{n} c(k)h(n-k) & n \geq 1
 \end{cases}
\end{displaymath}
and the output will contain
\begin{displaymath}
   h(0),h(1),h(2),\dots,h(L -1)
\end{displaymath}
The format of input and output format is float.
\end{qsection}

\begin{options}
	\argm{m}{M_1}{order of cepstrum}{25}
	\argm{M}{M_2}{length of impulse response}{L-1}
	\argm{l}{L}{order of impulse response}{256}
	\argm{i}{}{input minimum phase sequence}{FALSE}
	\desc{If the number of cepstral coefficients $M_1$ is not
             assigned and the order of the cepstral analysis
             is less then $L$, then the number of coefficients
             read is made equal to $M_1$.}
\end{options}

\begin{qsection}{EXAMPLE}
The output file {\em data.ir} contains the impulse response
in the range $n = 0 \sim 99$ obtained from the 30-th order cepstral
coefficients file {\em data.cep}, in float format:
 \begin{quote}
  \verb!c2ir -l 100 -m 30 data.cep > data.ir!
 \end{quote}
\end{qsection}

\begin{qsection}{SEE ALSO}
\hyperlink{c2sp}{c2sp},
\hyperlink{c2acr}{c2acr}
\end{qsection}
