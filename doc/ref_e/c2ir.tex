%  ---------------------------------------------------------------  %
%            Speech Signal Processing Toolkit (SPTK)                %
%                      SPTK Working Group                           %
%                                                                   %
%                  Department of Computer Science                   %
%                  Nagoya Institute of Technology                   %
%                               and                                 %
%   Interdisciplinary Graduate School of Science and Engineering    %
%                  Tokyo Institute of Technology                    %
%                                                                   %
%                     Copyright (c) 1984-2007                       %
%                       All Rights Reserved.                        %
%                                                                   %
%  Permission is hereby granted, free of charge, to use and         %
%  distribute this software and its documentation without           %
%  restriction, including without limitation the rights to use,     %
%  copy, modify, merge, publish, distribute, sublicense, and/or     %
%  sell copies of this work, and to permit persons to whom this     %
%  work is furnished to do so, subject to the following conditions: %
%                                                                   %
%    1. The source code must retain the above copyright notice,     %
%       this list of conditions and the following disclaimer.       %
%                                                                   %
%    2. Any modifications to the source code must be clearly        %
%       marked as such.                                             %
%                                                                   %
%    3. Redistributions in binary form must reproduce the above     %
%       copyright notice, this list of conditions and the           %
%       following disclaimer in the documentation and/or other      %
%       materials provided with the distribution.  Otherwise, one   %
%       must contact the SPTK working group.                        %
%                                                                   %
%  NAGOYA INSTITUTE OF TECHNOLOGY, TOKYO INSTITUTE OF TECHNOLOGY,   %
%  SPTK WORKING GROUP, AND THE CONTRIBUTORS TO THIS WORK DISCLAIM   %
%  ALL WARRANTIES WITH REGARD TO THIS SOFTWARE, INCLUDING ALL       %
%  IMPLIED WARRANTIES OF MERCHANTABILITY AND FITNESS, IN NO EVENT   %
%  SHALL NAGOYA INSTITUTE OF TECHNOLOGY, TOKYO INSTITUTE OF         %
%  TECHNOLOGY, SPTK WORKING GROUP, NOR THE CONTRIBUTORS BE LIABLE   %
%  FOR ANY SPECIAL, INDIRECT OR CONSEQUENTIAL DAMAGES OR ANY        %
%  DAMAGES WHATSOEVER RESULTING FROM LOSS OF USE, DATA OR PROFITS,  %
%  WHETHER IN AN ACTION OF CONTRACT, NEGLIGENCE OR OTHER TORTUOUS   %
%  ACTION, ARISING OUT OF OR IN CONNECTION WITH THE USE OR          %
%  PERFORMANCE OF THIS SOFTWARE.                                    %
%                                                                   %
%  ---------------------------------------------------------------  %
%
\hypertarget{c2ir}{}
\name{c2ir}{cepstrum to minimum phase impulse response}%
{speech parameter transformation}

\begin{synopsis}
 \item[c2ir] [ --l $L$ ] [ --m $M_1$ ] [ --M $M_2$ ] [ --i ] [ {\em infile} ]
\end{synopsis}

\begin{qsection}{DESCRIPTION}
{\em c2ir} calcullates the minimum phase impulse response 
from minimum phase cepstral coefficients 
from {\em infile} (or standard input), 
sending the result to standard output.
For example, if the input sequence is
\begin{displaymath}
   c(0),c(1),c(2),\dots,c(M_1)
\end{displaymath}
then the impulse response is calculated as
\begin{displaymath}
 h(n)= \begin{cases}
 \;\; h(0)=\exp(c(0)) & \\
 \;\; h(n)=\displaystyle \sum_{k=1}^{M_1} \frac{k}{n} c(k)h(n-k) & n \geq 1
 \end{cases}
\end{displaymath}
and the output will contain
\begin{displaymath}
   h(0),h(1),h(2),\dots,h(L -1)
\end{displaymath}
The format of input and output format is float.
\end{qsection}

\begin{options}
	\argm{l}{L}{order of impulse response}{256}
	\argm{m}{M_1}{order of cepstrum}{25}
	\argm{M}{M_2}{length of impulse response}{L-1}
	\argm{i}{}{input minimum phase sequence}{FALSE}
	\desc{If the number of cepstral coefficients $M_1$ is not
             assigned and the order of the cepstral analysis
             is less then $L$, then the number of coefficients
             read is made equal to $M_1$.}
\end{options}

\begin{qsection}{EXAMPLE}
The output file {\em data.ir} contains the impulse response
in the range $n = 0 \sim 99$ obtained from the 30-th order cepstral
coefficients file {\em data.cep}, in float format:
 \begin{quote}
  \verb!c2ir -l 100 -m 30 data.cep > data.ir!
 \end{quote}
\end{qsection}

\begin{qsection}{SEE ALSO}
\hyperlink{c2sp}{c2sp},
\hyperlink{c2acr}{c2acr}
\end{qsection}
