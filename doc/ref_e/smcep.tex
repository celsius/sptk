\name[ref:smcep-IEICE]{smcep}%
{2nd order mel-cepstrum analysis}{speech analysis}

\begin{synopsis}
\item [smcep] [ --a $A$ ] [ --t $T$ ] [ --m $M$ ] [ --l $L$ ]
\item [\ ~~~~] [ --i $I$ ] [ --j $J$ ] [ --d $D$ ] [ --e $E$ ] [ {\em infile} ]
\end{synopsis}

\begin{qsection}{DESCRIPTION}
This command uses the following 2nd-order
all pass function raised to 1/2 
\begin{displaymath}
A(z)=
\left(
\frac{z^{-2}-2\alpha \cos \theta z^{-1}+\alpha^2}
	{1-2\alpha \cos \theta z^{-1}+\alpha^2 z^{-2}}
\right)^{\frac{1}{2}}
\tilde{z}^{-1} = \frac{z^{-1}-\alpha}{1-\alpha z^{-1}}
\end{displaymath}
to undertake mel-cepstrum analysis,
and writes into the standard output the mel-cepstrum
coefficients $c(m)$.
Let's assume that input data sequence is
\begin{displaymath}
  x(0),x(1),\ldots,x(L-1)
\end{displaymath}
\par
Input and output data are in float format.
\par
In the mel-cepstrum analysis using
a 2nd-order all pass function,
the speech spectrum is modeled as  $m$ order cepstrum
coefficients $c(m)$ as follows.
\begin{displaymath}
H(z) = \exp \sum_{m=0}^{M} c(m)\,B_m(e^{j\omega})
\end{displaymath}
where
\begin{displaymath}
\mbox{Re}\left[B_m(e^{j\omega})\right]
	=\left\{A^m(e^{j\omega})+A^m(e^{-j\omega})\right\}/2
\end{displaymath}
\par The Newton--Raphson method is applied to calculate
the mel-cepstrum coefficients through the minimization
of the cost function.
\end{qsection}

\newpage
\begin{options}
	\argm{a}{A}{all-pass constant $\alpha$}{0.35}
	\argm{t}{T}{emphasized frequency $\theta*\pi$(rad)}{0}
	\argm{m}{M}{order of mel cepstrum}{25}
	\argm{l}{L_1}{frame length}{256}
	\argm{L}{L_2}{ifft size for making matrices}{256}
	\desc[1zh]{Usually, the options below do not need to be assigned.}
	\argm{i}{I}{minimum iteration of Newton-Raphson method}{2}
	\argm{j}{J}{maximum iteration of Newton-Raphson method}{30}
	\argm{d}{D}{end condition of Newton-Raphson}{0.001}
	\argm{e}{E}{small value added to periodgram}{0.0}
\end{options}

\begin{qsection}{EXAMPLE}
In the example below, speech data is read in float format
from {\em data.f}, analyzed, and resulting mel-cepstrum
coefficients are written to {\em data.mcep}:
\begin{quote}
 \verb!frame < data.f | window | smcep > data.mcep !
\end{quote}
\end{qsection}

\begin{qsection}{SEE ALSO}
 uels, gcep, mcep, mgcep, mlsadf
\end{qsection}
