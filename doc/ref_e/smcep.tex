% ----------------------------------------------------------------
%       Speech Signal Processing Toolkit (SPTK): version 3.0
%                      SPTK Working Group
% 
%                Department of Computer Science
%                Nagoya Institute of Technology
%                             and
%   Interdisciplinary Graduate School of Science and Engineering
%                Tokyo Institute of Technology
%                   Copyright (c) 1984-2000
%                     All Rights Reserved.
% 
% Permission is hereby granted, free of charge, to use and
% distribute this software and its documentation without
% restriction, including without limitation the rights to use,
% copy, modify, merge, publish, distribute, sublicense, and/or
% sell copies of this work, and to permit persons to whom this
% work is furnished to do so, subject to the following conditions:
% 
%   1. The code must retain the above copyright notice, this list
%      of conditions and the following disclaimer.
% 
%   2. Any modifications must be clearly marked as such.
%                                                                        
% NAGOYA INSTITUTE OF TECHNOLOGY, TOKYO INSITITUTE OF TECHNOLOGY,
% SPTK WORKING GROUP, AND THE CONTRIBUTORS TO THIS WORK DISCLAIM
% ALL WARRANTIES WITH REGARD TO THIS SOFTWARE, INCLUDING ALL
% IMPLIED WARRANTIES OF MERCHANTABILITY AND FITNESS, IN NO EVENT
% SHALL NAGOYA INSTITUTE OF TECHNOLOGY, TOKYO INSITITUTE OF
% TECHNOLOGY, SPTK WORKING GROUP, NOR THE CONTRIBUTORS BE LIABLE
% FOR ANY SPECIAL, INDIRECT OR CONSEQUENTIAL DAMAGES OR ANY
% DAMAGES WHATSOEVER RESULTING FROM LOSS OF USE, DATA OR PROFITS,
% WHETHER IN AN ACTION OF CONTRACT, NEGLIGENCE OR OTHER TORTIOUS
% ACTION, ARISING OUT OF OR IN CONNECTION WITH THE USE OR
% PERFORMANCE OF THIS SOFTWARE.
% ----------------------------------------------------------------
%
\hypertarget{smcep}{}
\name[ref:smcep-IEICE,ref:smcep-SPCOM]{smcep}%
{2nd order mel-cepstral analysis}{speech analysis}
 
\begin{synopsis}
\item [smcep] [ --a $A$ ] [ --t $T$ ] [ --m $M$ ] [ --l $L$ ]
\item [\ ~~~~] [ --i $I$ ] [ --j $J$ ] [ --d $D$ ] [ --e $E$ ] [ {\em infile} ]
\end{synopsis}

\begin{qsection}{DESCRIPTION}
{\em smcep} calculates the mel-cepstral coefficients 
from $L$-length framed windowed input data 
from {\em infile} (or standard input), 
sending the result to standard output. 
The analysis uses a second-order all-pass function raised to the 1/2 power: 
\begin{align}
A(z) &=
\left(
\frac{z^{-2}-2\alpha \cos \theta z^{-1}+\alpha^2}
	{1-2\alpha \cos \theta z^{-1}+\alpha^2 z^{-2}}
\right)^{\frac{1}{2}}, \notag \\
\tilde{z}^{-1} &= \frac{z^{-1}-\alpha}{1-\alpha z^{-1}}. \notag
\end{align}

Input and output data are in float format.

In the mel-cepstral analysis using
a 2nd-order all pass function,
the speech spectrum is modeled as  $m$ order cepstral
coefficients $c(m)$ as follows.
\begin{displaymath}
H(z) = \exp \sum_{m=0}^{M} c(m)\,B_m(e^{j\omega})
\end{displaymath}
where
\begin{displaymath}
\mathrm{Re}\left[B_m(e^{j\omega})\right]
	=\left\{A^m(e^{j\omega})+A^m(e^{-j\omega})\right\}/2
\end{displaymath}
\par The Newton--Raphson method is applied to calculate
the mel-cepstral coefficients through the minimization
of the cost function.
\end{qsection}

\newpage
\begin{options}
	\argm{a}{A}{all-pass constant $\alpha$}{0.35}
	\argm{t}{T}{emphasized frequency $\theta*\pi$(rad)}{0}
	\argm{m}{M}{order of mel cepstrum}{25}
	\argm{l}{L_1}{frame length}{256}
	\argm{L}{L_2}{ifft size for making matrices}{256}
	\desc[1ex]{Usually, the options below do not need to be assigned.}
	\argm{i}{I}{minimum iteration of Newton-Raphson method}{2}
	\argm{j}{J}{maximum iteration of Newton-Raphson method}{30}
	\argm{d}{D}{end condition of Newton-Raphson}{0.001}
	\argm{e}{E}{small value added to periodgram}{0.0}
\end{options}

\begin{qsection}{EXAMPLE}
In the example below, speech data is read in float format
from {\em data.f}, analyzed, and resulting mel-cepstral
coefficients are written to {\em data.mcep}:
\begin{quote}
 \verb!frame < data.f | window | smcep > data.mcep !
\end{quote}
\end{qsection}

\begin{qsection}{SEE ALSO}
\hyperlink{uels}{uels},
\hyperlink{gcep}{gcep},
\hyperlink{mcep}{mcep},
\hyperlink{mgcep}{mgcep},
\hyperlink{mlsadf}{mlsadf}
\end{qsection}
