% ----------------------------------------------------------------- %
%             The Speech Signal Processing Toolkit (SPTK)           %
%             developed by SPTK Working Group                       %
%             http://sp-tk.sourceforge.net/                         %
% ----------------------------------------------------------------- %
%                                                                   %
%  Copyright (c) 1984-2007  Tokyo Institute of Technology           %
%                           Interdisciplinary Graduate School of    %
%                           Science and Engineering                 %
%                                                                   %
%                1996-2010  Nagoya Institute of Technology          %
%                           Department of Computer Science          %
%                                                                   %
% All rights reserved.                                              %
%                                                                   %
% Redistribution and use in source and binary forms, with or        %
% without modification, are permitted provided that the following   %
% conditions are met:                                               %
%                                                                   %
% - Redistributions of source code must retain the above copyright  %
%   notice, this list of conditions and the following disclaimer.   %
% - Redistributions in binary form must reproduce the above         %
%   copyright notice, this list of conditions and the following     %
%   disclaimer in the documentation and/or other materials provided %
%   with the distribution.                                          %
% - Neither the name of the SPTK working group nor the names of its %
%   contributors may be used to endorse or promote products derived %
%   from this software without specific prior written permission.   %
%                                                                   %
% THIS SOFTWARE IS PROVIDED BY THE COPYRIGHT HOLDERS AND            %
% CONTRIBUTORS "AS IS" AND ANY EXPRESS OR IMPLIED WARRANTIES,       %
% INCLUDING, BUT NOT LIMITED TO, THE IMPLIED WARRANTIES OF          %
% MERCHANTABILITY AND FITNESS FOR A PARTICULAR PURPOSE ARE          %
% DISCLAIMED. IN NO EVENT SHALL THE COPYRIGHT OWNER OR CONTRIBUTORS %
% BE LIABLE FOR ANY DIRECT, INDIRECT, INCIDENTAL, SPECIAL,          %
% EXEMPLARY, OR CONSEQUENTIAL DAMAGES (INCLUDING, BUT NOT LIMITED   %
% TO, PROCUREMENT OF SUBSTITUTE GOODS OR SERVICES; LOSS OF USE,     %
% DATA, OR PROFITS; OR BUSINESS INTERRUPTION) HOWEVER CAUSED AND ON %
% ANY THEORY OF LIABILITY, WHETHER IN CONTRACT, STRICT LIABILITY,   %
% OR TORT (INCLUDING NEGLIGENCE OR OTHERWISE) ARISING IN ANY WAY    %
% OUT OF THE USE OF THIS SOFTWARE, EVEN IF ADVISED OF THE           %
% POSSIBILITY OF SUCH DAMAGE.                                       %
% ----------------------------------------------------------------- %
\hypertarget{smcep}{}
\name[ref:smcep-IEICE,ref:smcep-SPCOM]{smcep}%
{mel-cepstral analysis using 2nd order all-pass filter}{speech analysis}
 
\begin{synopsis}
\item [smcep] [ --a $A$ ] [ --t $T$ ] [ --m $M$ ] [ --l $L$ ] [ --q $Q$ ]
\item [\ ~~~~] [ --i $I$ ] [ --j $J$ ] [ --d $D$ ] [ --e $E$ ] [ --f $F$ ] [ {\em infile} ]
\end{synopsis}

\begin{qsection}{DESCRIPTION}
{\em smcep} calculates the mel-cepstral coefficients 
from $L$-length framed windowed input data 
from {\em infile} (or standard input), 
sending the result to standard output. 
The analysis uses a second-order all-pass function raised to the $1/2$ power: 
\begin{align}
A(z) &=
\left(
\frac{z^{-2}-2\alpha \cos \theta z^{-1}+\alpha^2}
	{1-2\alpha \cos \theta z^{-1}+\alpha^2 z^{-2}}
\right)^{\frac{1}{2}}, \notag \\
\tilde{z}^{-1} &= \frac{z^{-1}-\alpha}{1-\alpha z^{-1}}. \notag
\end{align}

Input and output data are in float format.

In the mel-cepstral analysis using
a 2nd-order all pass function,
the speech spectrum is modeled as $m$-th order cepstral
coefficients $c(m)$ as follows.
\begin{displaymath}
H(z) = \exp \sum_{m=0}^{M} c(m)\,B_m(e^{j\omega})
\end{displaymath}
where
\begin{displaymath}
\mathrm{Re}\left[B_m(e^{j\omega})\right]
	= \frac{A^m(e^{j\omega})+A^m(e^{-j\omega})}{2}
\end{displaymath}
\par The Newton-Raphson method is applied to calculate
the mel-cepstral coefficients through the minimization
of the cost function.
\end{qsection}

\newpage
\begin{options}
	\argm{a}{A}{all-pass constant $\alpha$}{0.35}
	\argm{t}{T}{emphasized frequency $\theta*\pi$ (rad)}{0}
	\argm{m}{M}{order of mel cepstrum}{25}
	\argm{l}{L_1}{frame length}{256}
	\argm{L}{L_2}{ifft size for making matrices}{1024}
	\argm{q}{Q}{input data style\\
        \begin{tabular}{ll} \\[-1ex]
            $Q=0$ & windowed data sequence \\
            $Q=1$ & $20 \times \log |f(w)|$ \\
            $Q=2$ & $\ln |f(w)|$ \\
            $Q=3$ & $|f(w)|$ \\
            $Q=4$ & $|f(w)|^2$ \\[1ex]
        \end{tabular}\\\hspace*{\fill}}{0}
	\desc[1ex]{Usually, the options below do not need to be assigned.}
	\argm{i}{I}{minimum iteration of Newton-Raphson method}{2}
	\argm{j}{J}{maximum iteration of Newton-Raphson method}{30}
	\argm{d}{D}{end condition of Newton-Raphson}{0.001}
	\argm{e}{E}{small value added to periodgram}{0}
	\argm{f}{F}{mimimum value of the determinant of the normal matrix}{0.000001}
\end{options}

\begin{qsection}{EXAMPLE}
In the example below, speech data is read in float format
from {\em data.f}, analyzed, and resulting mel-cepstral
coefficients are written to {\em data.mcep}:
\begin{quote}
 \verb!frame +f < data.f | window | smcep > data.mcep !
\end{quote}
\end{qsection}

\begin{qsection}{SEE ALSO}
\hyperlink{uels}{uels},
\hyperlink{gcep}{gcep},
\hyperlink{mcep}{mcep},
\hyperlink{mgcep}{mgcep},
\hyperlink{mlsadf}{mlsadf}
\end{qsection}
