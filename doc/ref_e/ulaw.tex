% ----------------------------------------------------------------
%       Speech Signal Processing Toolkit (SPTK): version 3.0
%                      SPTK Working Group
% 
%                Department of Computer Science
%                Nagoya Institute of Technology
%                             and
%   Interdisciplinary Graduate School of Science and Engineering
%                Tokyo Institute of Technology
%                   Copyright (c) 1984-2000
%                     All Rights Reserved.
% 
% Permission is hereby granted, free of charge, to use and
% distribute this software and its documentation without
% restriction, including without limitation the rights to use,
% copy, modify, merge, publish, distribute, sublicense, and/or
% sell copies of this work, and to permit persons to whom this
% work is furnished to do so, subject to the following conditions:
% 
%   1. The code must retain the above copyright notice, this list
%      of conditions and the following disclaimer.
% 
%   2. Any modifications must be clearly marked as such.
%                                                                        
% NAGOYA INSTITUTE OF TECHNOLOGY, TOKYO INSITITUTE OF TECHNOLOGY,
% SPTK WORKING GROUP, AND THE CONTRIBUTORS TO THIS WORK DISCLAIM
% ALL WARRANTIES WITH REGARD TO THIS SOFTWARE, INCLUDING ALL
% IMPLIED WARRANTIES OF MERCHANTABILITY AND FITNESS, IN NO EVENT
% SHALL NAGOYA INSTITUTE OF TECHNOLOGY, TOKYO INSITITUTE OF
% TECHNOLOGY, SPTK WORKING GROUP, NOR THE CONTRIBUTORS BE LIABLE
% FOR ANY SPECIAL, INDIRECT OR CONSEQUENTIAL DAMAGES OR ANY
% DAMAGES WHATSOEVER RESULTING FROM LOSS OF USE, DATA OR PROFITS,
% WHETHER IN AN ACTION OF CONTRACT, NEGLIGENCE OR OTHER TORTIOUS
% ACTION, ARISING OUT OF OR IN CONNECTION WITH THE USE OR
% PERFORMANCE OF THIS SOFTWARE.
% ----------------------------------------------------------------
%
\name{ulaw}{$\mu$-law compress/decompress}{signal processing}

\begin{synopsis}
\item[ulaw] [ --v $V$ ] [ --u $U$ ] [ --c ] [ --d ] [ {\em infile} ]
\end{synopsis}

\begin{qsection}{DESCRIPTION}
{\em ulaw} converts data between 8-bit $\mu$-law and 16-bit linear formats.
The input data is {\rm infile} (or standard input), 
and the output is sent to standard output.

If the input is $x(n)$, the output is $y(n)$,
the largest value of input data is $V$, compression coefficients is $U$,
then the compression is made through the following equation.
\begin{displaymath}
y(n) = sgn(x(n)) V \frac{\log(1 + U \frac{|x(n)|}{V} )}{\log(1+U)}
\end{displaymath}
And decompression is
\begin{displaymath}
y(n) = sgn(x(n)) V \frac{(1+u)^{|x(n)|/V} - 1}{U}
\end{displaymath}
\end{qsection}

\begin{options}
	\argm{v}{V}{maximum of input}{32768}
	\argm{u}{U}{compression ratio}{256}
	\argm{c}{}{coder mode}{TRUE}
	\argm{d}{}{decoder mode}{FALSE}
\end{options}

\begin{qsection}{EXAMPLE}
In the following, 16 bit data read from {\em data.s}
is compressed to 8 bit ulaw format, and written to {\em data.ulaw}
\begin{quote}
  \verb!x2x +sf data.s | ulaw | sopr -d 256 | x2x +fc -r > data.ulaw!
\end{quote}
\end{qsection}

%\begin{qsection}{SEE ALSO}
%none
%\end{qsection}
