% ----------------------------------------------------------------- %
%             The Speech Signal Processing Toolkit (SPTK)           %
%             developed by SPTK Working Group                       %
%             http://sp-tk.sourceforge.net/                         %
% ----------------------------------------------------------------- %
%                                                                   %
%  Copyright (c) 1984-2007  Tokyo Institute of Technology           %
%                           Interdisciplinary Graduate School of    %
%                           Science and Engineering                 %
%                                                                   %
%                1996-2015  Nagoya Institute of Technology          %
%                           Department of Computer Science          %
%                                                                   %
% All rights reserved.                                              %
%                                                                   %
% Redistribution and use in source and binary forms, with or        %
% without modification, are permitted provided that the following   %
% conditions are met:                                               %
%                                                                   %
% - Redistributions of source code must retain the above copyright  %
%   notice, this list of conditions and the following disclaimer.   %
% - Redistributions in binary form must reproduce the above         %
%   copyright notice, this list of conditions and the following     %
%   disclaimer in the documentation and/or other materials provided %
%   with the distribution.                                          %
% - Neither the name of the SPTK working group nor the names of its %
%   contributors may be used to endorse or promote products derived %
%   from this software without specific prior written permission.   %
%                                                                   %
% THIS SOFTWARE IS PROVIDED BY THE COPYRIGHT HOLDERS AND            %
% CONTRIBUTORS "AS IS" AND ANY EXPRESS OR IMPLIED WARRANTIES,       %
% INCLUDING, BUT NOT LIMITED TO, THE IMPLIED WARRANTIES OF          %
% MERCHANTABILITY AND FITNESS FOR A PARTICULAR PURPOSE ARE          %
% DISCLAIMED. IN NO EVENT SHALL THE COPYRIGHT OWNER OR CONTRIBUTORS %
% BE LIABLE FOR ANY DIRECT, INDIRECT, INCIDENTAL, SPECIAL,          %
% EXEMPLARY, OR CONSEQUENTIAL DAMAGES (INCLUDING, BUT NOT LIMITED   %
% TO, PROCUREMENT OF SUBSTITUTE GOODS OR SERVICES; LOSS OF USE,     %
% DATA, OR PROFITS; OR BUSINESS INTERRUPTION) HOWEVER CAUSED AND ON %
% ANY THEORY OF LIABILITY, WHETHER IN CONTRACT, STRICT LIABILITY,   %
% OR TORT (INCLUDING NEGLIGENCE OR OTHERWISE) ARISING IN ANY WAY    %
% OUT OF THE USE OF THIS SOFTWARE, EVEN IF ADVISED OF THE           %
% POSSIBILITY OF SUCH DAMAGE.                                       %
% ----------------------------------------------------------------- %
\hypertarget{ulaw}{}
\name{ulaw}{$\mu$-law compress/decompress}{signal processing}

\begin{synopsis}
\item[ulaw] [ --v $V$ ] [ --u $U$ ] [ --c ] [ --d ] [ {\em infile} ]
\end{synopsis}

\begin{qsection}{DESCRIPTION}
{\em ulaw} converts data between 8-bit $\mu$-law and 16-bit linear formats.
The input data is {\rm infile} (or standard input),
and the output is sent to standard output.

If the input is $x(n)$, the output is $y(n)$,
the largest value of input data is $V$, the compression coefficients vector is $U$,
then the compression will be performed using made through the following equation.
\begin{displaymath}
y(n) = sgn(x(n)) V \frac{\log(1 + U \frac{|x(n)|}{V} )}{\log(1+U)}
\end{displaymath}
Likewise, the decompression can be performed by applying the following:
\begin{displaymath}
y(n) = sgn(x(n)) V \frac{(1+u)^{|x(n)|/V} - 1}{U}
\end{displaymath}
\end{qsection}

\begin{options}
	\argm{v}{V}{maximum value of input}{32768}
	\argm{u}{U}{compression ratio}{256}
	\argm{c}{}{coder mode}{TRUE}
	\argm{d}{}{decoder mode}{FALSE}
\end{options}

\begin{qsection}{EXAMPLE}
In the following, 16-bit data read from {\em data.s}
is compressed to 8-bit ulaw format, and output to {\em data.ulaw}
\begin{quote}
  \verb!x2x +sf data.s | ulaw | sopr -d 256 | x2x +fc -r > data.ulaw!
\end{quote}
\end{qsection}

%\begin{qsection}{SEE ALSO}
%none
%\end{qsection}
