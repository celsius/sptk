\name{ulaw}{$\mu$-law compress/decompress}{signal processing}

\begin{synopsis}
\item[ulaw] [ --v $V$ ] [ --u $U$ ] [ --c ] [ --d ] [ {\em infile} ]
\end{synopsis}

\begin{qsection}{DESCRIPTION}
This command compresses or decompresses data into
$\mu$-law format.
If the input is $x(n)$, the output is $y(n)$,
the largest value of input data is $V$, compression coefficients is $U$,
then the compression is made through the following equation.
\begin{displaymath}
y(n) = sgn(x(n)) V \frac{\log(1 + U \frac{|x(n)|}{V} )}{\log(1+U)}
\end{displaymath}
And decompression is
\begin{displaymath}
y(n) = sgn(x(n)) V \frac{(1+u)^{|x(n)|/V} - 1}{U}
\end{displaymath}
\end{qsection}

\begin{options}
	\argm{v}{V}{maximum of input}{32768}
	\argm{u}{U}{compression ratio}{256}
	\argm{c}{}{coder mode}{TRUE}
	\argm{d}{}{decoder mode}{FALSE}
\end{options}

\begin{qsection}{EXAMPLE}
In the following, 16 bit data read from {\em data.s}
is compressed to 8 bit ulaw format, and written to {\em data.ulaw}
\begin{quote}
  \verb!x2x +sf data.s | ulaw | sopr -d 256 | x2x +fc -r > data.ulaw!
\end{quote}
\end{qsection}

%\begin{qsection}{SEE ALSO}
%none
%\end{qsection}
