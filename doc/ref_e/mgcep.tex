\name[ref:mgcep-IEICE,ref:mgcep-ICSLP94]{mgcep}%
{mel-generalized cepstral analysis}{speech analysis}

\begin{synopsis}
\item[mgcep]   [ --a $A$ ] [ --g $G$ ] [ --m $M$ ] [ --l $L$ ] 
	       [ --o $O$ ]
\item[\ ~~~~~~~] [ --i $I$ ] [ --j $J$ ] [ --d $D$ ] [ --p $P$ ] [ -- e $E$ ] 
		 [ {\em infile} ]
\end{synopsis}

\begin{qsection}{DESCRIPTION}

This command undertakes the mel-generalized cepstrum analysis.
The resulting analysis is outputed to the standard output
taking into consideration the option assigned to {\bf --o}.
When input signal has length $L$,
then the time sequence is given by
\begin{displaymath}
  x(0),x(1),\ldots,x(L-1)
\end{displaymath}
\par
Input and output data are in float format.
\par
In the mel-generalized cepstrum analysis, the spectrum of the speech signal
is modeled by $M$ order mel-generalized cepstrum
coefficients $c_{\alpha, \gamma}(m)$
as follows.
\begin{eqnarray*}
H(z) &=& s_\gamma^{-1}\left(
	\sum_{m=0}^M c_{\alpha, \gamma}(m)z^{-m} \right) \\
     &=& \left\{ \begin{array}{ll} \displaystyle
	\left( 1+\gamma\sum_{m=1}^M c_{\alpha, \gamma}(m)\tilde{z}^{-m}
		\right)^{1/\gamma}, & -1 \leq \gamma < 0 \\
	\displaystyle \exp \sum_{m=1}^M c_{\alpha, \gamma}(m)\tilde{z}^{-m}, 
		& \gamma=0
	\end{array} \right.
\end{eqnarray*}
For this command ``mcep'', it is applied a cost function
based on the unbiased estimation log spectrum method.
The valiable $\tilde{z}^{-1}$ can be expressed as the following
first order all-pass function
\begin{displaymath}
\tilde{z}^{-1} = \frac{z^{-1}-\alpha}{1-\alpha z^{-1}}
\end{displaymath}
The phase characteristic is given by the valiable $\alpha$.
For a sampling rate 10kHz, $\alpha$ is made equal to $0.35$.
For a smmpling rate 8kHz, $\alpha$ is mde equal to $0.31$.
By making these choices for $\alpha$,
the mel-scale becomes the good approximation to human
sensitivity to the roudness speech sound.
\par
The Newton-Raphson method is used to minimize the cost function
when evaluating mel-cepstrum coefficients.
\par
The mel-generalized cepstrum analysis includes several other
methods to analyze speech, depending on the values of $\alpha$
and $\gamma$(refer to figure \ref{fig:mgcep_overview}).

\setcounter{figure}{0}
\begin{figure}
\begin{center}
  \setlength{\unitlength}{1mm}
  \begin{picture}(140,100)
    \thicklines
    \put(70,47.5){\oval(140,95)[b]}
    \put(45,47.5){\oval(90,95)[tl]}
    \put(95,47.5){\oval(90,95)[tr]}
    \put(70,95){\makebox(0,0){$|\alpha|<1,\hspace{1em}-1\leq\gamma\leq 0$}}
    \put(42.5,47.5){\oval(65,75)[b]}
    \put(35,47.5){\oval(50,75)[tl]}
    \put(50,47.5){\oval(50,75)[tr]}
    \put(42.5,85){\makebox(0,0){$\alpha=0$}}
    \put(75,55){\oval(110,20)[b]}
    \put(90,55){\oval(140,20)[tl]}
    \put(110,55){\oval(40,20)[tr]}
    \put(100,65){\makebox(0,0){$\gamma=-1$}}
    \put(75,30){\oval(110,20)[b]}
    \put(90,30){\oval(140,20)[tl]}
    \put(110,30){\oval(40,20)[tr]}
    \put(100,40){\makebox(0,0){$\gamma=0$}}
    \put(42.5,75){\makebox(0,0){{\gt generalized cepstrum analysis}}}
    \put(47.5,55){\makebox(0,0){{\gt LPC analysis}}}
    \put(47.5,30){\makebox(0,0){
      \shortstack{{\gt unbiased estimation}\\{\gt of log spectrum}}}}
    \put(107.5,80){\makebox(0,0){
	\underline{\gt mel-generalized cepstrum analysis}}}
    \put(102.5,55){\makebox(0,0){{\gt mel-LPC analysis}}}
    \put(102.5,30){\makebox(0,0){
			{\gt mel-cepstrum analysis}}}
  \end{picture}
\caption{mel-generalized cepstrum analysis and other method relations}
\label{fig:mgcep_overview}
\end{center}
\end{figure}
\end{qsection}

\newpage
\begin{options}
	\argm{a}{A}{alpha $\alpha$}{0.35}
	\argm{g}{G}{power parameter of generalized cepstrum $\gamma$��\\
			 if $G>1.0$ then $\gamma=-1/G$.}{0}
	\argm{m}{M}{order of mel-generalized cepstrum}{25}
	\argm{l}{L}{frame length power of 2}{256}
	\argm{o}{O}{output data style\\
                        $O = 0$:\\
			  $c_{\alpha, \gamma}(0), c_{\alpha, \gamma}(1),
			  \ldots, c_{\alpha, \gamma}(M)$\\
			$O = 1$:\\
			  $b_\gamma(0), b_\gamma(1), \ldots, b_\gamma(M)$\\
			$O = 2$:\\
			  $K_\alpha, c_{\alpha, \gamma}'(1), 
			  \ldots, c_{\alpha, \gamma}'(M)$\\
			$O = 3$:\\
			  $K, b_\gamma'(1), \ldots, b_\gamma'(M)$\\
			$O = 4$:\\
			  $K_\alpha, \gamma\,c_{\alpha, \gamma}'(1), \ldots,
			\gamma\,c_{\alpha, \gamma}'(M)$\\
			$O = 5$:\\
			  $K, \gamma\,b_\gamma'(1), \ldots, 
			  \gamma\,b_\gamma'(M)$
			}{0}
	\desc[1zh]{Usually, the options below do not need to be assigned.}
	\argm{i}{I}{minimum iteration of Newton-Raphson method}{2}
	\argm{j}{J}{maximum iteration of Newton-Raphson method}{30}
	\argm{d}{D}{end condition of Newton-Raphson method}{0.001}
	\argm{p}{P}{order of recursions}{$L-1$}
	\argm{e}{E}{small value added to periodgram}{0}	
\end{options}

\begin{qsection}{EXAMPLE}
In the following speech data in float format is read
from {\em data.f} and analyzed with $\gamma=0$, $\alpha=0$
(which corresspond to UELS method for log spectrum estimation)
and the resulting cepstrum coefficients are written {\em data.cep}:
\begin{quote}
  \verb!frame < data.f | window | mgcep > data.cep !
\end{quote}
\par
In the same way if we want mel-cepstrum coefficients:
\begin{quote}
 \verb!frame < data.f | window | mgcep -a 0.35 > data.mcep !
\end{quote}
\par
If we want linear prediction coefficients:
\begin{quote}
  \verb!frame < data.f | window | mgcep -g -1 -o 5 > data.lpc !
\end{quote}
In this case the linear prediction coefficients are written
in the following representation.
\begin{displaymath}
  K, a(1), a(2), \ldots, a(M)
\end{displaymath}
\end{qsection}

\begin{qsection}{SEE ALSO}
 uels, gcep, mcep, freqt, gc2gc, mgc2mgc, gnorm, mglsadf
\end{qsection}
