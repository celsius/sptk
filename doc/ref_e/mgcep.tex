% ----------------------------------------------------------------- %
%             The Speech Signal Processing Toolkit (SPTK)           %
%             developed by SPTK Working Group                       %
%             http://sp-tk.sourceforge.net/                         %
% ----------------------------------------------------------------- %
%                                                                   %
%  Copyright (c) 1984-2007  Tokyo Institute of Technology           %
%                           Interdisciplinary Graduate School of    %
%                           Science and Engineering                 %
%                                                                   %
%                1996-2008  Nagoya Institute of Technology          %
%                           Department of Computer Science          %
%                                                                   %
% All rights reserved.                                              %
%                                                                   %
% Redistribution and use in source and binary forms, with or        %
% without modification, are permitted provided that the following   %
% conditions are met:                                               %
%                                                                   %
% - Redistributions of source code must retain the above copyright  %
%   notice, this list of conditions and the following disclaimer.   %
% - Redistributions in binary form must reproduce the above         %
%   copyright notice, this list of conditions and the following     %
%   disclaimer in the documentation and/or other materials provided %
%   with the distribution.                                          %
% - Neither the name of the SPTK working group nor the names of its %
%   contributors may be used to endorse or promote products derived %
%   from this software without specific prior written permission.   %
%                                                                   %
% THIS SOFTWARE IS PROVIDED BY THE COPYRIGHT HOLDERS AND            %
% CONTRIBUTORS "AS IS" AND ANY EXPRESS OR IMPLIED WARRANTIES,       %
% INCLUDING, BUT NOT LIMITED TO, THE IMPLIED WARRANTIES OF          %
% MERCHANTABILITY AND FITNESS FOR A PARTICULAR PURPOSE ARE          %
% DISCLAIMED. IN NO EVENT SHALL THE COPYRIGHT OWNER OR CONTRIBUTORS %
% BE LIABLE FOR ANY DIRECT, INDIRECT, INCIDENTAL, SPECIAL,          %
% EXEMPLARY, OR CONSEQUENTIAL DAMAGES (INCLUDING, BUT NOT LIMITED   %
% TO, PROCUREMENT OF SUBSTITUTE GOODS OR SERVICES; LOSS OF USE,     %
% DATA, OR PROFITS; OR BUSINESS INTERRUPTION) HOWEVER CAUSED AND ON %
% ANY THEORY OF LIABILITY, WHETHER IN CONTRACT, STRICT LIABILITY,   %
% OR TORT (INCLUDING NEGLIGENCE OR OTHERWISE) ARISING IN ANY WAY    %
% OUT OF THE USE OF THIS SOFTWARE, EVEN IF ADVISED OF THE           %
% POSSIBILITY OF SUCH DAMAGE.                                       %
% ----------------------------------------------------------------- %
\hypertarget{mgcep}{}
\name[ref:mgcep-IEICE,ref:mgcep-ICSLP94]{mgcep}%
{mel-generalized cepstral analysis}{speech analysis}

\begin{synopsis}
\item[mgcep]   [ --a $A$ ] [ --g $G$ ] [ --m $M$ ] [ --l $L$ ] 
	       [ --q $Q$ ] [ --o $O$ ]
\item[\ ~~~~~~~] [ --i $I$ ] [ --j $J$ ] [ --d $D$ ] [ --p $P$ ] [ -- e $E$ ] [ --f $F$ ] 
		 [ {\em infile} ]
\end{synopsis}

\begin{qsection}{DESCRIPTION}
{\em mgcep} uses mel-generalized cepstral analysis 
to calculate mel-generalized cepstral coefficients 
from $L$-length framed windowed input data 
from {\em infile} (or standard input), 
sending the result to standard output. 
There are several different output formats,
controlled by the --o option.

When input signal has length $L$,
then the time sequence is given by
\begin{displaymath}
  x(0),x(1),\dots,x(L-1)
\end{displaymath}

Input and output data are in float format.

In the mel-generalized cepstral analysis, the spectrum of the speech signal
is modeled by $M$-th order mel-generalized cepstral
coefficients $c_{\alpha, \gamma}(m)$
as follows.
\begin{align}
H(z) &= s_\gamma^{-1}\left(
	\sum_{m=0}^M c_{\alpha, \gamma}(m)z^{-m} \right) \notag \\
     &= \begin{cases} \;\; \displaystyle
	\left( 1+\gamma\sum_{m=1}^M c_{\alpha, \gamma}(m)\tilde{z}^{-m}
		\right)^{1/\gamma}, & -1 \leq \gamma < 0 \\
	\;\; \displaystyle \exp \sum_{m=1}^M c_{\alpha, \gamma}(m)\tilde{z}^{-m}, 
		& \gamma=0
	\end{cases} \notag
\end{align}
For this command ``mgcep'', it is applied a cost function
based on the unbiased estimation log spectrum method.
The variable $\tilde{z}^{-1}$ can be expressed as the following
first order all-pass function
\begin{displaymath}
\tilde{z}^{-1} = \frac{z^{-1}-\alpha}{1-\alpha z^{-1}}
\end{displaymath}
The phase characteristic is given by the variable $\alpha$.
For a sampling rate 10kHz, $\alpha$ is made equal to $0.35$.
For a sampling rate 8kHz, $\alpha$ is made equal to $0.31$.
By making these choices for $\alpha$,
the mel-scale becomes the good approximation to human
sensitivity to the loudness speech sound.

The Newton-Raphson method is used to minimize the cost function
when evaluating mel-cepstral coefficients.

The mel-generalized cepstral analysis includes several other
methods to analyze speech, depending on the values of $\alpha$
and $\gamma$ (refer to figure \ref{fig:mgcep_overview}).

\setcounter{figure}{0}
\begin{figure}
\begin{center}
  \setlength{\unitlength}{1mm}
  \begin{picture}(140,100)
    \thicklines
    \put(70,47.5){\oval(140,95)[b]}
    \put(45,47.5){\oval(90,95)[tl]}
    \put(95,47.5){\oval(90,95)[tr]}
    \put(70,95){\makebox(0,0){$|\alpha|<1,\hspace{1em}-1\leq\gamma\leq 0$}}
    \put(42.5,47.5){\oval(65,75)[b]}
    \put(35,47.5){\oval(50,75)[tl]}
    \put(50,47.5){\oval(50,75)[tr]}
    \put(42.5,85){\makebox(0,0){$\alpha=0$}}
    \put(75,55){\oval(110,20)[b]}
    \put(90,55){\oval(140,20)[tl]}
    \put(110,55){\oval(40,20)[tr]}
    \put(100,65){\makebox(0,0){$\gamma=-1$}}
    \put(75,30){\oval(110,20)[b]}
    \put(90,30){\oval(140,20)[tl]}
    \put(110,30){\oval(40,20)[tr]}
    \put(100,40){\makebox(0,0){$\gamma=0$}}
    \put(42.5,75){\makebox(0,0){generalized cepstral analysis}}
    \put(47.5,55){\makebox(0,0){LPC analysis}}
    \put(47.5,30){\makebox(0,0){
      \shortstack{unbiased estimation\\of log spectrum}}}
    \put(107.5,80){\makebox(0,0){
	\underline{mel-generalized cepstral analysis}}}
    \put(102.5,55){\makebox(0,0){mel-LPC analysis}}
    \put(102.5,30){\makebox(0,0){mel-cepstral analysis}}
  \end{picture}
\caption{mel-generalized cepstral analysis and other method relations}
\label{fig:mgcep_overview}
\end{center}
\end{figure}
\end{qsection}

\newpage
\begin{options}
	\argm{a}{A}{alpha $\alpha$}{0.35}
	\argm{g}{G}{power parameter of generalized cepstrum $\gamma$\\
			 if $G \ge 1.0$ then $\gamma=-1/G$}{0}
	\argm{m}{M}{order of mel-generalized cepstrum}{25}
	\argm{l}{L}{frame length power of 2}{256}
    \argm{q}{Q}{input data style\\
        \begin{tabular}{ll} \\[-1ex]
            $Q=0$ & windowed data sequence \\
            $Q=1$ & $20 \times \log |f(w)|$ \\
            $Q=2$ & $\ln |f(w)|$ \\
            $Q=3$ & $|f(w)|$ \\
            $Q=4$ & $|f(w)|^2$ \\[1ex]
        \end{tabular}\\\hspace*{\fill}}{0}
	\argm{o}{O}{output format \\
		\begin{tabular}{ll} \\[-1ex]
			$O = 0$ & $c_{\alpha, \gamma}(0), c_{\alpha, \gamma}(1), \dots, c_{\alpha, \gamma}(M)$\\
			$O = 1$ & $b_\gamma(0), b_\gamma(1), \dots, b_\gamma(M)$\\
			$O = 2$ & $K_\alpha, c_{\alpha, \gamma}'(1), \dots, c_{\alpha, \gamma}'(M)$\\
			$O = 3$ & $K, b_\gamma'(1), \dots, b_\gamma'(M)$\\
			$O = 4$ & $K_\alpha, \gamma\,c_{\alpha, \gamma}'(1), \dots, \gamma\,c_{\alpha, \gamma}'(M)$\\
			$O = 5$ & $K, \gamma\,b_\gamma'(1), \dots, \gamma\,b_\gamma'(M)$
		\end{tabular}\\\hspace*{\fill}}{0}
	\desc[1ex]{Usually, the options below do not need to be assigned.}
	\argm{i}{I}{minimum iteration of Newton-Raphson method}{2}
	\argm{j}{J}{maximum iteration of Newton-Raphson method}{30}
	\argm{d}{D}{end condition of Newton-Raphson method}{0.001}
	\argm{p}{P}{order of recursions}{$L-1$}
	\argm{e}{E}{small value added to periodgram}{0}	
	\argm{f}{F}{mimimum value of the determinant of the normal matrix}{0.000001}
\end{options}
 
\begin{qsection}{EXAMPLE}
In the following speech data in float format is read
from {\em data.f} and analyzed with $\gamma=0$, $\alpha=0$
(which correspond to UELS method for log spectrum estimation)
and the resulting cepstral coefficients are written {\em data.cep}:
\begin{quote}
  \verb!frame +f < data.f | window | mgcep > data.cep !
\end{quote}

In the same way if we want mel-cepstral coefficients:
\begin{quote}
 \verb!frame +f < data.f | window | mgcep -a 0.35 > data.mcep !
\end{quote}

If we want linear prediction coefficients:
\begin{quote}
  \verb!frame +f < data.f | window | mgcep -g -1 -o 5 > data.lpc !
\end{quote}
In this case the linear prediction coefficients are written
in the following representation.
\begin{displaymath}
  K, a(1), a(2), \dots, a(M)
\end{displaymath}
\end{qsection}

In the following, speech data in float format is read
from {\em data.f}, and analyzed with $\gamma=0$, $\alpha=0$
(which correspond to UELS method for log spectrum estimation)
and the resulting cepstral coefficients are written {\em data.cep}:
\begin{quote}
  \verb!frame +f < data.f | window | \! \\
  \verb!fftr -A -H | mgcep -q 3 > data.cep!
\end{quote}

\begin{qsection}{SEE ALSO}
\hyperlink{uels}{uels},
\hyperlink{gcep}{gcep},
\hyperlink{mcep}{mcep},
\hyperlink{freqt}{freqt},
\hyperlink{gc2gc}{gc2gc},
\hyperlink{mgc2mgc}{mgc2mgc},
\hyperlink{gnorm}{gnorm},
\hyperlink{mglsadf}{mglsadf}
\end{qsection}
