% ----------------------------------------------------------------- %
%             The Speech Signal Processing Toolkit (SPTK)           %
%             developed by SPTK Working Group                       %
%             http://sp-tk.sourceforge.net/                         %
% ----------------------------------------------------------------- %
%                                                                   %
%  Copyright (c) 1984-2007  Tokyo Institute of Technology           %
%                           Interdisciplinary Graduate School of    %
%                           Science and Engineering                 %
%                                                                   %
%                1996-2016  Nagoya Institute of Technology          %
%                           Department of Computer Science          %
%                                                                   %
% All rights reserved.                                              %
%                                                                   %
% Redistribution and use in source and binary forms, with or        %
% without modification, are permitted provided that the following   %
% conditions are met:                                               %
%                                                                   %
% - Redistributions of source code must retain the above copyright  %
%   notice, this list of conditions and the following disclaimer.   %
% - Redistributions in binary form must reproduce the above         %
%   copyright notice, this list of conditions and the following     %
%   disclaimer in the documentation and/or other materials provided %
%   with the distribution.                                          %
% - Neither the name of the SPTK working group nor the names of its %
%   contributors may be used to endorse or promote products derived %
%   from this software without specific prior written permission.   %
%                                                                   %
% THIS SOFTWARE IS PROVIDED BY THE COPYRIGHT HOLDERS AND            %
% CONTRIBUTORS "AS IS" AND ANY EXPRESS OR IMPLIED WARRANTIES,       %
% INCLUDING, BUT NOT LIMITED TO, THE IMPLIED WARRANTIES OF          %
% MERCHANTABILITY AND FITNESS FOR A PARTICULAR PURPOSE ARE          %
% DISCLAIMED. IN NO EVENT SHALL THE COPYRIGHT OWNER OR CONTRIBUTORS %
% BE LIABLE FOR ANY DIRECT, INDIRECT, INCIDENTAL, SPECIAL,          %
% EXEMPLARY, OR CONSEQUENTIAL DAMAGES (INCLUDING, BUT NOT LIMITED   %
% TO, PROCUREMENT OF SUBSTITUTE GOODS OR SERVICES; LOSS OF USE,     %
% DATA, OR PROFITS; OR BUSINESS INTERRUPTION) HOWEVER CAUSED AND ON %
% ANY THEORY OF LIABILITY, WHETHER IN CONTRACT, STRICT LIABILITY,   %
% OR TORT (INCLUDING NEGLIGENCE OR OTHERWISE) ARISING IN ANY WAY    %
% OUT OF THE USE OF THIS SOFTWARE, EVEN IF ADVISED OF THE           %
% POSSIBILITY OF SUCH DAMAGE.                                       %
% ----------------------------------------------------------------- %
\hypertarget{pitch_mark}{}
\name{pitch\_mark}{pitch mark extraction}{signal processing,speech analysis and synthesis}

\begin{synopsis}
 \item[pitch\_mark] [ --s $S$ ] [ --t $T$ ] [ --L $Lo$ ] [ --H $Hi$ ] [ --o $O$ ][ {\em infile} ]
\end{synopsis}

\begin{qsection}{DESCRIPTION}
{\em pitch\_mark} extracts the pitch mark from {\em infile} (or standard input),
sending the result to standard output.
The REAPER \cite{ref:pitch-REAPER} algorithm is adopted for pitch mark extraction.
The output format (pulse sequence, second when a pitch mark appears or sample number when a pitch mark appears) can be specified by --o option.

Both input and output files are in float format.

\end{qsection}

\begin{options}
 \argm{s}{S}{sampling frequency (kHz)}{16.0}
 \argm{t}{T}{voiced/unvoiced threshold \\
    value of $t$ should be $-0.5 < t < 1.6$}{0.9}
 \argm{L}{Lo}{minimum fundamental frequency to search for (Hz)}{60.0}
 \argm{H}{Hi}{maximum fundamental frequency to search for (Hz)}{240.0}
 \argm{o}{O}{output format\\
 \begin{tabular}{ll} \\[-1ex]
  $O=0$ & ~~~pulse sequence: if a current sample is a pitch mark,\\
        & ~~~1 or -1 is outputted considering polarity,\\
        & ~~~otherwise 0 is outputted.\\
  $O=1$ & ~~~second when a pitch mark appears\\
  $O=2$ & ~~~sample number when a pitch mark appears\\
 \end{tabular}\\\hspace*{\fill}}{0}
\end{options}

\begin{qsection}{EXAMPLE}
 In the example below, speech data in float format
 is read from {\em data.f} and the pitch mark data is extracted
 under the condition that sampling frequency is 16kHz
 and the minimum and maximum fundamental frequency are
 80 and 165 Hz, respectively.
 Then, the output is written to {\em data.pitch\_mark}:
 \begin{quote}
  \verb!pitch_mark -s 16 -L 80 -H 165 data.f > data.pitch_mark!
 \end{quote}
\end{qsection}

\begin{qsection}{NOTICE}
Regarding -t option, when the threshold is raised, the number of pitch marks increases.
\end{qsection}

\begin{qsection}{SEE ALSO}
\hyperlink{excite}{excite},
\hyperlink{pitch}{pitch}
\end{qsection}
