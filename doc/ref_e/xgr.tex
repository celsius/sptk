% ----------------------------------------------------------------
%       Speech Signal Processing Toolkit (SPTK): version 3.0
%                      SPTK Working Group
% 
%                Department of Computer Science
%                Nagoya Institute of Technology
%                             and
%   Interdisciplinary Graduate School of Science and Engineering
%                Tokyo Institute of Technology
%                   Copyright (c) 1984-2000
%                     All Rights Reserved.
% 
% Permission is hereby granted, free of charge, to use and
% distribute this software and its documentation without
% restriction, including without limitation the rights to use,
% copy, modify, merge, publish, distribute, sublicense, and/or
% sell copies of this work, and to permit persons to whom this
% work is furnished to do so, subject to the following conditions:
% 
%   1. The code must retain the above copyright notice, this list
%      of conditions and the following disclaimer.
% 
%   2. Any modifications must be clearly marked as such.
%                                                                        
% NAGOYA INSTITUTE OF TECHNOLOGY, TOKYO INSITITUTE OF TECHNOLOGY,
% SPTK WORKING GROUP, AND THE CONTRIBUTORS TO THIS WORK DISCLAIM
% ALL WARRANTIES WITH REGARD TO THIS SOFTWARE, INCLUDING ALL
% IMPLIED WARRANTIES OF MERCHANTABILITY AND FITNESS, IN NO EVENT
% SHALL NAGOYA INSTITUTE OF TECHNOLOGY, TOKYO INSITITUTE OF
% TECHNOLOGY, SPTK WORKING GROUP, NOR THE CONTRIBUTORS BE LIABLE
% FOR ANY SPECIAL, INDIRECT OR CONSEQUENTIAL DAMAGES OR ANY
% DAMAGES WHATSOEVER RESULTING FROM LOSS OF USE, DATA OR PROFITS,
% WHETHER IN AN ACTION OF CONTRACT, NEGLIGENCE OR OTHER TORTIOUS
% ACTION, ARISING OUT OF OR IN CONNECTION WITH THE USE OR
% PERFORMANCE OF THIS SOFTWARE.
% ----------------------------------------------------------------
%
\hypertarget{xgr}{}
\name{xgr}{XY-plotter simulator for X-window system}{plotting graphs}

\begin{synopsis}
 \item[xgr]   [ --s {\em S} ] [ --l ] [ --rv ] [ --m ] [ --bg {\em BG} ]
              [ --hl {\em HL} ] [ --bd {\em BD} ] 
 \item[\ ~~~~] [ --ms {\em MS} ] [ --g {\em G} ] [ --d {\em D} ]
              [ --t {\em T} ] [ {\em infile} ]
\end{synopsis} 

\begin{qsection}{DESCRIPTION}
{\em xgr} plots a graph from a sequence of FP5301 plotter commands, 
displaying the output on the screen in a new X window.

When the X window is created, 
the keyboard focus is initially assigned to that new window, 
which responds to a limited set of user interactions:
\begin{itemize}
\item Changing the window size truncates or expands the area 
	in which the graph is displayed, 
	but the graph stays the same size; 
	it is not rescaled to fit the new window size.
\item If the graph is larger than the window, 
	the position within the window can be changed with 
	``vi'' cursor movement commands:
\begin{quote}
		h: left scroll\\
		j: down scroll\\
		k: up scroll\\
		l: right scroll
\end{quote}
\item To delete the window, type one of the following:
	``q'',``Ctrl-c'',``Ctrl-d''
\end{itemize}
\end{qsection}

\begin{options}
	\argm{s}{S}{shrink}{3.38667}
	\argm{l}{}{landscape}{FALSE}
	\argm{rv}{}{reverse mode}{FALSE}
	\argm{m}{}{monochrome display mode}{FALSE}
	\argm{bg}{BG}{background color}{white}
	\argm{hl}{HL}{highlight color}{blue}
	\argm{bd}{BD}{border color}{blue}
	\argm{ms}{MS}{mouse color}{red}
	\argm{g}{G}{geometry}{NULL}
	\argm{d}{D}{display}{NULL}
	\argm{t}{T}{window title}{xgr}
\end{options}
\begin{qsection}{EXAMPLE}
The following example uses ``fdrw'' to draw a graph based on data read
from {\em data.f}, and sends the output in a X-Window environment:
\begin{quote}
 \verb!fdrw < data.f | xgr!
\end{quote}
\end{qsection}
\begin{qsection}{BUGS}
\begin{itemize}
\item If the display server does not contain backing store function,
then the hidden part of virtual screen is erased.

\item To lessen the waiting time to display graphs,
a image of virtual screen is copied to the memory.
If the size assigned by the --g option is too small
or if during the time the graph is being plotted an another window
is put above the virtual screen, then a part of virtual screen
will be erased.
The --s option is suggested whenever the size of
the virtual screen should be reduced.
\end{itemize}

\end{qsection}
\begin{qsection}{SEE ALSO}
\hyperlink{fig}{fig},
\hyperlink{fdrw}{fdrw}
\end{qsection}
