\name{xgr}{XY-plotter simulator for X-window system}{graph}

\begin{synopsis}
 \item[xgr]   [ --s {\em S} ] [ --l ] [ --rv ] [ --m ] [ --bg {\em BG} ]
              [ --hl {\em HL} ] [ --bd {\em BD} ] 
 \item[\ ~~~~] [ --ms {\em MS} ] [ --g {\em G} ] [ --d {\em D} ]
              [ --t {\em T} ] [ {\em infile} ]
\end{synopsis} 

\begin{qsection}{DESCRIPTION}
This command reads a sequence of plot command from standard input,
and prints the corresponding graph to the screen in a X-Window
environment.
\begin{itemize}
\item Even though the window manager has function to
change the window size, the graph plotted in a virtual screen
with ``xgr'' dose not allow for zooming.
\item  In the case window size is smaller then the virtual screen,
then scroll function can be used in virtual screen(use vi commands).
\begin{quote}
		h: left scroll\\
		j: down scroll\\
		k: up scroll\\
		l: right scroll
\end{quote}
\item To delete the virtual screen, one of the following can be used:
"q","Ctrl-c","Ctrl-d"
\end{itemize}
\end{qsection}

\begin{options}
	\argm{s}{S}{shrink}{3.38667}
	\argm{l}{}{landscape}{FALSE}
	\argm{rv}{}{reverse mode}{FALSE}
	\argm{m}{}{monochrome display mode}{FALSE}
	\argm{bg}{BG}{background color}{white}
	\argm{hl}{HL}{highlight color}{blue}
	\argm{bd}{BD}{border color}{blue}
	\argm{ms}{MS}{mouse color}{red}
	\argm{g}{G}{geometry}{NULL}
	\argm{d}{D}{display}{NULL}
	\argm{t}{T}{window title}{xgr}
\end{options}
\begin{qsection}{EXAMPLE}
The following example uses ``fdrw'' to draw a graph based on data read
from {\em data.f}, and sends the output in a X-Window environment:
\begin{quote}
 \verb!fdrw < data.f | xgr!
\end{quote}
\end{qsection}
\begin{qsection}{BUGS}
\begin{itemize}
\item If the display server does not contain backing store function,
then the hidden part of virtual screen is erased.

\item To lessen the waiting time to display graphs,
a image of virtual screen is copyed to the memory.
If the size assigned by the --g option is too small
or if during the time the graph is being plotted an another window
is put above the virtual screen, then a part of virtual screen
will be erased.
The --s option is suggested whenever the size of
the virtual screen should be reduced.
\end{itemize}

\end{qsection}
\begin{qsection}{SEE ALSO}
fig, fdrw
\end{qsection}
