% ----------------------------------------------------------------
%       Speech Signal Processing Toolkit (SPTK): version 3.0
%                      SPTK Working Group
% 
%                Department of Computer Science
%                Nagoya Institute of Technology
%                             and
%   Interdisciplinary Graduate School of Science and Engineering
%                Tokyo Institute of Technology
%                   Copyright (c) 1984-2000
%                     All Rights Reserved.
% 
% Permission is hereby granted, free of charge, to use and
% distribute this software and its documentation without
% restriction, including without limitation the rights to use,
% copy, modify, merge, publish, distribute, sublicense, and/or
% sell copies of this work, and to permit persons to whom this
% work is furnished to do so, subject to the following conditions:
% 
%   1. The code must retain the above copyright notice, this list
%      of conditions and the following disclaimer.
% 
%   2. Any modifications must be clearly marked as such.
%                                                                        
% NAGOYA INSTITUTE OF TECHNOLOGY, TOKYO INSITITUTE OF TECHNOLOGY,
% SPTK WORKING GROUP, AND THE CONTRIBUTORS TO THIS WORK DISCLAIM
% ALL WARRANTIES WITH REGARD TO THIS SOFTWARE, INCLUDING ALL
% IMPLIED WARRANTIES OF MERCHANTABILITY AND FITNESS, IN NO EVENT
% SHALL NAGOYA INSTITUTE OF TECHNOLOGY, TOKYO INSITITUTE OF
% TECHNOLOGY, SPTK WORKING GROUP, NOR THE CONTRIBUTORS BE LIABLE
% FOR ANY SPECIAL, INDIRECT OR CONSEQUENTIAL DAMAGES OR ANY
% DAMAGES WHATSOEVER RESULTING FROM LOSS OF USE, DATA OR PROFITS,
% WHETHER IN AN ACTION OF CONTRACT, NEGLIGENCE OR OTHER TORTIOUS
% ACTION, ARISING OUT OF OR IN CONNECTION WITH THE USE OR
% PERFORMANCE OF THIS SOFTWARE.
% ----------------------------------------------------------------
%
\hypertarget{gwave}{}
\name{gwave}{draw a waveform}{plotting graphs}

\begin{synopsis}
\item[gwave] [ --s $S$ ] [ --e $E$ ] [ --n $N$ ] [ --i $I$ ] [ --y $ymax$ ]
	       [ --y2 $ymin$ ] [ --p $P$ ] 
\item[\ ~~~~~~~~] [ +{\em type} ]  [ {\em infile} ]

\end{synopsis}

\begin{qsection}{DESCRIPTION}
{\em gwave} converts converts speech waveform data 
from {\em infile} (or standard input) to FP5301 plot format, 
sending the result to standard output. 
The output can viewed with ``xgr''.

{\em gwave} is implemented as a shell script 
that uses the ``fig'' and ``fdrw'' commands.
\end{qsection}

\begin{options}
	\argm{s}{S}{start point}{0}
	\argm{e}{E}{end point}{EOF}
	\argm{n}{N}{data number of one screen\\
		    if this option is omitted,
                    all of the data is plotted on one screen.}{N/A}
	\argm{i}{I}{number of screen}{5}
	\argm{y}{ymax}{maximum amplitude\\
                       if this option is omitted,
                       ymax is maximum value of the input data.}{N/A}
	\argm{y2}{ymin}{minimum amplitude}{-YMAX}
	\argm{p}{P}{pen type($1 \sim 10$)}{1}
	\argp{t}{Input data format\\ 
		\begin{tabular}{llcll} \\[-1ex]
			s & short (2bytes) & \quad &
			i & int (4bytes) \\
			f & float (4bytes) & \quad &
			d & double (8bytes) \\
		\end{tabular}}{f}
\end{options}

\begin{qsection}{EXAMPLE}
This example reads speech waveform file in float format from
{\em data.f} and writes the output in Postscript format to
{\em data.ps}.
\begin{quote}
 \verb!gwave < data.f | psgr > data.ps!
 \end{quote}
\end{qsection}

\begin{qsection}{SEE ALSO}
\hyperlink{fig}{fig},
\hyperlink{fdrw}{fdrw},
\hyperlink{xgr}{xgr},
\hyperlink{psgr}{psgr},
\hyperlink{glogsp}{glogsp},
\hyperlink{grlogsp}{grlogsp}
\end{qsection}
