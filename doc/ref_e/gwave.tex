% ----------------------------------------------------------------- %
%             The Speech Signal Processing Toolkit (SPTK)           %
%             developed by SPTK Working Group                       %
%             http://sp-tk.sourceforge.net/                         %
% ----------------------------------------------------------------- %
%                                                                   %
%  Copyright (c) 1984-2007  Tokyo Institute of Technology           %
%                           Interdisciplinary Graduate School of    %
%                           Science and Engineering                 %
%                                                                   %
%                1996-2014  Nagoya Institute of Technology          %
%                           Department of Computer Science          %
%                                                                   %
% All rights reserved.                                              %
%                                                                   %
% Redistribution and use in source and binary forms, with or        %
% without modification, are permitted provided that the following   %
% conditions are met:                                               %
%                                                                   %
% - Redistributions of source code must retain the above copyright  %
%   notice, this list of conditions and the following disclaimer.   %
% - Redistributions in binary form must reproduce the above         %
%   copyright notice, this list of conditions and the following     %
%   disclaimer in the documentation and/or other materials provided %
%   with the distribution.                                          %
% - Neither the name of the SPTK working group nor the names of its %
%   contributors may be used to endorse or promote products derived %
%   from this software without specific prior written permission.   %
%                                                                   %
% THIS SOFTWARE IS PROVIDED BY THE COPYRIGHT HOLDERS AND            %
% CONTRIBUTORS "AS IS" AND ANY EXPRESS OR IMPLIED WARRANTIES,       %
% INCLUDING, BUT NOT LIMITED TO, THE IMPLIED WARRANTIES OF          %
% MERCHANTABILITY AND FITNESS FOR A PARTICULAR PURPOSE ARE          %
% DISCLAIMED. IN NO EVENT SHALL THE COPYRIGHT OWNER OR CONTRIBUTORS %
% BE LIABLE FOR ANY DIRECT, INDIRECT, INCIDENTAL, SPECIAL,          %
% EXEMPLARY, OR CONSEQUENTIAL DAMAGES (INCLUDING, BUT NOT LIMITED   %
% TO, PROCUREMENT OF SUBSTITUTE GOODS OR SERVICES; LOSS OF USE,     %
% DATA, OR PROFITS; OR BUSINESS INTERRUPTION) HOWEVER CAUSED AND ON %
% ANY THEORY OF LIABILITY, WHETHER IN CONTRACT, STRICT LIABILITY,   %
% OR TORT (INCLUDING NEGLIGENCE OR OTHERWISE) ARISING IN ANY WAY    %
% OUT OF THE USE OF THIS SOFTWARE, EVEN IF ADVISED OF THE           %
% POSSIBILITY OF SUCH DAMAGE.                                       %
% ----------------------------------------------------------------- %
\hypertarget{gwave}{}
\name{gwave}{draw a waveform}{plotting graphs}

\begin{synopsis}
\item[gwave] [ --F $F$] [ --s $S$ ] [ --e $E$ ] [ --n $N$ ] [ --i $I$ ] [ --y $ymax$ ]
               [ --y2 $ymin$ ] [ --p $P$ ] 
\item[\ ~~~~~~~~] [ +{\em type} ]  [ {\em infile} ]

\end{synopsis}

\begin{qsection}{DESCRIPTION}
{\em gwave} converts speech waveform data 
from {\em infile} (or standard input) to FP5301 plot format, 
sending the result to standard output. 
The output can viewed with \hyperlink{xgr}{xgr}.

{\em gwave} is implemented as a shell script 
that uses the \hyperlink{fig}{fig} and \hyperlink{fdrw}{fdrw} commands.
\end{qsection}

\begin{options}
        \argm{F}{F}{factor}{1}
        \argm{s}{S}{start point}{0}
        \argm{e}{E}{end point}{EOF}
        \argm{n}{N}{data number of one screen\\
                    if this option is omitted,
                    all of the data is plotted on one screen.}{N/A}
        \argm{i}{I}{number of screen}{5}
        \argm{y}{ymax}{maximum amplitude\\
                       if this option is omitted,
                       ymax is maximum value of the input data.}{N/A}
        \argm{y2}{ymin}{minimum amplitude}{-YMAX}
        \argm{p}{P}{pen type($1 \sim 10$)}{1}
        \argp{t}{Input data format\\ 
                \begin{tabular}{llcll} \\[-1ex]
                        c & char (1 byte) & \quad &
                        C & unsigned char (1 byte) \\
                        s & short (2 bytes) & \quad &
                        S & unsigned short (2 bytes) \\
                        i3 & int (3 bytes) & \quad &
                        I3 & unsigned int (3 bytes) \\
                        i & int (4 bytes) & \quad &
                        I & unsigned int (4 bytes) \\
                        l & long (4 bytes) & \quad &
                        L & unsigned long (4 bytes) \\
                        le & long long (8 bytes) & \quad &
                        LE & unsigned long long (8 bytes) \\
                        f & float (4 bytes) & \quad &
                        d & double (8 bytes) \\
                        de & long double (12 bytes) & \quad &
                \end{tabular}}{f}
\end{options}

\begin{qsection}{EXAMPLE}
This example reads speech waveform file in float format from
{\em data.f} and writes the output in Postscript format to
{\em data.ps}.
\begin{quote}
 \verb!gwave +f < data.f | psgr > data.ps!
 \end{quote}
\end{qsection}

\begin{qsection}{NOTICE}
\begin{itemize}
\item If options of amplitude are not used, value of amplitude is
automatically determined.
\item If --n option is not used, entire waveform is displayed.
\end{itemize}
\end{qsection}

\begin{qsection}{SEE ALSO}
\hyperlink{fig}{fig},
\hyperlink{fdrw}{fdrw},
\hyperlink{xgr}{xgr},
\hyperlink{psgr}{psgr},
\hyperlink{glogsp}{glogsp},
\hyperlink{grlogsp}{grlogsp}
\end{qsection}
