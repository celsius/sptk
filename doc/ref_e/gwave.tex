\name{gwave}{draw a waveform}{plotting graphs}

\begin{synopsis}
\item[gwave] [ --s $S$ ] [ --e $E$ ] [ --n $N$ ] [ --i $I$ ] [ --y $ymax$ ]
	       [ --y2 $ymin$ ] [ --p $P$ ] 
\item[\ ~~~~~~~~] [ +{\em type} ]  [ {\em infile} ]

\end{synopsis}

\begin{qsection}{DESCRIPTION}
The {\em gwave} command reads speech waveform data from
the standard input and generates an output suitable for
plotting.
This command can be used in connection with commands such as ``xgr''.
\par
Actually, it uses {\em fig} and {\em fdrw} commands through a
shell script.
\end{qsection}

\begin{options}
	\argm{s}{S}{start point}{0}
	\argm{e}{E}{end point}{EOF}
	\argm{n}{N}{data number of one screen\\
		    if this option is omitted,
                    all of the data is plotted on one screen.}{N/A}
	\argm{i}{I}{number of screen}{5}
	\argm{y}{ymax}{maximum amplitude\\
                       if this option is omitted,
                       ymax is maximum value of the input data.}{N/A}
	\argm{y2}{ymin}{minimum amplitude}{-YMAX}
	\argm{p}{P}{pen type($1 \sim 10$)}{1}
	\argp{t}{Input data format\\ 
		\begin{tabular}{llcll} \\[-1zh]
			s & short (2bytes) & \quad &
			i & int (4bytes) \\
			f & float (4bytes) & \quad &
			d & double (8bytes) \\
		\end{tabular}}{f}
\end{options}

\begin{qsection}{EXAMPLE}
This example reads speech waveform file in float format from
{\em data.f} and writes the output in Postscript format to
{\em data.ps}.
\begin{quote}
 \verb!gwave < data.f | psgr > data.ps!
 \end{quote}
\end{qsection}

\begin{qsection}{SEE ALSO}
 fig, fdrw, xgr, psgr, glogsp, grlogsp
\end{qsection}
