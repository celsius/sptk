%  ---------------------------------------------------------------  %
%            Speech Signal Processing Toolkit (SPTK)                %
%                      SPTK Working Group                           %
%                                                                   %
%                  Department of Computer Science                   %
%                  Nagoya Institute of Technology                   %
%                               and                                 %
%   Interdisciplinary Graduate School of Science and Engineering    %
%                  Tokyo Institute of Technology                    %
%                                                                   %
%                     Copyright (c) 1984-2007                       %
%                       All Rights Reserved.                        %
%                                                                   %
%  Permission is hereby granted, free of charge, to use and         %
%  distribute this software and its documentation without           %
%  restriction, including without limitation the rights to use,     %
%  copy, modify, merge, publish, distribute, sublicense, and/or     %
%  sell copies of this work, and to permit persons to whom this     %
%  work is furnished to do so, subject to the following conditions: %
%                                                                   %
%    1. The source code must retain the above copyright notice,     %
%       this list of conditions and the following disclaimer.       %
%                                                                   %
%    2. Any modifications to the source code must be clearly        %
%       marked as such.                                             %
%                                                                   %
%    3. Redistributions in binary form must reproduce the above     %
%       copyright notice, this list of conditions and the           %
%       following disclaimer in the documentation and/or other      %
%       materials provided with the distribution.  Otherwise, one   %
%       must contact the SPTK working group.                        %
%                                                                   %
%  NAGOYA INSTITUTE OF TECHNOLOGY, TOKYO INSTITUTE OF TECHNOLOGY,   %
%  SPTK WORKING GROUP, AND THE CONTRIBUTORS TO THIS WORK DISCLAIM   %
%  ALL WARRANTIES WITH REGARD TO THIS SOFTWARE, INCLUDING ALL       %
%  IMPLIED WARRANTIES OF MERCHANTABILITY AND FITNESS, IN NO EVENT   %
%  SHALL NAGOYA INSTITUTE OF TECHNOLOGY, TOKYO INSTITUTE OF         %
%  TECHNOLOGY, SPTK WORKING GROUP, NOR THE CONTRIBUTORS BE LIABLE   %
%  FOR ANY SPECIAL, INDIRECT OR CONSEQUENTIAL DAMAGES OR ANY        %
%  DAMAGES WHATSOEVER RESULTING FROM LOSS OF USE, DATA OR PROFITS,  %
%  WHETHER IN AN ACTION OF CONTRACT, NEGLIGENCE OR OTHER TORTUOUS   %
%  ACTION, ARISING OUT OF OR IN CONNECTION WITH THE USE OR          %
%  PERFORMANCE OF THIS SOFTWARE.                                    %
%                                                                   %
%  ---------------------------------------------------------------  %
%
\hypertarget{cdist}{}
\name{cdist}{calculation of cepstral distance}{data processing}

\begin{synopsis}
\item [cdist] [ --m $M$ ] [ --o $O$ ] [ --f ] {\em cfile}
 	    [ {\em infile} ] 
\end{synopsis}

\begin{qsection}{DESCRIPTION}
{\em cdist} calculates the cepstral distance 
between the cepstral coefficients 
in {\em infile} (or standard input) and {\em cfile}, 
sending the result to standard output.
For example, if the cepstral coefficients of the {\em infile} at
frame $t$ are 
\begin{displaymath}
   c_{1,t}(0), c_{1,t}(1), c_{1,t}(2), \dots, c_{1,t}(M)
\end{displaymath}
and the cepstral coefficients of the {\em cfile} at frame $t$ are
\begin{displaymath}
   c_{2,t}(0), c_{2,t}(1), c_{2,t}(2), \dots, c_{2,t}(M)
\end{displaymath}
then the squared cepstrum distance for every frame is
\begin{displaymath}
   d(t)=\sum_{k=1}^{M} (c_{1,t}(k)-c_{2,t}(k))^2
\end{displaymath}
and the total cepstral distance between both files is
\begin{displaymath}
   d=\frac{1}{T} \sum_{t=0}^{T} d(t)
\end{displaymath}

If the number of frames in {\em infile} or {\em cfile} is less
then $T$, then evaluation is undertaken for the smallest number of frames.
\end{qsection}

\begin{options}
	\argm{m}{M}{order of minimum-phase cepstrum}{25}
	\argm{o}{O}{output format\\
		\begin{tabular}{ll} \\[-1ex]
		$O=0$ & $  \frac{10}{\ln 10} \sqrt{2d(t)}$ \ \ \ \ [db]\\
		$O=1$ & $  d(t)$ \\
		$O=2$ & $  \sqrt{d(t)}$ \\[1ex]
		\end{tabular}\\
						}{0}
	\argm{f}{}{output frame by frame}{FALSE}
\end{options}

\begin{qsection}{EXAMPLE}
In the example below, the squared spectral distance of the 15-th order
cepstrum files {\em data1.cep} and {\em data2.cep},
both in float formats, is evaluated and displayed:
\begin{quote}
\verb! cdist -m 15 data1.cep data2.cep | dmp !
\end{quote}
\end{qsection}

\begin{qsection}{SEE ALSO}
\hyperlink{acep}{acep},
\hyperlink{agcep}{agcep},
\hyperlink{amcep}{amcep},
\hyperlink{mcep}{mcep}
\end{qsection}
