% ----------------------------------------------------------------- %
%             The Speech Signal Processing Toolkit (SPTK)           %
%             developed by SPTK Working Group                       %
%             http://sp-tk.sourceforge.net/                         %
% ----------------------------------------------------------------- %
%                                                                   %
%  Copyright (c) 1984-2007  Tokyo Institute of Technology           %
%                           Interdisciplinary Graduate School of    %
%                           Science and Engineering                 %
%                                                                   %
%                1996-2010  Nagoya Institute of Technology          %
%                           Department of Computer Science          %
%                                                                   %
% All rights reserved.                                              %
%                                                                   %
% Redistribution and use in source and binary forms, with or        %
% without modification, are permitted provided that the following   %
% conditions are met:                                               %
%                                                                   %
% - Redistributions of source code must retain the above copyright  %
%   notice, this list of conditions and the following disclaimer.   %
% - Redistributions in binary form must reproduce the above         %
%   copyright notice, this list of conditions and the following     %
%   disclaimer in the documentation and/or other materials provided %
%   with the distribution.                                          %
% - Neither the name of the SPTK working group nor the names of its %
%   contributors may be used to endorse or promote products derived %
%   from this software without specific prior written permission.   %
%                                                                   %
% THIS SOFTWARE IS PROVIDED BY THE COPYRIGHT HOLDERS AND            %
% CONTRIBUTORS "AS IS" AND ANY EXPRESS OR IMPLIED WARRANTIES,       %
% INCLUDING, BUT NOT LIMITED TO, THE IMPLIED WARRANTIES OF          %
% MERCHANTABILITY AND FITNESS FOR A PARTICULAR PURPOSE ARE          %
% DISCLAIMED. IN NO EVENT SHALL THE COPYRIGHT OWNER OR CONTRIBUTORS %
% BE LIABLE FOR ANY DIRECT, INDIRECT, INCIDENTAL, SPECIAL,          %
% EXEMPLARY, OR CONSEQUENTIAL DAMAGES (INCLUDING, BUT NOT LIMITED   %
% TO, PROCUREMENT OF SUBSTITUTE GOODS OR SERVICES; LOSS OF USE,     %
% DATA, OR PROFITS; OR BUSINESS INTERRUPTION) HOWEVER CAUSED AND ON %
% ANY THEORY OF LIABILITY, WHETHER IN CONTRACT, STRICT LIABILITY,   %
% OR TORT (INCLUDING NEGLIGENCE OR OTHERWISE) ARISING IN ANY WAY    %
% OUT OF THE USE OF THIS SOFTWARE, EVEN IF ADVISED OF THE           %
% POSSIBILITY OF SUCH DAMAGE.                                       %
% ----------------------------------------------------------------- %
\hypertarget{cdist}{}
\name{cdist}{calculation of cepstral distance}{data processing}

\begin{synopsis}
\item [cdist] [ --m $M$ ] [ --o $O$ ] [ --f ] {\em cfile}
 	    [ {\em infile} ] 
\end{synopsis}

\begin{qsection}{DESCRIPTION}
{\em cdist} calculates the cepstral distance 
between the cepstral coefficients 
in {\em infile} (or standard input) and {\em cfile}, 
sending the result to standard output.
For example, if the cepstral coefficients of the {\em infile} at
frame $t$ are 
\begin{displaymath}
   c_{1,t}(0), c_{1,t}(1), c_{1,t}(2), \dots, c_{1,t}(M)
\end{displaymath}
and the cepstral coefficients of the {\em cfile} at frame $t$ are
\begin{displaymath}
   c_{2,t}(0), c_{2,t}(1), c_{2,t}(2), \dots, c_{2,t}(M)
\end{displaymath}
then the squared cepstrum distance for every frame is
\begin{displaymath}
   d(t)=\sum_{k=1}^{M} (c_{1,t}(k)-c_{2,t}(k))^2
\end{displaymath}
and the total cepstral distance between both files is
\begin{displaymath}
   d=\frac{1}{T} \sum_{t=0}^{T} d(t)
\end{displaymath}

If the number of frames in {\em infile} or {\em cfile} is less
then $T$, then evaluation is undertaken for the smallest number of frames.
\end{qsection}

\begin{options}
	\argm{m}{M}{order of minimum-phase cepstrum}{25}
	\argm{o}{O}{output format\\
		\begin{tabular}{ll} \\[-1ex]
		$O=0$ & $  \frac{10}{\ln 10} \sqrt{2d(t)}$ \ \ \ \ [db]\\
		$O=1$ & $  d(t)$ \\
		$O=2$ & $  \sqrt{d(t)}$ \\[1ex]
		\end{tabular}\\
						}{0}
	\argm{f}{}{output frame by frame}{FALSE}
\end{options}

\begin{qsection}{EXAMPLE}
In the example below, the squared spectral distance of the 15-th order
cepstrum files {\em data1.cep} and {\em data2.cep},
both in float formats, is evaluated and displayed:
\begin{quote}
\verb! cdist -m 15 data1.cep data2.cep | dmp +f!
\end{quote}
\end{qsection}

\begin{qsection}{SEE ALSO}
\hyperlink{acep}{acep},
\hyperlink{agcep}{agcep},
\hyperlink{amcep}{amcep},
\hyperlink{mcep}{mcep}
\end{qsection}
