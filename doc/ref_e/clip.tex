% ----------------------------------------------------------------
%       Speech Signal Processing Toolkit (SPTK): version 3.0
%                      SPTK Working Group
% 
%                Department of Computer Science
%                Nagoya Institute of Technology
%                             and
%   Interdisciplinary Graduate School of Science and Engineering
%                Tokyo Institute of Technology
%                   Copyright (c) 1984-2000
%                     All Rights Reserved.
% 
% Permission is hereby granted, free of charge, to use and
% distribute this software and its documentation without
% restriction, including without limitation the rights to use,
% copy, modify, merge, publish, distribute, sublicense, and/or
% sell copies of this work, and to permit persons to whom this
% work is furnished to do so, subject to the following conditions:
% 
%   1. The code must retain the above copyright notice, this list
%      of conditions and the following disclaimer.
% 
%   2. Any modifications must be clearly marked as such.
%                                                                        
% NAGOYA INSTITUTE OF TECHNOLOGY, TOKYO INSITITUTE OF TECHNOLOGY,
% SPTK WORKING GROUP, AND THE CONTRIBUTORS TO THIS WORK DISCLAIM
% ALL WARRANTIES WITH REGARD TO THIS SOFTWARE, INCLUDING ALL
% IMPLIED WARRANTIES OF MERCHANTABILITY AND FITNESS, IN NO EVENT
% SHALL NAGOYA INSTITUTE OF TECHNOLOGY, TOKYO INSITITUTE OF
% TECHNOLOGY, SPTK WORKING GROUP, NOR THE CONTRIBUTORS BE LIABLE
% FOR ANY SPECIAL, INDIRECT OR CONSEQUENTIAL DAMAGES OR ANY
% DAMAGES WHATSOEVER RESULTING FROM LOSS OF USE, DATA OR PROFITS,
% WHETHER IN AN ACTION OF CONTRACT, NEGLIGENCE OR OTHER TORTIOUS
% ACTION, ARISING OUT OF OR IN CONNECTION WITH THE USE OR
% PERFORMANCE OF THIS SOFTWARE.
% ----------------------------------------------------------------
%
\name{clip}{data clipping}{data processing}

\begin{synopsis}
\item[clip] [ --y $y_{min} \; y_{max}$ ] [ --ymin $y_{min}$ ] [ --ymax $y_{max}$ ] [ {\em infile} ]
\end{synopsis}

\begin{qsection}{DESCRIPTION}
This command reads the input file {\em infile}
as well as the minimum and maximum values,
clips the data, and sends it to the standard output.
The input file must be in float format
and if no input file is specified, the standard input is used instead.
\end{qsection}

\begin{options}
	\argm{y}{y_{min} \; y_{max}}{lower bound \& upper bound}
		{$-1.0 \, 1.0$}
	\argm{ymin} {y_{min}} {lower bound (ymax = inf)}{N/A}
	\argm{ymax} {y_{max}} {upper bound (ymin = -inf)}{N/A}
\end{options}

\begin{qsection}{EXAMPLE}
Suppose that the data of the file {\em data.f} is in float format
with the following values,
\begin{displaymath}
 1.0, 2.0, 3.0, 4.0, 5.0, 6.0
\end{displaymath}
If we type the command
\begin{quote}
 \verb!clip -y 2.5 5.5 < data.f > data.clip!
\end{quote}
the output {\em data.clip} will contain the values below.
\begin{displaymath}
 2.5, 2.5, 3.0, 4.0, 5.0, 5.5
\end{displaymath}
\end{qsection}
