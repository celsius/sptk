\name{clip}{data clipping}{data processing}

\begin{synopsis}
\item[clip] [ --y $y_{min} \; y_{max}$ ] [ --ymin $y_{min}$ ] [ --ymax $y_{max}$ ] [ {\em infile} ]
\end{synopsis}

\begin{qsection}{DESCRIPTION}
This command reads the input file {\em infile}
as well as the minimum and maximum values,
clips the data, and send it to the standard output.
The input file must be in float format
and if no input file is specified, the starndard input is used instead.
\end{qsection}

\begin{options}
	\argm{y}{y_{min} \; y_{max}}{lower bound & upper bound}
		{$-1.0 \, 1.0$}
	\argm{ymin} {y_{min}} {lower bound (ymax = inf)}{N/A}
	\argm{ymax} {y_{max}} {upper bound (ymin = -inf)}{N/A}
\end{options}

\begin{qsection}{EXAMPLE}
Suppose that the data of the file {\em data.f} is in float format
with the following values,
\begin{displaymath}
 1.0, 2.0, 3.0, 4.0, 5.0, 6.0
\end{displaymath}
If we type the following command
\begin{quote}
 \verb!clip -y 2.5 5.5 < data.f > data.clip!
\end{quote}
the output {\em data.clip} will contain these values
\begin{displaymath}
 2.5, 2.5, 3.0, 4.0, 5.0, 5.5
\end{displaymath}
\end{qsection}
