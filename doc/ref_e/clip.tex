% ----------------------------------------------------------------- %
%             The Speech Signal Processing Toolkit (SPTK)           %
%             developed by SPTK Working Group                       %
%             http://sp-tk.sourceforge.net/                         %
% ----------------------------------------------------------------- %
%                                                                   %
%  Copyright (c) 1984-2007  Tokyo Institute of Technology           %
%                           Interdisciplinary Graduate School of    %
%                           Science and Engineering                 %
%                                                                   %
%                1996-2016  Nagoya Institute of Technology          %
%                           Department of Computer Science          %
%                                                                   %
% All rights reserved.                                              %
%                                                                   %
% Redistribution and use in source and binary forms, with or        %
% without modification, are permitted provided that the following   %
% conditions are met:                                               %
%                                                                   %
% - Redistributions of source code must retain the above copyright  %
%   notice, this list of conditions and the following disclaimer.   %
% - Redistributions in binary form must reproduce the above         %
%   copyright notice, this list of conditions and the following     %
%   disclaimer in the documentation and/or other materials provided %
%   with the distribution.                                          %
% - Neither the name of the SPTK working group nor the names of its %
%   contributors may be used to endorse or promote products derived %
%   from this software without specific prior written permission.   %
%                                                                   %
% THIS SOFTWARE IS PROVIDED BY THE COPYRIGHT HOLDERS AND            %
% CONTRIBUTORS "AS IS" AND ANY EXPRESS OR IMPLIED WARRANTIES,       %
% INCLUDING, BUT NOT LIMITED TO, THE IMPLIED WARRANTIES OF          %
% MERCHANTABILITY AND FITNESS FOR A PARTICULAR PURPOSE ARE          %
% DISCLAIMED. IN NO EVENT SHALL THE COPYRIGHT OWNER OR CONTRIBUTORS %
% BE LIABLE FOR ANY DIRECT, INDIRECT, INCIDENTAL, SPECIAL,          %
% EXEMPLARY, OR CONSEQUENTIAL DAMAGES (INCLUDING, BUT NOT LIMITED   %
% TO, PROCUREMENT OF SUBSTITUTE GOODS OR SERVICES; LOSS OF USE,     %
% DATA, OR PROFITS; OR BUSINESS INTERRUPTION) HOWEVER CAUSED AND ON %
% ANY THEORY OF LIABILITY, WHETHER IN CONTRACT, STRICT LIABILITY,   %
% OR TORT (INCLUDING NEGLIGENCE OR OTHERWISE) ARISING IN ANY WAY    %
% OUT OF THE USE OF THIS SOFTWARE, EVEN IF ADVISED OF THE           %
% POSSIBILITY OF SUCH DAMAGE.                                       %
% ----------------------------------------------------------------- %
\hypertarget{clip}{}
\name{clip}{data clipping}{data processing}

\begin{synopsis}
\item[clip] [ --y $y_{min} \; y_{max}$ ] [ --ymin $y_{min}$ ] [ --ymax $y_{max}$ ] [ {\em infile} ]
\end{synopsis}

\begin{qsection}{DESCRIPTION}
{\em clip} clips the data from {\em infile} (or standard input) 
between the minimum and maximum values specified on the command line, 
sending the result to standard output.

Input and output data are in float format.
\end{qsection}

\begin{options}
	\argm{y}{y_{min} \; y_{max}}{lower bound \& upper bound}
		{$-1.0 \, 1.0$}
	\argm{ymin} {y_{min}} {lower bound (ymax = inf)}{N/A}
	\argm{ymax} {y_{max}} {upper bound (ymin = -inf)}{N/A}
\end{options}

\begin{qsection}{EXAMPLE}
Suppose that the data in {\em data.f} is in float format
and presents the following values,
\begin{displaymath}
 1.0, 2.0, 3.0, 4.0, 5.0, 6.0
\end{displaymath}
If we type the command
\begin{quote}
 \verb!clip -y 2.5 5.5 < data.f > data.clip!
\end{quote}
then the output {\em data.clip} will contain the following values.
\begin{displaymath}
 2.5, 2.5, 3.0, 4.0, 5.0, 5.5
\end{displaymath}
\end{qsection}
