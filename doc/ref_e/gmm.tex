% ----------------------------------------------------------------- %
%             The Speech Signal Processing Toolkit (SPTK)           %
%             developed by SPTK Working Group                       %
%             http://sp-tk.sourceforge.net/                         %
% ----------------------------------------------------------------- %
%                                                                   %
%  Copyright (c) 1984-2007  Tokyo Institute of Technology           %
%                           Interdisciplinary Graduate School of    %
%                           Science and Engineering                 %
%                                                                   %
%                1996-2012  Nagoya Institute of Technology          %
%                           Department of Computer Science          %
%                                                                   %
% All rights reserved.                                              %
%                                                                   %
% Redistribution and use in source and binary forms, with or        %
% without modification, are permitted provided that the following   %
% conditions are met:                                               %
%                                                                   %
% - Redistributions of source code must retain the above copyright  %
%   notice, this list of conditions and the following disclaimer.   %
% - Redistributions in binary form must reproduce the above         %
%   copyright notice, this list of conditions and the following     %
%   disclaimer in the documentation and/or other materials provided %
%   with the distribution.                                          %
% - Neither the name of the SPTK working group nor the names of its %
%   contributors may be used to endorse or promote products derived %
%   from this software without specific prior written permission.   %
%                                                                   %
% THIS SOFTWARE IS PROVIDED BY THE COPYRIGHT HOLDERS AND            %
% CONTRIBUTORS "AS IS" AND ANY EXPRESS OR IMPLIED WARRANTIES,       %
% INCLUDING, BUT NOT LIMITED TO, THE IMPLIED WARRANTIES OF          %
% MERCHANTABILITY AND FITNESS FOR A PARTICULAR PURPOSE ARE          %
% DISCLAIMED. IN NO EVENT SHALL THE COPYRIGHT OWNER OR CONTRIBUTORS %
% BE LIABLE FOR ANY DIRECT, INDIRECT, INCIDENTAL, SPECIAL,          %
% EXEMPLARY, OR CONSEQUENTIAL DAMAGES (INCLUDING, BUT NOT LIMITED   %
% TO, PROCUREMENT OF SUBSTITUTE GOODS OR SERVICES; LOSS OF USE,     %
% DATA, OR PROFITS; OR BUSINESS INTERRUPTION) HOWEVER CAUSED AND ON %
% ANY THEORY OF LIABILITY, WHETHER IN CONTRACT, STRICT LIABILITY,   %
% OR TORT (INCLUDING NEGLIGENCE OR OTHERWISE) ARISING IN ANY WAY    %
% OUT OF THE USE OF THIS SOFTWARE, EVEN IF ADVISED OF THE           %
% POSSIBILITY OF SUCH DAMAGE.                                       %
% ----------------------------------------------------------------- %
\hypertarget{gmm}{}
\name{gmm}{GMM parameter estimation}{model training}

\begin{synopsis}
\item [gmm] [ --l $L$ ] [ --m $M$ ] [ --t $T$ ] [ --s $S$ ] [ --a $A$ ] [ --b $B$ ]
        [ --e $E$ ] [ --v $V$ ] [ --w $W$ ] [ --f ]
\item [\ ~~~~]  [ --F $gmmfile$ ] [ {\em infile} ]
\end{synopsis}

\newfont{\bggg}{cmr10 scaled\magstep3}
\newcommand{\bigzerol}{\smash{\hbox{\bggg 0}}}
\newcommand{\bigzerou}{\smash{\lower1.3ex\hbox{\bggg 0}}}

\begin{qsection}{DESCRIPTION}
{\em gmm} uses the expectation maximization (EM) algorithm to estimate
Gaussian mixture model (GMM) parameters with diagonal covariance
matrices, from a sequence of vectors in the {\em infile} (or standard
input), sending the result to standard output.

The input sequence $\bX$ consists of $T$ float vectors $\bx$, each of
size $L$:
\begin{align}
 &\bX=\left[\bx(0), \bx(1), \dots, \bx(T-1)\right]\notag,\\
 &\bx(t)=\left[x_t(0), x_t(1), \ldots, x_t(L-1)\right].\notag
\end{align}
The result is GMM parameters $\lambda$ consisting of $M$ mixture weights
$\bw$ and $M$ Gaussians with mean vector $\bmu$ and variance vector
$\bv$, each of length $L$:
\begin{align}
 \lambda =
 \left[\bw,\right.&\left.\bmu(0),\bv(0), \bmu(1), \bv(1),
 \ldots, \bmu(M-1), \bv(M-1)\right],\notag\\[2mm]
 \bw &=\left[ w(0), w(1), \ldots, w(M-1) \right],\notag\\
 \bmu(m) &=\left[\mu_m(0), \mu_m(1), \ldots, \mu_m(L-1)\right],\notag\\
 \bv(m) &=\left[\sigma_m^2(0), \sigma_m^2(1), \ldots,
 \sigma_m^2(L-1)\right],\notag
\end{align}
where
\begin{displaymath}
 \sum_{m=0}^{M-1}w(m)=1.
\end{displaymath}

The GMM parameter set $\lambda$ is initialized by an LBG algorithm and
the following EM steps are used iteratively to obtain the new parameter set
$\hat{\lambda}$:
\begin{align}
   \hat{w}(m) & = \frac{1}{T}
   \sum_{t=0}^{T-1}p(m\mid\bx(t),\lambda),\notag\\
  \hat{\bmu}(m) & 
 = \frac{\sum_{t=0}^{T-1}p(m\mid\bx(t),\lambda)\bx(t)}
 {\sum_{t=0}^{T-1}p(m\mid\bx(t),\lambda)}, \notag\\[2mm]
  \hat{\sigma}_m^2(l)&
  =\frac{\sum_{t=0}^{T-1}p(m\mid\bx(t),\lambda)x_t^2(l)}
  {\sum_{t=0}^{T-1}p(m\mid\bx(t),\lambda)}
 - \hat{\mu}_m^2(l),\notag
\end{align}
where $p(m\mid\bx(t),\lambda)$ is the posterior probability of being in
the $m$-th component at time $t$ and is given by:
\begin{displaymath}
 p(m\mid\bx(t),\lambda)
 = \frac{w(m){\cal N}(\bx(t)\mid \bmu(m),\bv(m))}
 {\sum_{k=0}^{M-1}w(k){\cal N}(\bx(t)\mid \bmu(k),\bv(k))},
\end{displaymath}
 where
\begin{align}
 {\cal N}(\bx(t)\mid\bmu(m),\bv(m))%
 &=\frac{1}{(2\pi)^{L/2}\,|\Sigma(m)|^{1/2}}%
 \exp{\left\{-\frac{1}{2}%
 (\bx(t)-\bmu(m))'\,\Sigma(m)^{-1}\,%
 (\bx(t)-\bmu(m))\right\}}\notag\\
 &=\frac{1}{(2\pi)^{L/2}\prod_{l=0}^{L-1}\sigma_m(l)}%
 \exp{\left\{-\frac{1}{2}%
 \sum_{l=0}^{L-1}
 \frac{\left(x_t(l)-\mu_m(l)\right)^2}%
 {\sigma_m^2(l)}\right\}},\notag
\end{align}
and $\Sigma(m)$ is a diagonal matrix with diagonal elements
 $\bv(m)$:
\begin{displaymath}
 \Sigma(m)=\left[
 \begin{array}{cccc}
  \sigma_m^2(0) & & &\bigzerou\\
  & \sigma_m^2(1) & &\\
  & & \ddots &\\
  \bigzerol & & & \sigma_m^2(L-1)\\
 \end{array}\right].\notag
\end{displaymath}

Also, the Average log-likelihood for training data $X$
\begin{displaymath}
  \log P(\bX)
 =\frac{1}{T}\sum_{t=0}^{T-1}
 \log\sum_{m=0}^{M-1}w(m){\cal N}(\bx(t)\mid\bmu(m),\bv(m))
\end{displaymath}
is increased by iterating the above steps. The average log-probability $\log
P(\bX)$ at each iterative step is printed on the standard error output.
The EM steps are iterated at least $A$ times and stopped at the $B$-th
iteration or when there is a small absolute change in $\log P(\bX)~(\leq
E)$.

\end{qsection}

\begin{options}
 \argm{l}{L}{length of vector}{26}
 \argm{m}{M}{number of Gaussian components}{16}
 \argm{t}{T}{number of training vectors}{N/A}
 \argm{s}{S}{seed of random variable for LBG algorithm}{1}
 \argm{a}{A}{minimum number of EM iterations}{0}
 \argm{b}{B}{maximum number of EM iterations (A$\leq$ B)}{20}
 \argm{e}{E}{end condition for EM iteration}{0.00001}
 \argm{v}{V}{flooring value for variances}{0.001}
 \argm{w}{W}{flooring value for weights (1/M)*W}{0.001}
 \argm{f}{}{full covariance}{FALSE}
 \argm{F}{fn}{GMM initial parameter file}{N/A}
\end{options}

\begin{qsection}{EXAMPLE}
In the following example, a GMM with 8 Gaussian components is generated
from training vectors {\em data.f} in float format, and GMM parameters
are written to {\em gmm.f}.
\begin{quote}
\verb! gmm -m 8 data.f > gmm.f!
\end{quote}
If one wants to model GMMs with full covariances,
 one can use the -f option.
\begin{quote}
\verb! gmm -m 8 -f data.f > gmm.f! 
\end{quote}
If one wants to specify GMM initial parameter or resume GMM training,
 one can use the -F option specifying GMM initial parameter file {\em gmm.init}.
\begin{quote}
\verb! gmm -m 8 -f data.f -F gmm.init > gmm.f! 
\end{quote}

\end{qsection}

\begin{qsection}{SEE ALSO}
\hyperlink{gmmp}{gmmp},
\hyperlink{lbg}{lbg}
\end{qsection}
