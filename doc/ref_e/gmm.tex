% ----------------------------------------------------------------- %
%             The Speech Signal Processing Toolkit (SPTK)           %
%             developed by SPTK Working Group                       %
%             http://sp-tk.sourceforge.net/                         %
% ----------------------------------------------------------------- %
%                                                                   %
%  Copyright (c) 1984-2007  Tokyo Institute of Technology           %
%                           Interdisciplinary Graduate School of    %
%                           Science and Engineering                 %
%                                                                   %
%                1996-2008  Nagoya Institute of Technology          %
%                           Department of Computer Science          %
%                                                                   %
% All rights reserved.                                              %
%                                                                   %
% Redistribution and use in source and binary forms, with or        %
% without modification, are permitted provided that the following   %
% conditions are met:                                               %
%                                                                   %
% - Redistributions of source code must retain the above copyright  %
%   notice, this list of conditions and the following disclaimer.   %
% - Redistributions in binary form must reproduce the above         %
%   copyright notice, this list of conditions and the following     %
%   disclaimer in the documentation and/or other materials provided %
%   with the distribution.                                          %
% - Neither the name of the SPTK working group nor the names of its %
%   contributors may be used to endorse or promote products derived %
%   from this software without specific prior written permission.   %
%                                                                   %
% THIS SOFTWARE IS PROVIDED BY THE COPYRIGHT HOLDERS AND            %
% CONTRIBUTORS "AS IS" AND ANY EXPRESS OR IMPLIED WARRANTIES,       %
% INCLUDING, BUT NOT LIMITED TO, THE IMPLIED WARRANTIES OF          %
% MERCHANTABILITY AND FITNESS FOR A PARTICULAR PURPOSE ARE          %
% DISCLAIMED. IN NO EVENT SHALL THE COPYRIGHT OWNER OR CONTRIBUTORS %
% BE LIABLE FOR ANY DIRECT, INDIRECT, INCIDENTAL, SPECIAL,          %
% EXEMPLARY, OR CONSEQUENTIAL DAMAGES (INCLUDING, BUT NOT LIMITED   %
% TO, PROCUREMENT OF SUBSTITUTE GOODS OR SERVICES; LOSS OF USE,     %
% DATA, OR PROFITS; OR BUSINESS INTERRUPTION) HOWEVER CAUSED AND ON %
% ANY THEORY OF LIABILITY, WHETHER IN CONTRACT, STRICT LIABILITY,   %
% OR TORT (INCLUDING NEGLIGENCE OR OTHERWISE) ARISING IN ANY WAY    %
% OUT OF THE USE OF THIS SOFTWARE, EVEN IF ADVISED OF THE           %
% POSSIBILITY OF SUCH DAMAGE.                                       %
% ----------------------------------------------------------------- %
\hypertarget{gmm}{}
\name{gmm}{EM estimation of GMM}{data processing}

\begin{synopsis}
\item[gmm] [ --l $L$ ] [ --m $M$ ] [ --t $T$ ] [ --a A ] [ --b N ]
           [ --e $E$ ] [ --v $V$ ] [ --w $W$ ] [ {\em infile} ]
\end{synopsis}

\begin{qsection}{DESCRIPTION}
{\em gmm} estimates the Gaussian mixture model (GMM) which consists of $M$ Gaussian components 
from {\em infile} (or standard input), sending the estimated GMM parameters to standard output.

For $L$-dimensional input vectors $\bx_1, \dots, \bx_T$,
{\em gmm} estimates a GMM $\lambda$ with $M$ Gaussian components with the $L \times 1$ mean vectors 
$\{ \bmu_m \}_{m=1}^M$ and the $L\times L$ diagonal covariance matrices $\{ \bSigma_m \}_{m=1}^M$
so as to maximize its likelihood to the training data using the EM algorithm. 
That is, 
\begin{align}
   P( \bx_1, \dots, \bx_T \mid \lambda ) &= \prod_{t=1}^{T} 
   \sum_{m=1}^{M} w_m \mathcal{N} \left( \bx_t \; ; \; \bmu_m, \bSigma_m \right) \notag \\
   \hat{\lambda} &= \arg \max_\lambda P( \bx_1, \dots, \bx_T \mid \lambda ) \notag
\end{align}
The output format is 
\begin{align}
    & w_1, \dots, w_M, \notag\\
    & \mu_1(1), \dots, \mu_1(L),
      \sigma^2_1(1), \dots, \sigma^2_1(L), \notag \\
    & \mu_2(1), \dots, \mu_2(L),
      \sigma^2_2(1), \dots, \sigma^2_2(L), \notag \\
    & 
      \qquad \qquad \vdots \notag \\
    & \mu_M(1), \dots, \mu_M(L),
      \sigma^2_M(1), \dots, \sigma^2_M(L) \notag
 \end{align}
\end{qsection}

\begin{options}
	\argm{l}{L}{length of vector}{26}
	\argm{m}{M}{number of Gaussian components}{16}
	\argm{t}{T}{number of training vectors}{N/A}
	\argm{a}{A}{minimum number of EM iterations}{0}
	\argm{b}{B}{maximum number of EM iterations}{20}
	\argm{e}{E}{convergence factor}{1e-05}
	\argm{v}{V}{flooring value for variance}{0.001}
	\argm{v}{V}{flooring value for weights (1/M)*w}{0.001}
\end{options}

\begin{qsection}{EXAMPLE}
The output file {\em data.gmm} contains the weights, means and diagonal elements of covariance matrices 
of Gaussian components estimated from the whole data in {\em data.f} read in float format.
\begin{quote}
  \verb!gmm data.f > data.stat!
\end{quote}

In the example below, an 8-mixture GMM estimated from 15-th order coefficients vector from
{\em data.f} is sent to {\em data.gmm}:
\begin{quote}
  \verb!gmm -l 15 -m 8 data.f > data.gmm!
\end{quote}
\end{qsection}

\begin{qsection}{SEE ALSO}
\hyperlink{average}{average},
\hyperlink{vsum}{vsum}
\hyperlink{vstat}{vstat}
\hyperlink{vq}{vq}
\hyperlink{msvq}{msvq}
\end{qsection}
