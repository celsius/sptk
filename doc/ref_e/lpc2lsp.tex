% ----------------------------------------------------------------- %
%             The Speech Signal Processing Toolkit (SPTK)           %
%             developed by SPTK Working Group                       %
%             http://sp-tk.sourceforge.net/                         %
% ----------------------------------------------------------------- %
%                                                                   %
%  Copyright (c) 1984-2007  Tokyo Institute of Technology           %
%                           Interdisciplinary Graduate School of    %
%                           Science and Engineering                 %
%                                                                   %
%                1996-2013  Nagoya Institute of Technology          %
%                           Department of Computer Science          %
%                                                                   %
% All rights reserved.                                              %
%                                                                   %
% Redistribution and use in source and binary forms, with or        %
% without modification, are permitted provided that the following   %
% conditions are met:                                               %
%                                                                   %
% - Redistributions of source code must retain the above copyright  %
%   notice, this list of conditions and the following disclaimer.   %
% - Redistributions in binary form must reproduce the above         %
%   copyright notice, this list of conditions and the following     %
%   disclaimer in the documentation and/or other materials provided %
%   with the distribution.                                          %
% - Neither the name of the SPTK working group nor the names of its %
%   contributors may be used to endorse or promote products derived %
%   from this software without specific prior written permission.   %
%                                                                   %
% THIS SOFTWARE IS PROVIDED BY THE COPYRIGHT HOLDERS AND            %
% CONTRIBUTORS "AS IS" AND ANY EXPRESS OR IMPLIED WARRANTIES,       %
% INCLUDING, BUT NOT LIMITED TO, THE IMPLIED WARRANTIES OF          %
% MERCHANTABILITY AND FITNESS FOR A PARTICULAR PURPOSE ARE          %
% DISCLAIMED. IN NO EVENT SHALL THE COPYRIGHT OWNER OR CONTRIBUTORS %
% BE LIABLE FOR ANY DIRECT, INDIRECT, INCIDENTAL, SPECIAL,          %
% EXEMPLARY, OR CONSEQUENTIAL DAMAGES (INCLUDING, BUT NOT LIMITED   %
% TO, PROCUREMENT OF SUBSTITUTE GOODS OR SERVICES; LOSS OF USE,     %
% DATA, OR PROFITS; OR BUSINESS INTERRUPTION) HOWEVER CAUSED AND ON %
% ANY THEORY OF LIABILITY, WHETHER IN CONTRACT, STRICT LIABILITY,   %
% OR TORT (INCLUDING NEGLIGENCE OR OTHERWISE) ARISING IN ANY WAY    %
% OUT OF THE USE OF THIS SOFTWARE, EVEN IF ADVISED OF THE           %
% POSSIBILITY OF SUCH DAMAGE.                                       %
% ----------------------------------------------------------------- %
\hypertarget{lpc2lsp}{}
\name{lpc2lsp}{transform LPC to LSP}{speech parameter transformation}

\begin{synopsis}
\item [lpc2lsp] [ --m $M$ ] [ --s $S$ ] [ --k ] [ --L ] [ --o $O$ ] [ --n $N$ ]
		[ --p $P$ ] [ --q $Q$ ] [ --d $D$ ] 
\item [\ ~~~~~~~~] [ {\em infile} ] 
\end{synopsis}

\begin{qsection}{DESCRIPTION}
{\em lpc2lsp} calculates line spectral pair (LSP) coefficients 
from $M$-th order linear prediction (LPC) coefficients 
from {\em infile} (or standard input),
sending the result to standard output.

Although the gain $K$ is included in the LPC input vectors as follows
\begin{displaymath}
  K, a(1), \dots, a(M)
\end{displaymath}
$K$ is not used in the calculation of the LSP coefficients.

The $M$-th order polynomial linear prediction equation $A(z)$ is
\begin{displaymath}
  A_M(z) = 1 + \sum_{m=1}^M a(m) z^{-m}
\end{displaymath}
The PARCOR coefficients satisfy the following equations.
\begin{align}
  A_m(z) &= A_{m-1}(z) - k(m) B_{m-1}(z) \notag \\
  B_m(z) &= z^{-1} (B_{m-1}(z) - k(m) A_{m-1}(z)) \notag
\end{align}
Also, the initial conditions are set as follows,
\begin{align}
  A_0(z) &= 1 \notag \\
  B_0(z) &= z^{-1}.
\end{align}
When the linear prediction polynomial equation
of $M$-th order $A_M(z)$ are given, and the evaluation of $A_{M+1}(z)$
is obtained with the value of $k(M+1)$ set to $1$ or $-1$, then
$P(z)$ and $Q(z)$ are defined as follow.
\begin{align}
  P(z) &= A_M(z) - B_M(z) \notag \\
  Q(z) &= A_M(z) + B_M(z) \notag
\end{align}
Making $k(M+1)$ equal to $\pm 1$ means that,
regarding PARCOR coefficients,
the boundary condition for the glottis of the fixed vocal tract model
satisfies a perfect reflection characteristic.
Also, $A_M(z)$ can be written as
\begin{displaymath}
  A_M(z) =  \frac{P(z) + Q(z)}{2}.
\end{displaymath}
Also, to make sure the roots of $A_M(z)=0$ will all be inside
the unit circle, i.e. to make sure $A_M(z)$ is stable, the following
conditions must be met.
\begin{itemize}
\item All of the roots of $P(z)=0$ and $Q(z)=0$ are on the unit circle
      line.
\item the roots of $P(z)=0$ and $Q(z)=0$ should be above the unit
      circle line and intercalate.
\end{itemize}
\par
If we assume that $M$ is an even number, then
$P(z)$ and $Q(z)$ can be factorized as follows.
\begin{align}
  P(z) &= ( 1 - z^{-1} ) \prod_{i=2,4,\dots,M}
          ( 1 - 2 z^{-1} \cos \omega_i + z^{-2} ) \notag \\
  Q(z) &= ( 1 + z^{-1} ) \prod_{i=1,3,\dots,M-1}
          ( 1 - 2 z^{-1} \cos \omega_i + z^{-2} ) \notag
\end{align}
Also, the values of $\omega_i$ will satisfy the following ordering condition.
\begin{displaymath}
  0 < \omega_1 < \omega_2 < \dots < \omega_{M-1} < \omega_M < \pi
\end{displaymath}
If $M$ is an odd number, a solution can be found in a similar way.

The coefficients $\omega_i$ obtained through factorization are called
LSP coefficients.
\end{qsection}

\begin{options}
	\argm{m}{M}{order of LPC}{25}
	\argm{s}{S}{sampling frequency (kHz)}{10.0}
	\argm{k}{}{output gain}{TRUE}
	\argm{L}{}{output log gain instead of linear gain}{FALSE}
	\argm{o}{O}{output format \\
		\begin{tabular}{ll} \\[-1ex]
			$0$ & normalized frequency $(0 \dots \pi)$ \\
			$1$ & normalized frequency $(0 \dots 0.5)$ \\
			$2$ & frequency (kHz) \\
			$3$ & frequency (Hz)  \\
		\end{tabular}\\\hspace*{\fill}}{0}
	\desc[0.6ex]{Usually, the options below do not need to be assigned.}
	\argm{n}{N}{split number of unit circle}{128}
	\argm{p}{P}{maximum number of interpolation}{4}
	\argm{d}{D}{end condition of interpolation}{1e-06}
\end{options}

\begin{qsection}{EXAMPLE}
In the following example, speech data is read in float format from
{\em data.f}, 10-th order LPC coefficients are calculated,
and the LSP coefficients are evaluated and written to {\em data.lsp}:
\begin{quote}
\verb!frame < data.f | window | lpc -m 10 |\!\\
\verb!lpc2lsp -m 10 > data.lsp!
\end{quote}
\end{qsection}

\begin{qsection}{SEE ALSO}
\hyperlink{lpc}{lpc},
\hyperlink{lsp2lpc}{lsp2lpc},
\hyperlink{lspdf}{lspdf}
\end{qsection}
