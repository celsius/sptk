% ----------------------------------------------------------------
%       Speech Signal Processing Toolkit (SPTK): version 3.0
%                      SPTK Working Group
% 
%                Department of Computer Science
%                Nagoya Institute of Technology
%                             and
%   Interdisciplinary Graduate School of Science and Engineering
%                Tokyo Institute of Technology
%                   Copyright (c) 1984-2000
%                     All Rights Reserved.
% 
% Permission is hereby granted, free of charge, to use and
% distribute this software and its documentation without
% restriction, including without limitation the rights to use,
% copy, modify, merge, publish, distribute, sublicense, and/or
% sell copies of this work, and to permit persons to whom this
% work is furnished to do so, subject to the following conditions:
% 
%   1. The code must retain the above copyright notice, this list
%      of conditions and the following disclaimer.
% 
%   2. Any modifications must be clearly marked as such.
%                                                                        
% NAGOYA INSTITUTE OF TECHNOLOGY, TOKYO INSITITUTE OF TECHNOLOGY,
% SPTK WORKING GROUP, AND THE CONTRIBUTORS TO THIS WORK DISCLAIM
% ALL WARRANTIES WITH REGARD TO THIS SOFTWARE, INCLUDING ALL
% IMPLIED WARRANTIES OF MERCHANTABILITY AND FITNESS, IN NO EVENT
% SHALL NAGOYA INSTITUTE OF TECHNOLOGY, TOKYO INSITITUTE OF
% TECHNOLOGY, SPTK WORKING GROUP, NOR THE CONTRIBUTORS BE LIABLE
% FOR ANY SPECIAL, INDIRECT OR CONSEQUENTIAL DAMAGES OR ANY
% DAMAGES WHATSOEVER RESULTING FROM LOSS OF USE, DATA OR PROFITS,
% WHETHER IN AN ACTION OF CONTRACT, NEGLIGENCE OR OTHER TORTIOUS
% ACTION, ARISING OUT OF OR IN CONNECTION WITH THE USE OR
% PERFORMANCE OF THIS SOFTWARE.
% ----------------------------------------------------------------
%
\name{lpc2lsp}{transform LPC to LSP}{speech parameter transformation}

\begin{synopsis}
\item [lpc2lsp] [ --m $M$ ] [ --s $S$ ] [ --k ] [ --o $O$ ] [ --n $N$ ]
		[ --p $P$ ] [ --q $Q$ ] [ --d $D$ ] 
\item [\ ~~~~~~~~] [ {\em infile} ] 
\end{synopsis}

\begin{qsection}{DESCRIPTION}
The {\em lpc2lsp} command reads the linear prediction coefficients
\begin{displaymath}
  K, a(1), \ldots, a(M)
\end{displaymath}
from the assigned file, calculates LSP coefficients and
sends the result to the standard output.
Note that the gain $K$ is included but it is not used
in the evaluation of the LSP coefficients.
\par
The $M$ order polynomial linear prediction equation $A(z)$ is
\begin{displaymath}
  A_M(z) = 1 + \sum_{m=1}^M a(m) z^{-m}
\end{displaymath}
The PARCOR coefficients satisfy the following equations.
\begin{eqnarray*}
  A_m(z) &=& A_{m-1}(z) - k(m) B_{m-1}(z) \\
  B_m(z) &=& z^{-1} (B_{m-1}(z) - k(m) A_{m-1}(z))
\end{eqnarray*}
Also, the initial conditions are set as follows,
\begin{displaymath}
  A_0(z) = 1,~~~B_0(z) = z^{-1}
\end{displaymath}
When we are given the linear prediction polynomial equation
of $M$ order $A_M(z)$, and when the evaluation of $A_{M+1}(z)$
is obtained with the value of $k(M+1)$ equal to $1$ or $-1$, 
$P(z)$ and $Q(z)$ are defined as follow.
\begin{eqnarray*}
  P(z) &=& A_M(z) - B_M(z) \\
  Q(z) &=& A_M(z) + B_M(z)
\end{eqnarray*}
Making $k(M+1)$ equal to $\pm 1$ is means that,
with respect PARCOR coefficients,
the boundary condition for the glottis of the fixed vocal tract model
satisfies a perfect reflection characteristic.
Also, $A_M(z)$ can be expressed as
\begin{displaymath}
  A_M(z) = ( P(z) + Q(z) ) / 2.
\end{displaymath}
When we express $A_M(z)$ in this way,
$A_M(z)$ is stable.
That is for the roots of $A_M(z)=0$ to be all inside
the unit circle a necessary and sufficient condition is given
in the following.
\begin{itemize}
\item All of the roots of $P(z)=0$ and $Q(z)=0$ are on the unit circle
      line.
\item the roots of $P(z)=0$ and $Q(z)=0$ should be above the unit
      circle line and intercalate.
\end{itemize}
In other words, if  the roots of $P(z)=0$ and $Q(z)=0$ satisfy the
above condition, then $A_M(z)$ is stable.
\par
If we assume that $M$ is a even number, then
$P(z)$ and $Q(z)$ can be factorized as follows.
\begin{eqnarray*}
  P(z) &=& ( 1 - z^{-1} ) \prod_{i=2,4,\cdots,M}
           ( 1 - 2 z^{-1} \cos \omega_i + z^{-2} ) \\
  Q(z) &=& ( 1 + z^{-1} ) \prod_{i=1,3,\cdots,M-1}
           ( 1 - 2 z^{-1} \cos \omega_i + z^{-2} )
\end{eqnarray*}
Also, the values of $\omega_i$ satisfy the following ordering condition.
\begin{displaymath}
  0 < \omega_1 < \omega_2 < \cdots < \omega_{M-1} < \omega_M < \pi
\end{displaymath}
In the case, $M$ is odd number solution can be found in a similar way.
The coefficients $\omega_i$ obtained through factorization are called
LSP coefficients.
\end{qsection}

\begin{options}
	\argm{m}{M}{order of LPC}{25}
	\argm{s}{S}{sampling frequency (kHz)}{10}
	\argm{k}{}{output gain}{TRUE}
                 0 (normalized frequency [0...pi])
                 1 (normalized frequency [0...0.5])
                 2 (frequency [kHz])
                 3 (frequency [Hz])
	\argm{o}{O}{output format \\
		\begin{tabular}{ll} \\[-1zh]
			$0$ & normalized frequency $(0 \ldots \pi)$ \\
			$1$ & normalized frequency $(0 \ldots 0.5)$ \\
			$2$ & frequency (kHz) \\
			$3$ & frequency (Hz)  \\
		\end{tabular}\\\hspace*{\fill}}{0}
	\desc[0.6zh]{Usually, the options below do not need to be assigned.}
	\argm{n}{N}{split number of unit circle}{128}
	\argm{p}{P}{maximum number of interpolation for $P(z)$}{4}
	\argm{q}{Q}{maximum number of interpolation for $Q(z)$}{15}
	\argm{d}{D}{end condition of interpolation}{1e-06}
\end{options}

\begin{qsection}{EXAMPLE}
In the following example, speech data is read in float format from
{\em data.f}, 10 order LPC coefficients are calculated,
and the LSP coefficients are evaluated and written to {\em data.lsp}:
\begin{quote}
\verb!frame < data.f | window | lpc -m 10 |\!\\
\verb!lpc2lsp -m 10 > data.lsp!
\end{quote}
\end{qsection}

\begin{qsection}{SEE ALSO}
lpc, lsp2lpc, lspdf
\end{qsection}
