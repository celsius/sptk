% ----------------------------------------------------------------
%       Speech Signal Processing Toolkit (SPTK): version 3.0
%                      SPTK Working Group
% 
%                Department of Computer Science
%                Nagoya Institute of Technology
%                             and
%   Interdisciplinary Graduate School of Science and Engineering
%                Tokyo Institute of Technology
%                   Copyright (c) 1984-2000
%                     All Rights Reserved.
% 
% Permission is hereby granted, free of charge, to use and
% distribute this software and its documentation without
% restriction, including without limitation the rights to use,
% copy, modify, merge, publish, distribute, sublicense, and/or
% sell copies of this work, and to permit persons to whom this
% work is furnished to do so, subject to the following conditions:
% 
%   1. The code must retain the above copyright notice, this list
%      of conditions and the following disclaimer.
% 
%   2. Any modifications must be clearly marked as such.
%                                                                        
% NAGOYA INSTITUTE OF TECHNOLOGY, TOKYO INSITITUTE OF TECHNOLOGY,
% SPTK WORKING GROUP, AND THE CONTRIBUTORS TO THIS WORK DISCLAIM
% ALL WARRANTIES WITH REGARD TO THIS SOFTWARE, INCLUDING ALL
% IMPLIED WARRANTIES OF MERCHANTABILITY AND FITNESS, IN NO EVENT
% SHALL NAGOYA INSTITUTE OF TECHNOLOGY, TOKYO INSITITUTE OF
% TECHNOLOGY, SPTK WORKING GROUP, NOR THE CONTRIBUTORS BE LIABLE
% FOR ANY SPECIAL, INDIRECT OR CONSEQUENTIAL DAMAGES OR ANY
% DAMAGES WHATSOEVER RESULTING FROM LOSS OF USE, DATA OR PROFITS,
% WHETHER IN AN ACTION OF CONTRACT, NEGLIGENCE OR OTHER TORTIOUS
% ACTION, ARISING OUT OF OR IN CONNECTION WITH THE USE OR
% PERFORMANCE OF THIS SOFTWARE.
% ----------------------------------------------------------------
%
\hypertarget{lpc2c}{}
\name{lpc2c}{transform LPC to cepstrum}{speech parameter transformation}

\begin{synopsis}
 \item[lpc2c] [ --m $M_1$ ] [ --M $M_2$ ] [ {\em infile} ]
\end{synopsis}
\begin{qsection}{DESCRIPTION}
{\em lpc2c} calculates LPC cepstrum coefficients 
from linear prediction (LPC) coefficients 
from {\em infile} (or standard input), 
sending the result to standard output.
That is, when the input sequence is 
\begin{displaymath} 
   \sigma, a(1), a(2), \dots, a(p) 
\end{displaymath}
where
\begin{displaymath}
   H(z)=\frac{\sigma}{A(z)}=\frac{\sigma}{\displaystyle 1+\sum_{k=1}^P a(k) z^{-k}}
\end{displaymath}
then the LPC cepstrum coefficients are evaluated as follows.
\begin{displaymath}
   c(n) = \begin{cases}
 \;\; \ln(h),&n=0\\
 \;\; \displaystyle -a(n)=-\sum^{n-1}_{k=1}\frac{k}{n}c(k) a(n-k),&1\leq n\leq P\\ 
 \;\; \displaystyle -\sum_{k=n-P}^{n-1}\frac{k}{n}c(k) a(n-k),& n>P
\end{cases}
\end{displaymath}
And the sequence of cepstrum coefficients
\begin{displaymath}
   c(0), c(1), \dots, c(M)
\end{displaymath}
is outputed.
Input and output data are in float format.
\end{qsection}

\begin{options}
	\argm{m}{M_1}{order of LPC}{25}
	\argm{M}{M_2}{order of cepstrum}{25}
\end{options}

\begin{qsection}{EXAMPLE}
In the example below, a 10 order LPC analysis is undertaken after
passing the speech data {\em data.f} in float format through a window,
15 order LPC cepstrum coefficients are calculated,
and the result is written in {\em data.cep}.
\begin{quote}
 \verb!frame < data.f | window | lpc -m 10 |\!\\
 \verb!lpc2c -m 10 -M 15 > data.cep!
\end{quote}
\end{qsection}

\begin{qsection}{SEE ALSO}
\hyperlink{lpc}{lpc},
\hyperlink{gc2gc}{gc2gc},
\hyperlink{mgc2mgc}{mgc2mgc},
\hyperlink{freqt}{freqt}
\end{qsection}
