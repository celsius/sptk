\name{lpc2c}{transform LPC to cepstrum}{speech parameter transformation}

\begin{synopsis}
 \item[lpc2c] [ --m $M_1$ ] [ --M $M_2$ ] [ {\em infile} ]
\end{synopsis}
\begin{qsection}{DESCRIPTION}
The {\em lpc2c} command reads LPC coefficients from the assigned
{\em infile} and output LPC cepstrum coefficients.
That is, when the input sequence is 
\[ \sigma, a(1), a(2), \cdots, a(p) \]
where
\[
H(z)=\frac{\sigma}{A(z)}=\frac{\sigma}{\displaystyle 1+\sum_{k=1}^P a(k) z^{-k}}
\]
then the LPC cepstrum coefficients are evaluated as follows.
\[
c(n) =\left\{
\begin{array}{lc}
 ln(h),&n=0\\
\displaystyle -a(n)=-\sum^{n-1}_{k=1}\frac{k}{n}c(k) a(n-k),&1\leq n\leq P\\
\displaystyle -\sum_{k=n-P}^{n-1}\frac{k}{n}c(k) a(n-k),& n>P
\end{array}
\right.
\]
And the sequence of cepstrum coefficients
\[ c(0), c(1), \cdots, c(M) \]
is outputed.
Input and output data are in float format.
\end{qsection}

\begin{options}
	\argm{m}{M_1}{order of LPC}{25}
	\argm{M}{M_2}{order of cepstrum}{25}
\end{options}

\begin{qsection}{EXAMPLE}
In the example below, a 10 order LPC analysis is undertaken after
passing the speech data {\em data.f} in float format through a window,
15 order LPC cepstrum coefficients are calculated,
and the result is written in {\em data.cep}.
\begin{quote}
 \verb!frame < data.f | window | lpc -m 10 |\!\\
 \verb!lpc2c -m 10 -M 15 > data.cep!
\end{quote}
\end{qsection}

\begin{qsection}{SEE ALSO}
lpc, gc2gc, mgc2mgc, freqt
\end{qsection}
