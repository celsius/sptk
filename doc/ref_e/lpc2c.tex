% ----------------------------------------------------------------- %
%             The Speech Signal Processing Toolkit (SPTK)           %
%             developed by SPTK Working Group                       %
%             http://sp-tk.sourceforge.net/                         %
% ----------------------------------------------------------------- %
%                                                                   %
%  Copyright (c) 1984-2007  Tokyo Institute of Technology           %
%                           Interdisciplinary Graduate School of    %
%                           Science and Engineering                 %
%                                                                   %
%                1996-2011  Nagoya Institute of Technology          %
%                           Department of Computer Science          %
%                                                                   %
% All rights reserved.                                              %
%                                                                   %
% Redistribution and use in source and binary forms, with or        %
% without modification, are permitted provided that the following   %
% conditions are met:                                               %
%                                                                   %
% - Redistributions of source code must retain the above copyright  %
%   notice, this list of conditions and the following disclaimer.   %
% - Redistributions in binary form must reproduce the above         %
%   copyright notice, this list of conditions and the following     %
%   disclaimer in the documentation and/or other materials provided %
%   with the distribution.                                          %
% - Neither the name of the SPTK working group nor the names of its %
%   contributors may be used to endorse or promote products derived %
%   from this software without specific prior written permission.   %
%                                                                   %
% THIS SOFTWARE IS PROVIDED BY THE COPYRIGHT HOLDERS AND            %
% CONTRIBUTORS "AS IS" AND ANY EXPRESS OR IMPLIED WARRANTIES,       %
% INCLUDING, BUT NOT LIMITED TO, THE IMPLIED WARRANTIES OF          %
% MERCHANTABILITY AND FITNESS FOR A PARTICULAR PURPOSE ARE          %
% DISCLAIMED. IN NO EVENT SHALL THE COPYRIGHT OWNER OR CONTRIBUTORS %
% BE LIABLE FOR ANY DIRECT, INDIRECT, INCIDENTAL, SPECIAL,          %
% EXEMPLARY, OR CONSEQUENTIAL DAMAGES (INCLUDING, BUT NOT LIMITED   %
% TO, PROCUREMENT OF SUBSTITUTE GOODS OR SERVICES; LOSS OF USE,     %
% DATA, OR PROFITS; OR BUSINESS INTERRUPTION) HOWEVER CAUSED AND ON %
% ANY THEORY OF LIABILITY, WHETHER IN CONTRACT, STRICT LIABILITY,   %
% OR TORT (INCLUDING NEGLIGENCE OR OTHERWISE) ARISING IN ANY WAY    %
% OUT OF THE USE OF THIS SOFTWARE, EVEN IF ADVISED OF THE           %
% POSSIBILITY OF SUCH DAMAGE.                                       %
% ----------------------------------------------------------------- %
\hypertarget{lpc2c}{}
\name{lpc2c}{transform LPC to cepstrum}{speech parameter transformation}

\begin{synopsis}
 \item[lpc2c] [ --m $M_1$ ] [ --M $M_2$ ] [ {\em infile} ]
\end{synopsis}
\begin{qsection}{DESCRIPTION}
{\em lpc2c} calculates LPC cepstral coefficients 
from linear prediction (LPC) coefficients 
from {\em infile} (or standard input), 
sending the result to standard output.
That is, when the input sequence is 
\begin{displaymath} 
   \sigma, a(1), a(2), \dots, a(p) 
\end{displaymath}
where
\begin{displaymath}
   H(z)=\frac{\sigma}{A(z)}=\frac{\sigma}{\displaystyle 1+\sum_{k=1}^P a(k) z^{-k}}
\end{displaymath}
then the LPC cepstral coefficients are evaluated as follows.
\begin{displaymath}
   c(n) = \begin{cases}
 \;\; \ln(h),&n=0\\
 \;\; \displaystyle -a(n)=-\sum^{n-1}_{k=1}\frac{k}{n}c(k) a(n-k),&1\leq n\leq P\\ 
 \;\; \displaystyle -\sum_{k=n-P}^{n-1}\frac{k}{n}c(k) a(n-k),& n>P
\end{cases}
\end{displaymath}
And the sequence of cepstral coefficients
\begin{displaymath}
   c(0), c(1), \dots, c(M)
\end{displaymath}
is given as output.
Input and output data are in float format.
\end{qsection}

\begin{options}
	\argm{m}{M_1}{order of LPC}{25}
	\argm{M}{M_2}{order of cepstrum}{25}
\end{options}

\begin{qsection}{EXAMPLE}
In the example below, a 10-th order LPC analysis is undertaken after
passing the speech data {\em data.f} in float format through a window,
15-th order LPC cepstral coefficients are calculated,
and the result is written to {\em data.cep}.
\begin{quote}
 \verb!frame +f < data.f | window | lpc -m 10 |\!\\
 \verb!lpc2c -m 10 -M 15 > data.cep!
\end{quote}
\end{qsection}

\begin{qsection}{SEE ALSO}
\hyperlink{lpc}{lpc},
\hyperlink{gc2gc}{gc2gc},
\hyperlink{mgc2mgc}{mgc2mgc},
\hyperlink{freqt}{freqt}
\end{qsection}
