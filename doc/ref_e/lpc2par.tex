\name{lpc2par}{transform LPC to PARCOR}{speech parameter transformation}

\begin{synopsis}
\item [lpc2par] [ --m $M$ ] [ --g $G$ ] [ --s ] [ {\em infile} ] 
\end{synopsis}

\begin{qsection}{DESCRIPTION}
This command evaluates PARCOR coefficients from LPC coefficients.
It is reads from the assigned file $M$ order linear prediction
coefficients,
\begin{displaymath}
  K, a(1),\ldots, a(M)
\end{displaymath}
evaluates the corresponding PARCOR coefficients
\begin{displaymath}
  K, k(1),\ldots, k(M)
\end{displaymath}
and sends the results to the standard output.
If the --s option is assigned, the stability of the filter is analyzed.
If the filter is stable, then 0 is returned.
If the filter is not stable, then 1 is returned to the standard output.
\par
Input and output data are in float format.
\par
The transformation from LPC coefficients to PARCOR coefficients
is undertaken as follows.
\begin{eqnarray*} 
k(m) &=& a^{(m)}(m) \\
a^{(m-1)}(i) &=& \frac{a^{(m)}(i)+a^{(m)}(m)a^{(m)}(m-i)}{1-k^2(m)},
\end{eqnarray*}
where $1 \leq i \leq m-1$, $m=p, p-1, \ldots, 1$.
The initial condition is
\begin{displaymath}
a^{(M)}(m) = a(m), ~~~~~1 \leq m \leq M.
\end{displaymath}
If we use the --g option, then the input contains normalized generalized
cepstrum coefficients with power parameter $\gamma$
and the output contains the corresponding PARCOR coefficients.
In other words, the input is 
\begin{displaymath}
K,c_\gamma'(1),\ldots,c_\gamma'(M)
\end{displaymath}
and the initial condition is
\begin{displaymath}
a^{(M)}(m) = \gamma c_\gamma'(M), ~~~~~1 \leq m \leq M.
\end{displaymath}

Also with respect to the stability analysis,
the PARCOR coefficients are checked through the following equation.
\begin{displaymath}
-1 < k(m) < 1
\end{displaymath}
If this condition satisfy then the filter is stable.

\end{qsection}

\begin{options}
	\argm{m}{M}{order of LPC}{25}
	\argm{g}{G}{gamma of generalized cepstum\\
			If $G>1.0$ tnen $\gamma=-1/G$.}{1}
	\argm{s}{}{check stable or unstable}{FALSE}
\end{options}

\begin{qsection}{EXAMPLE}
In the example below, a linear prediction analysis is
done in the input file {\em data.f} in float format,
the LPC coefficients are then transformed into PARCOR coefficients,
and the output is written to {\em data.rc}:
\begin{quote}
 \verb!frame < data.f | window | lpc | lpc2par > data.rc!
\end{quote} 
\end{qsection}

\begin{qsection}{SEE ALSO}
 acorr, levdur, lpc, par2lpc, ltcdf
\end{qsection}
