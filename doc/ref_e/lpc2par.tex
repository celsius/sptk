%  ---------------------------------------------------------------  %
%            Speech Signal Processing Toolkit (SPTK)                %
%                      SPTK Working Group                           %
%                                                                   %
%                  Department of Computer Science                   %
%                  Nagoya Institute of Technology                   %
%                               and                                 %
%   Interdisciplinary Graduate School of Science and Engineering    %
%                  Tokyo Institute of Technology                    %
%                                                                   %
%                     Copyright (c) 1984-2007                       %
%                       All Rights Reserved.                        %
%                                                                   %
%  Permission is hereby granted, free of charge, to use and         %
%  distribute this software and its documentation without           %
%  restriction, including without limitation the rights to use,     %
%  copy, modify, merge, publish, distribute, sublicense, and/or     %
%  sell copies of this work, and to permit persons to whom this     %
%  work is furnished to do so, subject to the following conditions: %
%                                                                   %
%    1. The source code must retain the above copyright notice,     %
%       this list of conditions and the following disclaimer.       %
%                                                                   %
%    2. Any modifications to the source code must be clearly        %
%       marked as such.                                             %
%                                                                   %
%    3. Redistributions in binary form must reproduce the above     %
%       copyright notice, this list of conditions and the           %
%       following disclaimer in the documentation and/or other      %
%       materials provided with the distribution.  Otherwise, one   %
%       must contact the SPTK working group.                        %
%                                                                   %
%  NAGOYA INSTITUTE OF TECHNOLOGY, TOKYO INSTITUTE OF TECHNOLOGY,   %
%  SPTK WORKING GROUP, AND THE CONTRIBUTORS TO THIS WORK DISCLAIM   %
%  ALL WARRANTIES WITH REGARD TO THIS SOFTWARE, INCLUDING ALL       %
%  IMPLIED WARRANTIES OF MERCHANTABILITY AND FITNESS, IN NO EVENT   %
%  SHALL NAGOYA INSTITUTE OF TECHNOLOGY, TOKYO INSTITUTE OF         %
%  TECHNOLOGY, SPTK WORKING GROUP, NOR THE CONTRIBUTORS BE LIABLE   %
%  FOR ANY SPECIAL, INDIRECT OR CONSEQUENTIAL DAMAGES OR ANY        %
%  DAMAGES WHATSOEVER RESULTING FROM LOSS OF USE, DATA OR PROFITS,  %
%  WHETHER IN AN ACTION OF CONTRACT, NEGLIGENCE OR OTHER TORTUOUS   %
%  ACTION, ARISING OUT OF OR IN CONNECTION WITH THE USE OR          %
%  PERFORMANCE OF THIS SOFTWARE.                                    %
%                                                                   %
%  ---------------------------------------------------------------  %
%
\hypertarget{lpc2par}{}
\name{lpc2par}{transform LPC to PARCOR}{speech parameter transformation}

\begin{synopsis}
\item [lpc2par] [ --m $M$ ] [ --g $G$ ] [ --s ] [ {\em infile} ] 
\end{synopsis}

\begin{qsection}{DESCRIPTION}
{\em lpc2par} calculates PARCOR coefficients 
from $M$-th order linear prediction (LPC) coefficients 
from {\em infile} (or standard input), 
sending the result to standard output.

The LPC input format is
\begin{displaymath}
  K, a(1),\dots, a(M), 
\end{displaymath}
and the PARCOR output format is
\begin{displaymath}
  K, k(1),\dots, k(M).
\end{displaymath}
If the --s option is assigned, the stability of the filter is analyzed.
If the filter is stable, then 0 is returned.
If the filter is not stable, then 1 is returned to the standard output.
\par
Input and output data are in float format.
\par
The transformation from LPC coefficients to PARCOR coefficients
is undertaken as follows:
\begin{align} 
k(m) &= a^{(m)}(m) \notag \\
a^{(m-1)}(i) &= \frac{a^{(m)}(i)+a^{(m)}(m)a^{(m)}(m-i)}{1-k^2(m)}, \notag
\end{align}
where $1 \leq i \leq m-1$, $m=p, p-1, \dots, 1$.
The initial condition is
\begin{displaymath}  
a^{(M)}(m) = a(m), \qquad 1 \leq m \leq M.
\end{displaymath}
If we use the --g option, then the input contains normalized generalized
cepstral coefficients with power parameter $\gamma$
and the output contains the corresponding PARCOR coefficients.
In other words, the input is 
\begin{displaymath}
K,c_\gamma'(1),\dots,c_\gamma'(M)
\end{displaymath}
and the initial condition is
\begin{displaymath}
a^{(M)}(m) = \gamma c_\gamma'(M), \qquad 1 \leq m \leq M.
\end{displaymath}

Also with respect to the stability analysis,
the PARCOR coefficients are checked through the following equation.
\begin{displaymath}
-1 < k(m) < 1
\end{displaymath}
If this condition satisfy then the filter is stable.

\end{qsection}

\begin{options}
	\argm{m}{M}{order of LPC}{25}
	\argm{g}{G}{gamma of generalized cepstrum\\
			If $G>1.0$ then $\gamma=-1/G$.}{1}
	\argm{s}{}{check stable or unstable}{FALSE}
\end{options}

\begin{qsection}{EXAMPLE}
In the example below, a linear prediction analysis is
done in the input file {\em data.f} in float format,
the LPC coefficients are then transformed into PARCOR coefficients,
and the output is written to {\em data.rc}:
\begin{quote}
 \verb!frame +f < data.f | window | lpc | lpc2par > data.rc!
\end{quote} 
\end{qsection}

\begin{qsection}{SEE ALSO}
\hyperlink{acorr}{acorr},
\hyperlink{levdur}{levdur},
\hyperlink{lpc}{lpc},
\hyperlink{par2lpc}{par2lpc},
\hyperlink{ltcdf}{ltcdf}
\end{qsection}
