% ----------------------------------------------------------------- %
%             The Speech Signal Processing Toolkit (SPTK)           %
%             developed by SPTK Working Group                       %
%             http://sp-tk.sourceforge.net/                         %
% ----------------------------------------------------------------- %
%                                                                   %
%  Copyright (c) 1984-2007  Tokyo Institute of Technology           %
%                           Interdisciplinary Graduate School of    %
%                           Science and Engineering                 %
%                                                                   %
%                1996-2008  Nagoya Institute of Technology          %
%                           Department of Computer Science          %
%                                                                   %
% All rights reserved.                                              %
%                                                                   %
% Redistribution and use in source and binary forms, with or        %
% without modification, are permitted provided that the following   %
% conditions are met:                                               %
%                                                                   %
% - Redistributions of source code must retain the above copyright  %
%   notice, this list of conditions and the following disclaimer.   %
% - Redistributions in binary form must reproduce the above         %
%   copyright notice, this list of conditions and the following     %
%   disclaimer in the documentation and/or other materials provided %
%   with the distribution.                                          %
% - Neither the name of the SPTK working group nor the names of its %
%   contributors may be used to endorse or promote products derived %
%   from this software without specific prior written permission.   %
%                                                                   %
% THIS SOFTWARE IS PROVIDED BY THE COPYRIGHT HOLDERS AND            %
% CONTRIBUTORS "AS IS" AND ANY EXPRESS OR IMPLIED WARRANTIES,       %
% INCLUDING, BUT NOT LIMITED TO, THE IMPLIED WARRANTIES OF          %
% MERCHANTABILITY AND FITNESS FOR A PARTICULAR PURPOSE ARE          %
% DISCLAIMED. IN NO EVENT SHALL THE COPYRIGHT OWNER OR CONTRIBUTORS %
% BE LIABLE FOR ANY DIRECT, INDIRECT, INCIDENTAL, SPECIAL,          %
% EXEMPLARY, OR CONSEQUENTIAL DAMAGES (INCLUDING, BUT NOT LIMITED   %
% TO, PROCUREMENT OF SUBSTITUTE GOODS OR SERVICES; LOSS OF USE,     %
% DATA, OR PROFITS; OR BUSINESS INTERRUPTION) HOWEVER CAUSED AND ON %
% ANY THEORY OF LIABILITY, WHETHER IN CONTRACT, STRICT LIABILITY,   %
% OR TORT (INCLUDING NEGLIGENCE OR OTHERWISE) ARISING IN ANY WAY    %
% OUT OF THE USE OF THIS SOFTWARE, EVEN IF ADVISED OF THE           %
% POSSIBILITY OF SUCH DAMAGE.                                       %
% ----------------------------------------------------------------- %
\hypertarget{lpc2par}{}
\name{lpc2par}{transform LPC to PARCOR}{speech parameter transformation}

\begin{synopsis}
\item [lpc2par] [ --m $M$ ] [ --g $G$ ] [ --c $C$ ] [ --s ] [ {\em infile} ] 
\end{synopsis}

\begin{qsection}{DESCRIPTION}
{\em lpc2par} calculates PARCOR coefficients 
from $M$-th order linear prediction (LPC) coefficients 
from {\em infile} (or standard input), 
sending the result to standard output.

The LPC input format is
\begin{displaymath}
  K, a(1),\dots, a(M), 
\end{displaymath}
and the PARCOR output format is
\begin{displaymath}
  K, k(1),\dots, k(M).
\end{displaymath}
If the --s option is assigned, the stability of the filter is analyzed.
If the filter is stable, then 0 is returned.
If the filter is not stable, then 1 is returned to the standard output.
\par
Input and output data are in float format.
\par
The transformation from LPC coefficients to PARCOR coefficients
is undertaken as follows:
\begin{align} 
k(m) &= a^{(m)}(m) \notag \\
a^{(m-1)}(i) &= \frac{a^{(m)}(i)+a^{(m)}(m)a^{(m)}(m-i)}{1-k^2(m)}, \notag
\end{align}
where $1 \leq i \leq m-1$, $m=p, p-1, \dots, 1$.
The initial condition is
\begin{displaymath}  
a^{(M)}(m) = a(m), \qquad 1 \leq m \leq M.
\end{displaymath}
If we use the --g option, then the input contains normalized generalized
cepstral coefficients with power parameter $\gamma$
and the output contains the corresponding PARCOR coefficients.
In other words, the input is 
\begin{displaymath}
K,c_\gamma'(1),\dots,c_\gamma'(M)
\end{displaymath}
and the initial condition is
\begin{displaymath}
a^{(M)}(m) = \gamma c_\gamma'(M), \qquad 1 \leq m \leq M.
\end{displaymath}

Also with respect to the stability analysis,
the PARCOR coefficients are checked through the following equation.
\begin{displaymath}
-1 < k(m) < 1
\end{displaymath}
If this condition satisfy then the filter is stable.

\end{qsection}

\begin{options}
        \argm{m}{M}{order of LPC}{25}
        \argm{g}{G}{gamma of generalized cepstrum\\
                         $\gamma=G$}{0}
        \argm{c}{C}{gamma of generalized cepstrum\\
                        $\gamma =-1 / $(int)$ C$\\
                        $C$ must be $C \geq 1$}{}
        \argm{s}{}{check stable or unstable}{FALSE}
\end{options}

\begin{qsection}{EXAMPLE}
In the example below, a linear prediction analysis is
done in the input file {\em data.f} in float format,
the LPC coefficients are then transformed into PARCOR coefficients,
and the output is written to {\em data.rc}:
\begin{quote}
 \verb!frame +f < data.f | window | lpc | lpc2par > data.rc!
\end{quote} 
\end{qsection}

\begin{qsection}{SEE ALSO}
\hyperlink{acorr}{acorr},
\hyperlink{levdur}{levdur},
\hyperlink{lpc}{lpc},
\hyperlink{par2lpc}{par2lpc},
\hyperlink{ltcdf}{ltcdf}
\end{qsection}
