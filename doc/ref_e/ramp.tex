% ----------------------------------------------------------------
%       Speech Signal Processing Toolkit (SPTK): version 3.0
%                      SPTK Working Group
% 
%                Department of Computer Science
%                Nagoya Institute of Technology
%                             and
%   Interdisciplinary Graduate School of Science and Engineering
%                Tokyo Institute of Technology
%                   Copyright (c) 1984-2000
%                     All Rights Reserved.
% 
% Permission is hereby granted, free of charge, to use and
% distribute this software and its documentation without
% restriction, including without limitation the rights to use,
% copy, modify, merge, publish, distribute, sublicense, and/or
% sell copies of this work, and to permit persons to whom this
% work is furnished to do so, subject to the following conditions:
% 
%   1. The code must retain the above copyright notice, this list
%      of conditions and the following disclaimer.
% 
%   2. Any modifications must be clearly marked as such.
%                                                                        
% NAGOYA INSTITUTE OF TECHNOLOGY, TOKYO INSITITUTE OF TECHNOLOGY,
% SPTK WORKING GROUP, AND THE CONTRIBUTORS TO THIS WORK DISCLAIM
% ALL WARRANTIES WITH REGARD TO THIS SOFTWARE, INCLUDING ALL
% IMPLIED WARRANTIES OF MERCHANTABILITY AND FITNESS, IN NO EVENT
% SHALL NAGOYA INSTITUTE OF TECHNOLOGY, TOKYO INSITITUTE OF
% TECHNOLOGY, SPTK WORKING GROUP, NOR THE CONTRIBUTORS BE LIABLE
% FOR ANY SPECIAL, INDIRECT OR CONSEQUENTIAL DAMAGES OR ANY
% DAMAGES WHATSOEVER RESULTING FROM LOSS OF USE, DATA OR PROFITS,
% WHETHER IN AN ACTION OF CONTRACT, NEGLIGENCE OR OTHER TORTIOUS
% ACTION, ARISING OUT OF OR IN CONNECTION WITH THE USE OR
% PERFORMANCE OF THIS SOFTWARE.
% ----------------------------------------------------------------
%
\hypertarget{ramp}{}
\name{ramp}{generate ramp sequence}{signal generation}

\begin{synopsis}
\item[ramp] [ --l $L$ ] [ --n $N$ ] [ --s $S$ ] [ --e $E$ ] [ --t $T$ ]
\end{synopsis}

\begin{qsection}{DESCRIPTION}
{\em ramp} generates ramp sequences of length $L$, 
sending the result to standard output. 
The output is as follows.
\begin{displaymath}
\underbrace{S, S+T, S+2T,  \dots, S+(L-1)T}_{L}
\end{displaymath}
\par
Output format is in float format.
In the case the last value is assigned 
the generated sequence is,
\begin{displaymath}
\underbrace{S, S+T, S+2T,  \dots, E}_{(E-S)/T}
\end{displaymath}
\par
If the --l option, --e option and --n option are used
in the same time, then only the last option are taken into account.
\end{qsection}

\begin{options}
	\argm{l}{L}{length of ramp sequence\\
                    In the case $L \le 0$ then ramp values will be
                    generated indefinitely.}{256}
	\argm{n}{N}{order of ramp sequence}{L-1}
	\argm{s}{S}{start value}{0}
	\argm{e}{E}{end value}{N/A}
	\argm{t}{T}{step size}{1}
\end{options}

\begin{qsection}{EXAMPLE}
The Following example output the sequence 
\begin{displaymath}
  y(n)=\exp(-n)
\end{displaymath}
\begin{quote}
\verb!ramp | sopr -m -1 -E | dmp!
\end{quote}
\end{qsection}

\begin{qsection}{SEE ALSO}
\hyperlink{impulse}{impulse},
\hyperlink{step}{step},
\hyperlink{train}{train},
\hyperlink{sin}{sin}
\end{qsection}
