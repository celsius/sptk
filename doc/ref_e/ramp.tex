\name{ramp}{generate ramp sequence}{signal generation}

\begin{synopsis}
\item[ramp] [ --l $L$ ] [ --n $N$ ] [ --s $S$ ] [ --e $E$ ] [ --t $T$ ]
\end{synopsis}

\begin{qsection}{DESCRIPTION}
This command sends to the standard output a ramp sequence.
That is, the following sequence is outputed.
\begin{displaymath}
\underbrace{S, S+T, S+2T,  \ldots, S+(L-1)T}_{L}
\end{displaymath}
\par
Output format is in float format.
In the case the last value is assigned 
the generated sequence is,
\begin{displaymath}
\underbrace{S, S+T, S+2T,  \ldots, E}_{(E-S)/T}
\end{displaymath}
\par
If the --l option, --e option and --n option are used
in the same time, then only the last option are taken into account.
\end{qsection}

\begin{options}
	\argm{l}{L}{length of ramp sequence\\
                    In the case $L \le 0$ then ramp values will be
                    generated indefinitely.}{256}
	\argm{n}{N}{order of ramp sequence}{L-1}
	\argm{s}{S}{start value}{0}
	\argm{e}{E}{end value}{N/A}
	\argm{t}{T}{step size}{1}
\end{options}

\begin{qsection}{EXAMPLE}
The Following example output the sequence 
\begin{displaymath}
  y(n)=\exp(-n)
\end{displaymath}
\begin{quote}
\verb!ramp | sopr -m -1 -E | dmp!
\end{quote}
\end{qsection}

\begin{qsection}{SEE ALSO}
  impulse, step, train, sin
\end{qsection}
