% ----------------------------------------------------------------- %
%             The Speech Signal Processing Toolkit (SPTK)           %
%             developed by SPTK Working Group                       %
%             http://sp-tk.sourceforge.net/                         %
% ----------------------------------------------------------------- %
%                                                                   %
%  Copyright (c) 1984-2007  Tokyo Institute of Technology           %
%                           Interdisciplinary Graduate School of    %
%                           Science and Engineering                 %
%                                                                   %
%                1996-2009  Nagoya Institute of Technology          %
%                           Department of Computer Science          %
%                                                                   %
% All rights reserved.                                              %
%                                                                   %
% Redistribution and use in source and binary forms, with or        %
% without modification, are permitted provided that the following   %
% conditions are met:                                               %
%                                                                   %
% - Redistributions of source code must retain the above copyright  %
%   notice, this list of conditions and the following disclaimer.   %
% - Redistributions in binary form must reproduce the above         %
%   copyright notice, this list of conditions and the following     %
%   disclaimer in the documentation and/or other materials provided %
%   with the distribution.                                          %
% - Neither the name of the SPTK working group nor the names of its %
%   contributors may be used to endorse or promote products derived %
%   from this software without specific prior written permission.   %
%                                                                   %
% THIS SOFTWARE IS PROVIDED BY THE COPYRIGHT HOLDERS AND            %
% CONTRIBUTORS "AS IS" AND ANY EXPRESS OR IMPLIED WARRANTIES,       %
% INCLUDING, BUT NOT LIMITED TO, THE IMPLIED WARRANTIES OF          %
% MERCHANTABILITY AND FITNESS FOR A PARTICULAR PURPOSE ARE          %
% DISCLAIMED. IN NO EVENT SHALL THE COPYRIGHT OWNER OR CONTRIBUTORS %
% BE LIABLE FOR ANY DIRECT, INDIRECT, INCIDENTAL, SPECIAL,          %
% EXEMPLARY, OR CONSEQUENTIAL DAMAGES (INCLUDING, BUT NOT LIMITED   %
% TO, PROCUREMENT OF SUBSTITUTE GOODS OR SERVICES; LOSS OF USE,     %
% DATA, OR PROFITS; OR BUSINESS INTERRUPTION) HOWEVER CAUSED AND ON %
% ANY THEORY OF LIABILITY, WHETHER IN CONTRACT, STRICT LIABILITY,   %
% OR TORT (INCLUDING NEGLIGENCE OR OTHERWISE) ARISING IN ANY WAY    %
% OUT OF THE USE OF THIS SOFTWARE, EVEN IF ADVISED OF THE           %
% POSSIBILITY OF SUCH DAMAGE.                                       %
% ----------------------------------------------------------------- %
\hypertarget{ds}{}
\name{ds}{down-sampling}%
{sampling rate transformation}
 
\begin{synopsis}
\item[ds] [ --s $S$ ] [ {\em infile} ]
\end{synopsis}

\begin{qsection}{DESCRIPTION}
{\em ds} down-samples data from {\em infile} (or standard input), 
and sends the result to standard output.

Both input and output files are in float format.

The following filter coefficients can be used.

\begin{tabular}{ll} \\[-1ex]
	$S=21$ & \$SPTK/lib/lpfcoef.2to1 \\
	$S=32$ & \$SPTK/lib/lpfcoef.3to2 \\
	$S=43$ & \$SPTK/lib/lpfcoef.4to3 \\
	$S=52,s=54$ & \$SPTK/lib/lpfcoef.5to2up \\
	& \$SPTK/lib/lpfcoef.5to2dn \\
        &(\$SPTK is the directory where toolkit was installed.)
\end{tabular}

Filter coefficients are in ASCII format.
\end{qsection}

\begin{options}
	\argm{s}{S}{conversion type\\
		\begin{tabular}{ll} \\[-1ex]
			$S=21$ & downsampling by $2:1$ \\
			$S=32$ & downsampling by $3:2$ \\
			$S=43$ & downsampling by $4:3$ \\
			$S=52$ & downsampling by $5:2$ \\
			$S=54$ & downsampling by $5:4$
		\end{tabular}\\\hspace*{\fill}}{21}
\end{options}

\begin{qsection}{EXAMPLE}
In this example, the speech data in the input file {\em data.16},
which was sampled at 16 kHz in float format, is downsampled to 8 kHz:
\begin{quote}
\verb! ds data.16 > data.8 !
\end{quote}
\end{qsection}

\begin{qsection}{SEE ALSO}
 \hyperlink{us}{us},
 \hyperlink{uscd}{uscd},
 \hyperlink{us16}{us16}
\end{qsection}
