\name{ds}{Sampling rate conversion (down sampling)}%
{sampling rate conversion}

\begin{synopsis}
\item[ds] [ --s $S$ ] [ {\em infile} ]
\end{synopsis}

\begin{qsection}{DESCRIPTION}
This command undertakes down sampling.
\par
Input and output data is in float format.
The following filter coefficient can be used.
\begin{tabular}{ll} \\[-1zh]
	$S=21$ & \$SPTK/lib/lpfcoef.2to1 \\
	$S=43$ & \$SPTK/lib/lpfcoef.4to3 \\
	$S=52,s=54$ & \$SPTK/lib/lpfcoef.5to2up \\
	& \$SPTK/lib/lpfcoef.5to2dn \\
        &(\$SPTK is the directory where toolkit was installed.)
\end{tabular}

Filter coefficients are in ascii formats.
\end{qsection}

\begin{options}
	\argm{s}{S}{conversion type\\
		\begin{tabular}{ll} \\[-1zh]
			$S=21$ & down sampling by $2:1$ \\
			$S=43$ & down sampling by $4:3$ \\
			$S=52$ & down sampling by $5:2$ \\
			$S=54$ & down sampling by $5:4$
		\end{tabular}\\\hspace*{\fill}}{21}
\end{options}

\begin{qsection}{EXAMPLE}
In this example, the speech data in the input file {\em data.16},
which was sampled at 16kHz in float format, is converted to
sampling rate of 8kHz:
\begin{quote}
\verb! ds data.16 > data.8 !
\end{quote}
\end{qsection}

%\begin{qsection}{SEE ALSO}
%
%\end{qsection}
