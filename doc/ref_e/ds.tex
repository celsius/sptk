%  ---------------------------------------------------------------  %
%            Speech Signal Processing Toolkit (SPTK)                %
%                      SPTK Working Group                           %
%                                                                   %
%                  Department of Computer Science                   %
%                  Nagoya Institute of Technology                   %
%                               and                                 %
%   Interdisciplinary Graduate School of Science and Engineering    %
%                  Tokyo Institute of Technology                    %
%                                                                   %
%                     Copyright (c) 1984-2007                       %
%                       All Rights Reserved.                        %
%                                                                   %
%  Permission is hereby granted, free of charge, to use and         %
%  distribute this software and its documentation without           %
%  restriction, including without limitation the rights to use,     %
%  copy, modify, merge, publish, distribute, sublicense, and/or     %
%  sell copies of this work, and to permit persons to whom this     %
%  work is furnished to do so, subject to the following conditions: %
%                                                                   %
%    1. The source code must retain the above copyright notice,     %
%       this list of conditions and the following disclaimer.       %
%                                                                   %
%    2. Any modifications to the source code must be clearly        %
%       marked as such.                                             %
%                                                                   %
%    3. Redistributions in binary form must reproduce the above     %
%       copyright notice, this list of conditions and the           %
%       following disclaimer in the documentation and/or other      %
%       materials provided with the distribution.  Otherwise, one   %
%       must contact the SPTK working group.                        %
%                                                                   %
%  NAGOYA INSTITUTE OF TECHNOLOGY, TOKYO INSTITUTE OF TECHNOLOGY,   %
%  SPTK WORKING GROUP, AND THE CONTRIBUTORS TO THIS WORK DISCLAIM   %
%  ALL WARRANTIES WITH REGARD TO THIS SOFTWARE, INCLUDING ALL       %
%  IMPLIED WARRANTIES OF MERCHANTABILITY AND FITNESS, IN NO EVENT   %
%  SHALL NAGOYA INSTITUTE OF TECHNOLOGY, TOKYO INSTITUTE OF         %
%  TECHNOLOGY, SPTK WORKING GROUP, NOR THE CONTRIBUTORS BE LIABLE   %
%  FOR ANY SPECIAL, INDIRECT OR CONSEQUENTIAL DAMAGES OR ANY        %
%  DAMAGES WHATSOEVER RESULTING FROM LOSS OF USE, DATA OR PROFITS,  %
%  WHETHER IN AN ACTION OF CONTRACT, NEGLIGENCE OR OTHER TORTUOUS   %
%  ACTION, ARISING OUT OF OR IN CONNECTION WITH THE USE OR          %
%  PERFORMANCE OF THIS SOFTWARE.                                    %
%                                                                   %
%  ---------------------------------------------------------------  %
%
\hypertarget{ds}{}
\name{ds}{down-sampling}%
{sampling rate transformation}
 
\begin{synopsis}
\item[ds] [ --s $S$ ] [ {\em infile} ]
\end{synopsis}

\begin{qsection}{DESCRIPTION}
{\em ds} down-samples data from {\em infile} (or standard input), 
sending the result to standard output.

The format of input and output data is float.
The following filter coefficients can be used.

\begin{tabular}{ll} \\[-1ex]
	$S=21$ & \$SPTK/lib/lpfcoef.2to1 \\
	$S=32$ & \$SPTK/lib/lpfcoef.3to2 \\
	$S=43$ & \$SPTK/lib/lpfcoef.4to3 \\
	$S=52,s=54$ & \$SPTK/lib/lpfcoef.5to2up \\
	& \$SPTK/lib/lpfcoef.5to2dn \\
        &(\$SPTK is the directory where toolkit was installed.)
\end{tabular}

Filter coefficients are in ASCII format.
\end{qsection}

\begin{options}
	\argm{s}{S}{conversion type\\
		\begin{tabular}{ll} \\[-1ex]
			$S=21$ & down sampling by $2:1$ \\
			$S=32$ & down sampling by $3:2$ \\
			$S=43$ & down sampling by $4:3$ \\
			$S=52$ & down sampling by $5:2$ \\
			$S=54$ & down sampling by $5:4$
		\end{tabular}\\\hspace*{\fill}}{21}
\end{options}

\begin{qsection}{EXAMPLE}
In this example, the speech data in the input file {\em data.16},
which was sampled at 16 kHz in float format, is converted to
an 8 kHz sampling rate:
\begin{quote}
\verb! ds data.16 > data.8 !
\end{quote}
\end{qsection}

%\begin{qsection}{SEE ALSO}
%
%\end{qsection}
