% ----------------------------------------------------------------- %
%             The Speech Signal Processing Toolkit (SPTK)           %
%             developed by SPTK Working Group                       %
%             http://sp-tk.sourceforge.net/                         %
% ----------------------------------------------------------------- %
%                                                                   %
%  Copyright (c) 1984-2007  Tokyo Institute of Technology           %
%                           Interdisciplinary Graduate School of    %
%                           Science and Engineering                 %
%                                                                   %
%                1996-2010  Nagoya Institute of Technology          %
%                           Department of Computer Science          %
%                                                                   %
% All rights reserved.                                              %
%                                                                   %
% Redistribution and use in source and binary forms, with or        %
% without modification, are permitted provided that the following   %
% conditions are met:                                               %
%                                                                   %
% - Redistributions of source code must retain the above copyright  %
%   notice, this list of conditions and the following disclaimer.   %
% - Redistributions in binary form must reproduce the above         %
%   copyright notice, this list of conditions and the following     %
%   disclaimer in the documentation and/or other materials provided %
%   with the distribution.                                          %
% - Neither the name of the SPTK working group nor the names of its %
%   contributors may be used to endorse or promote products derived %
%   from this software without specific prior written permission.   %
%                                                                   %
% THIS SOFTWARE IS PROVIDED BY THE COPYRIGHT HOLDERS AND            %
% CONTRIBUTORS "AS IS" AND ANY EXPRESS OR IMPLIED WARRANTIES,       %
% INCLUDING, BUT NOT LIMITED TO, THE IMPLIED WARRANTIES OF          %
% MERCHANTABILITY AND FITNESS FOR A PARTICULAR PURPOSE ARE          %
% DISCLAIMED. IN NO EVENT SHALL THE COPYRIGHT OWNER OR CONTRIBUTORS %
% BE LIABLE FOR ANY DIRECT, INDIRECT, INCIDENTAL, SPECIAL,          %
% EXEMPLARY, OR CONSEQUENTIAL DAMAGES (INCLUDING, BUT NOT LIMITED   %
% TO, PROCUREMENT OF SUBSTITUTE GOODS OR SERVICES; LOSS OF USE,     %
% DATA, OR PROFITS; OR BUSINESS INTERRUPTION) HOWEVER CAUSED AND ON %
% ANY THEORY OF LIABILITY, WHETHER IN CONTRACT, STRICT LIABILITY,   %
% OR TORT (INCLUDING NEGLIGENCE OR OTHERWISE) ARISING IN ANY WAY    %
% OUT OF THE USE OF THIS SOFTWARE, EVEN IF ADVISED OF THE           %
% POSSIBILITY OF SUCH DAMAGE.                                       %
% ----------------------------------------------------------------- %
\hypertarget{gc2gc}{}
\name{gc2gc}{generalized cepstral transformation}{speech parameter transformation}

\begin{synopsis}
\item [gc2gc] [ --m $M_1$ ] [ --g $G_1$ ] [ --c $C_1$ ] [ --n ] [ --u ] 
\item [\ ~~~~~~]  [ --M $M_2$ ] [ --G $G_2$ ] [ --C $C_2$ ] [ --N ] [ --U ] [ {\em infile} ]
\end{synopsis}

\begin{qsection}{DESCRIPTION}
{\em gc2gc} uses a regressive equation 
to transform a sequence of generalized cepstral coefficients 
with power parameter $\gamma_1$ from {\em infile} (or standard input)
into generalized cepstral coefficients with power parameter $\gamma_2$, 
sending the result to standard output.

Input and output data are in float format.

The regressive equation for the generalized cepstral coefficients 
follows.
\begin{displaymath}
  c_{\gamma_2}(m) = c_{\gamma_1}(m) + \sum_{k=1}^{m-1}\frac{k}{m}
                      (\gamma_2 c_{\gamma_1}(k)c_{\gamma_2}(m-k)
                  -\gamma_1 c_{\gamma_2}(k)
                        c_{\gamma_1}(m-k)), \qquad m>0.
\end{displaymath}
For the above equation, in case $\gamma_1=-1, \gamma_2=0$,
then LPC cepstral coefficients are obtained from the LPC coefficients,
in case $\gamma_1=0, \gamma_2=1$, minimum phase impulse response is
obtained from cepstral coefficients.

If the coefficients $c_\gamma(m)$ have not been normalized,
then the input and output have following form.
\begin{displaymath}
1+\gamma c_\gamma(0), \gamma c_\gamma(1), \dots, \gamma c_\gamma(M)
\end{displaymath}
The following applies to the case the coefficients are normalized,
\begin{displaymath}
K_\alpha,\gamma c_\gamma'(1),\dots, \gamma c_\gamma'(M)
\end{displaymath}

\end{qsection}

\begin{options} 
        \argm{m}{M_1}{order of generalized cepstrum (input)}{25}
        \argm{g}{G_1}{gamma of generalized cepstrum (input)\\
                        $\gamma_1 = G_1$}{0}
        \argm{c}{C_1}{gamma of generalized cepstrum (input)\\
                        $\gamma_1 =-1 / $(int)$ C_1$\\
                        $C_1$ must be $C_1 \geq 1$}{}
        \argm{n}{}{regard input as normalized cepstrum}{FALSE}
        \argm{u}{}{regard input as multiplied by $\gamma_1$}{FALSE}
        \argm{M}{M_2}{order of generalized cepstrum (output)}{25}
        \argm{G}{G_2}{gamma of generalized cepstrum (output)\\
                        $\gamma_2 = G_2$}{1}
        \argm{C}{C_2}{gamma of mel-generalized cepstrum (output)\\
                        $\gamma_2 =-1 / $(int)$ G_2$\\
                        $C_2$ must be $C_2 \geq 1$}{}
        \argm{N}{}{regard output as normalized cepstrum}{FALSE}
        \argm{U}{}{regard output as multiplied by $\gamma_1$}{FALSE}
\end{options}

\begin{qsection}{EXAMPLE} 
In the following example, generalized cepstral coefficients
with $M=10$ and $\gamma_1=-0.5$ are read in float format from
{\em data.gcep} file, transformed into 30-th order cepstral coefficients,
and written to {\em data.cep}:
\begin{quote}
 \verb!gc2gc -m 10 -c 2 -M 30 -G 0 < data.gcep > data.cep!
\end{quote} 
\end{qsection}

\begin{qsection}{SEE ALSO}
\hyperlink{gcep}{gcep},
\hyperlink{mgcep}{mgcep},
\hyperlink{freqt}{freqt},
\hyperlink{mgc2mgc}{mgc2mgc},
\hyperlink{lpc2c}{lpc2c}
\end{qsection}
