% ----------------------------------------------------------------
%       Speech Signal Processing Toolkit (SPTK): version 3.0
%                      SPTK Working Group
% 
%                Department of Computer Science
%                Nagoya Institute of Technology
%                             and
%   Interdisciplinary Graduate School of Science and Engineering
%                Tokyo Institute of Technology
%                   Copyright (c) 1984-2000
%                     All Rights Reserved.
% 
% Permission is hereby granted, free of charge, to use and
% distribute this software and its documentation without
% restriction, including without limitation the rights to use,
% copy, modify, merge, publish, distribute, sublicense, and/or
% sell copies of this work, and to permit persons to whom this
% work is furnished to do so, subject to the following conditions:
% 
%   1. The code must retain the above copyright notice, this list
%      of conditions and the following disclaimer.
% 
%   2. Any modifications must be clearly marked as such.
%                                                                        
% NAGOYA INSTITUTE OF TECHNOLOGY, TOKYO INSITITUTE OF TECHNOLOGY,
% SPTK WORKING GROUP, AND THE CONTRIBUTORS TO THIS WORK DISCLAIM
% ALL WARRANTIES WITH REGARD TO THIS SOFTWARE, INCLUDING ALL
% IMPLIED WARRANTIES OF MERCHANTABILITY AND FITNESS, IN NO EVENT
% SHALL NAGOYA INSTITUTE OF TECHNOLOGY, TOKYO INSITITUTE OF
% TECHNOLOGY, SPTK WORKING GROUP, NOR THE CONTRIBUTORS BE LIABLE
% FOR ANY SPECIAL, INDIRECT OR CONSEQUENTIAL DAMAGES OR ANY
% DAMAGES WHATSOEVER RESULTING FROM LOSS OF USE, DATA OR PROFITS,
% WHETHER IN AN ACTION OF CONTRACT, NEGLIGENCE OR OTHER TORTIOUS
% ACTION, ARISING OUT OF OR IN CONNECTION WITH THE USE OR
% PERFORMANCE OF THIS SOFTWARE.
% ----------------------------------------------------------------
%
\name{gc2gc}{generalized cepstral transformation}{speech parameter transformation}

\begin{synopsis}
\item [gc2gc] [ --m $M_1$ ] [ --g $G_1$ ] [ --n ] [ --u ] 
\item [\ ~~~~~~]  [ --M $M_2$ ] [ --G $G_2$ ] [ --N ] [ --U ] [ {\em infile} ]
\end{synopsis}

\begin{qsection}{DESCRIPTION}
{\em gc2gc} uses a regressive equation 
to transform a sequence of generalized cepstral coeffcients 
with power parameter $\gamma_1$ from {\em infile} (or standard input)
into generalized cepstral coefficients with power parameter $\gamma_2$, 
sending the result to standard output.

Input and output data are in float format.

The regressive equation for the generalized cepstral coefficients 
follows.
\begin{displaymath}
  c_{\gamma_2}(m) = c_{\gamma_1}(m) + \sum_{k=1}^{m-1}\frac{k}{m}
		      (\gamma_2 c_{\gamma_1}(k)c_{\gamma_2}(m-k)
     		  -\gamma_1 c_{\gamma_2}(k)
			c_{\gamma_1}(m-k)), \qquad m>0.
\end{displaymath}
For the above equation, in case $\gamma_1=-1, \gamma_2=0$,
then LPC cepstral coefficients are obtained from the LPC coefficients,
in case $\gamma_1=0, \gamma_2=1$, minimum phase impulse response is
obtained from cepstral coefficients.

If the coefficients $c_\gamma(m)$ have not been normalized,
then the input and output have following form.
\begin{displaymath}
1+\gamma c_\gamma(0), \gamma c_\gamma(1), \dots, \gamma c_\gamma(M)
\end{displaymath}
The following applies to the case the coefficients are normalized,
\begin{displaymath}
K_\alpha,\gamma c_\gamma'(1),\dots, \gamma c_\gamma'(M)
\end{displaymath}

\end{qsection}

\begin{options}
       -m m  :   [25]
       -g g  :   [0]
       -n    :     [FALSE]
       -u    :     [FALSE]
       -M M  :  [25]
       -G G  :  [1]
       -N    :    [FALSE]
       -U    :    [FALSE]
	\argm{m}{M_1}{order of generalized cepstrum (input)}{25}
	\argm{g}{G_1}{gamma of generalized cepstrum (input)
  	 	      If $G_1 > 1.0$ then $\gamma_1=-1 / G_1$.}{0}
	\argm{n}{}{regard input as normalized cepstrum}{FALSE}
	\argm{u}{}{regard input as multiplied by $\gamma_1$}{FALSE}
	\argm{M}{M_2}{order of generalized cepstrum (output)}{25}
	\argm{G}{G_2}{gamma of generalized cepstrum (output)
		      If $G_2 > 1.0$ then $\gamma_2=-1 / G_2$.}{1}
	\argm{N}{}{regard output as normalized cepstrum}{FALSE}
	\argm{U}{}{regard output as multiplied by $\gamma_1$}{FALSE}
\end{options}

\begin{qsection}{EXAMPLE}
In the following example, generalized cepstral coefficients
with $M=10$ and $\gamma_1=-0.5$ are read in float format from
{\em data.gcep} file, transformed into 30 order cepstral coefficients,
and written to {\em data.cep}:
\begin{quote}
 \verb!gc2gc -m 10 -g 2 -M 30 -G 0 < data.gcep > data.cep!
\end{quote} 
\end{qsection}

\begin{qsection}{SEE ALSO}
gcep, mgcep, freqt, mgc2mgc, lpc2c
\end{qsection}
