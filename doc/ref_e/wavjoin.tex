% ----------------------------------------------------------------- %
%             The Speech Signal Processing Toolkit (SPTK)           %
%             developed by SPTK Working Group                       %
%             http://sp-tk.sourceforge.net/                         %
% ----------------------------------------------------------------- %
%                                                                   %
%  Copyright (c) 1984-2007  Tokyo Institute of Technology           %
%                           Interdisciplinary Graduate School of    %
%                           Science and Engineering                 %
%                                                                   %
%                1996-2014  Nagoya Institute of Technology          %
%                           Department of Computer Science          %
%                                                                   %
% All rights reserved.                                              %
%                                                                   %
% Redistribution and use in source and binary forms, with or        %
% without modification, are permitted provided that the following   %
% conditions are met:                                               %
%                                                                   %
% - Redistributions of source code must retain the above copyright  %
%   notice, this list of conditions and the following disclaimer.   %
% - Redistributions in binary form must reproduce the above         %
%   copyright notice, this list of conditions and the following     %
%   disclaimer in the documentation and/or other materials provided %
%   with the distribution.                                          %
% - Neither the name of the SPTK working group nor the names of its %
%   contributors may be used to endorse or promote products derived %
%   from this software without specific prior written permission.   %
%                                                                   %
% THIS SOFTWARE IS PROVIDED BY THE COPYRIGHT HOLDERS AND            %
% CONTRIBUTORS "AS IS" AND ANY EXPRESS OR IMPLIED WARRANTIES,       %
% INCLUDING, BUT NOT LIMITED TO, THE IMPLIED WARRANTIES OF          %
% MERCHANTABILITY AND FITNESS FOR A PARTICULAR PURPOSE ARE          %
% DISCLAIMED. IN NO EVENT SHALL THE COPYRIGHT OWNER OR CONTRIBUTORS %
% BE LIABLE FOR ANY DIRECT, INDIRECT, INCIDENTAL, SPECIAL,          %
% EXEMPLARY, OR CONSEQUENTIAL DAMAGES (INCLUDING, BUT NOT LIMITED   %
% TO, PROCUREMENT OF SUBSTITUTE GOODS OR SERVICES; LOSS OF USE,     %
% DATA, OR PROFITS; OR BUSINESS INTERRUPTION) HOWEVER CAUSED AND ON %
% ANY THEORY OF LIABILITY, WHETHER IN CONTRACT, STRICT LIABILITY,   %
% OR TORT (INCLUDING NEGLIGENCE OR OTHERWISE) ARISING IN ANY WAY    %
% OUT OF THE USE OF THIS SOFTWARE, EVEN IF ADVISED OF THE           %
% POSSIBILITY OF SUCH DAMAGE.                                       %
% ----------------------------------------------------------------- %
\hypertarget{wavjoin}{}
\name{wavjoin}{join two monaural WAV files}{data operation}

\begin{synopsis}
\item[wavjoin] [ --i $I$ ] [ --o $O$]
\end{synopsis}

\begin{qsection}{DESCRIPTION}
{\em wavjoin} makes a stereo WAV file by joining two monaural WAV files.

\end{qsection}

\begin{options}
	\argm{i}{I}{Input WAV files or directories}{}
        \argm{o}{O}{Output WAV file or directory}{}
\end{options}

\begin{qsection}{EXAMPLE}
 In the following command, {\em wavjoin} joins the monaural WAV files {\em file0.wav} and {\em file1.wav} 
 and outputs the stereo WAV file {\em file0\_file1.wav}.
\begin{quote}
 \verb!wavjoin -i file0.wav file1.wav -o file0_file1.wav!
\end{quote}
 If input directories are specified, {\em wavjoin} joins all the WAV files 
 that have common names between the directories.
\begin{quote}
  \verb!wavjoin -i input_directory0 input_directory1 -o output_directory!
\end{quote}
\end{qsection}

\begin{qsection}{NOTICE}
 {\em wavjoin} does not distinguish between small and capital letters of the file extension.
 The first input WAV file or directory is related to channel 0, and the other is related to channel 1.
\end{qsection}

\begin{qsection}{SEE ALSO}
\hyperlink{raw2wav}{raw2wav},
\hyperlink{wavsplit}{wavsplit}
\end{qsection}
