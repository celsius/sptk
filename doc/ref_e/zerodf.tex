% ----------------------------------------------------------------
%       Speech Signal Processing Toolkit (SPTK): version 3.0
%                      SPTK Working Group
% 
%                Department of Computer Science
%                Nagoya Institute of Technology
%                             and
%   Interdisciplinary Graduate School of Science and Engineering
%                Tokyo Institute of Technology
%                   Copyright (c) 1984-2000
%                     All Rights Reserved.
% 
% Permission is hereby granted, free of charge, to use and
% distribute this software and its documentation without
% restriction, including without limitation the rights to use,
% copy, modify, merge, publish, distribute, sublicense, and/or
% sell copies of this work, and to permit persons to whom this
% work is furnished to do so, subject to the following conditions:
% 
%   1. The code must retain the above copyright notice, this list
%      of conditions and the following disclaimer.
% 
%   2. Any modifications must be clearly marked as such.
%                                                                        
% NAGOYA INSTITUTE OF TECHNOLOGY, TOKYO INSITITUTE OF TECHNOLOGY,
% SPTK WORKING GROUP, AND THE CONTRIBUTORS TO THIS WORK DISCLAIM
% ALL WARRANTIES WITH REGARD TO THIS SOFTWARE, INCLUDING ALL
% IMPLIED WARRANTIES OF MERCHANTABILITY AND FITNESS, IN NO EVENT
% SHALL NAGOYA INSTITUTE OF TECHNOLOGY, TOKYO INSITITUTE OF
% TECHNOLOGY, SPTK WORKING GROUP, NOR THE CONTRIBUTORS BE LIABLE
% FOR ANY SPECIAL, INDIRECT OR CONSEQUENTIAL DAMAGES OR ANY
% DAMAGES WHATSOEVER RESULTING FROM LOSS OF USE, DATA OR PROFITS,
% WHETHER IN AN ACTION OF CONTRACT, NEGLIGENCE OR OTHER TORTIOUS
% ACTION, ARISING OUT OF OR IN CONNECTION WITH THE USE OR
% PERFORMANCE OF THIS SOFTWARE.
% ----------------------------------------------------------------
%
\name{zerodf}{all zero digital filter for speech synthesis}{filters for
speech synthesis}

\begin{synopsis}
\item[zerodf] [ --m $M$ ] [ --p $P$ ] [ --i $I$ ] [ --t ] [ --k ]
		{\em bfile} [ {\em infile} ]
\end{synopsis}

\begin{qsection}{DESCRIPTION}
This commands reads data from the assigned {\em infile},
passes it through a FIR filter in standard form
built from the coefficients $b(0), b(1), \ldots, b(M)$
read from {\em bfile},
and sends the results to the standard output.
\par
Input and output data are in float format.
\par
The transfer function $H(z)$ of a FIR filter in standard form is 
\begin{displaymath}
H(z) = \sum_{m=0}^{M} b(m) z^{-m}
\end{displaymath}
\end{qsection}

\begin{options}
	\argm{m}{M}{order of coefficients}{25}
	\argm{p}{P}{frame period}{100}
	\argm{i}{I}{interpolation period}{1}
	\argm{t}{}{transpose filter}{FALSE}
	\argm{k}{}{filtering without gain}{FALSE}
\end{options}

\begin{qsection}{EXAMPLE}
Excitation is generated from pitch information read in float format
from {\em data.pitch}, passes through a FIR filter with
coefficients read from {\em data.b},
and the synthesized speech is written to {\em data.syn}:
\begin{quote}
  \verb!excite < data.pitch  | zerodf data.b > data.syn!
\end{quote}
\end{qsection}

\begin{qsection}{SEE ALSO}
  poledf, lmadf
\end{qsection}
