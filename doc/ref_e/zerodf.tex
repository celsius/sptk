\name{zerodf}{all zero digital filter for speech synthesis}{digital filter}

\begin{synopsis}
\item[zerodf] [ --m $M$ ] [ --p $P$ ] [ --i $I$ ] [ --t ] [ --k ]
		{\em bfile} [ {\em infile} ]
\end{synopsis}

\begin{qsection}{DESCRIPTION}
This commands reads data from the assigned {\em infile},
passes it through a FIR filter in standard form
built from the coefficients $b(0), b(1), \ldots, b(M)$
read from {\em bfile},
and sends the results to the standard output.
\par
Input and output data are in float format.
\par
The transfer function $H(z)$ of a FIR filter in standard form is 
\begin{displaymath}
H(z) = \sum_{m=0}^{M} b(m) z^{-m}
\end{displaymath}
\end{qsection}

\begin{options}
	\argm{m}{M}{order of coefficients}{25}
	\argm{p}{P}{frame period}{100}
	\argm{i}{I}{interpolation period}{1}
	\argm{t}{}{transpose filter}{FALSE}
	\argm{k}{}{filtering without gain}{FALSE}
\end{options}

\begin{qsection}{EXAMPLE}
Excitation is generated from pitch information read in float format
from {\em data.pitch}, passes through a FIR filter with
coefficients read from {\em data.b},
and the synthesized speech is written to {\em data.syn}:
\begin{quote}
  \verb!excite < data.pitch  | zerodf data.b > data.syn!
\end{quote}
\end{qsection}

\begin{qsection}{SEE ALSO}
  poledf, lmadf
\end{qsection}
