\name{uscd}{sampling rate conversion from
8|10|12|16kHz to 11.025|22.05|44.1kHz}{sampling rate transformation}

\begin{synopsis}
\item [uscd] [ --s $S$ $S$] [ +{\em type} ] [ {\em infile} ] [ {\em outfile} ]
\item [uscd] [ --s $S$ $S$] [ +{\em type} ] {\em infile1} $\cdots$ [ {\em infileN}] {\em outdir} 
\end{synopsis}

\begin{qsection}{DESCRIPTION}
The command {\em uscd} converts from
8, 10, 12, 16kHz to 11.025, 22.05, 44.1kHz sampling rate.
If input and/or output files are not assigned,
then the standard input and/or output are used, respectively.
Also, several files can be assigned at same time.
In this case the output is automatically written in the assigned
{\em outdir} directory for every input file
changing the corresponding suffix.

Also, the assignment the following sampling rates 11.025��22.05��44.1kHz
can be done by 11, 22, 44, respectively.
\par
\end{qsection}

\begin{options}
	\argm{s}{S1~S2}{input/output sampling frequency\\
		\begin{tabular}{lll} \\[-1zh]
			S1 & $8|10|12|16$ & sampling rate of input data\\
			S2 & $11.025|22.05|44.1$ & sampling rate of output data
		\end{tabular}
                where the assignment the following sampling rates 11.025��
                22.05��44.1kHz can be done by 11, 22, 44, respectively.}
                {$10~11.025$}
	\argp{t}{input data format\\
		\begin{tabular}{ll} \\[-1zh]
			s & short (2bytes) \\
			f & float (4bytes)
		\end{tabular}\\}{s}
\end{options}

\begin{qsection}{EXAMPLE}
In the example below, speech data sampled at 16kHz in short format
is read from {\em data.16}, up sampling to 22.05kHz is undertaken,
and the results are written to {data.22}:
\begin{quote}
\verb!uscd -s 16 22.05 < data.16 > data.22!
\end{quote}
\end{qsection}

%\begin{qsection}{BUGS}
%none
%\end{qsection}

%\begin{qsection}{SEE ALSO}
%none
%\end{qsection}
