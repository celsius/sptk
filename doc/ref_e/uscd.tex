% ----------------------------------------------------------------
%       Speech Signal Processing Toolkit (SPTK): version 3.0
%                      SPTK Working Group
% 
%                Department of Computer Science
%                Nagoya Institute of Technology
%                             and
%   Interdisciplinary Graduate School of Science and Engineering
%                Tokyo Institute of Technology
%                   Copyright (c) 1984-2000
%                     All Rights Reserved.
% 
% Permission is hereby granted, free of charge, to use and
% distribute this software and its documentation without
% restriction, including without limitation the rights to use,
% copy, modify, merge, publish, distribute, sublicense, and/or
% sell copies of this work, and to permit persons to whom this
% work is furnished to do so, subject to the following conditions:
% 
%   1. The code must retain the above copyright notice, this list
%      of conditions and the following disclaimer.
% 
%   2. Any modifications must be clearly marked as such.
%                                                                        
% NAGOYA INSTITUTE OF TECHNOLOGY, TOKYO INSITITUTE OF TECHNOLOGY,
% SPTK WORKING GROUP, AND THE CONTRIBUTORS TO THIS WORK DISCLAIM
% ALL WARRANTIES WITH REGARD TO THIS SOFTWARE, INCLUDING ALL
% IMPLIED WARRANTIES OF MERCHANTABILITY AND FITNESS, IN NO EVENT
% SHALL NAGOYA INSTITUTE OF TECHNOLOGY, TOKYO INSITITUTE OF
% TECHNOLOGY, SPTK WORKING GROUP, NOR THE CONTRIBUTORS BE LIABLE
% FOR ANY SPECIAL, INDIRECT OR CONSEQUENTIAL DAMAGES OR ANY
% DAMAGES WHATSOEVER RESULTING FROM LOSS OF USE, DATA OR PROFITS,
% WHETHER IN AN ACTION OF CONTRACT, NEGLIGENCE OR OTHER TORTIOUS
% ACTION, ARISING OUT OF OR IN CONNECTION WITH THE USE OR
% PERFORMANCE OF THIS SOFTWARE.
% ----------------------------------------------------------------
%
\name{uscd}{sampling rate conversion from
8|10|12|16kHz to 11.025|22.05|44.1kHz}{sampling rate transformation}

\begin{synopsis}
\item [uscd] [ --s $S$ $S$] [ +{\em type} ] [ {\em infile} ] [ {\em outfile} ]
\item [uscd] [ --s $S$ $S$] [ +{\em type} ] {\em infile1} $\cdots$ [ {\em infileN}] {\em outdir} 
\end{synopsis}

\begin{qsection}{DESCRIPTION}
{\em uscd} converts the sample rate from one of 8, 10, 12, or 16kHz 
to one of 11.025, 22.04, or 44.1kHz.
If {\em infile} and {\em outfile} arguments are not given, 
standard input and standard output are used.
If the last argument names a directory, 
each of the preceding argument files is resampled. 
The results are stored in multiple files in that directory, 
with base names the same as the input file base names, 
but with suffixes reflecting the new sample rate.
\end{qsection}

\begin{options}
%	\argm{s}{S1~S2}{input/output sampling frequency\\
%		\begin{tabular}{lll} \\[-1ex]
%			S1 & $8|10|12|16$ & sampling rate of input data\\
%			S2 & $11.025|22.05|44.1$ & sampling rate of output data
%		\end{tabular}
%                where the assignment the following sampling rates 11.025, 
%                22.05, 44.1kHz can be done by 11, 22, 44, respectively.}
%                {$10~11.025$}
	\argm{s}{S1}{input sampling frequency  $8|10|12|16$}{10}
	\argm{S}{S2}{output sampling frequency  $11.025|22.05|44.1$\\
		$S2$ can be abbreviated as 11, 22, or 44. \\
		If the last command line argument is a directory name, 
		the suffix for the output files is 
		either ``.11'', ``.22'', or ``.44.''}
		{44.1}
	\argp{t}{input data format\\
		\begin{tabular}{ll} \\[-1ex]
			s & short (2bytes) \\
			f & float (4bytes)
		\end{tabular}\\}{s}
\end{options}

\begin{qsection}{EXAMPLE}
In the example below, speech data sampled at 16kHz in short format
is read from {\em data.16}, up sampling to 22.05kHz is undertaken,
and the results are written to {data.22}:
\begin{quote}
\verb!uscd -s 16 22.05 < data.16 > data.22!
\end{quote}
\end{qsection}

%\begin{qsection}{BUGS}
%none
%\end{qsection}

%\begin{qsection}{SEE ALSO}
%none
%\end{qsection}
