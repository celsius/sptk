% ----------------------------------------------------------------- %
%             The Speech Signal Processing Toolkit (SPTK)           %
%             developed by SPTK Working Group                       %
%             http://sp-tk.sourceforge.net/                         %
% ----------------------------------------------------------------- %
%                                                                   %
%  Copyright (c) 1984-2007  Tokyo Institute of Technology           %
%                           Interdisciplinary Graduate School of    %
%                           Science and Engineering                 %
%                                                                   %
%                1996-2011  Nagoya Institute of Technology          %
%                           Department of Computer Science          %
%                                                                   %
% All rights reserved.                                              %
%                                                                   %
% Redistribution and use in source and binary forms, with or        %
% without modification, are permitted provided that the following   %
% conditions are met:                                               %
%                                                                   %
% - Redistributions of source code must retain the above copyright  %
%   notice, this list of conditions and the following disclaimer.   %
% - Redistributions in binary form must reproduce the above         %
%   copyright notice, this list of conditions and the following     %
%   disclaimer in the documentation and/or other materials provided %
%   with the distribution.                                          %
% - Neither the name of the SPTK working group nor the names of its %
%   contributors may be used to endorse or promote products derived %
%   from this software without specific prior written permission.   %
%                                                                   %
% THIS SOFTWARE IS PROVIDED BY THE COPYRIGHT HOLDERS AND            %
% CONTRIBUTORS "AS IS" AND ANY EXPRESS OR IMPLIED WARRANTIES,       %
% INCLUDING, BUT NOT LIMITED TO, THE IMPLIED WARRANTIES OF          %
% MERCHANTABILITY AND FITNESS FOR A PARTICULAR PURPOSE ARE          %
% DISCLAIMED. IN NO EVENT SHALL THE COPYRIGHT OWNER OR CONTRIBUTORS %
% BE LIABLE FOR ANY DIRECT, INDIRECT, INCIDENTAL, SPECIAL,          %
% EXEMPLARY, OR CONSEQUENTIAL DAMAGES (INCLUDING, BUT NOT LIMITED   %
% TO, PROCUREMENT OF SUBSTITUTE GOODS OR SERVICES; LOSS OF USE,     %
% DATA, OR PROFITS; OR BUSINESS INTERRUPTION) HOWEVER CAUSED AND ON %
% ANY THEORY OF LIABILITY, WHETHER IN CONTRACT, STRICT LIABILITY,   %
% OR TORT (INCLUDING NEGLIGENCE OR OTHERWISE) ARISING IN ANY WAY    %
% OUT OF THE USE OF THIS SOFTWARE, EVEN IF ADVISED OF THE           %
% POSSIBILITY OF SUCH DAMAGE.                                       %
% ----------------------------------------------------------------- %
\hypertarget{uscd}{}
\name{uscd}{up/down-sampling from 8, 10, 12, or 16 kHz to 11.025, 22.05, or
44.1 kHz}{sampling rate transformation}
 
\begin{synopsis}
\item [uscd] [ --s $S$ $S$] [ {\em infile} ] [ {\em outfile} ]
\item [uscd] [ --s $S$ $S$] {\em infile1} $\dots$ [ {\em infileN}] {\em outdir} 
\end{synopsis}

\begin{qsection}{DESCRIPTION}
{\em uscd} converts the sample rate from one of 8, 10, 12, or 16 kHz 
to one of 11.025, 22.04, or 44.1 kHz.
If {\em infile} and {\em outfile} arguments are not given, 
standard input and output are used.
If the last argument given names a directory, 
each of the preceding argument files is re-sampled. 
The results are stored in multiple files in that directory, 
with base names the same as the input file base names, 
but with extensions indicating the new sample rate.
\end{qsection}

\begin{options}
	\argm{s}{S1}{input sampling frequency (one of 8, 10, 12 or 16)}{10}
	\argm{S}{S2}{output sampling frequency (one of 11.025, 22.05, or 44.1)\\
		$S2$ can be abbreviated as 11, 22, or 44. \\
		If the last command line argument is a directory name, 
		the suffix for the output files is 
		either ``.11'', ``.22'', or ``.44.''}
		{11.025}
\end{options}

\begin{qsection}{EXAMPLE}
In the example below, speech data sampled at 16 kHz
is read from {\em data.16}, upsampled to 22.05 kHz,
and the results are written to {data.22}:
\begin{quote}
\verb!uscd -s 16 22.05 < data.16 > data.22!
\end{quote}
\end{qsection}

%\begin{qsection}{BUGS}
%none
%\end{qsection}

\begin{qsection}{SEE ALSO}
 \hyperlink{ds}{ds},
 \hyperlink{us}{us},
 \hyperlink{us16}{us16}
\end{qsection}
