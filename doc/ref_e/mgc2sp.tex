% ----------------------------------------------------------------
%       Speech Signal Processing Toolkit (SPTK): version 3.0
%                      SPTK Working Group
% 
%                Department of Computer Science
%                Nagoya Institute of Technology
%                             and
%   Interdisciplinary Graduate School of Science and Engineering
%                Tokyo Institute of Technology
%                   Copyright (c) 1984-2000
%                     All Rights Reserved.
% 
% Permission is hereby granted, free of charge, to use and
% distribute this software and its documentation without
% restriction, including without limitation the rights to use,
% copy, modify, merge, publish, distribute, sublicense, and/or
% sell copies of this work, and to permit persons to whom this
% work is furnished to do so, subject to the following conditions:
% 
%   1. The code must retain the above copyright notice, this list
%      of conditions and the following disclaimer.
% 
%   2. Any modifications must be clearly marked as such.
%                                                                        
% NAGOYA INSTITUTE OF TECHNOLOGY, TOKYO INSITITUTE OF TECHNOLOGY,
% SPTK WORKING GROUP, AND THE CONTRIBUTORS TO THIS WORK DISCLAIM
% ALL WARRANTIES WITH REGARD TO THIS SOFTWARE, INCLUDING ALL
% IMPLIED WARRANTIES OF MERCHANTABILITY AND FITNESS, IN NO EVENT
% SHALL NAGOYA INSTITUTE OF TECHNOLOGY, TOKYO INSITITUTE OF
% TECHNOLOGY, SPTK WORKING GROUP, NOR THE CONTRIBUTORS BE LIABLE
% FOR ANY SPECIAL, INDIRECT OR CONSEQUENTIAL DAMAGES OR ANY
% DAMAGES WHATSOEVER RESULTING FROM LOSS OF USE, DATA OR PROFITS,
% WHETHER IN AN ACTION OF CONTRACT, NEGLIGENCE OR OTHER TORTIOUS
% ACTION, ARISING OUT OF OR IN CONNECTION WITH THE USE OR
% PERFORMANCE OF THIS SOFTWARE.
% ----------------------------------------------------------------
%
\hypertarget{mgc2sp}{}
\name{mgc2sp}{transform mel-generalized cepstrum to spectrum}%
{speech parameter transformation}

\begin{synopsis}
\item[mgc2sp] [ --a $A$ ] [ --g $G$ ] [ --m $M$ ]
	       [ --n ] [ --u ] [ --l $L$ ] [ --p ]
\item[\ ~~~~~] [ --o $O$ ] [ {\em infile} ]
\end{synopsis}

\begin{qsection}{DESCRIPTION}
{\em mgc2sp} calculates the log magnitude spectrum 
from mel-generalized cepstral coefficients $c_{\alpha, \gamma}(m)$
from {\em infile} (or standard input),
sending the result to standard output.

Input and output data are in float format.

The mel-generalized cepstral coefficients $c_{\alpha, \gamma}(m)$
are transformed into mel-generalized log cepstral coefficients
(refer to mgc2mgc)
and then the log magnitude spectrum is calculated(refer to spec).

When the input data is normalized by the gain,
then it can be represented as follows.
\begin{align}
K_{\alpha} &= 
	s_{\gamma}^{-1}\left(c_{\alpha,\gamma}^{(0)}(0)\right), \notag \\
c_{\alpha,\gamma}'(m) &=
          c_{\alpha,\gamma}^{(0)}(m)/\left(1+\gamma\,
	  c_{\alpha,\gamma}^{(0)}(0)\right), \qquad m = 1,2,\dots, M \notag
\end{align}

In case we represent input with $\gamma$,
if the coefficients $c_{\alpha,\gamma}(m)$ are not normalized, then
the following representation is assumed
\begin{displaymath}
1+\gamma c_{\alpha,\gamma}(0), \gamma c_{\alpha,\gamma}(1), \dots, \gamma c_{\alpha,\gamma}(M)
\end{displaymath}
if they are normalized, then
the following representation is assumed
\begin{displaymath}
K_\alpha,\gamma c_{\alpha,\gamma}'(1),\dots, \gamma c_{\alpha,\gamma}'(M)
\end{displaymath}

\end{qsection}

\begin{options}
	\argm{a}{A}{alpha $\alpha$}{0}
	\argm{g}{G}{power parameter $\gamma$ of mel-generalized cepstrum\\
			 if $G>1.0$ then $\gamma = -1/G$.}{0}
	\argm{m}{M}{order of mel-generalized cepstrum}{25}
	\argm{n}{}{regard input as normalized cepstrum}{FALSE}
	\argm{u}{}{regard input as multiplied by $\gamma$}{FALSE}
	\argm{l}{L}{FFT length}{256}
	\argm{p}{}{output phase}{FALSE}
	\argm{o}{O}{output format \\
                    if the --p option is assigned, scale of output spectrum
                    can be assigned.\\
		\begin{tabular}{ll} \\[-1ex]
			$O=0$ & $20 \times \log |H(z)|$ \\
			$O=1$ & $\ln |H(z)|$ \\
			$O=2$ & $|H(z)|$ \\[1ex]
		\end{tabular} \\
		    if the --p option is not assigned, unit of output phase
                    can be assigned.\\
		\begin{tabular}{ll} \\[-1ex]
			$O=0$ & $\arg |H(z)| \div \pi \quad [\pi \; rad.]$ \\
			$O=1$ & $\arg |H(z)| \quad [rad.]$ \\
			$O=2$ & $\arg |H(z)| \times180\div\pi\quad[deg.]$ \\
		\end{tabular}\\\hspace*{\fill}}{0}

\end{options}

\begin{qsection}{EXAMPLE}
In the following example, mel-generalized cepstral coefficients
in float format are read from {\em data.mgcep}
($M=12, \alpha=0.35, \gamma=-0.5$)
and the log magnitude spectrum is evaluated and plotted:
\begin{quote}
 \verb!mgc2sp -m 12 -a 0.35 -r -0.5 < data.mgcep | glogsp | xgr!
\end{quote} 
\end{qsection}

\begin{qsection}{SEE ALSO}
\hyperlink{c2sp}{c2sp},
\hyperlink{mgc2mgc}{mgc2mgc},
\hyperlink{gc2gc}{gc2gc},
\hyperlink{freqt}{freqt},
\hyperlink{gnorm}{gnorm},
\hyperlink{lpc2c}{lpc2c}
\end{qsection}
