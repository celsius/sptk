\name{mgc2sp}{transform mel-generalized cepstrum to spectrum}%
{speech parameter transformation}

\begin{synopsis}
\item[mgc2sp] [ --a $A$ ] [ --g $G$ ] [ --m $M$ ]
	       [ --n ] [ --u ] [ --l $L$ ] [ --p ]
\item[\ ~~~~~] [ --o $O$ ] [ {\em infile} ]
\end{synopsis}

\begin{qsection}{DESCRIPTION}
This command reads mel-generalized cepstrum coefficients
$c_{\alpha, \gamma}(m)$ and evaluated the log magnitude
spectrum.
\par
Input and output data are in float format.
\par
The mel-generalized cepstrum coefficients $c_{\alpha, \gamma}(m)$
are transformed into mel-generalized log cepstrum coefficients
(refer to mgc2mgc)
and then the log magnitude spectrum is calculated(refer to spec).

When the input data is normalized by the gain,
then it can be represented as follows.
\begin{eqnarray*}
\hspace{-15mm}&&K_{\alpha} = 
	s_{\gamma}^{-1}\left(c_{\alpha,\gamma}^{(0)}(0)\right), 
	  \qquad\qquad \\ 
\hspace{-15mm}&&c_{\alpha,\gamma}'(m) =
          c_{\alpha,\gamma}^{(0)}(m)/\left(1+\gamma\,
	  c_{\alpha,\gamma}^{(0)}(0)\right), \quad m = 1,2,\ldots, M 
\end{eqnarray*}

In case we represent input with $\gamma$,
if the coefficients $c_{\alpha,\gamma}(m)$ are not normalized, then
the following representation is assumed
\begin{displaymath}
1+\gamma c_{\alpha,\gamma}(0), \gamma c_{\alpha,\gamma}(1), \ldots, \gamma c_{\alpha,\gamma}(M)
\end{displaymath}
if they are normalized, then
the following representation is assumed
\begin{displaymath}
K_\alpha,\gamma c_{\alpha,\gamma}'(1),\ldots, \gamma c_{\alpha,\gamma}'(M)
\end{displaymath}

\end{qsection}

\begin{options}
       -o o  : output format                       [0]
                 0 (20*log|H(z)|)
                 1 (ln|H(z)|)
                 2 (|H(z)|)
             -p option is specified
                 0 (arg|H(z)|/pi     [pi rad])
                 1 (arg|H(z)|        [rad])
                 2 (arg|H(z)|*180/pi [deg])

	\argm{a}{A}{alpha $\alpha$}{0}
	\argm{g}{G}{power parameter $\gamma$ of mel-generalized cepstrum\\
			 if $G>1.0$ then $\gamma = -1/G$.}{0}
	\argm{m}{M}{order of mel-genralized cepstrum}{25}
	\argm{n}{}{regard input as normalized cepstrum}{FALSE}
	\argm{u}{}{regard input as multiplied by $\gamma$}{FALSE}
	\argm{l}{L}{FFT length}{256}
	\argm{p}{}{output phase}{FALSE}
	\argm{o}{O}{output format \\
                    if the --p option is assigned, scale of output spectrum
                    can be assinged.\\
		\begin{tabular}{ll} \\[-1zh]
			$O=0$ & $20 \times \log |H(z)|$ \\
			$O=1$ & $\ln |H(z)|$ \\
			$O=2$ & $|H(z)|$ \\[1zh]
		\end{tabular} \\
		    if the --p option is not assigned, unit of output phase
                    can be assigned.\\
		\begin{tabular}{ll} \\[-1zh]
			$O=0$ & $\arg |H(z)| \div \pi \quad [\pi \; rad.]$ \\
			$O=1$ & $\arg |H(z)| \quad [rad.]$ \\
			$O=2$ & $\arg |H(z)| \times180\div\pi\quad[deg.]$ \\
		\end{tabular}\\\hspace*{\fill}}{0}

\end{options}

\begin{qsection}{EXAMPLE}
In the following example, mel-generalized cepstrum coefficients
in float format are read from {\em data.mgcep}
($M=12, \alpha=0.35, \gamma=-0.5$)
and the log magnitude spectrum is evaluated and plotted:
\begin{quote}
 \verb!mgc2sp -m 12 -a 0.35 -r -0.5 < data.mgcep | glogsp | xgr!
\end{quote} 
\end{qsection}

\begin{qsection}{SEE ALSO}
c2sp, mgc2mgc, gc2gc, freqt, gnorm, lpc2c, c2lpc
\end{qsection}
