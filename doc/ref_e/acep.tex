% ----------------------------------------------------------------- %
%             The Speech Signal Processing Toolkit (SPTK)           %
%             developed by SPTK Working Group                       %
%             http://sp-tk.sourceforge.net/                         %
% ----------------------------------------------------------------- %
%                                                                   %
%  Copyright (c) 1984-2007  Tokyo Institute of Technology           %
%                           Interdisciplinary Graduate School of    %
%                           Science and Engineering                 %
%                                                                   %
%                1996-2014  Nagoya Institute of Technology          %
%                           Department of Computer Science          %
%                                                                   %
% All rights reserved.                                              %
%                                                                   %
% Redistribution and use in source and binary forms, with or        %
% without modification, are permitted provided that the following   %
% conditions are met:                                               %
%                                                                   %
% - Redistributions of source code must retain the above copyright  %
%   notice, this list of conditions and the following disclaimer.   %
% - Redistributions in binary form must reproduce the above         %
%   copyright notice, this list of conditions and the following     %
%   disclaimer in the documentation and/or other materials provided %
%   with the distribution.                                          %
% - Neither the name of the SPTK working group nor the names of its %
%   contributors may be used to endorse or promote products derived %
%   from this software without specific prior written permission.   %
%                                                                   %
% THIS SOFTWARE IS PROVIDED BY THE COPYRIGHT HOLDERS AND            %
% CONTRIBUTORS "AS IS" AND ANY EXPRESS OR IMPLIED WARRANTIES,       %
% INCLUDING, BUT NOT LIMITED TO, THE IMPLIED WARRANTIES OF          %
% MERCHANTABILITY AND FITNESS FOR A PARTICULAR PURPOSE ARE          %
% DISCLAIMED. IN NO EVENT SHALL THE COPYRIGHT OWNER OR CONTRIBUTORS %
% BE LIABLE FOR ANY DIRECT, INDIRECT, INCIDENTAL, SPECIAL,          %
% EXEMPLARY, OR CONSEQUENTIAL DAMAGES (INCLUDING, BUT NOT LIMITED   %
% TO, PROCUREMENT OF SUBSTITUTE GOODS OR SERVICES; LOSS OF USE,     %
% DATA, OR PROFITS; OR BUSINESS INTERRUPTION) HOWEVER CAUSED AND ON %
% ANY THEORY OF LIABILITY, WHETHER IN CONTRACT, STRICT LIABILITY,   %
% OR TORT (INCLUDING NEGLIGENCE OR OTHERWISE) ARISING IN ANY WAY    %
% OUT OF THE USE OF THIS SOFTWARE, EVEN IF ADVISED OF THE           %
% POSSIBILITY OF SUCH DAMAGE.                                       %
% ----------------------------------------------------------------- %
\hypertarget{acep}{}
\name[ref:acep-IEICE,ref:acep-IEEESP]{acep}{adaptive cepstral analysis}%
{speech analysis}
\begin{synopsis}
 \item [acep] [ --m $M$ ] [ --l $L$ ] [ --t $T$ ] [ --k $K$ ]
	      [ --p $P$ ] [ --s ] [ --e $E$ ] [ --P $Pa$ ]
\item [\ ~~~~~~~] [ {\em pefile} ] $<$ {\em infile}
\end{synopsis}

\begin{qsection}{DESCRIPTION}
	{\em acep} uses adaptive cepstral analysis
	\cite{ref:acep-IEICE}, \cite{ref:acep-IEEESP},
	to calculate cepstral coefficients from 
	unframed float data from standard input,
	sending the result to standard output.  
	If {\em pefile} is given,
	{\em acep} writes the prediction error is written to that file.

	Both input and output files are in float format.

	The algorithm to calculate recursively the
        adaptive cepstral coefficients is 
 %
\begin{align}
  \bc^{(n+1)} &= \bc^{(n)} - \mu^{(n)} \hat{\nabla} \varepsilon_{\tau}^{(n)} \notag \\
  \hat{\nabla} \varepsilon_{0}^{(n)} &= -2 e(n) \be^{(n)} \qquad ( \tau = 0 ) \notag \\
  \hat{\nabla} \varepsilon_{\tau}^{(n)} &= -2 (1 - \tau) \sum_{i=-\infty}^{n}
  \tau^{n-i} e(i) \be^{(i)} \qquad ( 0 \le \tau < 1 ) \notag \\
  \hat{\nabla} \varepsilon_{\tau}^{(n)} &= \tau \hat{\nabla}
  \varepsilon_{\tau}^{(n-1)} - 2 (1 - \tau) e(n) \be^{(n)} \notag \\
  \mu^{(n)} &= \frac{k}{M \varepsilon^{(n)}} \notag \\
  \varepsilon^{(n)} &= \lambda \varepsilon^{(n-1)}
     + (1-\lambda)e^2(n) \notag
\end{align}	
	where 
	$\bc=[c(1),\ldots,c(M)]^\top$,
	$\be^{(n)}=[e(n-1),\ldots,e(n-M)]^\top$.
	Also, the gain is expressed by $c(0)$ as follows: 
%
 \begin{displaymath}
	  c(0) = \frac{1}{2} \log \varepsilon^{(n)}
 \end{displaymath}
	In Figure \ref{fig:acep_block}, the system for adaptive cepstral
        analysis is shown.
\setcounter{figure}{0}
 \begin{figure}[h]
	\setlength{\unitlength}{.5mm}
  \begin{center}
   \begin{picture}(100,50)(0,0)
	\put(12,39){LMA filter}
	\put(-17,30){$x(n)$}
	\put(100,30){$e(n)$}
	\thicklines
	\put(0,25){\line(1,0){20}}
	\put(20,15){\framebox(55,20){
%		\( 1/D(z) = \exp \displaystyle\sum_{m=1}^{M} -c(m)\,z^{-m}\)
		$1/D(z)$}}
	\put(75,25){\vector(1,0){25}}
	\put(85,25){\circle*{1.5}}
	\put(85,25){\line(0,-1){20}}
	\put(85,5){\line(-1,0){47.5}}
	\put(37.5,5){\line(1,2){5}}
	\put(52.5,35){\vector(1,2){5}}
   \end{picture}
  \end{center}
	\caption{Adaptive cepstral analysis system}
	\label{fig:acep_block}
 \end{figure}
\end{qsection}
%
%\newpage

\begin{options}
	\argm{m}{M}{order of cepstrum}{25}
	\argm{l}{L}{leakage factor $\lambda$}{0.98}
	\argm{t}{T}{momentum constant $\tau$}{0.9}
	\argm{k}{K}{step size $k$}{0.1}
	\argm{p}{P}{output period of cepstrum}{1}
	\argm{s}{}{output smoothed cepstrum}{FALSE}
	\argm{e}{E}{minimum value for $\varepsilon^{(n)}$}{0.0}
	\argm{P}{Pa}{number of coefficients of the LMA filter
                     using the Pad\'e approximation. $Pa$ should be 4 or 5.}{4}
\end{options}

\begin{qsection}{EXAMPLE}
        In this example, the speech data is in the file {\em data.f} in
        float format and the prediction error can be found in
        {\em data.er}. The cepstral coefficients are written to the file
        {\em data.acep} for every block of 100 samples.
 \begin{quote}
	\verb!acep -m 15 -p 100 data.er < data.f > data.acep!
 \end{quote} 
\end{qsection}

\begin{qsection}{NOTICE}
$Pa$ = 4 or 5
\end{qsection}

\begin{qsection}{SEE ALSO}
\hyperlink{uels}{uels}, 
\hyperlink{gcep}{gcep}, 
\hyperlink{mcep}{mcep}, 
\hyperlink{mgcep}{mgcep},
\hyperlink{amcep}{amcep},
\hyperlink{agcep}{agcep},
\hyperlink{lmadf}{lmadf}
\end{qsection}
