% ----------------------------------------------------------------- %
%             The Speech Signal Processing Toolkit (SPTK)           %
%             developed by SPTK Working Group                       %
%             http://sp-tk.sourceforge.net/                         %
% ----------------------------------------------------------------- %
%                                                                   %
%  Copyright (c) 1984-2007  Tokyo Institute of Technology           %
%                           Interdisciplinary Graduate School of    %
%                           Science and Engineering                 %
%                                                                   %
%                1996-2010  Nagoya Institute of Technology          %
%                           Department of Computer Science          %
%                                                                   %
% All rights reserved.                                              %
%                                                                   %
% Redistribution and use in source and binary forms, with or        %
% without modification, are permitted provided that the following   %
% conditions are met:                                               %
%                                                                   %
% - Redistributions of source code must retain the above copyright  %
%   notice, this list of conditions and the following disclaimer.   %
% - Redistributions in binary form must reproduce the above         %
%   copyright notice, this list of conditions and the following     %
%   disclaimer in the documentation and/or other materials provided %
%   with the distribution.                                          %
% - Neither the name of the SPTK working group nor the names of its %
%   contributors may be used to endorse or promote products derived %
%   from this software without specific prior written permission.   %
%                                                                   %
% THIS SOFTWARE IS PROVIDED BY THE COPYRIGHT HOLDERS AND            %
% CONTRIBUTORS "AS IS" AND ANY EXPRESS OR IMPLIED WARRANTIES,       %
% INCLUDING, BUT NOT LIMITED TO, THE IMPLIED WARRANTIES OF          %
% MERCHANTABILITY AND FITNESS FOR A PARTICULAR PURPOSE ARE          %
% DISCLAIMED. IN NO EVENT SHALL THE COPYRIGHT OWNER OR CONTRIBUTORS %
% BE LIABLE FOR ANY DIRECT, INDIRECT, INCIDENTAL, SPECIAL,          %
% EXEMPLARY, OR CONSEQUENTIAL DAMAGES (INCLUDING, BUT NOT LIMITED   %
% TO, PROCUREMENT OF SUBSTITUTE GOODS OR SERVICES; LOSS OF USE,     %
% DATA, OR PROFITS; OR BUSINESS INTERRUPTION) HOWEVER CAUSED AND ON %
% ANY THEORY OF LIABILITY, WHETHER IN CONTRACT, STRICT LIABILITY,   %
% OR TORT (INCLUDING NEGLIGENCE OR OTHERWISE) ARISING IN ANY WAY    %
% OUT OF THE USE OF THIS SOFTWARE, EVEN IF ADVISED OF THE           %
% POSSIBILITY OF SUCH DAMAGE.                                       %
% ----------------------------------------------------------------- %
\hypertarget{amcep}{}
\name[ref:amcep-IEICE,ref:amcep-ICASSP92]{amcep}{adaptive mel-cepstral analysis}%
{speech analysis}

\begin{synopsis}
\item [amcep] [ --m $M$ ] [ --a $A$ ] [ --l $L$ ] [ --t $T$ ] [ --k $K$ ]
	      [ --p $P$ ] [ --s ] [ --e $E$ ]
\item [\ ~~~~~~~]  [--P $Pa$ ] [ {\em pefile} ] $<$ {\em infile}
\end{synopsis}

\begin{qsection}{DESCRIPTION}
	{\em amcep} uses adaptive mel-cepstral analysis 
	to calculate mel-cepstral coefficients $c_{\alpha}(m)$ 
	from unframed float data from standard input, 
	sending the result to standard output. 
	If {\em pefile} is given, 
	{\em amcep} writes the prediction error to that file.

	The format for input and output data is float.

	The algorithm to calculate recursively the
        adaptive mel-cepstral coefficients $b(m)$ is 
\begin{align}
  \bc^{(n+1)} &= \bb^{(n)} - \mu^{(n)} \hat{\nabla} \varepsilon_{\tau}^{(n)} \notag \\
  \hat{\nabla} \varepsilon_0^{(n)} &= -2 e(n) \be^{(n)}_{\Phi} \qquad ( \tau = 0 )\notag\\
  \hat{\nabla} \varepsilon_{\tau}^{(n)} &= -2 (1 - \tau) \sum_{i=-\infty}^{n} \tau^{n-i} e(i) \be^{(i)}_{\Phi} \qquad ( 0 \le \tau < 1 )\notag\\
  \hat{\nabla} \varepsilon_{\tau}^{(n)} &= \tau \hat{\nabla} \varepsilon_{\tau}^{(n-1)} - 2 (1 - \tau) e(n) \be^{(n)}_{\Phi} \notag\\
  \mu^{(n)} &= \frac{k}{M \varepsilon^{(n)}} \notag\\
  \varepsilon^{(n)} &= \lambda \varepsilon^{(n-1)}
     + (1-\lambda)e^2(n) \notag
\end{align}	

\setcounter{figure}{0}
\begin{figure}[h]
\begin{center}
\setlength{\unitlength}{1.5mm}
1\begin{picture}(50,24)(-3,-8)
  \thicklines
  \put(0,0){\line(0,1){4}}		%right triangle
  \put(0,0){\line(3,2){3}}
  \put(0,4){\line(3,-2){3}}
  \put(2,6){\makebox(0,0){$1-\alpha ^2$}}
  
  \put(13.5,7){\line(0,1){4}}  		%left triangle
  \put(13.5,7){\line(-3,2){3}}
  \put(13.5,11){\line(-3,-2){3}}
  \put(12.5,9){\makebox(0,0){\small $\alpha$}}

  \put(31,10){\line(1,0){4}}  		%down triangle
  \put(31,10){\line(2,-3){2}}
  \put(35,10){\line(-2,-3){2}}
  \put(33,9){\makebox(0,0){\small $\alpha$}}

  \put(45,10){\line(1,0){4}}  		%down triangle
  \put(45,10){\line(2,-3){2}}
  \put(49,10){\line(-2,-3){2}}
  \put(47,9){\makebox(0,0){\small $\alpha$}}

  \put(10,0){\framebox(4,4){$z^{-1}$}}
  \put(24,0){\framebox(4,4){$z^{-1}$}}
  \put(38,0){\framebox(4,4){$z^{-1}$}}
  \put(52,0){\framebox(4,4){$z^{-1}$}}

  \put(-3,-2){\makebox(0,0){$e(n)$}}
  \put(-4,2){\line(1,0){4}}
  \put(3,2){\vector(1,0){2.5}}
  \put(6.5,2){\circle{2}}
  \put(6.5,2){\makebox(0,0){\scriptsize $+$}}
%  \put(5,0.5){\makebox(0,0){\scriptsize $+$}}
%  \put(5,3.5){\makebox(0,0){\scriptsize $+$}}
  \put(7.5,2){\line(1,0){2.5}}
  \put(14,2){\line(1,0){10}}
  \put(17.5,2){\line(0,1){7}}
  \put(17.5,9){\line(-1,0){4}}
  \put(10.5,9){\line(-1,0){4}}
  \put(6.5,9){\vector(0,-1){6}}

  \put(17.5,2){\circle*{0.7}}
  \put(22,2){\circle*{0.7}}

  \put(28,2){\vector(1,0){4}}
  \put(34,2){\line(1,0){4}}
  \put(33,2){\circle{2}}
  \put(33,2){\makebox(0,0){\scriptsize $+$}}
%  \put(31.5,0.5){\makebox(0,0){\scriptsize $+$}}
%  \put(31.5,3.5){\makebox(0,0){\scriptsize $+$}}
  \put(36,2){\circle*{0.7}}

  \put(42,2){\vector(1,0){4}}
  \put(48,2){\line(1,0){4}}
  \put(47,2){\circle{2}}
  \put(47,2){\makebox(0,0){\scriptsize $+$}}
%  \put(45.5,0.5){\makebox(0,0){\scriptsize $+$}}
%  \put(45.5,3.5){\makebox(0,0){\scriptsize $+$}}
  \put(44,2){\circle*{0.7}}
  \put(50,2){\circle*{0.7}}

  \put(22,2){\line(0,1){2}}
  \put(22,4){\vector(1,1){10.2}}
  \put(33,14){\line(0,-1){4}}
  \put(33,7){\vector(0,-1){4}}
  \put(44,2){\line(0,1){2}}
  \put(44,4){\vector(-1,1){10.2}}
  \put(33,15){\circle{2}}
  \put(33,15){\makebox(0,0){\scriptsize $+$}}
  \put(31,15){\makebox(0,0){\scriptsize $-$}}
%  \put(35,15){\makebox(0,0){\scriptsize $+$}}
  
  \put(36,2){\line(0,1){2}}
  \put(36,4){\vector(1,1){10.2}}
  \put(47,14){\line(0,-1){4}}
  \put(47,7){\vector(0,-1){4}}
  \put(58,2){\line(0,1){2}}
  \put(58,4){\vector(-1,1){10.2}}
  \put(47,15){\circle{2}}
  \put(47,15){\makebox(0,0){\scriptsize $+$}}
  \put(45,15){\makebox(0,0){\scriptsize $-$}}
%  \put(49,15){\makebox(0,0){\scriptsize $+$}}

  \put(50,2){\line(0,1){2}}
  \put(50,4){\vector(1,1){10.2}}
  \put(56,2){\vector(1,0){4}}
  \put(58,2){\circle*{0.7}}

  \put(22,2){\vector(0,-1){6}}
  \put(22,-7){\makebox(0,0){$e_1(n)$}}
  \put(36,2){\vector(0,-1){6}}
  \put(36,-7){\makebox(0,0){$e_2(n)$}}
  \put(50,2){\vector(0,-1){6}}
  \put(50,-7){\makebox(0,0){$e_3(n)$}}
\end{picture}
\caption{Filter $\Phi_m(z)$}
\label{fig:mcep_Phi}
\end{center}
\end{figure}
where
$\bb=[b(1), b(2), \ldots, b(M)]^\top$,
$\be_{\it\Phi}^{(n)} =[e_1(n) ,e_2(n), \ldots, e_{M}(n)]^T$,
 $e_m(n)$ is the output of the inverse filter,
which is obtained as shown
in Figure \ref{fig:mcep_Phi}, passing $e(n)$ through the filter
 ${\Phi}_m(z)$.
\par
The coefficients $b(m)$ are equivalent to the coefficients of
the MLSA filter. The mel-cepstral coefficients $c_{\alpha}(m)$
can be obtained from $b(m)$ through a linear transformation
(refer to \hyperlink{b2mc}{b2mc} and \hyperlink{mc2b}{mc2b}).

In Figure \ref{fig:mcep_block}, the adaptive mel-cepstral 
analysis system is shown.

The filter $1/D(z)$ is realized by a MLSA filter.

\begin{figure}[h]
\begin{center}
\setlength{\unitlength}{1.0mm}
\begin{picture}(100,38)(0,0)
  \thicklines
  \put(20,10){\framebox(60,16){\small
	$\displaystyle1/D(z)=\exp \sum_{m=1}^{M}-b(m) \,\Phi_m(z)$}}
  \put(0,18){\line(1,0){20}}
  \put(80,18){\vector(1,0){25}}
  \put(93,18){\circle*{1}}
  \put(93,18){\line(0,-1){16}}
  \put(93,2){\line(-1,0){17}}
  \put(61,-3){\framebox(15,10){\normalsize $\Phi_m(z)$}}
  \put(61,2){\line(-1,0){19}}
  \put(42,2){\line(1,2){4}}
  \put(52,26){\vector(1,2){4}}
  \put(5,20){\makebox(0,0)[lb]{$x(n)$}}
  \put(97,20){\makebox(0,0)[lb]{$e(n)$}}
\end{picture}
\caption{Adaptive mel-cepstral analysis system}
\label{fig:mcep_block}
\end{center}
\end{figure}

\end{qsection}

\begin{options}
	\argm{m}{M}{order of mel-cepstrum}{25}
	\argm{a}{A}{all-pass constant $\alpha$}{0.35}
	\argm{l}{L}{leakage factor $\lambda$}{0.98}
	\argm{t}{T}{momentum constant $\tau$}{0.9}
	\argm{k}{K}{step size $k$}{0.1}
	\argm{p}{P}{output period of mel-cepstrum}{1}
	\argm{s}{}{output smoothed mel-cepstrum}{FALSE}
	\argm{e}{E}{minimum value for $\varepsilon^{(n)}$}{0.0}
	\argm{P}{Pa}{number of coefficients of the MLSA filter
                     using the Pad\'e approximation. $Pa$ should be $4$ or $5$.}{4}
\end{options}

\begin{qsection}{EXAMPLE}
	In this example, the speech data is in the file {\em data.f}
        in float format, and the adaptive mel-cepstral coefficients
        are written to
        the file {\em data.amcep} for every block of 100 samples:
\begin{quote}
 \verb!amcep -m 15 -p 100  < data.f > data.amcep!
\end{quote} 
\end{qsection}

\begin{qsection}{SEE ALSO}
\hyperlink{acep}{acep}, 
\hyperlink{agcep}{agcep}, 
\hyperlink{mc2b}{mc2b}, 
\hyperlink{b2mc}{b2mc}, 
\hyperlink{mlsadf}{mlsadf}
\end{qsection}
