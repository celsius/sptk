% ----------------------------------------------------------------
%       Speech Signal Processing Toolkit (SPTK): version 3.0
%                      SPTK Working Group
% 
%                Department of Computer Science
%                Nagoya Institute of Technology
%                             and
%   Interdisciplinary Graduate School of Science and Engineering
%                Tokyo Institute of Technology
%                   Copyright (c) 1984-2000
%                     All Rights Reserved.
% 
% Permission is hereby granted, free of charge, to use and
% distribute this software and its documentation without
% restriction, including without limitation the rights to use,
% copy, modify, merge, publish, distribute, sublicense, and/or
% sell copies of this work, and to permit persons to whom this
% work is furnished to do so, subject to the following conditions:
% 
%   1. The code must retain the above copyright notice, this list
%      of conditions and the following disclaimer.
% 
%   2. Any modifications must be clearly marked as such.
%                                                                        
% NAGOYA INSTITUTE OF TECHNOLOGY, TOKYO INSITITUTE OF TECHNOLOGY,
% SPTK WORKING GROUP, AND THE CONTRIBUTORS TO THIS WORK DISCLAIM
% ALL WARRANTIES WITH REGARD TO THIS SOFTWARE, INCLUDING ALL
% IMPLIED WARRANTIES OF MERCHANTABILITY AND FITNESS, IN NO EVENT
% SHALL NAGOYA INSTITUTE OF TECHNOLOGY, TOKYO INSITITUTE OF
% TECHNOLOGY, SPTK WORKING GROUP, NOR THE CONTRIBUTORS BE LIABLE
% FOR ANY SPECIAL, INDIRECT OR CONSEQUENTIAL DAMAGES OR ANY
% DAMAGES WHATSOEVER RESULTING FROM LOSS OF USE, DATA OR PROFITS,
% WHETHER IN AN ACTION OF CONTRACT, NEGLIGENCE OR OTHER TORTIOUS
% ACTION, ARISING OUT OF OR IN CONNECTION WITH THE USE OR
% PERFORMANCE OF THIS SOFTWARE.
% ----------------------------------------------------------------
%
\name{ignorm}{inverse gain normalization}{signal processing}
\begin{synopsis}
\item [ignorm] [ --m $M$ ] [ --g $G$ ] [ {\em infile} ]
\end{synopsis}

\begin{qsection}{DESCRIPTION}
{\em ignorm} un-normalizes normalized generalized cepstrum coefficients
$c_\gamma(m)$ from {\em infile} (or standard input), 
sending the un-normalized generalized cepstrum coefficients 
to standard output.

Input and output data are in float format.

To convert normalized generalized cepstrum coefficients
$c_\gamma'(m)$ into not-normalized generalized cepstrum coefficients
$c_\gamma(m)$, the following equation can be used.
\begin{displaymath}
c_\gamma(m) = \left( c_\gamma'(0) \right)^{\gamma} c_\gamma'(m), \qquad m>0
\end{displaymath}
Also, the gain $K = c_\gamma(0)$ is
\begin{displaymath}
c_\gamma(0) = \begin{cases} \;\; \displaystyle
	  \frac{\Bigl(c_\gamma'(0)\Bigr)^{\gamma} - 1.0}{\gamma},
		& 0<|\gamma|\leq 1 \\ \;\; \displaystyle
	  \log c_\gamma'(0),  & \gamma=0
	\end{cases}
\end{displaymath}
\end{qsection}

\begin{options}
	\argm{m}{M}{order of generalized cepstrum}{25}
	\argm{g}{G}{power parameter $\gamma$ of generalized cepstrum\\
		    if $G>1.0$ then $\gamma=-1/G$.}{0}
\end{options}

\begin{qsection}{EXAMPLE}
In this example below,
normalized generalized cepstrum coefficients in
float format are read from {\em data.ngcep}$(M=15, \gamma=-0.5)$,
and the not-normalized generalized cepstrum coefficients
are outputed to {\em data.gcep}.
\begin{quote}
 \verb! ignorm -m 15 -g -0.5 < data.ngcep > data.gcep!
\end{quote} 
\end{qsection}

\begin{qsection}{SEE ALSO}
 gcep, mgcep, gc2gc, mgc2mgc, freqt
\end{qsection}
