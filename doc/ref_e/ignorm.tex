\name{ignorm}{inverse gain normalization}{signal processing}
\begin{synopsis}
\item [ignorm] [ --m $M$ ] [ --g $G$ ] [ {\em infile} ]
\end{synopsis}

\begin{qsection}{DESCRIPTION}
This command reads normalized generalized cepstrum coefficient
 $c_\gamma(m)$, and sends the not-normalized generalized
cepstrum coefficients to the standard output.
\par
Input and output data are in float format.
\par
To convert normalized generalized cepstrum coefficients
$c_\gamma'(m)$ into not-normalized generalized cepstrum coefficients
$c_\gamma(m)$, the following equation can be used.
\begin{displaymath}
c_\gamma(m) = \Bigl(c_\gamma'(0)\Bigr)^{\gamma} c_\gamma'(m), ~~~m>0
\end{displaymath}
Also, the gain $K = c_\gamma(0)$ is
\begin{displaymath}
c_\gamma(0) = \left\{
	\begin{array}{ll} \displaystyle
	  \frac{\Bigl(c_\gamma'(0)\Bigr)^{\gamma} - 1.0}{\gamma},
		& 0<|\gamma|\leq 1 \\ \displaystyle
	  \log c_\gamma'(0),  & \gamma=0
	\end{array} \right.
\end{displaymath}
\end{qsection}

\begin{options}
	\argm{m}{M}{order of generalized cepstrum}{25}
	\argm{g}{G}{power parameter $\gamma$ of generalized cepstrum\\
		    if $G>1.0$ then $\gamma=-1/G$.}{0}
\end{options}

\begin{qsection}{EXAMPLE}
In this example below,
normalized generalized cepstrum coefficients in
float format are read from {\em data.ngcep}$(M=15, \gamma=-0.5)$,
and the not-normalized generalized cepstrum coefficients
are outputed to {\em data.gcep}.
\begin{quote}
 \verb! ignorm -m 15 -g -0.5 < data.ngcep > data.gcep!
\end{quote} 
\end{qsection}

\begin{qsection}{SEE ALSO}
 gcep, mgcep, gc2gc, mgc2mgc, freqt
\end{qsection}
