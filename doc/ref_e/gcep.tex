\name[ref:gcep-IEICE,ref:gcep-IEEEASSP,ref:gcep-ICASSP90]{gcep}%
{generalized cepstral analysis}{speech analysis}

\begin{synopsis}
\item [gcep] [ --m $M$ ] [ --g $G$ ] [ --l $L$ ] [ --n ]
	     [ --i $I$ ] [ --j $J$ ] [ --d $D$ ] [ --e $E$ ]
\item [\ ~~~~~~~] [ {\em infile} ]
\end{synopsis}

\begin{qsection}{DESCRIPTION}
This command undertakes the generalized cepstral analysis,
evaluates normalized coefficient $c_\gamma'(m)$
and writes the result to the standard output.
The windowed input sequence of length $L$ is
\begin{displaymath}
  x(0),x(1),\ldots,x(L-1)
\end{displaymath}
\par
Input and output data are in float format.
\par
In the generalized cepstral analysis,
the speech spectrum is estimated by the $M$ order generalized
cepstrum $c_\gamma(m)$ or by normalized generalized cepstrum 
$c_\gamma'(m)$ using the log spectrum through the unbiased estimation
method.
\begin{eqnarray*}
H(z) &=& s_\gamma^{-1}\left(
	\sum_{m=0}^{M} c_\gamma(m)z^{-m} \right) \\
     &=& K \cdot s_\gamma^{-1}\left(
	\sum_{m=1}^{M} c_\gamma'(m)z^{-m} \right) \\
     &=& \left\{ \begin{array}{ll} \displaystyle
	K\cdot \left( 1+\gamma\sum_{m=1}^{M} c_\gamma'(m)z^{-m}
		\right)^{1/\gamma}, & -1 \leq \gamma < 0 \\
	\displaystyle K\cdot \exp \sum_{m=1}^{M} c_\gamma'(m)z^{-m}, 
		& \gamma=0
	\end{array} \right.
\end{eqnarray*}
To find the minimum value of cost function,
if $\gamma=-1$ then the linear prediction method is used,
while if $\gamma$ is different from $-1$, then
Newton--Raphson method applied.
\end{qsection}

\begin{options}
       -m m  :     [25]
       -g g  : gamma                            [0]
       -l l  :                      [256]
       -n    :        [FALSE]
     (level2)
       -i i  :                 [2]
       -j j  :                 [30]
       -d d  :                     [0.001]
       -e e  : small value added to periodgram  [0]
	\argm{m}{M}{order of generalized cepstrum}{25}
	\argm{g}{G}{$\gamma$. If $G>1.0$ then $\gamma=-1/G$.}{0}
	\argm{l}{L}{frame length}{256}
	\argm{n}{}{output normalized cepstrum}{FALSE}
	\desc[1zh]{Usually, the options below do not need to be assigned.}
	\argm{i}{I}{minimum iteration}{2}
	\argm{j}{J}{maximum iteration}{30}
	\argm{d}{D}{Newton-Raphson method end condition.
                    The default value is $D=0.001$.
                    In this case the end point is achieved when
                    the evaluation rate of $\varepsilon^{(i)}$ is
                    $0.001$, that is when its value changes less then $0.1\%$.}{0.001}
	\argm{e}{E}{small value added to periodgram}{0}
\end{options}

\begin{qsection}{EXAMPLE}
The following example read speech data in float format from {\em data.f}
file, undertakes the 15 order generalized cepstrum analysis,
and writes the results in {\em data.gcep}:
\begin{quote}
 \verb!frame < data.f | window | gcep -m 15 > data.gcep!
\end{quote} 
\end{qsection}

\begin{qsection}{SEE ALSO}
icep, uels, mcep, mgcep, glsadf
\end{qsection}
