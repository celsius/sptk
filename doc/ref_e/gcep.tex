% ----------------------------------------------------------------- %
%             The Speech Signal Processing Toolkit (SPTK)           %
%             developed by SPTK Working Group                       %
%             http://sp-tk.sourceforge.net/                         %
% ----------------------------------------------------------------- %
%                                                                   %
%  Copyright (c) 1984-2007  Tokyo Institute of Technology           %
%                           Interdisciplinary Graduate School of    %
%                           Science and Engineering                 %
%                                                                   %
%                1996-2011  Nagoya Institute of Technology          %
%                           Department of Computer Science          %
%                                                                   %
% All rights reserved.                                              %
%                                                                   %
% Redistribution and use in source and binary forms, with or        %
% without modification, are permitted provided that the following   %
% conditions are met:                                               %
%                                                                   %
% - Redistributions of source code must retain the above copyright  %
%   notice, this list of conditions and the following disclaimer.   %
% - Redistributions in binary form must reproduce the above         %
%   copyright notice, this list of conditions and the following     %
%   disclaimer in the documentation and/or other materials provided %
%   with the distribution.                                          %
% - Neither the name of the SPTK working group nor the names of its %
%   contributors may be used to endorse or promote products derived %
%   from this software without specific prior written permission.   %
%                                                                   %
% THIS SOFTWARE IS PROVIDED BY THE COPYRIGHT HOLDERS AND            %
% CONTRIBUTORS "AS IS" AND ANY EXPRESS OR IMPLIED WARRANTIES,       %
% INCLUDING, BUT NOT LIMITED TO, THE IMPLIED WARRANTIES OF          %
% MERCHANTABILITY AND FITNESS FOR A PARTICULAR PURPOSE ARE          %
% DISCLAIMED. IN NO EVENT SHALL THE COPYRIGHT OWNER OR CONTRIBUTORS %
% BE LIABLE FOR ANY DIRECT, INDIRECT, INCIDENTAL, SPECIAL,          %
% EXEMPLARY, OR CONSEQUENTIAL DAMAGES (INCLUDING, BUT NOT LIMITED   %
% TO, PROCUREMENT OF SUBSTITUTE GOODS OR SERVICES; LOSS OF USE,     %
% DATA, OR PROFITS; OR BUSINESS INTERRUPTION) HOWEVER CAUSED AND ON %
% ANY THEORY OF LIABILITY, WHETHER IN CONTRACT, STRICT LIABILITY,   %
% OR TORT (INCLUDING NEGLIGENCE OR OTHERWISE) ARISING IN ANY WAY    %
% OUT OF THE USE OF THIS SOFTWARE, EVEN IF ADVISED OF THE           %
% POSSIBILITY OF SUCH DAMAGE.                                       %
% ----------------------------------------------------------------- %
\hypertarget{gcep}{}
\name[ref:gcep-IEICE,ref:gcep-IEEEASSP,ref:gcep-ICSLP90]{gcep}%
{generalized cepstral analysis}{speech analysis}

\begin{synopsis}
\item [gcep] [ --m $M$ ] [ --g $G$ ] [ --c $C$ ] [ --l $L$ ] [ --q $Q$ ] [ --n ]
             [ --i $I$ ] [ --j $J$ ] [ --d $D$ ] [ --e $E$ ]
\item [\ ~~~~~~~] [ --f $F$ ] [ {\em infile} ]
\end{synopsis}

\begin{qsection}{DESCRIPTION}
{\em gcep} uses generalized cepstral analysis 
to calculate normalized cepstral coefficients $c_\gamma'(m)$ 
from $L$-length framed windowed input data 
from {\em infile} (or standard input), 
sending the result to standard output.
The windowed input sequence of length $L$ is of the form:
\begin{displaymath}
  x(0),x(1),\dots,x(L-1)
\end{displaymath}
\par
Input and output data are in float format.
\par
In the generalized cepstral analysis,
the speech spectrum is estimated by the $M$-th order generalized
cepstrum $c_\gamma(m)$ or by normalized generalized cepstrum 
$c_\gamma'(m)$ using the log spectrum through the unbiased estimation
method showed below.
\begin{align}
H(z) &= s_\gamma^{-1}\left(
        \sum_{m=0}^{M} c_\gamma(m)z^{-m} \right) \notag \\
     &= K \cdot s_\gamma^{-1}\left(
        \sum_{m=1}^{M} c_\gamma'(m)z^{-m} \right) \notag \\
     &= \begin{cases} \;\; \displaystyle
        K\cdot \left( 1+\gamma\sum_{m=1}^{M} c_\gamma'(m)z^{-m}
                \right)^{1/\gamma}, & -1 \leq \gamma < 0 \\
        \;\;\displaystyle K\cdot \exp \sum_{m=1}^{M} c_\gamma'(m)z^{-m}, 
                & \gamma=0
        \end{cases}\notag
\end{align}
In order to find the minimum value of the cost function,
the linear prediction method is used for $\gamma = -1$.
Otherwise, the Newton--Raphson method is applied.
\end{qsection}

\begin{options}
        \argm{m}{M}{order of generalized cepstrum}{25}
        \argm{g}{G}{gamma of generalized cepstrum\\
                         $\gamma=G$}{0}
        \argm{c}{C}{gamma of generalized cepstrum\\
                        $\gamma =-1 / $(int)$ C$\\
                        $C$ must be $C \geq 1$}{}
        \argm{l}{L}{frame length}{256}
        \argm{n}{}{output normalized cepstrum}{FALSE}
        \argm{q}{Q}{input data style\\
        \begin{tabular}{ll} \\[-1ex]
            $Q=0$ & windowed data sequence \\
            $Q=1$ & $20 \times \log |f(w)|$ \\
            $Q=2$ & $\ln |f(w)|$ \\
            $Q=3$ & $|f(w)|$ \\
            $Q=4$ & $|f(w)|^2$ \\[1ex]
        \end{tabular}\\\hspace*{\fill}}{0}
        \desc[1ex]{Usually, the options below do not need to be assigned.}
        \argm{i}{I}{minimum iteration}{2}
        \argm{j}{J}{maximum iteration}{30}
        \argm{d}{D}{Newton-Raphson method end condition.
                    The default value is $D=0.001$.
                    In this case, the end point is achieved when
                    the evaluation rate of $\varepsilon^{(i)}$ is
                    $0.001$, that is, when its value changes
                    in a rate smaller than $0.1\%$.}{0.001}
        \argm{e}{E}{small value added to periodgram}{0}
        \argm{f}{F}{mimimum value of the determinant of the normal matrix}{0.000001}
\end{options}

\begin{qsection}{EXAMPLE}
In the following example, speech data is read in float format from {\em data.f}
file, and a 15-th order generalized cepstral analysis is applied.
The results are written to {\em data.gcep}:
\begin{quote}
 \verb!frame +f < data.f | window | gcep -m 15 > data.gcep!
\end{quote} 
\end{qsection}

\begin{qsection}{SEE ALSO}
\hyperlink{uels}{uels},
\hyperlink{mcep}{mcep},
\hyperlink{mgcep}{mgcep},
\hyperlink{glsadf}{glsadf}
\end{qsection}
