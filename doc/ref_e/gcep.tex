% ----------------------------------------------------------------
%       Speech Signal Processing Toolkit (SPTK): version 3.0
%                      SPTK Working Group
% 
%                Department of Computer Science
%                Nagoya Institute of Technology
%                             and
%   Interdisciplinary Graduate School of Science and Engineering
%                Tokyo Institute of Technology
%                   Copyright (c) 1984-2000
%                     All Rights Reserved.
% 
% Permission is hereby granted, free of charge, to use and
% distribute this software and its documentation without
% restriction, including without limitation the rights to use,
% copy, modify, merge, publish, distribute, sublicense, and/or
% sell copies of this work, and to permit persons to whom this
% work is furnished to do so, subject to the following conditions:
% 
%   1. The code must retain the above copyright notice, this list
%      of conditions and the following disclaimer.
% 
%   2. Any modifications must be clearly marked as such.
%                                                                        
% NAGOYA INSTITUTE OF TECHNOLOGY, TOKYO INSITITUTE OF TECHNOLOGY,
% SPTK WORKING GROUP, AND THE CONTRIBUTORS TO THIS WORK DISCLAIM
% ALL WARRANTIES WITH REGARD TO THIS SOFTWARE, INCLUDING ALL
% IMPLIED WARRANTIES OF MERCHANTABILITY AND FITNESS, IN NO EVENT
% SHALL NAGOYA INSTITUTE OF TECHNOLOGY, TOKYO INSITITUTE OF
% TECHNOLOGY, SPTK WORKING GROUP, NOR THE CONTRIBUTORS BE LIABLE
% FOR ANY SPECIAL, INDIRECT OR CONSEQUENTIAL DAMAGES OR ANY
% DAMAGES WHATSOEVER RESULTING FROM LOSS OF USE, DATA OR PROFITS,
% WHETHER IN AN ACTION OF CONTRACT, NEGLIGENCE OR OTHER TORTIOUS
% ACTION, ARISING OUT OF OR IN CONNECTION WITH THE USE OR
% PERFORMANCE OF THIS SOFTWARE.
% ----------------------------------------------------------------
%
\hypertarget{gcep}{}
\name[ref:gcep-IEICE,ref:gcep-IEEEASSP,ref:gcep-ICSLP90]{gcep}%
{generalized cepstral analysis}{speech analysis}

\begin{synopsis}
\item [gcep] [ --m $M$ ] [ --g $G$ ] [ --l $L$ ] [ --n ]
	     [ --i $I$ ] [ --j $J$ ] [ --d $D$ ] [ --e $E$ ]
\item [\ ~~~~~~~] [ {\em infile} ]
\end{synopsis}

\begin{qsection}{DESCRIPTION}
{\em gcep} uses generalized cepstral analysis 
to calculate normalized cepstral coefficients $c_\gamma'(m)$ 
from $L$-length framed windowed input data 
from {\em infile} (or standard input), 
sending the result to standard output.
The windowed input sequence of length $L$ is
\begin{displaymath}
  x(0),x(1),\dots,x(L-1)
\end{displaymath}
\par
Input and output data are in float format.
\par
In the generalized cepstral analysis,
the speech spectrum is estimated by the $M$-th order generalized
cepstrum $c_\gamma(m)$ or by normalized generalized cepstrum 
$c_\gamma'(m)$ using the log spectrum through the unbiased estimation
method.
\begin{align}
H(z) &= s_\gamma^{-1}\left(
	\sum_{m=0}^{M} c_\gamma(m)z^{-m} \right) \notag \\
     &= K \cdot s_\gamma^{-1}\left(
	\sum_{m=1}^{M} c_\gamma'(m)z^{-m} \right) \notag \\
     &= \begin{cases} \;\; \displaystyle
	K\cdot \left( 1+\gamma\sum_{m=1}^{M} c_\gamma'(m)z^{-m}
		\right)^{1/\gamma}, & -1 \leq \gamma < 0 \\
	\;\;\displaystyle K\cdot \exp \sum_{m=1}^{M} c_\gamma'(m)z^{-m}, 
		& \gamma=0
	\end{cases}\notag
\end{align}
To find the minimum value of cost function,
if $\gamma=-1$ then the linear prediction method is used,
while if $\gamma$ is different from $-1$, then
Newton--Raphson method applied.
\end{qsection}

\begin{options}
	\argm{m}{M}{order of generalized cepstrum}{25}
	\argm{g}{G}{$\gamma$. If $G>1.0$ then $\gamma=-1/G$.}{0}
	\argm{l}{L}{frame length}{256}
	\argm{n}{}{output normalized cepstrum}{FALSE}
	\desc[1ex]{Usually, the options below do not need to be assigned.}
	\argm{i}{I}{minimum iteration}{2}
	\argm{j}{J}{maximum iteration}{30}
	\argm{d}{D}{Newton-Raphson method end condition.
                    The default value is $D=0.001$.
                    In this case the end point is achieved when
                    the evaluation rate of $\varepsilon^{(i)}$ is
                    $0.001$, that is when its value changes less then $0.1\%$.}{0.001}
	\argm{e}{E}{small value added to periodgram}{0}
\end{options}

\begin{qsection}{EXAMPLE}
The following example read speech data in float format from {\em data.f}
file, undertakes the 15-th order generalized cepstral analysis,
and writes the results in {\em data.gcep}:
\begin{quote}
 \verb!frame < data.f | window | gcep -m 15 > data.gcep!
\end{quote} 
\end{qsection}

\begin{qsection}{SEE ALSO}
\hyperlink{uels}{uels},
\hyperlink{mcep}{mcep},
\hyperlink{mgcep}{mgcep},
\hyperlink{glsadf}{glsadf}
\end{qsection}
