% ----------------------------------------------------------------- %
%             The Speech Signal Processing Toolkit (SPTK)           %
%             developed by SPTK Working Group                       %
%             http://sp-tk.sourceforge.net/                         %
% ----------------------------------------------------------------- %
%                                                                   %
%  Copyright (c) 1984-2007  Tokyo Institute of Technology           %
%                           Interdisciplinary Graduate School of    %
%                           Science and Engineering                 %
%                                                                   %
%                1996-2011  Nagoya Institute of Technology          %
%                           Department of Computer Science          %
%                                                                   %
% All rights reserved.                                              %
%                                                                   %
% Redistribution and use in source and binary forms, with or        %
% without modification, are permitted provided that the following   %
% conditions are met:                                               %
%                                                                   %
% - Redistributions of source code must retain the above copyright  %
%   notice, this list of conditions and the following disclaimer.   %
% - Redistributions in binary form must reproduce the above         %
%   copyright notice, this list of conditions and the following     %
%   disclaimer in the documentation and/or other materials provided %
%   with the distribution.                                          %
% - Neither the name of the SPTK working group nor the names of its %
%   contributors may be used to endorse or promote products derived %
%   from this software without specific prior written permission.   %
%                                                                   %
% THIS SOFTWARE IS PROVIDED BY THE COPYRIGHT HOLDERS AND            %
% CONTRIBUTORS "AS IS" AND ANY EXPRESS OR IMPLIED WARRANTIES,       %
% INCLUDING, BUT NOT LIMITED TO, THE IMPLIED WARRANTIES OF          %
% MERCHANTABILITY AND FITNESS FOR A PARTICULAR PURPOSE ARE          %
% DISCLAIMED. IN NO EVENT SHALL THE COPYRIGHT OWNER OR CONTRIBUTORS %
% BE LIABLE FOR ANY DIRECT, INDIRECT, INCIDENTAL, SPECIAL,          %
% EXEMPLARY, OR CONSEQUENTIAL DAMAGES (INCLUDING, BUT NOT LIMITED   %
% TO, PROCUREMENT OF SUBSTITUTE GOODS OR SERVICES; LOSS OF USE,     %
% DATA, OR PROFITS; OR BUSINESS INTERRUPTION) HOWEVER CAUSED AND ON %
% ANY THEORY OF LIABILITY, WHETHER IN CONTRACT, STRICT LIABILITY,   %
% OR TORT (INCLUDING NEGLIGENCE OR OTHERWISE) ARISING IN ANY WAY    %
% OUT OF THE USE OF THIS SOFTWARE, EVEN IF ADVISED OF THE           %
% POSSIBILITY OF SUCH DAMAGE.                                       %
% ----------------------------------------------------------------- %
\hypertarget{c2acr}{}
\name{c2acr}{transform cepstrum to autocorrelation}%
{speech parameter transformation}

\begin{synopsis}
\item[c2acr] [ --m $M_1$ ] [ --M $M_2$ ] [ --l $L$ ] [ {\em infile} ]
\end{synopsis}

\begin{qsection}{DESCRIPTION}
{\em c2acr} calculates $M_2$-th order autocorrelation coefficients 
from $M_1$-th order cepstral coefficients from {\em infile} (or standard input),
 writing the result to standard output.
Give cepstral coefficients
\begin{displaymath}
c(0), c(1), \ldots, c(M_1)
\end{displaymath}
the corresponding autocorrelation coefficients are
\begin{displaymath}
r(0), r(1), \ldots, r(M_2)
\end{displaymath}

The format of input and output format is float.

The power spectrum is calculated from the logarithm spectrum,
which is obtained from the Fourier transform of the $M_1$-th
order cepstral analysis.
The autocorrelation coefficients are obtained through the inverse
Fourier transform of the power spectrum.
\end{qsection}

\begin{options}
        \argm{m}{M_1}{order of cepstrum}{25}
        \argm{M}{M_2}{order of autocorrelation}{25}
        \argm{l}{L}{FFT length}{256}
\end{options}

\begin{qsection}{EXAMPLE}
The output file {\em data.lpc} contains the 15-th order linear prediction
coefficients calculated from the autocorrelation coefficients,
which were obtained from the 30-th order cepstral coefficients
file {\em data.cep}:
\begin{quote}
  \verb!c2acr -m 30 -M 15 < data.cep | levdur -m 15 > data.lpc!
\end{quote}
\end{qsection}

\begin{qsection}{SEE ALSO}
\hyperlink{uels}{uels},
\hyperlink{c2sp}{c2sp},
\hyperlink{c2ir}{c2ir},
\hyperlink{lpc2c}{lpc2c}
\end{qsection}
