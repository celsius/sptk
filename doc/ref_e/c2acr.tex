% ----------------------------------------------------------------
%       Speech Signal Processing Toolkit (SPTK): version 3.0
%                      SPTK Working Group
% 
%                Department of Computer Science
%                Nagoya Institute of Technology
%                             and
%   Interdisciplinary Graduate School of Science and Engineering
%                Tokyo Institute of Technology
%                   Copyright (c) 1984-2000
%                     All Rights Reserved.
% 
% Permission is hereby granted, free of charge, to use and
% distribute this software and its documentation without
% restriction, including without limitation the rights to use,
% copy, modify, merge, publish, distribute, sublicense, and/or
% sell copies of this work, and to permit persons to whom this
% work is furnished to do so, subject to the following conditions:
% 
%   1. The code must retain the above copyright notice, this list
%      of conditions and the following disclaimer.
% 
%   2. Any modifications must be clearly marked as such.
%                                                                        
% NAGOYA INSTITUTE OF TECHNOLOGY, TOKYO INSITITUTE OF TECHNOLOGY,
% SPTK WORKING GROUP, AND THE CONTRIBUTORS TO THIS WORK DISCLAIM
% ALL WARRANTIES WITH REGARD TO THIS SOFTWARE, INCLUDING ALL
% IMPLIED WARRANTIES OF MERCHANTABILITY AND FITNESS, IN NO EVENT
% SHALL NAGOYA INSTITUTE OF TECHNOLOGY, TOKYO INSITITUTE OF
% TECHNOLOGY, SPTK WORKING GROUP, NOR THE CONTRIBUTORS BE LIABLE
% FOR ANY SPECIAL, INDIRECT OR CONSEQUENTIAL DAMAGES OR ANY
% DAMAGES WHATSOEVER RESULTING FROM LOSS OF USE, DATA OR PROFITS,
% WHETHER IN AN ACTION OF CONTRACT, NEGLIGENCE OR OTHER TORTIOUS
% ACTION, ARISING OUT OF OR IN CONNECTION WITH THE USE OR
% PERFORMANCE OF THIS SOFTWARE.
% ----------------------------------------------------------------
%
\hypertarget{c2acr}{}
\name{c2acr}{transform cepstrum to autocorrelation}%
{speech parameter transformation}

\begin{synopsis}
\item[c2acr] [ --m $M_1$ ] [ --M $M_2$ ] [ --l $L$ ] [ {\em infile} ]
\end{synopsis}

\begin{qsection}{DESCRIPTION}
{\em c2acr} calculates $M_2$-order autocorrelation coefficients 
from $M_1$-order cepstrum coefficients from {\em infile} (or standard input), 
writing the result to standard output.
Give cepstrum coefficients
\begin{displaymath}
c(0), c(1), \ldots, c(M_1)
\end{displaymath}
the corresponding autocorrelation coefficients are
\begin{displaymath}
r(0), r(1), \ldots, r(M_2)
\end{displaymath}

The format of input and output format is float.

The power spectrum is calculated from the logarithm spectrum,
which is obtained from the Fourier transform of the $M_1$
order cepstrum analysis.
The autocorrelation coefficients are obtained through the inverse
Fourier transform of the power spectrum.
\end{qsection}

\begin{options}
	\argm{m}{M_1}{order of cepstrum}{25}
	\argm{M}{M_2}{order of autocorrelation}{25}
	\argm{l}{L}{FFT length}{256}
\end{options}

\begin{qsection}{EXAMPLE}
The output file {\em data.lpc} contains the 15th order linear prediction
coefficients calculated from the autocorrelation coefficients,
which were obtained from the 30th order cepstrum coefficients
file {\em data.cep}:
\begin{quote}
  \verb!c2acr -m 30 -M 15 < data.cep | lev_dur -n 15 > data.lpc!
\end{quote}
\end{qsection}

\begin{qsection}{SEE ALSO}
\hyperlink{uels}{uels},
\hyperlink{c2sp}{c2sp},
\hyperlink{c2ir}{c2ir},
\hyperlink{lpc2c}{lpc2c}
\end{qsection}
