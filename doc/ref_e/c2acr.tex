\name{c2acr}{Evaluation of autocorrelation coefficients from cepstrum
coefficients}%
{$B2;@<%Q%i%a!<%?JQ49(B}

\begin{synopsis}
\item[c2acr] [ --m $M_1$ ] [ --M $M_2$ ] [ --l $L$ ] [ {\em infile} ]
\end{synopsis}

\begin{qsection}{DESCRIPTION}
If the following $M_1$ cepstrum coefficients
are read from the standard input
\begin{displaymath}
c(0), c(1), \ldots, c(M_1)
\end{displaymath}
then the following $M_2$ autocorrelation coefficients
\begin{displaymath}
r(0), r(1), \ldots, r(M_2)
\end{displaymath}
are evaluated.
\par
Input and output formats are float$B!%(B
\par
The power spectrum is calculated from logarithm spectrum,
which is obtained from the Fourier transform of the $M_1$
order cepstrum analysis.
The autocorrelation coefficients are obtained through the inverse
Fourier transform of the power spectrum.
\end{qsection}

\begin{options}
	\argm{m}{M_1}{order of cepstrum}{25}
	\argm{M}{M_2}{order of autocorrelation}{25}
	\argm{l}{L}{FFT length}{256}
\end{options}

\begin{qsection}{EXAMPLE}
The output file {\em data.lpc} contains the 15th order linear prediction
coefficients calculated from the autocorrelation coefficients,
which were obtained from 30th order cepstrum coefficients
file {\em data.cep}:
\begin{quote}
  \verb!c2acr -m 30 -M 15 < data.cep | lev_dur -n 15 > data.lpc!
\end{quote}
\end{qsection}

\begin{qsection}{SEE ALSO}
  uels, c2sp, c2ir, lpc2c 
\end{qsection}
