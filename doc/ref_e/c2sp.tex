% ----------------------------------------------------------------
%       Speech Signal Processing Toolkit (SPTK): version 3.0
%                      SPTK Working Group
% 
%                Department of Computer Science
%                Nagoya Institute of Technology
%                             and
%   Interdisciplinary Graduate School of Science and Engineering
%                Tokyo Institute of Technology
%                   Copyright (c) 1984-2000
%                     All Rights Reserved.
% 
% Permission is hereby granted, free of charge, to use and
% distribute this software and its documentation without
% restriction, including without limitation the rights to use,
% copy, modify, merge, publish, distribute, sublicense, and/or
% sell copies of this work, and to permit persons to whom this
% work is furnished to do so, subject to the following conditions:
% 
%   1. The code must retain the above copyright notice, this list
%      of conditions and the following disclaimer.
% 
%   2. Any modifications must be clearly marked as such.
%                                                                        
% NAGOYA INSTITUTE OF TECHNOLOGY, TOKYO INSITITUTE OF TECHNOLOGY,
% SPTK WORKING GROUP, AND THE CONTRIBUTORS TO THIS WORK DISCLAIM
% ALL WARRANTIES WITH REGARD TO THIS SOFTWARE, INCLUDING ALL
% IMPLIED WARRANTIES OF MERCHANTABILITY AND FITNESS, IN NO EVENT
% SHALL NAGOYA INSTITUTE OF TECHNOLOGY, TOKYO INSITITUTE OF
% TECHNOLOGY, SPTK WORKING GROUP, NOR THE CONTRIBUTORS BE LIABLE
% FOR ANY SPECIAL, INDIRECT OR CONSEQUENTIAL DAMAGES OR ANY
% DAMAGES WHATSOEVER RESULTING FROM LOSS OF USE, DATA OR PROFITS,
% WHETHER IN AN ACTION OF CONTRACT, NEGLIGENCE OR OTHER TORTIOUS
% ACTION, ARISING OUT OF OR IN CONNECTION WITH THE USE OR
% PERFORMANCE OF THIS SOFTWARE.
% ----------------------------------------------------------------
%
\name{c2sp}{transform cepstrum to spectrum}
{speech parameter transformation}

\begin{synopsis}
\item[c2sp] [ --m $M$ ] [ --l $L$ ] [ --p ] [ --o $O$ ] [ {\em infile} ]
\end{synopsis}

\begin{qsection}{DESCRIPTION}
This command evaluates the spectrum from the minimum phase
 cepstrum.
 Input and output data are in float format.
\end{qsection}

\begin{options}
	\argm{m}{M}{order of cepstrum}{25}
	\argm{l}{L}{frame length}{256}
	\argm{p}{}{output phase}{FALSE}
	\argm{o}{O}{output format\\
                if the ``--p'' option is not assigned then
                \\
		\begin{tabular}{ll} \\[-1zh]
			$O=0$ & $20 \times \log |H(z)|$ \\
			$O=1$ & $\ln |H(z)|$ \\
			$O=2$ & $|H(z)|$ \\[1zh]
		\end{tabular}\\
		if the ``--p'' option is assigned then
		\\
		\begin{tabular}{ll}\\[-1zh]
			$O=0$ & $\arg |H(z)| \div \pi \quad [\pi \; rad.]$ \\
			$O=1$ & $\arg |H(z)| \quad [rad.]$ \\
			$O=2$ & $\arg |H(z)| \times180\div\pi\quad[deg.]$ \\
		\end{tabular}\\\hspace*{\fill}}{0}
\end{options}

\begin{qsection}{EXAMPLE}
The example below takes the 15 order cepstrum from the file
 {\em data.cep} in float format, evaluates the running spectrum,
 and presents it in the screen:
\begin{quote}
 \verb! c2sp -m 15 data.cep | grlogsp | xgr ! 
\end{quote}
\end{qsection}

\begin{qsection}{SEE ALSO}
uels, mgc2sp
\end{qsection}
