%  ---------------------------------------------------------------  %
%            Speech Signal Processing Toolkit (SPTK)                %
%                      SPTK Working Group                           %
%                                                                   %
%                  Department of Computer Science                   %
%                  Nagoya Institute of Technology                   %
%                               and                                 %
%   Interdisciplinary Graduate School of Science and Engineering    %
%                  Tokyo Institute of Technology                    %
%                                                                   %
%                     Copyright (c) 1984-2007                       %
%                       All Rights Reserved.                        %
%                                                                   %
%  Permission is hereby granted, free of charge, to use and         %
%  distribute this software and its documentation without           %
%  restriction, including without limitation the rights to use,     %
%  copy, modify, merge, publish, distribute, sublicense, and/or     %
%  sell copies of this work, and to permit persons to whom this     %
%  work is furnished to do so, subject to the following conditions: %
%                                                                   %
%    1. The source code must retain the above copyright notice,     %
%       this list of conditions and the following disclaimer.       %
%                                                                   %
%    2. Any modifications to the source code must be clearly        %
%       marked as such.                                             %
%                                                                   %
%    3. Redistributions in binary form must reproduce the above     %
%       copyright notice, this list of conditions and the           %
%       following disclaimer in the documentation and/or other      %
%       materials provided with the distribution.  Otherwise, one   %
%       must contact the SPTK working group.                        %
%                                                                   %
%  NAGOYA INSTITUTE OF TECHNOLOGY, TOKYO INSTITUTE OF TECHNOLOGY,   %
%  SPTK WORKING GROUP, AND THE CONTRIBUTORS TO THIS WORK DISCLAIM   %
%  ALL WARRANTIES WITH REGARD TO THIS SOFTWARE, INCLUDING ALL       %
%  IMPLIED WARRANTIES OF MERCHANTABILITY AND FITNESS, IN NO EVENT   %
%  SHALL NAGOYA INSTITUTE OF TECHNOLOGY, TOKYO INSTITUTE OF         %
%  TECHNOLOGY, SPTK WORKING GROUP, NOR THE CONTRIBUTORS BE LIABLE   %
%  FOR ANY SPECIAL, INDIRECT OR CONSEQUENTIAL DAMAGES OR ANY        %
%  DAMAGES WHATSOEVER RESULTING FROM LOSS OF USE, DATA OR PROFITS,  %
%  WHETHER IN AN ACTION OF CONTRACT, NEGLIGENCE OR OTHER TORTUOUS   %
%  ACTION, ARISING OUT OF OR IN CONNECTION WITH THE USE OR          %
%  PERFORMANCE OF THIS SOFTWARE.                                    %
%                                                                   %
%  ---------------------------------------------------------------  %
%
\hypertarget{c2sp}{}
\name{c2sp}{transform cepstrum to spectrum}
{speech parameter transformation}

\begin{synopsis}
\item[c2sp] [ --m $M$ ] [ --l $L$ ] [ --p ] [ --o $O$ ] [ {\em infile} ]
\end{synopsis}

\begin{qsection}{DESCRIPTION}
{\em c2sp} calculates the spectrum from the minimum phase cepstrum 
from {\em infile} (or standard input), 
sending the result to standard output.
Input and output data are in float format.
\end{qsection}

\begin{options}
	\argm{m}{M}{order of cepstrum}{25}
	\argm{l}{L}{frame length}{256}
	\argm{p}{}{output phase}{FALSE}
	\argm{o}{O}{output format\\
                if the ``--p'' option is not assigned then
                \\
		\begin{tabular}{ll} \\[-1ex]
			$O=0$ & $20 \times \log |H(z)|$ \\
			$O=1$ & $\ln |H(z)|$ \\
			$O=2$ & $|H(z)|$ \\[1ex]
		\end{tabular}\\
		if the ``--p'' option is assigned then
		\\
		\begin{tabular}{ll}\\[-1ex]
			$O=0$ & $\arg |H(z)| \div \pi \quad [\pi \; rad.]$ \\
			$O=1$ & $\arg |H(z)| \quad [rad.]$ \\
			$O=2$ & $\arg |H(z)| \times180\div\pi\quad[deg.]$ \\
		\end{tabular}\\\hspace*{\fill}}{0}
\end{options}

\begin{qsection}{EXAMPLE}
The example below takes the 15-th order cepstrum from the file
 {\em data.cep} in float format, evaluates the running spectrum,
 and presents it in the screen:
\begin{quote}
 \verb! c2sp -m 15 data.cep | grlogsp | xgr ! 
\end{quote}
\end{qsection}

\begin{qsection}{SEE ALSO}
\hyperlink{uels}{uels},
\hyperlink{mgc2sp}{mgc2sp}
\end{qsection}
